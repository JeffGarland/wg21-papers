\newcommand\wgTitle{\texttt{= delete} should have a reason}
\newcommand\wgName{Matthias Kretz <m.kretz@gsi.de>, Augustin?, LWG ?}
\newcommand\wgDocumentNumber{DXXXXR0}
\newcommand\wgPreviousDocumentNumber{N4186}
\newcommand\wgGroup{EWG-I}
\newcommand\wgTarget{\CC{}23}
%\newcommand\wgAcknowledgements{ }

\usepackage{mymacros}
\usepackage{wg21}
\usepackage{changelog}
\usepackage{underscore}

\newcommand\wglink[1]{\href{https://wg21.link/#1}{#1}}

\begin{document}
\selectlanguage{american}
\begin{wgTitlepage}
  Good library implementations care for helpful compiler error messages.
  Especially in template-heavy code, error messages can be harder to understand than guessing what might be wrong in the code (thanks to the line \& column pointer the compiler provides).
  The \code{static_assert} message is often used for this purpose but requires to be placed inside a function body and thus cannot be used in cases where SFINAE-friendlyness is important/necessary.
  A SFINAE-friendly approach is to provide a deleted overload that catches the invalid invocations.
  If \code{= delete(“had a reason”)} that could help library implementers who want to be helpful to their users.
\end{wgTitlepage}

\pagestyle{scrheadings}

\section{Changelog}
\begin{revision}
  \item Added Wording.
\end{revision}

\section{Straw Polls}
\subsection{LEWG telecon 2022-03-29}
\wgPoll
{Numeric traits can deviate from \code{numeric_limits}.}
{13&8&0&0&0}

\wgPoll
{Numeric traits should be based on representation rather than behavior (ignoring \code{reciprocal_overflow_threshold}).}
{7&5&2&0&0}

\wgPoll
{All numeric traits for bool should be disabled.}
{12&6&1&0&0}

\wgPoll
{The numeric traits that are not meaningful for \code{numeric_limits} (\code{denorm_min}, \code{epsilon}, etc) should be disabled for integral types.}
{14&3&0&0&0}

\wgPoll
{\code{max_digits10} should deviate from \code{numeric_limits} and yields \code{digits10_v + 1}.}
{6&5&2&0&0}

\subsection{LEWG telecon 2022-06-07}
\wgPoll
{Remove \code{reciprocal_overflow_threshold} from P1841.}
{6&4&1&0&0}


\section{Introduction}

\subsection{Previous work}
N4186 was reviewed in Urbana-Champaign 2014 and the summary on the wiki says “Reviewed, further work encouraged”.
The main concern in the review at the time was:
\begin{quote}{}
“I have the impression that we're heading towards a fully configurable language where for every single thing that could be ill-formed we'll have to provide this hook.
I don't think that's the right way to go for designing a language.”
\end{quote}
Since then we have voted \wglink{P1301R4} (\code{[[nodiscard(“should have a reason”)]]}) into the IS.
And it seems the prediction is right.
However, the question whether “thats' the right way to go for designing a language” did not get a new answer/idea in the meantime.
To the contrary, adding parenthesis and a string is the existing practice throughout \CC{} for adding custom diagnostic messages (at least for attributes).

\section{Existing practice}
clang's \code{__attribute__((enable_if(condition))) __attribute__((unavailable(“message”)))} achieves what a concepts requires clause together with a \code{= delete(“message”)} should do.

GCC's warning attributes and \code{\#pragma}.

ICC, MSVC?

\code{\#warning}

\section{Alternative approach}
\begin{lstlisting}[style=Vc]
[[diag("this turns into a warning")]];

[[diag("f only takes int and does not allow implicit conversions.")]]
template <class T> void f(T) = delete;

void f(int);

[[diag("oh well, you really should convert to int manually.")]]
void f(short);

void g() {
  f(1);        // ok
  f(short(1)); // "warning: oh well, you really should convert to int manually."
  f(1.);       // ill-formed, printing "error: f only takes int and does not allow implicit conversions."
}

constexpr int h(int i) {
  if (std::is_constant_evaluated()) {
    return i + 1;
  }
  [[diag("TODO: an optimized implementation at runtime is missing")]];
  return 1;
}

constexpr int a = h(1);  // no diagnostics
\end{lstlisting}

Pros:
\begin{itemize}
  \item Implementations may ignore the attribute if they believe those messages are useless noise for their users.
  \item An attribute provides a generic facility to provide custom diagnostics in more places than only deleted functions.
\end{itemize}
Cons:
\begin{itemize}
  \item Potentially, it is unclear when a warning or error is intended.
    Maybe it is better to add two attributes \code{warn} and \code{fail} (bikesheddable, of course) instead.
    However, it is unclear whether \code{[[fail("msg")]]} should make the expression it appertains to ill-formed.
    This would make it problematic for an implementation to ignore the attribute.
  \item More verbose than extending \code{= delete("msg")}
  \item The attribute is far apart from the \code{= delete}.
\end{itemize}

\section{Wording}
is waiting for guidance

\end{document}
% vim: sw=2 sts=2 ai et tw=0
