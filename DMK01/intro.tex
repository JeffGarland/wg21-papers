\section{Introduction}

For more information on \datapar see \cite{P0214R0} and \cite{Kretz2015}.

\subsection{Hardware Applicability}
The assumption that \datapar[<T>::size()] yields a constant expression on every conceivable hardware implementation for efficient data parallel execution is limiting.
There exist vector machines that advertise a variable vector length.
There may be research going into future hardware to loosen the vector width on SIMD instructions.
Consequently, the static member functions \code{static constexpr size_t size()} in \datapar and \mask may inhibit or at least reduce the applicability of the programming model to the widest range of available and future hardware.

\subsection{Optimization Opportunities}
Consider the following code snippet:
\smallskip\begin{lstlisting}[style=Vc]
void f(const float *input, float *output, size_t N) {
  using V = datapar<float>;
  for (int i = 0; i + V::size() <= N; i += V::size()) {
    const V x = load<V>(&input[i]);
    store(x + 1, &output[i]);
  }
  // process the rest
}
\end{lstlisting}
The compiler sees a loop that reads \code{V::size()} elements from \code{input} and writes the same number of elements to \code{output} before going into the next iteration.
The compiler traditionally would do alias analysis if it wants to determine whether it can reorder the loads and stores, as they could be dependent.
With \datapar there is an opportunity to loosen the rules.
Consider that \datapar[<float>] returns a different number for \code{size()} depending on the target system and available SIMD instruction set.
The type did not require any specific number of elements.
If the compiler thus were allowed to assume \type V to have any \code{size()}, it could assume \code{size() == N}.
In this case it would not have to do the alias analysis and could unroll and reorder more freely.

Enabling this optimization has a major impact.
Consider x86 where the \float add instruction has a latency of 3 cycles and a throughput of 1 cycle.
If the sequence of instructions must follow \textit{load x0; x0 += 1; store x0; load x1; x1 += 1; store x1; \ldots}, then a single loop iteration requires the sum of latencies from \textit{load + add + store} (\textasciitilde{}5 cycles).
If, however, the sequence of operations can be reordered to e.g. \textit{load x0; load x1; x0 += 1; load x2; load x3; x1 += 1; load x4; load x5; x2 += 1; load x6; store x0; x3 += 1; store x1; \dots}, then the add instructions can be enqueued efficiently in the execution pipeline.
The efficiency of the loop can thus be lowered to \textasciitilde{}1 cycle per iteration.

However, consider the following calling code (in a different TU to make things worse):
\smallskip \begin{lstlisting}[style=Vc]
void g() {
  float data[1000] = ...;
  if (datapar<float>::size() <= 8) {
    f(data, data + 8, 1000 - 8);
  }
}
\end{lstlisting}
The user ensured that the load store dependencies are met.
The above optimization thus would break the users intention.

% vim: sw=2 sts=2 ai et tw=0 spell
