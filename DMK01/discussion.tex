\section{Discussion}

Having a \code{constexpr} \code{size()} in \datapar and \mask for ABI parameters that are not \type{fixed_size<N>} thus appears to inhibit the full potential of expressing data parallelism via the type system.
Is it possible to weaken the \code{size()} function to resolve the issues?
At this point I have no solution.
This paper is meant to set a starting point to either figuring out a solution or a well-informed decision to not pursue this as a design goal any further.

Considering that the inhibitor is restricted load-store reordering because of aliasing, the best and most general solution may be to enable \code{restrict} or something similar in \CC{}.

\subsection{No \code{sizeof}}
Let us start with the simple loop:
\smallskip \begin{lstlisting}[style=Vc]
using V = datapar<float>;
for(int i = 0; i < N; i += V::size()) {
  store(1 + load(&input[i]), &output[i]);
}
\end{lstlisting}
If the compiler shall generate code for hardware that determines the vector width only when executing the SIMD load/store/add instructions, then \code{V::size()} cannot be a constant expression.
Rather, the machine code would have to increment \code i using a value obtained from some register or instruction.
(If \code{V::size()} is a constant expression, then the increment can be encoded into the instruction itself.)
Would dropping \code{constexpr} from \code{size()} help?
At least it would decouple the number of elements from the type information.
However, as a consequence \code{sizeof(V)} could also not return any useful value, since one could infer the \code{constexpr size()} from \code{sizeof}.
Thus, the declaration of structures with \datapar members could not have a useful \code{sizeof}.

The major advantage of types with data-parallel semantics over the control-flow approach is the vectorization of data structures.
Suddenly this would not work anymore.
At this point this approach does not show much appeal.

\subsection{A \code{variable} ABI}
Consider a special ABI tag, called \code{datapar_abi::variable} in the following.
An instance of \datapar[<T, variable>] would not have a \code{constexpr size()} function, no \code{sizeof}, and using it as a data member would be ill-formed.
So what is it useable for?
\smallskip\begin{lstlisting}[style=Vc]
using V = datapar<float, datapar_abi::variable>;
for (int i = 0; i < N; i += V::size()) {
  store(1 + load(&input[i]), &output[i]);
}
\end{lstlisting}
In essence this would be equivalent to writing (using some hypothetical \code{simd_for} loop construct, which allows vector execution of the loop body):
\smallskip\begin{lstlisting}[style=Vc]
simd_for (int i = 0; i < N; ++i) {
  output[i] = 1 + input[i];
}
\end{lstlisting}

I cannot image any advantage of this construction over control-flow based vectorization except maybe the ability to mark functions as vectorizable.

In any case, calling it \datapar[<T, variable>] is asking for repeating the \std\type{vector<bool>} mistake.
Therefore, it should rather use a name such as \type{dyndatapar<T>}\footnote{Yes, it would likely repeat many of the discussions around \type{dynarray}.}

\subsection{Parallel Algorithms}
Consider writing the loop as
\smallskip\begin{lstlisting}[style=Vc]
transform(execution_policy::datapar,
          begin(input), end(input), begin(output) [](auto x) {
  return x + 1;
});
\end{lstlisting}
Here the use of \datapar is implicit.
The implementation chooses the number of elements.
The user has no influence on choosing the \code{datapar_abi} and thus has no guarantee for the \code{size()} outside of the loop.
(Inside the loop the code can still query \code{x.size()} and choose different branches.)
But at this point the compiler has all necessary information to use a variable vector width or to unroll the loop and reorder loads and stores.

Is this still true if there are function calls from inside the callable?
The following code can still be assumed to work correctly with an arbitrary number of elements in \datapar, because the \code{Abi} parameter is unconstrained:
\smallskip\begin{lstlisting}[style=Vc]
template <typename T, typename Abi>
datapar<T, Abi> f(datapar<T, Abi> x) { return x + 1; }

transform(execution_policy::datapar,
          begin(input), end(input), begin(output) [](auto x) {
  return f(x);
});
\end{lstlisting}

If, however, the \code{Abi} parameter is constrained, the compiler would have to be conservative.
It would have to assume that the maximum number of elements it may reorder is bounded by the maximum number of elements the overloads of \code g support.
\smallskip\begin{lstlisting}[style=Vc]
template <typename T>
datapar<T, datapar_abi::avx> g(datapar<T, datapar_abi::avx> x) { return x + 1; }
template <typename T>
datapar<T, datapar_abi::sse> g(datapar<T, datapar_abi::sse> x) { return x + 1; }
template <typename T>
datapar<T, datapar_abi::scalar> g(datapar<T, datapar_abi::scalar> x) { return x + 1; }

transform(execution_policy::datapar,
          begin(input), end(input), begin(output) [](auto x) {
  return g(x);
});
\end{lstlisting}
However, it is unlikely that the above code could work with all implementations of \code{transform}.
An implementation might choose to use a \datapar[<T, fixed_size<N>>], e.g. for prologue and epilogue.
Thus, the user would have to provide a \datapar[<T, fixed_size<N>>] overload:
\smallskip\begin{lstlisting}[style=Vc]
template <typename T, int N>
datapar<T, datapar_abi::fixed_size<N>> g(datapar<T, datapar_abi::fixed_size<N>> x) {
  return x + 1;
}
template <typename T>
datapar<T, datapar_abi::avx> g(datapar<T, datapar_abi::avx> x) { return x + 1; }
template <typename T>
datapar<T, datapar_abi::sse> g(datapar<T, datapar_abi::sse> x) { return x + 1; }
template <typename T>
datapar<T, datapar_abi::scalar> g(datapar<T, datapar_abi::scalar> x) { return x + 1; }

transform(execution_policy::datapar,
          begin(input), end(input), begin(output) [](auto x) {
  return g(x);
});
\end{lstlisting}
And suddenly the vector width is unbounded again.

% vim : sw = 2 sts = 2 ai et tw = 0 spell
