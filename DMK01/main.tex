\newcommand\wgTitle{Assumptions about the size of datapar}
\newcommand\wgName{Matthias Kretz <m.kretz@gsi.de>}
\newcommand\wgDocumentNumber{D0000R0}
\newcommand\wgGroup{SG1}
\newcommand\wgAcknowledgements{
  This work was supported by GSI Helmholtzzentrum für Schwerionenforschung
  and the Hessian LOEWE initiative through the Helmholtz International Center
  for FAIR (HIC for FAIR).
}

\usepackage{underscore}
\usepackage{typenames}
\usepackage{mymacros}
\usepackage{wg21}

\addbibresource{extra.bib}

\newcommand\datapar[1][]{\type{datapar#1}\xspace}
\newcommand\valuetype{\type{value\_type}\xspace}
\newcommand\simdcast{\code{datapar\_cast}\xspace}
\newcommand\mask[1][]{\type{mask#1}\xspace}
\newcommand\fixedsizeN{\type{datapar\_abi::fixed\_size<N>}\xspace}
\newcommand\fixedsizescoped{\type{datapar\_abi::fixed\_size}\xspace}
\newcommand\fixedsize{\type{fixed\_size}\xspace}

\begin{document}
\selectlanguage{american}
\begin{wgTitlepage}
  This paper discusses consequences of using \datapar for architectures with variable vector width.
  The assumptions a compiler can make about the \code{size()} of \datapar has consequences for optimizations.
  On the other hand, this requires restrictions on \datapar[::size()] that may be surprising to users.
\end{wgTitlepage}

\pagestyle{scrheadings}
\section{Introduction}

Parallel Algorithms enable implementations of the existing STL algorithms to use non-sequential semantics when executing the user-supplied code (explicit callable or implicit operator call).
The first argument to the algorithm function determines this change in execution semantics via an \emph{execution policy}.
This paper introduces a new execution policy, called \dataparEP.
\dataparEP requires user-provided function objects to be callable with \datapar[<T, Abi>] arguments instead of the \type T arguments the \seqEP variant would use.
The algorithm therefore processes chunks of \datapar[<T, Abi>::size()] objects concurrently.
The execution order of the chunks retains the sequential semantics of the non-parallel algorithms.

As a consequence, the applicability of the execution policy is limited to iterators where \datapar[<Iterator::value_type>] is a valid template instantiation of \datapar.
A future extension of \datapar may lift this restriction by allowing certain (or all) user-defined types as first template argument to \datapar.


\end{document}
% vim: sw=2 sts=2 ai et tw=0
