\wgSubsection{Class template \type{mask}}{mask}
\wgSubsubsection{Class template \mask overview}{mask.overview}
\lstinputlisting[]{mask.cpp}

\pnum The class template \mask[<T, Abi>] is a one-dimensional smart array of booleans.
The number of elements in the array is determined at compile time, equal to the number of elements in \datapar[<T, Abi>].

\pnum The first template argument \type T must be an integral or floating-point fundamental type.
The type \bool is not allowed.

\pnum The second template argument \type{Abi} must be a tag type from the \code{datapar_abi} namespace.

\begin{itemdecl}
typedef implementation_defined native_handle_type;
\end{itemdecl}
\begin{itemdescr}
  \pnum The \type{native_handle_type} member type is an alias for the \code{native_handle()} member function return type.
  It is used to expose an implementation-defined handle for implementation- and target-specific extensions.
\end{itemdescr}

\begin{itemdecl}
static constexpr size_type size();
\end{itemdecl}
\begin{itemdescr}
  \pnum\returns the number of boolean elements stored in objects of the given \mask[<T, Abi>] type.
\end{itemdescr}

\wgSubsubsection{\mask constructors}{mask.ctor}
\begin{itemdecl}
mask() = default;
\end{itemdecl}
\begin{itemdescr}
  \pnum\effects Constructs an object with all elements initialized to \code{bool()}.
  \wgNote{This zero-initializes the object.}
\end{itemdescr}

\begin{itemdecl}
mask(value_type);
\end{itemdecl}
\begin{itemdescr}
  \pnum\effects Constructs an object with each element initialized to the value of the argument.
\end{itemdescr}

\begin{itemdecl}
template <class U> mask(mask<U, Abi> x);
\end{itemdecl}
\begin{itemdescr}
  \pnum\remarks This constructor shall not participate in overload resolution unless
    \datapar[<U, Abi>] is implicitly convertible to \datapar[<T, Abi>].
  \pnum\effects Constructs an object of type \mask where the $i$-th element equals \code{x[i]} \foralli.
\end{itemdescr}

\wgSubsubsection{\mask{} subscript operators}{mask.subscr}
\begin{itemdecl}
reference operator[](size_type i);
\end{itemdecl}
\begin{itemdescr}
  \dataparElementReference
\end{itemdescr}

\begin{itemdecl}
value_type operator[](size_type) const;
\end{itemdecl}
\begin{itemdescr}
  \pnum\returns A copy of the $i$-th element.
\end{itemdescr}

\wgSubsubsection{\mask unary operators}{mask.unary}
\begin{itemdecl}
mask operator!() const;
\end{itemdecl}
\begin{itemdescr}
  \pnum\returns A mask object with the $i$-th element set to the logical negation \foralli.
\end{itemdescr}

\wgSubsubsection{\mask native handles}{mask.native}
\begin{itemdecl}
native_handle_type &native_handle();
\end{itemdecl}
\begin{itemdescr}
  \pnum\returns An lvalue reference to the implementation-specific object implementing the data-parallel semantics.
\end{itemdescr}

\begin{itemdecl}
const native_handle_type &native_handle() const;
\end{itemdecl}
\begin{itemdescr}
  \pnum\returns A \const lvalue reference to the implementation-specific object implementing the data-parallel semantics.
\end{itemdescr}

\wgSubsection{\type{mask} non-member operations}{mask.nonmembers}

\wgSubsubsection{\mask load function}{mask.load}
\begin{itemdecl}
template <class T, class Flags = flags::load_default> T load(const bool *, Flags = Flags{});
\end{itemdecl}
\begin{itemdescr}
  \pnum\remarks This function shall not participate in overload resolution unless \type{Flags} is one of the tag types in the \code{flags} namespace and \code{is_mask_v<T>}.
  \pnum\returns A new \mask object with each element $i$ initialized to \code{x[i]} \foralli.
  \pnum\remarks If \code{\type T::size()} returns a value greater than the number of values pointed to by the first argument, the behavior is undefined.
  \pnum\remarks If the \type{Flags} template parameter is of type \type{flags::aligned_tag} and the pointer value is not a multiple of \code{memory_alignment<\type T>}, the behavior is undefined.
\end{itemdescr}

\wgSubsubsection{\mask store functions}{mask.store}
\begin{itemdecl}
template <class T, class Abi, class Flags = flags::load_default>
void store(const mask<T, Abi> &x, bool *y, Flags = Flags{});
\end{itemdecl}
\begin{itemdescr}
  \pnum\remarks This function shall not participate in overload resolution unless \type{Flags} is one of the tag types in the \code{flags} namespace.
  \pnum\effects Copies each element \code{x[i]} to \code{y[i]} \foralli.
  \pnum\remarks If \mask[<T, Abi>]\code{::size()} is greater than the number of values pointed to by \code y, the behavior is undefined.
  \pnum\remarks If the \type{Flags} template parameter is of type \type{flags::aligned_tag} and the pointer value of \code y is not a multiple of \code{memory_alignment<\mask[<T, Abi>]>}, the behavior is undefined.
\end{itemdescr}

\begin{itemdecl}
template <class T0, class A0, class T1, class A1, class Flags = flags::load_default>
void store(const mask<T0, A0> &x, bool *y, const mask<T1, A1> &k, Flags = Flags{});
\end{itemdecl}
\begin{itemdescr}
  \pnum\remarks This function shall not participate in overload resolution unless \type{Flags} is one of the tag types in the \code{flags} namespace and \mask[<T1, A1>] is implicitly convertible to \mask[<T0, A0>].
  \pnum\effects Copies each element \code{x[i]} where \code{k[i]} is \true to \code{y[i]} \foralli.
  \pnum\remarks If the largest $i$ where \code{k[i]} is \true is greater than the number of values pointed to by \code y, the behavior is undefined.
  \wgNote{
    Masked stores only write to the bytes in memory selected by the \code k argument.
    This prohibits implementations that load, blend, and store the complete vector.
  }
  \pnum\remarks If the \type{Flags} template parameter is of type \type{flags::aligned_tag} and the pointer value of \code y is not a multiple of \code{memory_alignment<\datapar[<T0, A0>], \type U>}, the behavior is undefined.
\end{itemdescr}

\wgSubsubsection{\mask binary operators}{mask.binary}
\begin{itemdecl}
template <class T0, class A0, class T1, class A1> using mask_return_type = ...  // exposition only
template <class T0, class A0, class T1, class A1>
mask_return_type<T0, A0, T1, A1> operator&&(const mask<T0, A0> &, const mask<T1, A1> &);
template <class T0, class A0, class T1, class A1>
mask_return_type<T0, A0, T1, A1> operator||(const mask<T0, A0> &, const mask<T1, A1> &);
template <class T0, class A0, class T1, class A1>
mask_return_type<T0, A0, T1, A1> operator& (const mask<T0, A0> &, const mask<T1, A1> &);
template <class T0, class A0, class T1, class A1>
mask_return_type<T0, A0, T1, A1> operator| (const mask<T0, A0> &, const mask<T1, A1> &);
template <class T0, class A0, class T1, class A1>
mask_return_type<T0, A0, T1, A1> operator^ (const mask<T0, A0> &, const mask<T1, A1> &);
\end{itemdecl}
\comment{I think we need an additional \code{operator@(const mask<T, A> \&, const mask<T, A> \&)} overload to enable use of objects that are implicitly convertible to more than one \mask instantiation with equal priority.}
\begin{itemdescr}
  \newcommand\maskreturntype{\type{mask_return_type<T, Abi, U>}\xspace}
  \pnum\remarks The return type, \maskreturntype, shall be \mask[<\common{T0}{T1}, \commonabi{A0}{A1}{\common{T0}{T1}}>].
    The functions \code{\textit{common}} and \code{\textit{commonabi}} identify the functions described in \ref{unusual conversions}.

  \pnum\remarks Each of these operators only participate in overload resolution if both arguments are implicitly convertible to \maskreturntype.

  \pnum\returns A new \mask object initialized with the results of the component-wise application of the indicated operator.
\end{itemdescr}

\wgSubsubsection{\mask compares}{mask.comparison}
\begin{itemdecl}
template <class T0, class A0, class T1, class A1>
bool operator==(const mask<T0, A0> &, const mask<T1, A1> &);
\end{itemdecl}
\comment{ditto.}
\newcommand\maskCompareRemark{\pnum\remarks This operator only participates in overload resolution if \mask[<T0, A0>] is implicitly convertible to \mask[<T1, A1>] or \mask[<T1, A1>] is implicitly convertible to \mask[<T0, A0>].}
\begin{itemdescr}
  \maskCompareRemark
  \pnum\returns \true if all boolean elements of the first argument equal the corresponding element of the second argument.
  It returns \false otherwise.
\end{itemdescr}

\begin{itemdecl}
template <class T0, class A0, class T1, class A1>
bool operator!=(const mask<T0, A0> &a, const mask<T1, A1> &b);
\end{itemdecl}
\comment{ditto.}
\begin{itemdescr}
  \maskCompareRemark
  \pnum\returns \code{!operator==(a, b)}.
\end{itemdescr}

\wgSubsubsection{\mask reductions}{mask.reductions}
\begin{itemdecl}
template <class T, class Abi> bool  all_of(mask<T, Abi>);
constexpr bool  all_of(bool);
\end{itemdecl}
\begin{itemdescr}
  \pnum\returns \true if all boolean elements in the function argument equal \true, \false otherwise.
\end{itemdescr}

\begin{itemdecl}
template <class T, class Abi> bool  any_of(mask<T, Abi>);
constexpr bool  any_of(bool);
\end{itemdecl}
\begin{itemdescr}
  \pnum\returns \true if at least one boolean element in the function argument equals \true, \false otherwise.
\end{itemdescr}

\begin{itemdecl}
template <class T, class Abi> bool none_of(mask<T, Abi>);
constexpr bool none_of(bool);
\end{itemdecl}
\begin{itemdescr}
  \pnum\returns \true if none of the boolean element in the function argument equals \true, \false otherwise.
\end{itemdescr}

\begin{itemdecl}
template <class T, class Abi> bool some_of(mask<T, Abi>);
constexpr bool some_of(bool);
\end{itemdecl}
\begin{itemdescr}
  \pnum\returns \true if at least one of the boolean elements in the function argument equals \true and at least one of the boolean elements in the function argument equals \false, \false otherwise.
  \pnum\realnote \code{some_of(bool)} unconditionally returns \false.
\end{itemdescr}

\begin{itemdecl}
template <class T, class Abi> int popcount(mask<T, Abi>);
constexpr int popcount(bool);
\end{itemdecl}
\begin{itemdescr}
  \pnum\returns The number of boolean elements that are \true.
\end{itemdescr}

\begin{itemdecl}
template <class T, class Abi> int find_first_set(mask<T, Abi> m);
\end{itemdecl}
\begin{itemdescr}
  \pnum\returns The lowest element index \code i where \code{m[i] == true}.
  \pnum\remarks If \code{none_of(m) == true} the behavior is undefined.
\end{itemdescr}

\begin{itemdecl}
template <class T, class Abi> int find_last_set(mask<T, Abi> m);
\end{itemdecl}
\begin{itemdescr}
  \pnum\returns The highest element index \code i where \code{m[i] == true}.
  \pnum\remarks If \code{none_of(m) == true} the behavior is undefined.
\end{itemdescr}

\begin{itemdecl}
constexpr int find_first_set(bool);
constexpr int find_last_set(bool);
\end{itemdecl}
\begin{itemdescr}
  \pnum\returns 0 if the argument is \true.
\end{itemdescr}

\wgSubsubsection{Masked assigment}{mask.where}
\begin{itemdecl}
template <class T0, class A0, class T1, class A1>
implementation_defined where(const mask<T0, A0> &m, datapar<T1, A1> &v);
\end{itemdecl}
\begin{itemdescr}
  \pnum\remarks The function only participates in overload resolution if \mask[<T0, A0>] is implicitly convertible to \mask[<T1, A1>].

  \pnum\returns A temporary object with the following properties:
  \begin{itemize}
    \item The object is not \textit{CopyConstructible}.
    \item Assignment and compound assignment operators only participate in overload resolution if the corresponding operator for \datapar[<T1, A1>] is usable.
    \item \effects Assignment and compound assignment implement the same semantics as the corresponding operator for \datapar[<T1, A1>] with the exception that elements of \code v stay unmodified if the corresponding boolean element in \code m is \false.
    \item The assignment and compound assignment operators return \void.
  \end{itemize}
\end{itemdescr}

\begin{itemdecl}
template <class T> implementation_defined where(bool, T &);
\end{itemdecl}
\begin{itemdescr}
  \pnum\remarks The function only participates in overload resolution if \type T is a fundamental arithmetic type.

  \pnum\returns A temporary object with the following properties:
  \begin{itemize}
    \item The object is not \textit{CopyConstructible}.
    \item Assignment and compound assignment operators only participate in overload resolution if the corresponding operator for \type T is usable.
    \item \effects If the first argument is \false, the assignment operators do nothing.
      If the first argument is \true, the assignment operators forward to the corresponding builtin assignment operator.
    \item The assignment and compound assignment operators return \void.
  \end{itemize}
\end{itemdescr}
