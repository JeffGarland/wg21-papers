\newcommand\wgTitle{\code{std::integral\_constant} literals do not suffice --- \code{constexpr\_t}?}
\newcommand\wgName{Matthias Kretz <m.kretz@gsi.de>}
\newcommand\wgDocumentNumber{P2772R0}
\newcommand\wgGroup{LEWG-I, LEWG}
\newcommand\wgTarget{\CC{}26}
%\newcommand\wgAcknowledgements{ }

\usepackage{mymacros}
\usepackage{wg21}
%\setcounter{tocdepth}{2} % show sections and subsections in TOC
%\hypersetup{bookmarksdepth=5}
\usepackage{changelog}
\usepackage{underscore}
%\usepackage{comment}

\addbibresource{extra.bib}

\renewcommand{\lst}[1]{Listing~\ref{#1}}
\renewcommand{\sect}[1]{Section~\ref{#1}}
\renewcommand{\ttref}[1]{Tony~Table~\ref{#1}}

\begin{document}
\selectlanguage{american}
\begin{wgTitlepage}
  \textcite{P2725R0} proposes user-defined literals for simpler use of
  \code{std::integral_constant}, simplifying basically a notion of passing
  \code{constexpr} function arguments.
  I fully support the idea, but I believe it does not cover the complete
  problem \& design space.
  In this paper I show the use cases and solutions that I believe need to be
  considered at the same time.
\end{wgTitlepage}

\pagestyle{scrheadings}

%\section{Changelog}
\begin{revision}
  \item Added Wording.
\end{revision}

%\section{Straw Polls}
\subsection{LEWG telecon 2022-03-29}
\wgPoll
{Numeric traits can deviate from \code{numeric_limits}.}
{13&8&0&0&0}

\wgPoll
{Numeric traits should be based on representation rather than behavior (ignoring \code{reciprocal_overflow_threshold}).}
{7&5&2&0&0}

\wgPoll
{All numeric traits for bool should be disabled.}
{12&6&1&0&0}

\wgPoll
{The numeric traits that are not meaningful for \code{numeric_limits} (\code{denorm_min}, \code{epsilon}, etc) should be disabled for integral types.}
{14&3&0&0&0}

\wgPoll
{\code{max_digits10} should deviate from \code{numeric_limits} and yields \code{digits10_v + 1}.}
{6&5&2&0&0}

\subsection{LEWG telecon 2022-06-07}
\wgPoll
{Remove \code{reciprocal_overflow_threshold} from P1841.}
{6&4&1&0&0}


\section{Introduction}
I am convinced we need simpler and shorter syntax for passing constant
expressions to functions.
Especially if the function cannot easily resort to an NTTP (operator overload
or member function that often would require the \code{template} keyword when
called).

My solution idea started from \lst{lst:idea1}, which uses the variable template
\begin{lstlisting}[numbers=left,float={hbtp},label=lst:idea1,caption={
    \code{integral_constant} from variable template
}]
template <auto N>
inline constexpr std::integral_constant<decltype(N), N> Const = {};

template <typename T>
struct my_complex
{
  T re, im;
};

template <typename T>
struct X
{
  void f(auto c) {
    // c can be used in constant expressions here
  }
};

inline constexpr short foo = 2;

template <typename T>
void g(X<T> x) {
  x.f(Const<1>);
  x.f(Const<2uz>);
  x.f(Const<3.0>);
  x.f(Const<4.f>);
  x.f(Const<foo>); // P2725R0 doesn't solve this
  x.f(Const<my_complex(1.f, 1.f)>); // nor this
}
\end{lstlisting}
\code{Const} for constructing objects of type \code{integral_constant}.
When passed as deduced function parameter, the value can be used in constant
expressions in the function body.
In \lst{lst:idea1} the alternative is an NTTP to the function \code{f}, making
all calls in \code{g} look like \code{x.template f<1>()}.

The first four calls to \code{f} in \lst{lst:idea1} are possible with P2725R0,
but the last two are not.
P2725R0 can only turn integer literals into \code{integral_constant}s.
The problem space is larger than what P2725R0 solves.
Nevertheless, integer literals are a common case and therefore the solution of
P2725R0 seems what we want, just incomplete.
I think a viable outcome could be to add both `1ic` and `std::cnst<1>` at the
same time.\footnote{I'd like to write \code{std::const<1>}, but arrgh.
  `std::constant<1>` is a bit too long for my taste.}
I believe there is no good rationale for adding \emph{only}
\code{integral_constant} literals.
Simple tasks such as, how do I write an \code{integral_constant} for
\code{INT_MAX}, \code{std::numeric_limits<int>::max()}, or any other constexpr
variable?
Should \std\code{integral_constant<decltype(foo), foo>\{\}} really be our only answer?

\section{wait, what? \code{integral_constant<double>}?}
Oh, not to forget.
I instantiated \code{integral_constant<double>} (and \code{float} and
\code{my_complex}) in \lst{lst:idea1}.
\code{integral_constant} is misnamed nowadays.
Should it be constrained to integers (for no good reason other than the name)?
Or should we consider a new type so that our type names can still be used to carry intent?

A type for passing any possible NTTP could e.g. be named \code{constexpr_t}:
\medskip\begin{lstlisting}[style=Vc]
template <auto Value>
struct constexpr_t {
  using value_type = decltype(Value);
  using type = constexpr_t;
  static inline constexpr value_type value = Value;
  constexpr operator value_type() const noexcept { return Value; }
  static constexpr value_type operator()() noexcept { return Value; }
};
\end{lstlisting}

\section{A compile-time numeric type}
\textcite{P2725R0} proposes the addition of unary minus to \code{integral_constant}.
That's a breaking change, as shown by \lst{lst:breaking}.
\begin{lstlisting}[numbers=left,float={hbtp},label=lst:breaking,caption={
    Adding unary minus to \code{integral_constant} is a breaking change
}]
void f(std::same_as<int> auto);

void g(auto x) {
  f(-x); //valid now, ill-formed with P2725R0:
}

void h() {
  g(std::integral_constant<int, 1>());
}
\end{lstlisting}

In addition, the return type of unary minus is controversial.
\code{-short(1)} is of type \code{int}.
Whether you dislike integral promotions it or not, that would be inconsistent
with \code{integral_constant\MayBreak{}::\MayBreak{}operator-()} returning
\code{integral_constant<short, ...>}.

While the proposed return type seems to be an improvement, what about a
user-defined structural type that returns a different type on unary minus?
\code{bounded\MayBreak{}::\MayBreak{}integer} \cite{site.bounded-integer} is an example of such a
type (though not structural).
Example:
\medskip\begin{lstlisting}
  bounded::integer<1, 10> a;
  auto b = -a; // b is bounded::integer< -10, -1>
\end{lstlisting}
For such a case we need \code{integral_constant::operator-} to return
\code{decltype(-std\MayBreak{}::\MayBreak{}declval<T>())}.

Finally, adding only unary minus is inconsistent.
We should then also add unary plus and unary tilde (bit flip).

And why stop with unary operators?
Binary operators are also missing.

If we want an \code{integral_constant} type that implements unary minus, I
believe we need to have a new type.
E.g. \code{std::numeric_constant<auto value>} that requires the NTTP to have the properties of a numeric type.
Such a type would then overload all operators accordingly, similar to \code{boost::hana::integral_constant}.
See \url{https://godbolt.org/z/vdzzdKdKz}.


\section{Context \& unbaked explorations}
My original angle was the exploration of possible APIs for simple integration
of \code{std::simd} into the ranges and container world.

\begin{enumerate}
  \item \code{std::simd::size} shouldn't be a function, but an
    \code{integral_constant}.
    (You can still call it like a function.)
    \code{std::array::size} should also be changed to be an
    \code{integral_constant} (same for spans of static extent).

    Changing \code{array::size} should be a non-breaking change.
    The user code that could get broken is not allowed AFAIU.
    (“Moreover, the behavior of a C++ program is unspecified (possibly
      ill-formed) [\ldots] if it attempts to form a pointer-to-member
      designating [\ldots] a standard library non-static member function [\ldots].”
    \url{https://eel.is/c++draft/constraints#namespace.std-6})

  \item I've been playing with (needs my GCC branch for non-member
    \code{operator[]}):
\medskip\begin{lstlisting}
constexpr std::span<...>
operator[](std::ranges::contiguous_range auto&&,
           index_like auto first, index_like auto size)
\end{lstlisting}
    I.e. similar to \code{submdrange}'s \code{strided_index_range}, I was
    looking at index ranges.
    If you pass an \code{integral_constant} size, you get a \code{span} of
    static extent which can CTAD into a \code{std::simd}.
    Result:
\medskip\begin{lstlisting}[style=Vc]
std::vector<float> x = ...;
std::simd v = x[0, 8ic];
std::simd w = x[0, std::simd<float>::size];
\end{lstlisting}
    Literals as proposed by \textcite{P2725R0} make this code much simpler to read and write!
\end{enumerate}

For the second point, my prototype code is at:\\
\url{https://github.com/mattkretz/index-range-subscripting/blob/main/subscript.h#L71}

Given:
\medskip\begin{lstlisting}[style=Vc]
  std::vector<float> data;
  const auto& cdata = data;
\end{lstlisting}

I can write:
\begin{enumerate}
  \item \lstinline@data[1u, Const<4>]@ returning a \lstinline@span<float, 4>@

  \item \lstinline@cdata[1u, Const<4>]@ returning a \lstinline@span<const float, 4>@

  \item \lstinline@data[1, Const<4>]@ returning a \lstinline@span<float, dynamic_extent>@

  \item \lstinline@data[1u, Const<-4>]@ returning a \lstinline@span<float, dynamic_extent>@

  \item \lstinline@data[-1, Const<-4>]@ returning a \lstinline@span<float, dynamic_extent>@

  \item \lstinline@data[Const<-1>, Const<-4>]@ ERROR: index range results in negative size

  \item \lstinline@data[Const<-4>, Const<-1>]@ returning a \lstinline@span<float, 4>@

  \item \lstinline@data[Const<-1>, Const<4>]@ ERROR: index range exceeds bounds

  \item \lstinline@(data | transform(...))[1, 5]@ returning a \lstinline@subrange<...>@
\end{enumerate}

And, can I have more crazy\ldots? After I can write
\medskip\begin{lstlisting}[style=Vc]
  std::simd v = data[1u, std::simd<float>::size];
\end{lstlisting}

I want to write
\medskip\begin{lstlisting}[style=Vc]
  data[1u, v.size] = v;
\end{lstlisting}

This requires a new \code{span::operator=}.
Or a non-member \code{operator=} so that I can implement it as a hidden friend
in \code{simd}?

The more I think of it, the more I like the direction.
Originally, I wanted to only allow non-member \code{operator[]} and non-member
\code{operator?:} overloads.
But by now I'm ready to propose that we should simply make all operators the
same, i.e. allow non-member overloads for \code{[]}, \code{()}, \code{=}, and
\code{->} in addition to allowing member and non-member operator?: (even if I
  see little use for member \code{operator?:} --- but consistency wins over my
imagination).


\end{document}
% vim: sw=2 sts=2 ai et tw=0
