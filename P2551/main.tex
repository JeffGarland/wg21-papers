\newcommand\wgTitle{Clarify intent of P1841 numeric traits}
\newcommand\wgName{Matthias Kretz <m.kretz@gsi.de>\\Jonathan Wakely <cxx@kayari.org>}
\newcommand\wgDocumentNumber{D2551R0}
\newcommand\wgGroup{LEWG}
\newcommand\wgTarget{\CC{}23}
%\newcommand\wgAcknowledgements{ }

\usepackage{mymacros}
\usepackage{wg21}
\usepackage{changelog}
\usepackage{underscore}

\addbibresource{extra.bib}

\newcommand\wglink[1]{\href{https://wg21.link/#1}{#1}}

\begin{document}
\selectlanguage{american}
\begin{wgTitlepage}
  A list of design-related questions after implementation of \cite{P1841R2} “Wording for Individually Specializable Numeric Traits”.
\end{wgTitlepage}

\pagestyle{scrheadings}

\section{Introduction}
\cite{P1841R2} provides wording for numeric traits.
The last design paper was \cite{P0437R1} with additions from \cite{P1370R1}.

\section{Design Questions}
\begin{enumerate}
  \item When exactly is a trait disabled for a given numeric type?
  It seems the intent was for the \code{value} member to be defined whenever a representation for the desired constant exists.
  The wording needs to clarify whether any behavioral aspects play a role.
  For example, a \code{denorm_min} may be enabled independent of whether the execution environment flushes denormals to zero / treats denormals as zero.
  Even in the case of a processor that unconditionally zeros denormals; as long as a representation exists, is the trait enabled?
  Conversely, if a representation does not exist, is the trait disabled?
  Specifically, \code{denorm_min} should never have the value of \code{norm_min}?

  \item Please clarify whether we want to treat \code{bool} as a numeric type and enable the traits accordingly.
  The current wording in \cite{P1841R2} enables the traits for \code{bool}, which is consistent with \code{std::numeric_limits}.
  \code{std::numeric_limits<bool>} will still exist if needed.
  Numeric code does not use \code{bool} as a numeric type, despite it being technically an “arithmetic type” in the core language.

  \item Many of the numeric traits are motivated by floating-point and make little sense for integral types.
  Is it intended that all of the following numeric traits are enabled also for integral types?
  \begin{itemize}
    \item \code{denorm_min}
    \item \code{epsilon}
    \item \code{norm_min}
    \item \code{reciprocal_overflow_threshold}
    \item \code{round_error}
    \item \code{max_exponent}
    \item \code{max_exponent10}
    \item \code{min_exponent}
    \item \code{min_exponent10}
  \end{itemize}

  \item reciprocal_overflow_threshold yields a subnormal number for IEC559 types.
  How should this value change wrt. treat-denormals-as-zero?
  I.e. in a situation where the hardware treats subnormal operands as zero you get 1/0 -> inf, which does overflow.
  In which case it doesn't match the specification anymore (“The smallest positive value $x$ of type \type{T} such that \code{T(1)}$/x$ does not overflow”).
  This trait is specified by a behavior and as such may depend on processor state.
  As a compile-time constant this value must be independent from runtime behavior.
  But what is the correct value?

  \item \code{numeric_limits::max_digits10} is 0 for integral types.
  Is \code{max_digits10_v<int>} supposed to yield \code{digits10_v<int> + 1}?
  Or should it only be specialized for floating-point?
\end{enumerate}

\section{Suggested Straw Polls}
\wgPoll{Whether a numeric trait is enabled is independent of processor behavior and only reflects whether a representation for the requested trait exists (ignoring \code{reciprocal_overflow_threshold}).}{&&&&}

\wgPoll{All numeric traits for \code{bool} should be disabled.}{&&&&}

\wgPoll{The numeric traits listed item 3 in \wgDocumentNumber{} should be disabled for integral types.}{&&&&}

\wgPoll{\code{reciprocal_overflow_threshold} should be independent of processor behavior and only reflect the value range of possible representations of the given type.}{&&&&}

\wgPoll{\code{reciprocal_overflow_threshold} should reflect processor behavior if it is known at compile-time (e.g. the target hardware unconditionally treats denormals as zero), otherwise it should reflect the value range of possible representations of the given type.}{&&&&}

\wgPoll{\code{max_digits10} should deviate from \code{numeric_limits} and yields \code{digits10_v<T> + 1}.}{&&&&}

%\section{Changelog}
\begin{revision}
  \item Added Wording.
\end{revision}

%\section{Straw Polls}
\subsection{LEWG telecon 2022-03-29}
\wgPoll
{Numeric traits can deviate from \code{numeric_limits}.}
{13&8&0&0&0}

\wgPoll
{Numeric traits should be based on representation rather than behavior (ignoring \code{reciprocal_overflow_threshold}).}
{7&5&2&0&0}

\wgPoll
{All numeric traits for bool should be disabled.}
{12&6&1&0&0}

\wgPoll
{The numeric traits that are not meaningful for \code{numeric_limits} (\code{denorm_min}, \code{epsilon}, etc) should be disabled for integral types.}
{14&3&0&0&0}

\wgPoll
{\code{max_digits10} should deviate from \code{numeric_limits} and yields \code{digits10_v + 1}.}
{6&5&2&0&0}

\subsection{LEWG telecon 2022-06-07}
\wgPoll
{Remove \code{reciprocal_overflow_threshold} from P1841.}
{6&4&1&0&0}


\end{document}
% vim: sw=2 sts=2 ai et tw=0
