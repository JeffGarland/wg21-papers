\newcommand\wgTitle{Clarify intent of P1841 numeric traits}
\newcommand\wgName{Matthias Kretz <m.kretz@gsi.de>\\Jonathan Wakely <cxx@kayari.org>}
\newcommand\wgDocumentNumber{D2551R1}
\newcommand\wgGroup{LEWG}
\newcommand\wgTarget{\CC{}23}
%\newcommand\wgAcknowledgements{ }

\usepackage{mymacros}
\usepackage{wg21}
\usepackage{changelog}
\usepackage{underscore}

\addbibresource{extra.bib}

\newcommand\wglink[1]{\href{https://wg21.link/#1}{#1}}

\begin{document}
\selectlanguage{american}
\begin{wgTitlepage}
  A list of design-related questions after implementation of \cite{P1841R2} “Wording for Individually Specializable Numeric Traits”.
\end{wgTitlepage}

\pagestyle{scrheadings}

\addtocounter{section}{-1}
\section{Changelog}
\begin{revision}
  \item Added Wording.
\end{revision}

\section{Introduction}
\cite{P1841R2} provides wording for numeric traits.
The last design paper was \cite{P0437R1} with additions from \cite{P1370R1}.
Most of the open questions were answered in LEWG already.
The question on \code{reciprocal_overflow_threshold} was deferred to let the authors of \cite{P1370R1} and this paper determine the original intent and its consequences.


\section{Remaining Design Questions}
\begin{enumerate}
  \item Decision already taken in LEWG.

  \item Decision already taken in LEWG.

  \item Decision already taken in LEWG.

  \item \code{reciprocal_overflow_threshold} is currently defined as:
  \begin{wgText}[{P1841R2 [num.traits.val]}]
\begin{itemdecl}
  template <class T> struct reciprocal_overflow_threshold<T> { @\emph{see below}@ };
\end{itemdecl}
\begin{itemdescr}
\setcounter{Paras}{8}
  \pnum The smallest positive value $x$ of type \type{T} such that \code{T(1)}$/x$ does not overflow.
\end{itemdescr}
  \end{wgText}
  This yields a subnormal number for IEC559 types.
  How should this value change wrt. treat-denormals-as-zero?
  I.e. in a situation where the hardware treats subnormal operands as zero you get 1/0 -> inf, which does overflow.
  In which case it doesn't match the specification anymore.
  This trait is specified by a behavior and as such may depend on processor state.
  As a compile-time constant this value must be independent from runtime behavior.
  But what is the correct value?
  See \url{https://godbolt.org/z/eWxdnTYf8} for a demonstration of the problem.

  Update after consultation with Mark and Damien (the P1370R1 authors):
  \begin{itemize}
  \item It would be possible to decouple the specification from runtime behavior by specifying behavior of constant expressions only;
    i.e. that \code{T(1)}$/x$ does not overflow \emph{in a constant expression}.
  \item P1370R1 presented an algorithm to determine the value and it does not yield the “\emph{smallest} positive value $x$ of type \type{T} such that \code{T(1)}$/x$ does not overflow”.
  \item The P1370R1 algorithm seems to ensure that the value is never subnormal.
    Thus, the specification should have been “The smallest positive \emph{normal} value $x$ of type \type{T} such that \code{T(1)}$/x$ does not overflow”
  \item If \code{reciprocal_overflow_threshold} is limited to normal values we're not sure who would use the \code{reciprocal_overflow_threshold} constant.
    It seems simpler and safer to remove the constant from P1841.
  \end{itemize}

  \item Decision already taken in LEWG.
\end{enumerate}

\section{Suggested Straw Polls}

\wgPoll{Remove \code{reciprocal_overflow_threshold} from P1841.}{&&&&}

\noindent If the above poll doesn't reach consensus:\\

\wgPoll{Specify the behavior of \code{1 / reciprocal_overflow_threshold} only for constant expressions.}{&&&&}

\wgPoll{Require \code{reciprocal_overflow_threshold} to be a normal number.}{&&&&}

\section{Straw Polls}
\subsection{LEWG telecon 2022-03-29}
\wgPoll
{Numeric traits can deviate from \code{numeric_limits}.}
{13&8&0&0&0}

\wgPoll
{Numeric traits should be based on representation rather than behavior (ignoring \code{reciprocal_overflow_threshold}).}
{7&5&2&0&0}

\wgPoll
{All numeric traits for bool should be disabled.}
{12&6&1&0&0}

\wgPoll
{The numeric traits that are not meaningful for \code{numeric_limits} (\code{denorm_min}, \code{epsilon}, etc) should be disabled for integral types.}
{14&3&0&0&0}

\wgPoll
{\code{max_digits10} should deviate from \code{numeric_limits} and yields \code{digits10_v + 1}.}
{6&5&2&0&0}

\subsection{LEWG telecon 2022-06-07}
\wgPoll
{Remove \code{reciprocal_overflow_threshold} from P1841.}
{6&4&1&0&0}



\end{document}
% vim: sw=2 sts=2 ai et tw=0
