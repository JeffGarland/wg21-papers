\newcommand\wgTitle{Expected Feedback from \code{simd} in the Parallelism TS 2}
\newcommand\wgName{Matthias Kretz <m.kretz@gsi.de>}
\newcommand\wgDocumentNumber{DXXXXR0}
\newcommand\wgGroup{SG1 / LEWG}
%\newcommand\wgTarget{Parallelism TS 2}
%\newcommand\wgAcknowledgements{ }

\usepackage{mymacros}
\usepackage{wg21}
\usepackage{underscore}

\addbibresource{extra.bib}

\newcommand\simd[1][]{\type{simd#1}\xspace}
\newcommand\simdT{\type{simd<T>}\xspace}
\newcommand\valuetype{\type{value\_type}\xspace}
\newcommand\referencetype{\type{reference}\xspace}
\newcommand\whereexpression{\type{where\_expression}\xspace}
\newcommand\simdcast{\code{simd\_cast}\xspace}
\newcommand\mask[1][]{\type{simd\_mask#1}\xspace}
\newcommand\maskT{\type{simd\_mask<T>}\xspace}
\newcommand\fixedsizeN{\type{simd\_abi::fixed\_size<N>}\xspace}
\newcommand\fixedsizescoped{\type{simd\_abi::fixed\_size}\xspace}
\newcommand\fixedsize{\type{fixed\_size}\xspace}
\newcommand\simdEP{\code{execution::}\type{simd}\xspace}
\newcommand\seqEP{\code{execution::}\type{seq}\xspace}

\usepackage{pifont}

\newcommand\foralli[1][]{for all \code i $\in$ \code{[0, #1size())}\xspace}
\newcommand\forallmaskedi[1]{%
  for all \code i
  $\in \{j \in \mathbb{N}_0 | j < \code{size()} ⋀ \code{#1[}j\code{]}\}$%
  \xspace%
}
\newcommand\chck{\item[\color{black}\ensuremath{\checkmark}]}
\newcommand\todo{\item[\color{black}\ding{46}] \color{gray}}
\newcommand\itemheader[1]{\item[] \hfill \textcolor{gray}{\textsc{#1}}}

\begin{document}
\selectlanguage{american}
\begin{wgTitlepage}
  This paper collects the questions we hope to answer via the TS process.
\end{wgTitlepage}

\pagestyle{scrheadings}
\section{Introduction}
\textcite{P0214R6} introduces a type-based expression of data parallelism.
The paper targets the Parallelism TS 2.
Through the TS process we hope to get feedback on a number of issues that were not clear to the committee before hands-on experience.
This paper collects all the questions and design choices that SG1 / LEWG might want to revisit before moving \simdT into the IS.

\section{Questions}

\subsection{Names of \code{popcount}, \code{find_first}, \code{find_last}}
\url{https://github.com/mattkretz/wg21-papers/issues/63}

\subsection{Names \code{copy_from}, \code{copy_to}}
\url{https://github.com/mattkretz/wg21-papers/issues/62}

\subsection{Name of \code{where} function}
\url{https://github.com/mattkretz/wg21-papers/issues/61}

\subsection{Name of \simdT misleading?}
\url{https://github.com/mattkretz/wg21-papers/issues/60}

\subsection{Return types of split and concat unconditionally \fixedsizeN?}
\url{https://github.com/mattkretz/wg21-papers/issues/59}

\subsection{Allowed template arguments for (\code{static_})\code{simd_cast}}
\url{https://github.com/mattkretz/wg21-papers/issues/57}

\subsection{Should converting loads/stores be safer wrt. conversions}
\url{https://github.com/mattkretz/wg21-papers/issues/34}

Since implicit conversions on broadcasts are restricted to value-preserving conversions, the load/store interface is inconsistent with this strictness.
I.e.:

\begin{lstlisting}[style=Vc]
using V = simd<float>;
V(1.);  // error, double -> float conversion is not value-preserving
V(short(1));  // good, short -> float is value-preserving
double mem[V::size()] = {1.};
V(mem, flags::element_aligned);  // converting load, compiles (but shouldn't?)
short mem[V::size()] = {1};
V(mem, flags::element_aligned);  // value-preserving conversion, should compile?
\end{lstlisting}

The conversion is not obvious when you look at the line of code that requests the load.

Options:
\begin{enumerate}
  \item Drop converting loads and stores.
    (That would be unfortunate, since doing converting loads and stores efficiently and portably is otherwise a hard problem.)

  \item Use different functions, e.g. \code{v.copy_from(mem, flags)} requires \code{mem} to be a \valuetype array. \code{v.memload_cvt_safe(mem, flags)} allows value-preserving conversions.
    \code{v.memload_cvt_unsafe(mem, flags)} allows all conversions.

  \item Use a flag to enable conversions (without cvt flag, no conversions are allowed), e.g.
    \begin{lstlisting}[style=Vc]
    V(mem, flags::element_aligned | flags::safe_cvt);
    V(mem, flags::element_aligned | flags::any_cvt);
    V(mem, flags::element_aligned | flags::saturating_cvt);  // new feature
    \end{lstlisting}
    As a variation, safe conversions could be enabled per default and only unsafe conversions would require extra typing.

\end{enumerate}

\subsection{Consider more implicit conversions}
\url{https://github.com/mattkretz/wg21-papers/issues/26} and \url{https://github.com/mattkretz/wg21-papers/issues/3}

\subsection{Relation Operators}
\textcite{P0820R1} argues for a substantial change to the definition of relation operators.
\textcite{P0851R0} presents the reason for the status quo.
The choices are:
\begin{enumerate}
  \item All relops are defined and return \mask (status quo).
  \item Compares returning \mask are provided via new functions (member vs. non-member?)
    \begin{enumerate}
      \item No relops are defined.
      \item \code{operator==} and \code{operator!=} are defined and return \bool.
    \end{enumerate}
\end{enumerate}

\end{document}
% vim: sw=2 sts=2 ai et tw=0
