\section{Wording}

The following is a rough draft of possible wording that defines a basic set of data-parallel types.

\begin{wgText}
  \wgSection{Data-Parallel Types}{datapar.types}

  \wgSubsection{Header \code{<datapar>} synopsis}{datapar.syn}
  \lstinputlisting[]{synopsis.cpp}

  \pnum
  The header \code{<datapar>} defines two class templates (\datapar, and \mask), several tag types, and a series of related function templates for concurrent manipulation of the values in \datapar and \mask objects.

  \begin{itemdecl}
namespace datapar_abi {
  struct scalar {};
  // implementation-defined tag types, e.g. sse, avx, avx512, neon, ...
  typedef implementation_defined compatible;
  typedef implementation_defined native;
}
  \end{itemdecl}
  \begin{itemdescr}
    \pnum
    The ABI types are tag types to be used as the second template argument to \datapar and \mask.

    \pnum
    The \type{scalar} tag is present in all implementations and forces \datapar and \mask to store a single component (i.e. \datapar{}\type{<T, datapar_abi::scalar>::size()} returns \code 1).

    \pnum
    An implementation may choose to implement data-parallel execution for many different targets.
    \wgNote{There can certainly be more than one tag type per (micro-)architecture, e.g. to support different vector lengths or partial register usage.}
    All tag types an implementation supports shall be present independent of the target architecture determined at invocation of the compiler.

    \pnum
    The \type{datapar_abi::compatible} tag is defined by the implementation to alias the tag type with the most efficient data parallel execution that ensures the highest compatibility on the target architecture.

    \pnum
    The \type{datapar_abi::native} tag is defined by the implementation to alias the tag type with the most efficient data parallel execution that is supported on the target system.
  \end{itemdescr}

  \begin{itemdecl}
template <class T, size_t N> struct abi_for_size { typedef implementation_defined type; };
  \end{itemdecl}
  \begin{itemdescr}
    \pnum
    The \type{abi_for_size} class template defines the member type \type{type} to one of the tag types in \code{datapar_abi} or not at all, depending on the value of the template parameters.

    \pnum
    \code{datapar<T, abi_for_size_t<T, N>>::size()} must return \code N or result in a substitution failure.
  \end{itemdescr}

  \begin{itemdecl}
template <class T, class Abi = datapar_abi::compatible>
struct datapar_size : public integral_constant<size_t, implementation_defined> {};
  \end{itemdecl}
  \begin{itemdescr}
    \pnum The \type{datapar_size} class template inherits from \type{integral_constant} with a value that equals \datapar{}\code{<T, Abi>::size()}.

    \pnum \code{datapar_size<T, Abi>::value} shall result in a substitution failure if any of the template arguments \type T or \type{Abi} are invalid template arguments to \datapar.
  \end{itemdescr}

  \wgSubsection{Class template \datapar}{datapar}
  \wgSubsubsection{Class template \datapar overview}{datapar.overview}
  \lstinputlisting[]{datapar.cpp}

  \pnum The class template \datapar{}\type{<T, Abi>} is a one-dimensional smart array.
  In contrast to \type{valarray} (26.6), the number of elements in the array is determined at compile time, according to the \type{Abi} template parameter.

  \pnum The first template argument \type T must be an integral or floating-point fundamental type.
  The type \bool is not allowed.

  \pnum The second template argument \type{Abi} must be a tag type from the \code{datapar_abi} namespace.

  \begin{itemdecl}
typedef implementation_defined native_handle_type;
  \end{itemdecl}
  \begin{itemdescr}
    \pnum
    The \type{native_handle_type} member type is an alias for the \code{native_handle()} member function return type.
    It is used to expose an implementation-defined handle for implementation- and target-specific extensions.
  \end{itemdescr}

%  \begin{itemdecl}
%typedef implementation_defined register_value_type;
%  \end{itemdecl}
%  \begin{itemdescr}
%  \end{itemdescr}

  \begin{itemdecl}
static constexpr size_type size();
  \end{itemdecl}
  \begin{itemdescr}
    \pnum\returns the number of elements stored in objects of the given \datapar[<T, Abi>] type.
  \end{itemdescr}

  \wgSubsubsection{\datapar constructors}{datapar.ctor}
  \begin{itemdecl}
datapar() = default;
  \end{itemdecl}
  \begin{itemdescr}
    \pnum
    \effects
    Constructs an object with all elements initialized to \code{T()}.
    \wgNote{This zero-initializes the object.}
  \end{itemdescr}

  \begin{itemdecl}
datapar(value_type);
  \end{itemdecl}
  \begin{itemdescr}
    \pnum
    \effects
    Constructs an object with each element initialized to the value of the argument.
  \end{itemdescr}

  \begin{itemdecl}
template <class U> datapar(datapar<U, Abi> x);
  \end{itemdecl}
  \begin{itemdescr}
    \pnum\remarks This constructor shall not participate in overload resolution unless
                  \type U and \type T are different integral types and
                  \code{make_signed<U>::type} equals \code{make_signed<T>::type}.
    \pnum\effects Constructs an object of type \datapar.
    \pnum\postcondition The $i$-th element equals \code{static_cast<T>(x[i])} for all elements.
  \end{itemdescr}

  \wgSubsubsection{\datapar load functions}{datapar.load}
  \begin{itemdecl}
static datapar load(const value_type *x);
  \end{itemdecl}
  \begin{itemdescr}
    \pnum \effects Constructs an object with each element $i$ initialized to \code{x[i]} for all elements.
    \pnum \returns The constructed object.
    \pnum \remarks If \datapar{}\code{::size()} is greater than the number of values pointed to by the argument, the behavior is undefined.
  \end{itemdescr}

  \begin{itemdecl}
template <class Flags> static datapar load(const value_type *x, Flags);
  \end{itemdecl}
  \begin{itemdescr}
    \pnum\effects Constructs an object with each element $i$ initialized to \code{x[i]}.
    \pnum\returns The constructed object.
    \pnum\remarks If \datapar{}\code{::size()} is greater than the number of values pointed to by the first argument, the behavior is undefined.
    \pnum\remarks If the template parameter is of type \type{aligned_tag} and the pointer value is not a multiple of \code{memory_alignment<\type T>}, the behavior is undefined.
  \end{itemdescr}

  \begin{itemdecl}
template <class U, class Flags = unaligned_tag> static datapar load(const U *x, Flags = Flags());
  \end{itemdecl}
  \begin{itemdescr}
    \pnum\effects Constructs an object with each element $i$ initialized to \code{static_cast<T>(x[i])}.
    \pnum\returns The constructed object.
    \pnum\remarks If \datapar{}\code{::size()} is greater than the number of values pointed to by the first argument, the behavior is undefined.
    \pnum\remarks If the second template parameter is of type \type{aligned_tag} and the pointer value is not a multiple of \code{memory_alignment<\type U>}, the behavior is undefined.
  \end{itemdescr}

  \wgSubsubsection{\datapar store functions}{datapar.store}
  \begin{itemdecl}
void store(value_type *x);
  \end{itemdecl}
  \begin{itemdescr}
    \pnum\effects Copies each element such that the $i$-th element is stored to \code{x[i]}.
    \pnum\remarks If \datapar{}\code{::size()} is greater than the number of values pointed to by the first argument, the behavior is undefined.
  \end{itemdescr}

  \begin{itemdecl}
template <class Flags> void store(value_type *x, Flags);
  \end{itemdecl}
  \begin{itemdescr}
    \pnum\effects Copies each element such that the $i$-th element is stored to \code{x[i]}.
    \pnum\remarks If \datapar{}\code{::size()} is greater than the number of values pointed to by the first argument, the behavior is undefined.
    \pnum\remarks If the template parameter is of type \type{aligned_tag} and the pointer value is not a multiple of \code{memory_alignment<\type T>}, the behavior is undefined.
  \end{itemdescr}

  \begin{itemdecl}
template <class U, class Flags = unaligned_tag> void store(U *x, Flags = Flags());
  \end{itemdecl}
  \begin{itemdescr}
    \pnum\effects Copies each element such that the $i$-th element is first converted to \type U and then stored to \code{x[i]}.
    \pnum\remarks If \datapar{}\code{::size()} is greater than the number of values pointed to by the first argument, the behavior is undefined.
    \pnum\remarks If the second template parameter is of type \type{aligned_tag} and the pointer value is not a multiple of \code{memory_alignment<\type U>}, the behavior is undefined.
  \end{itemdescr}

  \begin{itemdecl}
void store(value_type *x, mask_type);
  \end{itemdecl}
  \begin{itemdescr}
    \pnum\effects Copies each element where the corresponding element in the second argument is \true such that the $i$-th element is stored to \code{x[i]}.
    \pnum\remarks If the largest $i$ where the second argument is \true is greater than the number of values pointed to by the first argument, the behavior is undefined.
  \end{itemdescr}

  \begin{itemdecl}
template <class Flags> void store(value_type *x, mask_type, Flags);
  \end{itemdecl}
  \begin{itemdescr}
    \pnum\effects Copies each element where the corresponding element in the second argument is \true such that the $i$-th element is stored to \code{x[i]}.
    \pnum\remarks If the largest $i$ where the second argument is \true is greater than the number of values pointed to by the first argument, the behavior is undefined.
    \pnum\remarks If the template parameter is of type \type{aligned_tag} and the pointer value is not a multiple of \code{memory_alignment<\type T>}, the behavior is undefined.
  \end{itemdescr}

  \begin{itemdecl}
template <class U, class Flags = unaligned_tag> void store(U *x, mask_type, Flags = Flags());
  \end{itemdecl}
  \begin{itemdescr}
    \pnum\effects Copies each element where the corresponding element in the second argument is \true such that the $i$-th element is first converted to \type U and then stored to \code{x[i]}.
    \pnum\remarks If the largest $i$ where the second argument is \true is greater than the number of values pointed to by the first argument, the behavior is undefined.
    \pnum\remarks If the template parameter is of type \type{aligned_tag} and the pointer value is not a multiple of \code{memory_alignment<\type U>}, the behavior is undefined.
  \end{itemdescr}

  \wgSubsubsection{\datapar subscript operators}{datapar.subscr}
  \begin{itemdecl}
reference operator[](size_type i);
  \end{itemdecl}
  \begin{itemdescr}
    \pnum\returns An lvalue reference to the $i$-th element.
    \pnum\postconditions Assignment of objects of type \type T modify the $i$-th element without aliasing violations.
    \pnum                Modification of \code{*this} does not invalidate references held to the return value.
    Subsequent reads from such references yield the new value of the $i$-th element.
  \end{itemdescr}

  \begin{itemdecl}
const_reference operator[](size_type) const;
  \end{itemdecl}
  \begin{itemdescr}
    \pnum\returns A \const lvalue reference to the $i$-th element.
    \pnum\postconditions Modification of \code{*this} does not invalidate references held to the return value.
    Subsequent reads from such references yield the new value of the $i$-th element.
  \end{itemdescr}

  \wgSubsubsection{\datapar unary operators}{datapar.unary}
  \begin{itemdecl}
datapar &operator++();
  \end{itemdecl}
  \begin{itemdescr}
    \pnum\effects Increments every element of \code{*this} by one.
    \pnum\returns An lvalue reference to \code{*this} after incrementing.
    \pnum\remarks Overflow semantics follow the same semantics as for \type T.
  \end{itemdescr}

  \begin{itemdecl}
datapar operator++(int);
  \end{itemdecl}
  \begin{itemdescr}
    \pnum\effects Increments every element of \code{*this} by one.
    \pnum\returns A copy of \code{*this} before incrementing.
    \pnum\remarks Overflow semantics follow the same semantics as for \type T.
  \end{itemdescr}

  \begin{itemdecl}
datapar &operator--();
  \end{itemdecl}
  \begin{itemdescr}
    \pnum\effects Decrements every element of \code{*this} by one.
    \pnum\returns An lvalue reference to \code{*this} after decrementing.
    \pnum\remarks Underflow semantics follow the same semantics as for \type T.
  \end{itemdescr}

  \begin{itemdecl}
datapar operator--(int);
  \end{itemdecl}
  \begin{itemdescr}
    \pnum\effects Decrements every element of \code{*this} by one.
    \pnum\returns A copy of \code{*this} before decrementing.
    \pnum\remarks Underflow semantics follow the same semantics as for \type T.
  \end{itemdescr}

  \begin{itemdecl}
mask_type operator!() const;
  \end{itemdecl}
  \begin{itemdescr}
    \pnum\returns A mask object with the $i$-th element set to \code{!operator[](i)} for all elements.
  \end{itemdescr}

  \begin{itemdecl}
datapar operator~() const;
  \end{itemdecl}
  \begin{itemdescr}
    \pnum\requires The first template argument \type T to \datapar must be an integral type.
    \pnum\effects Constructs an object where each bit of \code{*this} is inverted.
    \pnum\returns The new object.
    \pnum\remarks \datapar{}\code{::operator\textasciitilde{}()} shall not participate in overload resolution if \type T is a floating-point type.
  \end{itemdescr}

  \begin{itemdecl}
datapar operator+() const;
  \end{itemdecl}
  \begin{itemdescr}
    \pnum \returns A copy of \code{*this}
  \end{itemdescr}

  \begin{itemdecl}
datapar operator-() const;
  \end{itemdecl}
  \begin{itemdescr}
    \pnum\effects Constructs an object where the $i$-th element is initialized to \code{-operator[](i)} for all elements.
    \pnum\returns The new object.
  \end{itemdescr}

  \wgSubsubsection{\datapar native handles}{datapar.native}
  \begin{itemdecl}
native_handle_type &native_handle();
  \end{itemdecl}
  \begin{itemdescr}
    \pnum\returns An lvalue reference to the implementation-specific object implementing the data-parallel semantics.
  \end{itemdescr}

  \begin{itemdecl}
const native_handle_type &native_handle() const;
  \end{itemdecl}
  \begin{itemdescr}
    \pnum\returns A \const lvalue reference to the implementation-specific object implementing the data-parallel semantics.
  \end{itemdescr}

  \wgSubsection{\datapar non-member operations}{datapar.nonmembers}
  \wgSubsubsection{\datapar binary operators}{datapar.binary}
  \begin{itemdecl}
template <class L, class R> using datapar_return_type = ...;  // exposition only
template <class T, class Abi, class U>
datapar_return_type<datapar<T, Abi>, U> operator+ (datapar<T, Abi>, const U &);
template <class T, class Abi, class U>
datapar_return_type<datapar<T, Abi>, U> operator- (datapar<T, Abi>, const U &);
template <class T, class Abi, class U>
datapar_return_type<datapar<T, Abi>, U> operator* (datapar<T, Abi>, const U &);
template <class T, class Abi, class U>
datapar_return_type<datapar<T, Abi>, U> operator/ (datapar<T, Abi>, const U &);
template <class T, class Abi, class U>
datapar_return_type<datapar<T, Abi>, U> operator% (datapar<T, Abi>, const U &);
template <class T, class Abi, class U>
datapar_return_type<datapar<T, Abi>, U> operator& (datapar<T, Abi>, const U &);
template <class T, class Abi, class U>
datapar_return_type<datapar<T, Abi>, U> operator| (datapar<T, Abi>, const U &);
template <class T, class Abi, class U>
datapar_return_type<datapar<T, Abi>, U> operator^ (datapar<T, Abi>, const U &);
template <class T, class Abi, class U>
datapar_return_type<datapar<T, Abi>, U> operator<<(datapar<T, Abi>, const U &);
template <class T, class Abi, class U>
datapar_return_type<datapar<T, Abi>, U> operator>>(datapar<T, Abi>, const U &);
template <class T, class Abi, class U>
datapar_return_type<datapar<T, Abi>, U> operator+ (const U &, datapar<T, Abi>);
template <class T, class Abi, class U>
datapar_return_type<datapar<T, Abi>, U> operator- (const U &, datapar<T, Abi>);
template <class T, class Abi, class U>
datapar_return_type<datapar<T, Abi>, U> operator* (const U &, datapar<T, Abi>);
template <class T, class Abi, class U>
datapar_return_type<datapar<T, Abi>, U> operator/ (const U &, datapar<T, Abi>);
template <class T, class Abi, class U>
datapar_return_type<datapar<T, Abi>, U> operator% (const U &, datapar<T, Abi>);
template <class T, class Abi, class U>
datapar_return_type<datapar<T, Abi>, U> operator& (const U &, datapar<T, Abi>);
template <class T, class Abi, class U>
datapar_return_type<datapar<T, Abi>, U> operator| (const U &, datapar<T, Abi>);
template <class T, class Abi, class U>
datapar_return_type<datapar<T, Abi>, U> operator^ (const U &, datapar<T, Abi>);
template <class T, class Abi, class U>
datapar_return_type<datapar<T, Abi>, U> operator<<(const U &, datapar<T, Abi>);
template <class T, class Abi, class U>
datapar_return_type<datapar<T, Abi>, U> operator>>(const U &, datapar<T, Abi>);
  \end{itemdecl}
  \begin{itemdescr}
    \pnum\remarks The return type of these operators shall be deduced according to the following rules:
    \begin{itemize}
      \item If \code{is_datapar_v<U> == true}
        then the return type shall be determined from \type T and \type{U::value_type} according to the following paragraph.
      \item Otherwise, if \type T is integral and \type U is \intt
        the return type shall be \datapar{}\type{<T, Abi>}.
      \item Otherwise, if \type T is integral and \type U is \uint
        the return type shall be \datapar{}\code{<make_unsigned_t<T>, Abi>}.
      \item Otherwise, if \type U is a fundamental arithmetic type or \type U is convertible to \intt
        then the return type shall be determined from \type T and \type U according to the following paragraph.
      \item Otherwise, if \type U is implicitly convertible to \datapar{}\type{<V, Abi>}, where \type V is determined according to standard template type deduction,
        then the return type shall be determined from \type T and \type V according to the following paragraph.
      \item Otherwise, if \type U is implicitly convertible to \datapar{}\type{<T, Abi>},
        the return type shall be \datapar{}\type{<T, Abi>}.
      \item Otherwise no return type is defined (SFINAE).
    \end{itemize}

    \pnum\remarks Given the types \type T and \type{Abi} from the class template argument list and a third type \type U determined by the rules of the previous paragraph a return type is deduced according to the following rules:
    \begin{itemize}
      \item If \type U is not a fundamental arithmetic type then the return type shall be \datapar{}\type{<T, Abi>}.
      \item Otherwise, if at least one of the  types \type T and \type U is a floating-point type
        the return type shall be \datapar{}\type{<decltype(T() + U()), Abi>}.
      \item Otherwise, if \code{sizeof(T) < sizeof(U)} the return type shall be \datapar{}\type{<U, Abi>}.
      \item Otherwise, if \code{sizeof(T) > sizeof(U)} the return type shall be \datapar{}\type{<T, Abi>}.
      \item Otherwise, the type \type T or \type U that is farthest back in the list of \textit{standard integer types} (cf. [basic.fundamental]) is used as type \type V and
        the return type shall be \datapar{}\type{<V, Abi>} if both types \type T and \type U are signed, otherwise the return type shall be \datapar{}\type{<make_unsigned_t<V>, Abi>}.
    \end{itemize}

    \pnum\remarks Each of these operators only participates in overload resolution if all of the following hold:
    \begin{itemize}
      \item The indicated operator can be applied to objects of type \type{R::value_type}, with \type R the return type.
      \item \datapar{}\type{<T, Abi>} is implicitly convertible to the return type.
      \item \type U is implicitly convertible to the return type.
    \end{itemize}

    \pnum\remarks The operators with \type{const U \&} as first parameter shall not participate in overload resolution if \code{is_datapar_v<U> == true}.

    \pnum\effects Both arguments are first converted to the return type.
      Each of these operators subsequently performs the indicated operation component-wise on each of the elements of the first argument and the corresponding element of the second argument.
    \pnum\returns An object containing the results of the component-wise operator application.
  \end{itemdescr}

  \wgSubsubsection{\datapar compound assignment}{datapar.cassign}
  \begin{itemdecl}
template <class T, class Abi, class U> datapar<T, Abi> &operator+= (datapar<T, Abi> &, const U &);
template <class T, class Abi, class U> datapar<T, Abi> &operator-= (datapar<T, Abi> &, const U &);
template <class T, class Abi, class U> datapar<T, Abi> &operator*= (datapar<T, Abi> &, const U &);
template <class T, class Abi, class U> datapar<T, Abi> &operator/= (datapar<T, Abi> &, const U &);
template <class T, class Abi, class U> datapar<T, Abi> &operator%= (datapar<T, Abi> &, const U &);
template <class T, class Abi, class U> datapar<T, Abi> &operator&= (datapar<T, Abi> &, const U &);
template <class T, class Abi, class U> datapar<T, Abi> &operator|= (datapar<T, Abi> &, const U &);
template <class T, class Abi, class U> datapar<T, Abi> &operator^= (datapar<T, Abi> &, const U &);
template <class T, class Abi, class U> datapar<T, Abi> &operator<<=(datapar<T, Abi> &, const U &);
template <class T, class Abi, class U> datapar<T, Abi> &operator>>=(datapar<T, Abi> &, const U &);
  \end{itemdecl}
  \begin{itemdescr}
    \pnum\remarks Each of these operators only participates in overload resolution if all of the following hold:
    \begin{itemize}
      \item The indicated operator can be applied to objects of type \type{datapar_return_type<datapar<T, Abi>, U>::value_type}.
      \item \datapar{}\type{<T, Abi>} is implicitly convertible to \type{datapar_return_type<datapar<T, Abi>, U>}.
      \item \type U is implicitly convertible to \type{datapar_return_type<datapar<T, Abi>, U>}.
      \item \type{datapar_return_type<datapar<T, Abi>, U>} is implicitly convertible to \datapar{}\type{<T, Abi>}.
    \end{itemize}
    \pnum\effects Each of these operators performs the indicated operation component-wise on each of the elements of the first argument and the corresponding element of the second argument after conversion to \datapar{}\code{<T, Abi>}.
    \pnum\returns A reference to the first argument.
  \end{itemdescr}

  \wgSubsubsection{\datapar logical operators}{datapar.logical}
  \textit{TODO}

  \wgSubsubsection{\datapar compare operators}{datapar.comparison}
  \begin{itemdecl}
template <class T, class Abi, class U>
typename datapar_return_type<datapar<T, Abi>, U>::mask_type operator==(datapar<T, Abi>, const U &);
template <class T, class Abi, class U>
typename datapar_return_type<datapar<T, Abi>, U>::mask_type operator!=(datapar<T, Abi>, const U &);
template <class T, class Abi, class U>
typename datapar_return_type<datapar<T, Abi>, U>::mask_type operator>=(datapar<T, Abi>, const U &);
template <class T, class Abi, class U>
typename datapar_return_type<datapar<T, Abi>, U>::mask_type operator<=(datapar<T, Abi>, const U &);
template <class T, class Abi, class U>
typename datapar_return_type<datapar<T, Abi>, U>::mask_type operator> (datapar<T, Abi>, const U &);
template <class T, class Abi, class U>
typename datapar_return_type<datapar<T, Abi>, U>::mask_type operator< (datapar<T, Abi>, const U &);
template <class T, class Abi, class U>
typename datapar_return_type<datapar<T, Abi>, U>::mask_type operator==(const U &, datapar<T, Abi>);
template <class T, class Abi, class U>
typename datapar_return_type<datapar<T, Abi>, U>::mask_type operator!=(const U &, datapar<T, Abi>);
template <class T, class Abi, class U>
typename datapar_return_type<datapar<T, Abi>, U>::mask_type operator>=(const U &, datapar<T, Abi>);
template <class T, class Abi, class U>
typename datapar_return_type<datapar<T, Abi>, U>::mask_type operator<=(const U &, datapar<T, Abi>);
template <class T, class Abi, class U>
typename datapar_return_type<datapar<T, Abi>, U>::mask_type operator> (const U &, datapar<T, Abi>);
template <class T, class Abi, class U>
typename datapar_return_type<datapar<T, Abi>, U>::mask_type operator< (const U &, datapar<T, Abi>);
  \end{itemdecl}
  \begin{itemdescr}
    \pnum\remarks The return type of these operators shall be the \type{mask_type} member type of the type deduced according to the rules defined in [datapar.binary].

    \pnum\remarks Each of these operators only participates in overload resolution if all of the following hold:
    \begin{itemize}
      \item \datapar{}\type{<T, Abi>} is implicitly convertible to \type{datapar_return_type<datapar<T, Abi>, U>}.
      \item \type U is implicitly convertible to \type{datapar_return_type<datapar<T, Abi>, U>}.
    \end{itemize}

    \pnum\remarks The operators with \type{const U \&} as first parameter shall not participate in overload resolution if \code{is_datapar_v<U> == true}.

    \pnum\effects Both arguments are first converted to \type{datapar_return_type<datapar<T, Abi>, U>}.
      Each of these operators subsequently performs the indicated operation component-wise on each of the elements of the first argument and the corresponding element of the second argument.

    \pnum\returns An object containing the results of the component-wise operator application.
  \end{itemdescr}

  \wgSubsubsection{\datapar type traits}{datapar.traits}
  \begin{itemdecl}
template <class T> struct is_datapar;
  \end{itemdecl}
  \begin{itemdescr}
    \pnum\effects The \type{is_datapar} type derives from \type{true_type} if \type T is an instance of the \datapar class template.
    Otherwise it derives from \type{false_type}.
  \end{itemdescr}

  \wgSubsubsection{\datapar casts}{datapar.casts}
  \begin{itemdecl}
template <class T, class U, class... Us>
conditional_t<(T::size() == (U::size() + Us::size()...)), T,
              array<T, (U::size() + Us::size()...) / T::size()>> datapar_cast(U, Us...);
  \end{itemdecl}
  \begin{itemdescr}
    \pnum\remarks The \code{datapar_cast} function only participates in overload resolution if all of the following hold:
    \begin{itemize}
      \item \code{is_datapar_v<T>}
      \item \code{is_datapar_v<U>}
      \item All types in the template parameter pack \type{Us} are equal to \type U.
      \item \code{U::size() + Us::size()...} is an integral multiple of \code{T::size()}.
    \end{itemize}

    \pnum\effects All scalar elements \code{x\textsubscript{i}} of the function argument(s) are converted as if
    \code{y\textsubscript{i} = static_cast<typename T::value_type>(x\textsubscript{i})} is executed.
    The resulting \code{y\textsubscript{i}} are stored into the return object(s) of type \type T.
    \wgNote{%
      For \code{T::size() == 2 * U::size()} the following holds:
      \code{datapar_cast<T>(x0, x1)[i] == static_cast<typename T::value_type>(array<U, 2>\{x0, x1\}[i / U::size()][i \% U::size()])}.
      For \code{2 * T::size() == U::size()} the following holds:
      \code{datapar_cast<T>(x)[i][j] == static_cast<typename T::value_type>(x[i * T\MayBreak::size() + j])}.
    }

    \pnum\returns The converted values as one object of \type T or an array of \type T.
  \end{itemdescr}

  \wgSubsubsection{\datapar transcendentals}{datapar.transcend}
  \textit{TODO}

  \wgSubsection{Class template \mask}{mask}
  \wgSubsubsection{Class template \mask overview}{mask.overview}
  \lstinputlisting[]{mask.cpp}

  \pnum The class template \mask[<T, Abi>] is a one-dimensional smart array of booleans.
  The number of elements in the array is determined at compile time, equal to the number of elements in \datapar[<T, Abi>].

  \pnum The first template argument \type T must be an integral or floating-point fundamental type.
  The type \bool is not allowed.

  \pnum The second template argument \type{Abi} must be a tag type from the \code{datapar_abi} namespace.

  \begin{itemdecl}
typedef implementation_defined native_handle_type;
  \end{itemdecl}
  \begin{itemdescr}
    \pnum The \type{native_handle_type} member type is an alias for the \code{native_handle()} member function return type.
    It is used to expose an implementation-defined handle for implementation- and target-specific extensions.
  \end{itemdescr}

  \begin{itemdecl}
static constexpr size_type size();
  \end{itemdecl}
  \begin{itemdescr}
    \pnum\returns the number of boolean elements stored in objects of the given \mask[<T, Abi>] type.
  \end{itemdescr}

  \wgSubsubsection{\mask constructors}{mask.ctor}
  \begin{itemdecl}
mask() = default;
  \end{itemdecl}
  \begin{itemdescr}
    \pnum\effects Constructs an object with all elements initialized to \code{bool()}.
    \wgNote{This zero-initializes the object.}
  \end{itemdescr}

  \begin{itemdecl}
mask(value_type);
  \end{itemdecl}
  \begin{itemdescr}
    \pnum\effects Constructs an object with each element initialized to the value of the argument.
  \end{itemdescr}

  \begin{itemdecl}
template <class U> mask(mask<U, Abi> x);
  \end{itemdecl}
  \begin{itemdescr}
    \pnum\remarks This constructor shall not participate in overload resolution unless
      \datapar[<U, Abi>] is implicitly convertible to \datapar[<T, Abi>].
    \pnum\effects Constructs an object of type \mask.
    \pnum\postcondition The $i$-th element equals \code{x[i]} for all elements.
  \end{itemdescr}

  \wgSubsubsection{\mask load functions}{mask.load}
  \begin{itemdecl}
static mask load(const value_type *x);
  \end{itemdecl}
  \begin{itemdescr}
    \pnum \effects Constructs an object with each element $i$ initialized to \code{x[i]} for all elements.
    \pnum \returns The constructed object.
    \pnum \remarks If \mask{}\code{::size()} is greater than the number of values pointed to by the argument, the behavior is undefined.
  \end{itemdescr}

  \begin{itemdecl}
template <class Flags> static mask load(const value_type *x, Flags);
  \end{itemdecl}
  \begin{itemdescr}
    \pnum\effects Constructs an object with each element $i$ initialized to \code{x[i]} for all elements.
    \pnum\returns The constructed object.
    \pnum\remarks If \mask{}\code{::size()} is greater than the number of values pointed to by the first argument, the behavior is undefined.
    \pnum\remarks If the template parameter is of type \type{aligned_tag} and the pointer value is not a multiple of \code{memory_alignment<\type T>}, the behavior is undefined.
  \end{itemdescr}

  \wgSubsubsection{\mask{} store functions}{mask.store}
  \begin{itemdecl}
void store(value_type *x);
  \end{itemdecl}
  \begin{itemdescr}
    \pnum\effects Copies each element such that the $i$-th element is stored to \code{x[i]}.
    \pnum\remarks If \mask{}\code{::size()} is greater than the number of values pointed to by the first argument, the behavior is undefined.
  \end{itemdescr}

  \begin{itemdecl}
template <class Flags> void store(value_type *x, Flags);
  \end{itemdecl}
  \begin{itemdescr}
    \pnum\effects Copies each element such that the $i$-th element is stored to \code{x[i]}.
    \pnum\remarks If \mask{}\code{::size()} is greater than the number of values pointed to by the first argument, the behavior is undefined.
    \pnum\remarks If the template parameter is of type \type{aligned_tag} and the pointer value is not a multiple of \code{memory_alignment<\type T>}, the behavior is undefined.
  \end{itemdescr}

  \begin{itemdecl}
void store(value_type *x, mask);
  \end{itemdecl}
  \begin{itemdescr}
    \pnum\effects Copies each element where the corresponding element in the second argument is \true such that the $i$-th element is stored to \code{x[i]}.
    \pnum\remarks If the largest $i$ where the second argument is \true is greater than the number of values pointed to by the first argument, the behavior is undefined.
  \end{itemdescr}

  \begin{itemdecl}
template <class Flags> void store(value_type *x, mask, Flags);
  \end{itemdecl}
  \begin{itemdescr}
    \pnum\effects Copies each element where the corresponding element in the second argument is \true such that the $i$-th element is stored to \code{x[i]}.
    \pnum\remarks If the largest $i$ where the second argument is \true is greater than the number of values pointed to by the first argument, the behavior is undefined.
    \pnum\remarks If the template parameter is of type \type{aligned_tag} and the pointer value is not a multiple of \code{memory_alignment<\type T>}, the behavior is undefined.
  \end{itemdescr}

  \wgSubsubsection{\mask{} subscript operators}{mask.subscr}
  \begin{itemdecl}
reference operator[](size_type i);
  \end{itemdecl}
  \begin{itemdescr}
    \pnum\returns A temporary object acting as a smart reference wrapper to the $i$-th element.
    \pnum\postconditions Assignment of objects of type \bool modify the $i$-th element without aliasing violations.
    \pnum                Modification of \code{*this} does not invalidate references held to the return value.
    Subsequent reads from such references yield the new value of the $i$-th element.
  \end{itemdescr}

  \begin{itemdecl}
const_reference operator[](size_type) const;
  \end{itemdecl}
  \begin{itemdescr}
    \pnum\returns A \const lvalue reference to the $i$-th element.
    \pnum\postconditions Modification of \code{*this} does not invalidate references held to the return value.
    Subsequent reads from such references yield the new value of the $i$-th element.
  \end{itemdescr}

  \wgSubsubsection{\mask unary operators}{mask.unary}
  \begin{itemdecl}
mask operator!() const;
  \end{itemdecl}
  \begin{itemdescr}
    \pnum\returns A mask object with the $i$-th element set to the logical negation for all elements.
  \end{itemdescr}

  \wgSubsubsection{\mask native handles}{mask.native}
  \begin{itemdecl}
native_handle_type &native_handle();
  \end{itemdecl}
  \begin{itemdescr}
    \pnum\returns An lvalue reference to the implementation-specific object implementing the data-parallel semantics.
  \end{itemdescr}

  \begin{itemdecl}
const native_handle_type &native_handle() const;
  \end{itemdecl}
  \begin{itemdescr}
    \pnum\returns A \const lvalue reference to the implementation-specific object implementing the data-parallel semantics.
  \end{itemdescr}

  \wgSubsection{\mask{} non-member operations}{mask.nonmembers}
  \wgSubsubsection{\mask binary operators}{mask.binary}

\end{wgText}
