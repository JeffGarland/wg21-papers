\section{Wording}

The following is a rough draft of possible wording that defines a basic set of data-parallel types.

\begin{wgText}
  \wgSection{Data-Parallel Types}{datapar.types}

  \wgSubsection{Header \code{<datapar>} synopsis}{datapar.syn}
  \lstinputlisting[]{synopsis.cpp}

  \pnum
  The header \code{<datapar>} defines two class templates (\datapar, and \mask), several tag types, and a series of related function templates for concurrent manipulation of the values in \datapar and \mask objects.

  \begin{itemdecl}
namespace datapar_target {
  struct scalar {};
  // implementation-defined tag types, e.g. sse, avx, avx512, neon, ...
  typedef implementation_defined widest;
}
  \end{itemdecl}
  \begin{itemdescr}
    \pnum
    The target types are tag types to be used as the second template argument to \datapar and \mask.

    \pnum
    The \type{scalar} tag is present in all implementations and forces \datapar and \mask to store a single component (i.e. \datapar{}\type{<T, datapar_target::scalar>::size()} returns \code 1).

    \pnum
    An implementation may choose to implement data-parallel execution for many different targets.
    All tag types this implementation supports shall be present independent of the chosen target.
    %\wgNote{This means that if the type \type{datapar_target::abc} is implemented for the \emph{abc} target system, compilation for the target \emph{xyz} will also have the \type{datapar_target::abc} tag type.}

    \pnum
    The \type{datapar_target::widest} tag is defined by the implementation to alias the tag type with the most efficient data parallel execution.
  \end{itemdescr}

  \begin{itemdecl}
template <size_t> struct target_for_width {
  typedef implementation_defined type;
};
  \end{itemdecl}
  \begin{itemdescr}
    \pnum
    The \type{target_for_width} class template defines the member type \type{type} to one of the tag types in \code{datapar_target} or not at all, depending on the value of the template parameters.

    \pnum
    \code{datapar<T, target_for_width_t<T, N>>::size()} must return \code N or result in a substitution failure.
  \end{itemdescr}

  \wgSubsection{Class template \datapar}{datapar}
  \wgSubsubsection{Class template \datapar overview}{datapar.overview}
  \lstinputlisting[]{datapar.cpp}

  \pnum The class template \datapar{}\type{<T, Target>} is a one-dimensional smart array.
  In contrast to \type{valarray} (26.6), the number of elements in the array is determined at compile time, according to the \type{Target} template parameter.

  \begin{itemdecl}
typedef implementation_defined internal_type;
  \end{itemdecl}
  \begin{itemdescr}
    \pnum
    The \type{internal_type} member type is an alias for the \code{internal()} member function return type.
    It is used to expose an implementation-defined handle for implementation- and target-specific extensions.
  \end{itemdescr}

  \begin{itemdecl}
typedef implementation_defined register_value_type;
  \end{itemdecl}
  \begin{itemdescr}
  \end{itemdescr}

  \wgSubsubsection{\datapar constructors}{datapar.ctor}
  \begin{itemdecl}
datapar() = default;
  \end{itemdecl}
  \begin{itemdescr}
    \pnum
    \effects
    Constructs an object with all elements initialized to \code{T()}.
    \wgNote{This zero-initializes the object.}
  \end{itemdescr}

  \begin{itemdecl}
datapar(value_type);
  \end{itemdecl}
  \begin{itemdescr}
    \pnum
    \effects
    Constructs an object with all elements initialized to the value of the argument.
  \end{itemdescr}

  \begin{itemdecl}
template <typename U> datapar(datapar<U, Target>);
  \end{itemdecl}
  \begin{itemdescr}
    \pnum
    \effects
    Constructs an object from the argument, converting every element of type \type U to \type T.
  \end{itemdescr}

  \wgSubsubsection{\datapar load functions}{datapar.load}
  \begin{itemdecl}
static datapar load(const value_type *x);
  \end{itemdecl}
  \begin{itemdescr}
    \pnum
    \effects
    Constructs an object with each element $i$ initialized to \code{x[i]}.

    \pnum
    If \datapar{}\code{::size()} is greater than the number of values pointed to by the argument, the behavior is undefined.
  \end{itemdescr}

  \begin{itemdecl}
template <typename Flags> static datapar load(const value_type *x, Flags);
  \end{itemdecl}
  \begin{itemdescr}
    \pnum
    \effects
    Constructs an object with each element $i$ initialized to \code{x[i]}.

    \pnum
    If \datapar{}\code{::size()} is greater than the number of values pointed to by the first argument, the behavior is undefined.

    \pnum
    If the template parameter is of type \type{aligned_tag} and the pointer value is not a multiple of \code{memory_alignment<\type T>}, the behavior is undefined.
  \end{itemdescr}

  \begin{itemdecl}
template <typename U, typename Flags = unaligned_tag> static datapar load(const U *x, Flags = Flags());
  \end{itemdecl}
  \begin{itemdescr}
    \pnum
    \effects
    Constructs an object with each element $i$ initialized to \code{static_cast<T>(x[i])}.

    \pnum
    If \datapar{}\code{::size()} is greater than the number of values pointed to by the first argument, the behavior is undefined.

    \pnum
    If the second template parameter is of type \type{aligned_tag} and the pointer value is not a multiple of \code{memory_alignment<\type U>}, the behavior is undefined.
  \end{itemdescr}

  \wgSubsubsection{\datapar store functions}{datapar.store}
  \begin{itemdecl}
void store(value_type *x);
  \end{itemdecl}
  \begin{itemdescr}
    \pnum \effects Copies each element such that the $i$-th element is stored to \code{x[i]}.

    \pnum If \datapar{}\code{::size()} is greater than the number of values pointed to by the first argument, the behavior is undefined.
  \end{itemdescr}

  \begin{itemdecl}
template <typename Flags> void store(value_type *x, Flags);
  \end{itemdecl}
  \begin{itemdescr}
    \pnum \effects Copies each element such that the $i$-th element is stored to \code{x[i]}.

    \pnum If \datapar{}\code{::size()} is greater than the number of values pointed to by the first argument, the behavior is undefined.

    \pnum If the template parameter is of type \type{aligned_tag} and the pointer value is not a multiple of \code{memory_alignment<\type T>}, the behavior is undefined.
  \end{itemdescr}

  \begin{itemdecl}
template <typename U, typename Flags = unaligned_tag> void store(U *x, Flags = Flags());
  \end{itemdecl}
  \begin{itemdescr}
    \pnum \effects Copies each element such that the $i$-th element is first converted to \type U and then stored to \code{x[i]}.

    \pnum If \datapar{}\code{::size()} is greater than the number of values pointed to by the first argument, the behavior is undefined.

    \pnum If the second template parameter is of type \type{aligned_tag} and the pointer value is not a multiple of \code{memory_alignment<\type U>}, the behavior is undefined.
  \end{itemdescr}

  \begin{itemdecl}
void store(value_type *x, mask_type);
  \end{itemdecl}
  \begin{itemdescr}
    \pnum \effects Copies each element where the corresponding element in the second argument is \true such that the $i$-th element is stored to \code{x[i]}.

    \pnum If the largest $i$ where the second argument is \true \code{::size()} is greater than the number of values pointed to by the first argument, the behavior is undefined.
  \end{itemdescr}

  \begin{itemdecl}
template <typename Flags> void store(value_type *x, mask_type, Flags);
  \end{itemdecl}
  \begin{itemdescr}
    \pnum \effects Copies each element where the corresponding element in the second argument is \true such that the $i$-th element is stored to \code{x[i]}.

    \pnum If the largest $i$ where the second argument is \true \code{::size()} is greater than the number of values pointed to by the first argument, the behavior is undefined.

    \pnum If the template parameter is of type \type{aligned_tag} and the pointer value is not a multiple of \code{memory_alignment<\type T>}, the behavior is undefined.
  \end{itemdescr}

  \begin{itemdecl}
template <typename U, typename Flags = unaligned_tag> void store(U *x, mask_type, Flags = Flags());
  \end{itemdecl}
  \begin{itemdescr}
    \pnum \effects Copies each element where the corresponding element in the second argument is \true such that the $i$-th element is first converted to \type U and then stored to \code{x[i]}.

    \pnum If the largest $i$ where the second argument is \true \code{::size()} is greater than the number of values pointed to by the first argument, the behavior is undefined.

    \pnum If the template parameter is of type \type{aligned_tag} and the pointer value is not a multiple of \code{memory_alignment<\type U>}, the behavior is undefined.
  \end{itemdescr}

  \wgSubsubsection{\datapar subscript operators}{datapar.subscr}
  \begin{itemdecl}
  \end{itemdecl}
  \begin{itemdescr}
  \end{itemdescr}

  \begin{itemdecl}
  \end{itemdecl}
  \begin{itemdescr}
  \end{itemdescr}

  \begin{itemdecl}
  \end{itemdecl}
  \begin{itemdescr}
  \end{itemdescr}

  \wgSubsubsection{\datapar unary operators}{datapar.unary}
  \begin{itemdecl}
  \end{itemdecl}
  \begin{itemdescr}
  \end{itemdescr}

  \begin{itemdecl}
  \end{itemdecl}
  \begin{itemdescr}
  \end{itemdescr}

  \begin{itemdecl}
  \end{itemdecl}
  \begin{itemdescr}
  \end{itemdescr}

  \begin{itemdecl}
  \end{itemdecl}
  \begin{itemdescr}
  \end{itemdescr}

  \begin{itemdecl}
  \end{itemdecl}
  \begin{itemdescr}
  \end{itemdescr}

  \begin{itemdecl}
  \end{itemdecl}
  \begin{itemdescr}
  \end{itemdescr}

  \begin{itemdecl}
  \end{itemdecl}
  \begin{itemdescr}
  \end{itemdescr}

  \begin{itemdecl}
  \end{itemdecl}
  \begin{itemdescr}
  \end{itemdescr}

  \wgSubsubsection{\datapar internal handles}{datapar.intern}
  \begin{itemdecl}
  \end{itemdecl}
  \begin{itemdescr}
  \end{itemdescr}

  \begin{itemdecl}
  \end{itemdecl}
  \begin{itemdescr}
  \end{itemdescr}

  \wgSubsection{Class template \mask}{datapar.mask}
  \lstinputlisting[]{mask.cpp}

\end{wgText}
