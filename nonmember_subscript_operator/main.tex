\newcommand\wgTitle{Non-Member Subscript Operator}
\newcommand\wgName{Matthias Kretz <m.kretz@gsi.de>}
\newcommand\wgDocumentNumber{DNMSOR0}
\newcommand\wgGroup{EWG-I}
\newcommand\wgTarget{\CC{}23}

\usepackage{mymacros}
\usepackage{wg21}

\hypersetup{
  pdfkeywords={C++,operator overloading,subscripting}
}

\newlength\myIndent
\begin{document}
\begin{wgTitlepage}
  Currently the subscript operator can only be overloaded as a member function.
  This paper explores use-cases for a non-member overload and proposes to extend
  the standard accordingly.
\end{wgTitlepage}

\section{Problem}
The subscript operator is currently only overloadable via a member function.
Many other operators additionally allow overloads via non\hyp member functions.
That the subscript operator does not allow non\hyp member overloads appears like an arbitrary decision, apparently motivated by a conservative approach, to only add features when the use\hyp cases are clear.

Therefore, for the rest of this section I will give examples for non\hyp member subscript operators.
In general, non\hyp member subscript is needed if the class to be subscripted is independent of the type used for the subscript argument or if modifications of the subscripted type are not possible.
Consequently, several examples are very similar:
A generic container class only implements the \type{size\_t} subscript operator.
A user-defined type subsequently provides a good use\hyp case for an additional subscript overload for one or more container classes.

\subsection{Vector load/store/gather/scatter on std containers}
The use\hyp case that got me started was the design of a gather/scatter API for SIMD vector types.
Consider the classical container indexing operation:
\smallskip\begin{lstlisting}
std::vector<int> data = ...;
int i = ...;
int x = data[i];
\end{lstlisting}
If I want to carry this syntax over to data\hyp parallel types I would like to write:
\smallskip\begin{lstlisting}
std::vector<int> data = ...;
std::simd<int> i = ...;
auto x = data[i];
static_assert(std::is_simd_v<decltype(x)>);
\end{lstlisting}
This expression intuitively\footnote{All operators on data-parallel types act element-wise.} asks for four loads from \code{data} at \code{data[i[0]]}, \code{data[i[1]]}, \code{data[i[2]]}, and \code{data[i[3]]}.

The reverse operation is just as useful (scatter):
\smallskip\begin{lstlisting}
std::vector<int> data = ...;
std::simd<int> i = ...;
std::simd<int> x = ...;
data[i] = x;
\end{lstlisting}

\subsection{Vector load/store subscripts}
\begin{lstlisting}[morekeywords={[5]idx}]
std::vector<int> data = ...;
std::idx<std::simd<int>> i = 3; // the number of elements is given by simd<int>::size()
auto x = data[i];
// decltype(x) -> simd<int>
\end{lstlisting}

\TODO{
\todoitem needs non-fundamental types in the operator[] parameter type, thus a dependency on the type
\todoitem it makes a lot of sense to delay such declarations to somewhere later in the program to avoid pulling in all possible useful operator[] declarations
}

Obviously, this pattern is just as useful outside of standard library implementations, where a non-\code{std} object storing multiple indices is used for gather \& scatter subscripting.

\subsection{Subscripting with scoped enumerations}
\TODO{
  \todoitem This would better use \type{uint32\_t} as base for the enum and thus avoid the \code{ColorValues} array altogether.  The example is not very motivating this way.
}

\begin{lstlisting}[style=Vc]
enum class Color : char {
  Red,
  Green,
  Blue
};

using ColorArray = std::array<std::uint32_t, 3>;
const ColorArray ColorValues = {{
  0xFFFF0000,
  0xFF00FF00,
  0xFF0000FF,
}};

auto value = ColorValues[Color::Red]; // error: no implicit conversion

std::uint32_t operator[](const ColorArray &values, Color c) {
  return values[static_cast<std::size_t>(c)];
}

auto value = ColorValues[Color::Red]; // now it works
\end{lstlisting}

\subsection{Compatiblity layer}
Once P2128 is adopted, providing multi\hyp dimensional subscript operators for linear algebra (linalg) types (e.g. matrices) will be possible.
It is to be expected that linalg\hyp libaries will start providing these subscript operators, though adoption will be lagging until the respective C++ standard is in widespread use.
Therefore, early adopters likely won't benefit from P2128, unless they maintain their own linalg\hyp library.
Locally\footnote{This is somewhat similar to providing make_unique in C++11.} adding a non\hyp member subscript operator will allow early adopters to enable a more natural syntax until P2128 is in widespread use.

\begin{lstlisting}[style=Vc]
//linalg.h - external code, cannot be modified
namespace linalg {
  struct matrix {
    float & operator()(size_t row, size_t col);
  };
}

//utils.h - local utility header
namespace linalg {
  //! @attention to be removed once linalg provides multi-dim subscript
  inline float & operator[](matrix &self, size_t row, size_t col) {
    return self(row, col);
  }
}

linalg::matrix mat = ...;
mat[3, 3] = 1;

\end{lstlisting}

\section{Implementation}
Implementing non-member subscript overloads for GCC was as simple as removing one line of code (a special case).

\section{Wording}

The wording for non\hyp member subscript operators can be adopted from the wording for binary operators (§13.5.2 [over.binary]):

\begin{wgText}{§13.5.5 [over.sub]}
  operator[] shall be\wgAdd{ implemented either by} a non-static member function with\wgRemove{ exactly} one parameter\wgAdd{ or by a non-member function with two parameters}.
  It implements the subscripting syntax\\
  \hspace*{4em}\wgGrammar{postfix-expression \wgIdentifier [ expression \wgIdentifier ]}\\
  \hspace*{2em}or\\
  \hspace*{4em}\wgGrammar{postfix-expression \wgIdentifier [ braced-init-list \wgIdentifier ]}\\

  Thus, a subscripting expression \texttt{x[y]} is interpreted as\wgAdd{ either}
  \texttt{x.operator[](y)} \wgRemove{for a class object \texttt{x} of type \texttt{T} if \texttt{T::operator[](T1)} exists}\wgAdd{or \texttt{operator[](x, y)}}\wgRemove{
  and if the operator is selected as the best match function by the overload resolution mechanism (13.3.3)}.\wgAdd{
  If both forms of the operator function have been declared, the rules in 13.3.1.2 determine which, if any, interpretation is used.}
  [ \textit{Example:}\\
    \texttt{
      \settowidth{\myIndent}{X}%
      struct X \{\\
      \hspace*{2\myIndent}Z operator[](std::initializer\_list<int>);\\
      \};\\
      X x;\\
      x[\{1,2,3\}] = 7;\hspace*{7\myIndent}// \textrm{\itshape OK: meaning} x.operator[](\{1,2,3\})\\
      int a[10];\\
      a[\{1,2,3\}] = 7;\hspace*{7\myIndent}// \textrm{\itshape error: built-in subscript operator}\\
    }
  \textit{— end example} ]
\end{wgText}

In addition, Table 11 in §13.3.1.2 [over.match.oper] mentions the subscript operator and needs to be adjusted:
\begin{wgText}{§13.3.1.2 [over.match.oper]}
  \centering Table 11 — Relationship between operator and function call notation

  \hfil\begin{tabular}{|l|l|l|l|}
    \hline
    \bfseries Subclause & \bfseries Expression & \bfseries As member function & \bfseries As non-member function \\
    \hline
    \hline
    13.5.1 & @a   & (a).operator@()   & operator@(a) \\
    13.5.2 & a@b  & (a).operator@(b)  & operator@(a, b) \\
    13.5.3 & a=b  & (a).operator=(b)  & \\
    13.5.5 & a[b] & (a).operator[](b) & \wgAdd{operator[](a, b)} \\
    13.5.6 & a->  & (a).operator->()  & \\
    13.5.7 & a@   & (a).operator@(0)  & operator@(a, 0) \\
    \hline
  \end{tabular}
\end{wgText}

The following paragraph in [over.match.oper] needs to discuss the subscript operator in addition to unary and binary operators:
\begin{wgText}{§13.3.1.2 [over.match.oper]}
  For a unary operator \texttt{@} with an operand of a type whose cv-unqualified version is \texttt{T1}, for a binary operator \texttt{@} with a left operand of a type whose cv-unqualified version is \texttt{T1} and a right operand of a type whose cv-unqualified version is \texttt{T2}\wgAdd{, and for a subscript operator (where in the following \texttt{@} identifies \texttt{[]}) with a left operand of a type whose cv-unqualified version is \texttt{T1} and subscript operand of a type whose cv-unqualified version is \texttt{T2}}, three sets of candidate functions, designated \emph{member candidates}, \emph{non-member candidates} and \emph{built-in candidates}, are constructed as follows:
  \begin{itemize}
    \item If \texttt{T1} is a complete class type or a class currently being defined, the set of member candidates is the result of the qualified lookup of \texttt{T1::operator@} (13.3.1.1.1); otherwise, the set of member candidates is empty.
    \item The set of non-member candidates is the result of the unqualified lookup of \texttt{operator@} in the context of the expression according to the usual rules for name lookup in unqualified function calls (3.4.2) except that all member functions are ignored.
      However, if no operand has a class type, only those non-member functions in the lookup set that have a first parameter of type \texttt{T1} or “reference to (possibly cv-qualified) \texttt{T1}”, when \texttt{T1} is an enumeration type, or (if there is a right operand) a second parameter of type \texttt{T2} or “reference to (possibly cv-qualified) \texttt{T2}”, when \texttt{T2} is an enumeration type, are candidate functions.
    \item For the operator \texttt{,}, the unary operator \texttt{\&}, or the operator \texttt{->}, the built-in candidates set is empty.
      For all other operators, the built-in candidates include all of the candidate operator functions defined in 13.6 that, compared to the given operator,
      \begin{itemize}
        \item have the same operator name, and
        \item accept the same number of operands, and
        \item accept operand types to which the given operand or operands can be converted according to 13.3.3.1, and
        \item do not have the same parameter-type-list as any non-member candidate that is not a function template specialization.
      \end{itemize}
  \end{itemize}
\end{wgText}

\section{Evaluation}

Since, currently the declaration of non\hyp member subscript operators is ill\hyp formed, there are no backwards compatibility issues to expect.


\end{document}
