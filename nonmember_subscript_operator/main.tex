\newcommand\wgTitle{Non-Member Subscript Operator}
\newcommand\wgName{Matthias Kretz <m.kretz@gsi.de>}
\newcommand\wgDocumentNumber{DNMSOR0}
\newcommand\wgGroup{EWG-I, EWG}
\newcommand\wgTarget{\CC{}23}

\usepackage{mymacros}
\usepackage{wg21}

\hypersetup{
  pdfkeywords={C++,operator overloading,subscripting}
}

\newlength\myIndent
\begin{document}
\begin{wgTitlepage}
  Currently the subscript operator can only be overloaded as a member function.
  This paper explores use-cases for a non-member overload and proposes to extend
  the standard accordingly.
\end{wgTitlepage}

\section{Problem}
The subscript operator is currently only overloadable as a member.
Most other operators allow non\hyp member overloads.
That the subscript operator does not allow non\hyp member overloads seems
arbitrary, potentially motivated by a conservative approach only to add
features when the use\hyp cases are clear. Therefore, for the rest of this
section I will give examples for non\hyp member subscript operators.

In general, non\hyp member subscript is needed / preferable
\begin{itemize}
  \item if modification of the container type is not possible, or

  \item if the container type should not depend on the index type (e.g.
    \lstinline@#include <vector>@ should not implicitly
    \lstinline@#include <simd>@), or

  \item if the implementation of the subscript operation is a detail of the
    index type rather than the container type.
\end{itemize}

\subsection{Vector gather/scatter on std containers and C-arrays}\label{sec:gatherscatter}
The use\hyp case that got me started was the design of a gather/scatter API for SIMD vector types.
Consider a function \code{f} which modifies an element at a given index:
\smallskip\begin{lstlisting}
void f(std::vector<int>& data, int i) {
  int x = data[i];
  // do something with x
  data[i] = x;
}
\end{lstlisting}

Now, let us generalize \code{i} from one index value to multiple indices,
limiting the discussion to \code{std::simd}\footnote{Assuming
\code{std::experimental::simd} becomes \code{std::simd} once \cite{P1928R0}
gains traction again.} for the multi-index type. Obviously, \code{simd}
objects are not convertible to \type{size_t} and thus the existing subscript
operator is not viable. We therefore need a new member \code{operator[]}
overload in \code{std::vector} with a \code{simd} subscript argument.

While there's a non-zero chance for the standard library to adopt \code{simd}
integration throughout the library, it is unlikely that non-std containers will
do so. Furthermore, the gather and scatter operations on subscripting are
really a detail of \code{simd}, rather than \code{vector}. The necessary
interface to \code{vector} and all contiguous ranges already exists with
\code{std::ranges::data}. Last but not least, a hidden friend \code{operator[]}
inside \code{simd} would allow a single generic implementation for all
contiguous ranges and all supported \code{simd} element types. Without
non-member overloads the implementation must be copied into every type. (I.e.
without non-member \code{operator[]} generic programming in \CC{} has a hole.)

The alternative to adding additional member \code{operator[](simd)} overloads
is to rely on a named function instead. This is the most likely outcome if this
paper is not adopted. \ttref{gather} shows the significant improvement in
generic code if \code{simd} subscripts are available.
\begin{tonytable}{SIMD gather from a \code{std::vector}}\label{tt:gather}
\smaller\begin{lstlisting}
template<std::ranges::contiguous_range Range,
         class Idx>
void f(Range& data, Idx i) {
  std::conditional_t<std::is_simd_v<Idx>,
                     std::simd<int>, int> x;
  if constexpr (std::is_simd_v<Idx>)
    x = std::simd_gather(std::ranges::data(data), i);
  else
    x = data[i];
  // do something with x
  if constexpr (std::is_simd_v<Idx>)
    std::simd_scatter(std::ranges::data(data), i, x);
  else
    data[i] = x;
}
\end{lstlisting}
&
\smaller\begin{lstlisting}
template<std::ranges::contiguous_range Range,
         class Idx>
void f(Range& data, Idx i) {





  auto x = std::as_const(data)[i];
  // do something with x



  data[i] = x;
}
\end{lstlisting}
\end{tonytable}%

Note that the \code{simd} subscript expression intuitively\footnote{All
operators on data-parallel types act element-wise.} asks for
\code{simd<int>::size()} single-element loads from / stores to \code{data} at \code{data[i[0]]},
\code{data[i[1]]}, \code{data[i[2]]}, \ldots.

\subsubsection{Tangent: reference class decay}

The need for \code{as_const} in \ttref{gather} is unfortunate. The non-const
overload of a subscript has to return a reference class type that is able to
perform the scatter operation on assignment. Without \code{as_const} the type
of \code{x} would be the reference class. Since we don't want to have a
reference --- we wrote \code{auto} not \code{auto\&} --- we have to ensure the
\code{const} overload is used. Note the inconsistency with regard to a function
\code{f} that returns \code{int\&}: the type of \code{auto x = f()} isn't a
reference but a copy of the value (\code{decay_t<int\&>} is \code{int}). It
would be great to get language support to define user-defined reference
wrappers with decay semantics. See \cite{PTODO}.

\subsection{Vector load/store subscripts}\label{sec:loadstore}
Analogue to SIMD scatter \& gather subscripting, SIMD loads and stores can be implemented as subscripting into a contiguous range.
However, instead of a single index a SIMD load/store is an access via multiple contiguous indices.
This key difference must be reflected in the index type.
A \code{simd} index type therefore represents a contiguous index range (similar to a \code{ranges::iota_view<size_t>}).
In addition, a \code{simd} index type could determine the \code{simd} ABI tag, thus avoiding reference proxy types when subscripting \code{const} ranges.
The availability of SIMD loads and stores via \code{simd} index types improves generality and readability significantly.

\ttref{loadstore} shows a simple loop (i.e. without epilogue) over a given range of \code{float}s incrementing every value in the range by 1.
\begin{tonytable}{SIMD loop over a contiguous range}\label{tt:loadstore}
  \smaller\begin{lstlisting}
void f(std::ranges::contiguous_range<float>
       auto& data)
{
  using V = std::simd<float>;
  for (std::size_t i = 0;
         i + V::size() <= data.size();
         i += V::size())
  {
    const float* ptr
      = std::ranges::data(data) + i;
    V v(ptr);
    v += 1;
    v.copy_to(ptr);
  }
}
  \end{lstlisting}
  &
  \smaller\begin{lstlisting}
void f(std::ranges::contiguous_range<float>
       auto& data)
{

  for (std::simd<float>::index_type i = 0;
         i < data.size(); ++i)

  {
    data[i] += 1;




  }
}
  \end{lstlisting}
\end{tonytable}%
Without non-member \code{operator[]} overloads, there is little value in providing a \code{simd} index type.
Thus, code would have to resort to the low-level load and store operations of the \code{simd} type (left column in \ttref{loadstore}).
However, with suitable \code{operator[]} overloads, the \code{simd} index type allows concise, readable code which is also generic wrt. the index type%
\footnote{except for handling the epilogue in case the range isn't prepared have a matching size} (right column in \ttref{loadstore}).


\subsection{Vector gather/scatter or load/store on program-defined ranges}
As noted above, the SIMD gather, scatter, load, and store operations are implementation details of \code{simd} rather than the ranges where the memory access happens.
If subscript operators for SIMD were implemented in some ranges / containers (as members) the support would be inconsistent / patchy.
It is much more useful to define the subscript operators to apply to all ranges modelling \code{contiguous_range}, which \emph{requires a non-member \code{operator[]}} (preferably as hidden friend of the index type).
As a consequence we'd see seamless integration of program-defined ranges integrate with \code{simd}.

%For what it's worth, the duplication of \code{operator[](size_t)} in contiguous ranges could have been avoided with a generic non-member \code{operator[]}

\subsection{Multi-index subscripts}
The subscript overloads described in \sect{gatherscatter} and \sect{loadstore} are just as useful for enabling non-SIMD multi-index subscripts.
For example, a user could add a \code{vector<T>::operator[]} overload for a program-defined index type that returns a \code{span}.
Especially for contiguous ranges, which expose an array of data via \code{std::ranges::data}, it is a sensible (better) implementation choice to tie the “not \code{size_t}” \code{operator[]}s to the index type(s).
This requires non-member \code{operator[]} overloads.

\subsection{Subscripting with scoped enumerations}
The use of scoped enumerations can be cumbersome without the ability to use some of the operators that “just work” for unscoped enumerations.
Therefore, it is really useful that operators can be overloaded for scoped enumerations, effectively providing an opt-in for operators.
However, \code{operator[]} stands out.
There's no way to allow subscripting with scoped enumerations unless the necessary \code{operator[]} can be added as a member.
That's typically not going to happen, especially for standard library types like \code{array} and \code{vector}.

Allowing non-member \code{operator[]} fills that hole.
See \lst{colorsubscripts} for an example.
\begin{lstlisting}[numbers=left,float={hbtp},label=lst:colorsubscripts,caption={
  Scoped enumerations sometimes should be able to opt-in to being used as subscript index
}]
enum class Color : char {
  Red,
  Green,
  Blue
};

struct Hsv {
  short hue, sat, val;
};

using ArgbArray = std::array<std::uint32_t, 3>;
using HsvArray = std::array<Hsv, 3>;

const ArgbArray ArgbValue = {{
  0xFFFF0000,
  0xFF00FF00,
  0xFF0000FF,
}};

const HsvArray HsvValue = {{
  {0, 1000, 1000},
  {120, 1000, 1000},
  {240, 1000, 1000}
}};

auto rgb = ArgbValue[Color::Red];  // ERROR
auto hsv = HsvValue[Color::Green]; // ERROR

constexpr inline std::uint32_t operator[](const ArgbArray &values, Color c)
{ return values[static_cast<std::size_t>(c)]; }

constexpr inline Hsv operator[](const HsvArray &values, Color c)
{ return values[static_cast<std::size_t>(c)]; }

auto rgb = ArgbValue[Color::Red];  // now it works
auto hsv = HsvValue[Color::Green]; // now it works
\end{lstlisting}

\subsection{Compatiblity layer}
After P2128 was adopted for \CC{}23, providing multi\hyp dimensional subscript operators for linear algebra (linalg) types (e.g. matrices) will be possible.
It is to be expected that some linalg\hyp libaries will provide these subscript operators, but it is unlikely that every type where multi-dimensional indexing is possible will be updated.
Therefore, a non-member \code{operator[]} overload would allow programs to modernize the interface of such types without patching 3rd-party library types (e.g. \lst{linalg}).
This is similar to program-defined \code{make_unique} in C++11 sources.

\begin{lstlisting}[numbers=left,float={hbtp},label=lst:linalg,caption={
  Extending a type with missing multi-dimensional subscript operator
}]
// linalg.h - external code, cannot be modified
namespace linalg {
  struct matrix {
    float & operator()(size_t row, size_t col);
  };
}

// utils.h - local utility header
namespace linalg {
  inline float& operator[](matrix& self, size_t row, size_t col) {
    return self(row, col);
  }
}

linalg::matrix mat = ...;
mat[3, 3] = 1;
\end{lstlisting}

\section{Implementation}
Implementing non-member subscript overloads for GCC was as simple as removing two lines of code (two special cases) before P2128 was implemented.
After P2128 slightly more work was necessary because operators could only have up to three arguments before P2128 and the new subscript operator code did not look for non-member operators

\section{Wording}

The wording for non\hyp member subscript operators can be adopted from the wording for binary operators (§13.5.2 [over.binary]):

\begin{wgText}{§13.5.5 [over.sub]}
  operator[] shall be\wgAdd{ implemented either by} a non-static member function with\wgRemove{ exactly} one parameter\wgAdd{ or by a non-member function with two parameters}.
  It implements the subscripting syntax\\
  \hspace*{4em}\wgGrammar{postfix-expression \wgIdentifier [ expression \wgIdentifier ]}\\
  \hspace*{2em}or\\
  \hspace*{4em}\wgGrammar{postfix-expression \wgIdentifier [ braced-init-list \wgIdentifier ]}\\

  Thus, a subscripting expression \texttt{x[y]} is interpreted as\wgAdd{ either}
  \texttt{x.operator[](y)} \wgRemove{for a class object \texttt{x} of type \texttt{T} if \texttt{T::operator[](T1)} exists}\wgAdd{or \texttt{operator[](x, y)}}\wgRemove{
  and if the operator is selected as the best match function by the overload resolution mechanism (13.3.3)}.\wgAdd{
  If both forms of the operator function have been declared, the rules in 13.3.1.2 determine which, if any, interpretation is used.}
  [ \textit{Example:}\\
    \texttt{
      \settowidth{\myIndent}{X}%
      struct X \{\\
      \hspace*{2\myIndent}Z operator[](std::initializer\_list<int>);\\
      \};\\
      X x;\\
      x[\{1,2,3\}] = 7;\hspace*{7\myIndent}// \textrm{\itshape OK: meaning} x.operator[](\{1,2,3\})\\
      int a[10];\\
      a[\{1,2,3\}] = 7;\hspace*{7\myIndent}// \textrm{\itshape error: built-in subscript operator}\\
    }
  \textit{— end example} ]
\end{wgText}

In addition, Table 11 in §13.3.1.2 [over.match.oper] mentions the subscript operator and needs to be adjusted:
\begin{wgText}{§13.3.1.2 [over.match.oper]}
  \centering Table 11 — Relationship between operator and function call notation

  \hfil\begin{tabular}{|l|l|l|l|}
    \hline
    \bfseries Subclause & \bfseries Expression & \bfseries As member function & \bfseries As non-member function \\
    \hline
    \hline
    13.5.1 & @a   & (a).operator@()   & operator@(a) \\
    13.5.2 & a@b  & (a).operator@(b)  & operator@(a, b) \\
    13.5.3 & a=b  & (a).operator=(b)  & \\
    13.5.5 & a[b] & (a).operator[](b) & \wgAdd{operator[](a, b)} \\
    13.5.6 & a->  & (a).operator->()  & \\
    13.5.7 & a@   & (a).operator@(0)  & operator@(a, 0) \\
    \hline
  \end{tabular}
\end{wgText}

The following paragraph in [over.match.oper] needs to discuss the subscript operator in addition to unary and binary operators:
\begin{wgText}{§13.3.1.2 [over.match.oper]}
  For a unary operator \texttt{@} with an operand of a type whose cv-unqualified version is \texttt{T1}, for a binary operator \texttt{@} with a left operand of a type whose cv-unqualified version is \texttt{T1} and a right operand of a type whose cv-unqualified version is \texttt{T2}\wgAdd{, and for a subscript operator (where in the following \texttt{@} identifies \texttt{[]}) with a left operand of a type whose cv-unqualified version is \texttt{T1} and subscript operand of a type whose cv-unqualified version is \texttt{T2}}, three sets of candidate functions, designated \emph{member candidates}, \emph{non-member candidates} and \emph{built-in candidates}, are constructed as follows:
  \begin{itemize}
    \item If \texttt{T1} is a complete class type or a class currently being defined, the set of member candidates is the result of the qualified lookup of \texttt{T1::operator@} (13.3.1.1.1); otherwise, the set of member candidates is empty.
    \item The set of non-member candidates is the result of the unqualified lookup of \texttt{operator@} in the context of the expression according to the usual rules for name lookup in unqualified function calls (3.4.2) except that all member functions are ignored.
      However, if no operand has a class type, only those non-member functions in the lookup set that have a first parameter of type \texttt{T1} or “reference to (possibly cv-qualified) \texttt{T1}”, when \texttt{T1} is an enumeration type, or (if there is a right operand) a second parameter of type \texttt{T2} or “reference to (possibly cv-qualified) \texttt{T2}”, when \texttt{T2} is an enumeration type, are candidate functions.
    \item For the operator \texttt{,}, the unary operator \texttt{\&}, or the operator \texttt{->}, the built-in candidates set is empty.
      For all other operators, the built-in candidates include all of the candidate operator functions defined in 13.6 that, compared to the given operator,
      \begin{itemize}
        \item have the same operator name, and
        \item accept the same number of operands, and
        \item accept operand types to which the given operand or operands can be converted according to 13.3.3.1, and
        \item do not have the same parameter-type-list as any non-member candidate that is not a function template specialization.
      \end{itemize}
  \end{itemize}
\end{wgText}

\section{Evaluation}

Since, currently the declaration of non\hyp member subscript operators is ill\hyp formed, there are no backwards compatibility issues to expect.


\end{document}
