\wgSubsection{Header \code{<datapar>} synopsis}{datapar.syn}
\lstinputlisting[]{synopsis.cpp}

\pnum
The header \code{<datapar>} defines two class templates (\datapar, and \mask), several tag types, and a series of related function templates for concurrent manipulation of the values in \datapar and \mask objects.

\begin{itemdecl}
namespace datapar_abi {
struct scalar {};
// implementation-defined tag types, e.g. sse, avx, avx512, neon, ...
typedef implementation_defined compatible;
typedef implementation_defined native;
}
\end{itemdecl}
\begin{itemdescr}
  \pnum
  The ABI types are tag types to be used as the second template argument to \datapar and \mask.

  \pnum
  The \type{scalar} tag is present in all implementations and forces \datapar and \mask to store a single component (i.e. \datapar{}\type{<T, datapar_abi::scalar>::size()} returns \code 1).

  \pnum
  An implementation may choose to implement data-parallel execution for many different targets.
  \wgNote{There can certainly be more than one tag type per (micro-)architecture, e.g. to support different vector lengths or partial register usage.}
  All tag types an implementation supports shall be present independent of the target architecture determined at invocation of the compiler.

  \pnum
  The \type{datapar_abi::compatible} tag is defined by the implementation to alias the tag type with the most efficient data parallel execution that ensures the highest compatibility on the target architecture.

  \pnum
  The \type{datapar_abi::native} tag is defined by the implementation to alias the tag type with the most efficient data parallel execution that is supported on the target system.
\end{itemdescr}

\wgSubsubsection{\datapar type traits}{datapar.traits}
\begin{itemdecl}
template <class T> struct is_datapar;
\end{itemdecl}
\begin{itemdescr}
  \pnum The \type{is_datapar} type derives from \type{true_type} if \type T is an instance of the \datapar class template.
  Otherwise it derives from \type{false_type}.
\end{itemdescr}

\begin{itemdecl}
template <class T> struct is_mask;
\end{itemdecl}
\begin{itemdescr}
  \pnum The \type{is_mask} type derives from \type{true_type} if \type T is an instance of the \mask class template.
  Otherwise it derives from \type{false_type}.
\end{itemdescr}

\begin{itemdecl}
template <class T, size_t N> struct abi_for_size { typedef implementation_defined type; };
\end{itemdecl}
\begin{itemdescr}
  \pnum
  The \type{abi_for_size} class template defines the member type \type{type} to one of the tag types in \code{datapar_abi} or not at all, depending on the value of the template parameters.

  \pnum
  \code{datapar<T, abi_for_size_t<T, N>>::size()} shall return \code N or result in a substitution failure.
\end{itemdescr}

\begin{itemdecl}
template <class T, class Abi = datapar_abi::compatible>
struct datapar_size : public integral_constant<size_t, implementation_defined> {};
\end{itemdecl}
\begin{itemdescr}
  \pnum The \type{datapar_size} class template inherits from \type{integral_constant} with a value that equals \datapar{}\code{<T, Abi>::size()}.

  \pnum \code{datapar_size<T, Abi>::value} shall result in a substitution failure if any of the template arguments \type T or \type{Abi} are invalid template arguments to \datapar.
\end{itemdescr}

