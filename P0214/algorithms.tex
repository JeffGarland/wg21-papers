\section{Parallel Algorithms}

Parallel Algorithms allow implementations of the STL algorithms to use non-sequential semantics when executing the user-supplied code (explicit callable or implicit operator call).
The first argument to the algorithm function determines this change in execution semantics via an \emph{execution policy}.
This paper introduces a new execution policy, called \dataparEP.
\dataparEP requires user-provided function objects to be callable with \datapar[<T, Abi>] arguments instead of the \type T arguments the \seqEP variant would use.
%Otherwise, essentially retains sequential execution semantics in the user-supplied code.
%, except that  element access functions use \datapar load and stores whenever possible.

Consider the example in \lst{datapar foreach}.
\begin{lstlisting}[style=Vc,numbers=left,float,label=lst:datapar foreach,caption={
  Small example using \dataparEP with \code{iota} and \code{for_each}.
}]
vector<float> data;
data.resize(99);
iota(execution_policy::datapar, data.begin(), data.end(), 0.f);
for_each(execution_policy::datapar, data.begin(), data.end(), [](auto &x) {
  x *= x;
});
\end{lstlisting}
The \code{iota} and \code{for_each} functions each could create an internal \datapar iterator adaptor, depending on the iterator category.
The information about contiguous storage is most important in this context to enable vector loads and stores.
The \type{vector} iterators are \emph{contiguous iterators}.
An implementation could thus generate the equivalent of \lst{datapar iota implementation} and \lst{datapar foreach implementation}.
\lstinputlisting[style=Vc,numbers=left,float,label=lst:datapar iota implementation,caption={
  Pseudo-implementation of the \code{iota} function used in \lst{datapar foreach}.
}]{iota.cpp}
\lstinputlisting[style=Vc,numbers=left,float,label=lst:datapar foreach implementation,caption={
  Pseudo-implementation of the \code{for_each} function used in \lst{datapar foreach}.
}]{foreach.cpp}
