\newcommand\wgTitle{Data-Parallel Vector Types \& Operations}
\newcommand\wgName{Matthias Kretz <m.kretz@gsi.de>}
\newcommand\wgDocumentNumber{D0214R4}
\newcommand\wgGroup{LEWG}
\newcommand\wgTarget{Parallelism TS 2}
\newcommand\wgAcknowledgements{
  \begin{itemize}
    \item This work was supported by GSI Helmholtzzentrum für Schwerionenforschung
      and the Hessian LOEWE initiative through the Helmholtz International Center
      for FAIR (HIC for FAIR).

    \item Jens Maurer contributed important feedback and suggestions on the API.
      Thanks also for presenting the paper in Kona 2017.

    \item Thanks to Hartmut Kaiser for presenting in Issaquah 2016.

    \item Geoffrey Romer did a very helpful review of the wording.
  \end{itemize}
}

\usepackage{typenames}
\usepackage{mymacros}
\usepackage{wg21}
\usepackage{underscore}

\addbibresource{extra.bib}

\newcommand\datapar[1][]{\type{datapar#1}\xspace}
\newcommand\valuetype{\type{value\_type}\xspace}
\newcommand\referencetype{\type{reference}\xspace}
\newcommand\whereexpression{\type{where\_expression}\xspace}
\newcommand\simdcast{\code{datapar\_cast}\xspace}
\newcommand\mask[1][]{\type{mask#1}\xspace}
\newcommand\fixedsizeN{\type{datapar\_abi::fixed\_size<N>}\xspace}
\newcommand\fixedsizescoped{\type{datapar\_abi::fixed\_size}\xspace}
\newcommand\fixedsize{\type{fixed\_size}\xspace}
\newcommand\dataparEP{\code{execution\_policy::}\type{datapar}\xspace}
\newcommand\seqEP{\code{execution\_policy::}\type{seq}\xspace}

\newcommand\flagsRemarks[1]{
  \pnum\requires
  If the \type{Flags} template parameter is of type \type{flags::vector\_aligned\_tag}, the pointer value represents an address aligned to \code{memory\_alignment\_v<#1>}.
  If the \type{Flags} template parameter is of type \type{flags::overaligned\_tag<N>}, the pointer value represents an address aligned to \code N.
}

\newcommand\targetArch{target architecture\xspace}
\newcommand\targetArchs{target architectures\xspace}
\newcommand\currentTarget{currently targeted system\xspace}

\newcommand\realArithmeticType{arithmetic type except \type{bool}\xspace}

\usepackage{pifont}

\newcommand\foralli[1][]{for all \code i $\in$ \code{[0, #1size())}\xspace}
\newcommand\forallmaskedi[1]{%
  for all \code i
  $\in \{j \in \mathbb{N}_0 | j < \code{size()} ⋀ \code{#1[}j\code{]}\}$%
  \xspace%
}
\newcommand\chck{\item[\color{black}\ensuremath{\checkmark}]}
\newcommand\todo{\item[\color{black}\ding{46}] \color{gray}}
\newcommand\itemheader[1]{\item[] \hfill \textcolor{gray}{\textsc{#1}}}

\begin{document}
\selectlanguage{american}
\begin{wgTitlepage}
  This paper describes class templates for portable data-parallel (e.g. SIMD) programming via vector types.
\end{wgTitlepage}

\pagestyle{scrheadings}
\addtocounter{section}{-1}
\section{Remarks}
\begin{itemize}
  \item This documents talks about “vector” types/objects.
    In general this will not refer to the \std\type{vector} class template.
    References to the container type will explicitly call out the \code{std} prefix to avoid confusion.
  \item In the following, \VSize{T} denotes the number of scalar values (width) in a vector of type \type T (sometimes also called the number of SIMD lanes)
  \item \parencite{N4184}, \parencite{N4185}, and \parencite{N4395} provide more information on the rationale and design decisions.
    \parencite{N4454} discusses a matrix multiplication example.
    My PhD thesis \parencite{Kretz2015} contains a very thorough discussion of the topic.
  \item This paper is not supposed to specify a complete API for data-parallel types and operations.
    It is meant as a useful starting point.
    Once the foundation is settled on, higher level APIs will be proposed.
\end{itemize}

\section{Changelog}
\subsection{Changes from R0}
Previous revision: \parencite{P0214R0}.
\begin{itemize}
  \item Extended the \code{datapar_abi} tag types with a \code{fixed_size<N>} tag to handle arbitrarily sized vectors (\ref{sec:datapar.abi}).
  \item Converted \code{memory_alignment} into a non-member trait (\ref{sec:datapar.traits}).
  \item Extended implicit conversions to handle \fixedsizeN (\ref{sec:datapar.ctor}).
  \item Extended binary operators to convert correctly with \fixedsizeN (\ref{sec:datapar.binary}).
  \item Dropped the section on “\datapar logical operators”. Added a note that the omission is deliberate (\ref{sec:datapar.logical}).
  \item Added logical and bitwise operators to \mask (\ref{sec:mask.binary}).
  \item Modified \mask compares to work better with implicit conversions (\ref{sec:mask.comparison}).
  \item Modified \code{where} to support different Abi tags on the \mask and \datapar arguments (\ref{sec:mask.where}).
  \item Converted the load functions to non-member functions.
    SG1 asked for guidance from LEWG whether a load-expression or a template parameter to load is more appropriate.
  \item Converted the store functions to non-member functions to be consistent with the load functions.
  \item Added a note about masked stores not invoking out-of-bounds accesses for masked-off elements of the vector.
  \item Converted the return type of \datapar{}\code{::operator[]} to return a smart reference instead of an lvalue reference.
  \item Modified the wording of \mask{}\code{::operator[]} to match the reference type returned from \datapar{}\code{::operator[]}.
  \item Added non-trig/pow/exp/log math functions on \datapar.
  \item Added discussion on defaulting load/store flags.
  \item Added sum, product, min, and max reductions for \datapar.
  \item Added load constructor.
  \item Modified the wording of \code{native_handle()} to make the existence of the functions implementation-defined, instead of only the return type.
    Added a section in the discussion (cf. Section \ref{sec:native}).
  \item Fixed missing flag objects.
\end{itemize}
  %\todo Fix missing flag objects and specify the freedom for implementations to add additional flags and that \code{operator|} must work.
  %\todo Added usage examples

\subsection{Changes from R1}
Previous revision: \parencite{P0214R1}.
\begin{itemize}
    \item Fixed converting constructor synopsis of \datapar and \mask to also allow varying Abi types.
    \item Modified the wording of \code{\mask{}::native_handle()} to make the existence of the functions implementation-defined.
    \item Updated the discussion of member types to reflect the changes in R1.
    \item Added all previous SG1 straw poll results.
    \item Fixed \code{\textit{commonabi}} to not invent native Abi that makes the operator ill-formed.
    \item Dropped table of math functions.
    \item Be more explicit about the implementation-defined Abi types.
    \item Discussed resolution of the \fixedsizeN design (\ref{sec:fixedsize progress}).
    \item Made the \type{compatible} and \type{native} ABI aliases depend on \type T (\ref{sec:datapar.abi}).
    \item Added \code{max_fixed_size} constant (\ref{datapar.maxfixedsize.def}).
    \item Added masked loads.
    \item Added rationale for return type of \datapar[::operator-()] (\ref{sec:unary minus}).
  \color{black}\item[---] SG1 guidance:
    \item Dropped the default load / store flags.
    \item Renamed the (un)aligned flags to \code{element_aligned} and \code{vector_aligned}.
    \item Added an \code{overaligned<N>} load / store flag.
    \item Dropped the ampersand on \code{native_handle} (no strong preference).
    \item Completed the set of math functions (i.e. add trig, log, and exp).
  \color{black}\item[---] LEWG (small group) guidance:
    \item Dropped \code{native_handle} and add non-normative wording for supporting \code{static_cast} to implementation-defined SIMD extensions.
    \item Dropped non-member load and store functions.
    Instead have \code{copy_from} and \code{copy_to} member functions for loads and stores. (\ref{sec:datapar.load}, \ref{sec:datapar.store}, \ref{sec:mask.load}, \ref{sec:mask.store})
    (Did not use the \code{load} and \code{store} names because of the unfortunate inconsistency with \std\type{atomic}.)
    \item Added algorithm overloads for \datapar reductions.
    Integrate with \code{where} to enable masked reductions. (\ref{sec:datapar.reductions})
    This made it necessary to spell out the class \type{where_expression}.
\end{itemize}
\subsection{Changes from R2}
Previous revision: \parencite{P0214R2}.
\begin{itemize}
    \item Fixed return type of masked \code{reduce} (\ref{sec:datapar.reductions}).
    \item Added binary \code{min}, \code{max}, \code{minmax}, and \code{clamp} (\ref{sec:datapar.alg}).
    \item Moved member \code{min} and \code{max} to non-member \code{hmin} and \code{hmax}, which cannot easily be optimized from \code{reduce}, since no function object such as \code{std::plus} exists (\ref{sec:datapar.reductions}).
    \item Fixed neutral element of masked \code{hmin}/\code{hmax} and drop UB (\ref{sec:datapar.reductions}).
    \item Removed remaining reduction member functions in favor of non-member \code{reduce} (as requested by LEWG).
    \item Replaced \code{init} parameter of masked \code{reduce} with \code{neutral_element} (\ref{sec:datapar.reductions}).
    \item Extend \type{where_expression} to support \const \datapar objects (\ref{sec:mask.where}).
    \item Fixed missing \code{explicit} keyword on \code{mask(bool)} constructor (\ref{sec:mask.ctor}).
    \item Made binary operators for \datapar and \mask friend functions of \datapar and \mask, simplifying the SFINAE requirements considerably (\ref{sec:datapar.binary}, \ref{sec:mask.binary}).
    \item Restricted broadcasts to only allow non-narrowing conversions (\ref{sec:datapar.ctor}).
    \item Restricted datapar to datapar conversions to only allow non-narrowing conversions with \type{fixed_size} ABI (\ref{sec:datapar.ctor}).
    \item Added generator constructor (as discussed in LEWG in Issaquah) (\ref{sec:datapar.ctor}).
    \item Renamed \code{copy_from} to \code{memload} and \code{copy_to} to \code{memstore}. (\ref{sec:datapar.load}, \ref{sec:datapar.store}, \ref{sec:mask.load}, \ref{sec:mask.store})
    \item Documented effect of \type{overaligned_tag<N>} as \type{Flags} parameter to load/store. (\ref{sec:datapar.load}, \ref{sec:datapar.store}, \ref{sec:mask.load}, \ref{sec:mask.store})
    \item Clarified cv requirements on \type T parameter of \datapar and \mask.
    \item Allowed all implicit \mask conversions with \fixedsize ABI and equal size (\ref{sec:mask.ctor}).
    \item Made increment and decrement of \type{where_expression} return \type{void}.
    \item Added \code{static_datapar_cast} for simple casts (\ref{sec:datapar.casts}).
    \item Clarified default constructor (\ref{sec:datapar.overview}, \ref{sec:datapar.overview}).
    \item Clarified \datapar and \mask with invalid template parameters to be complete types with deleted constructors, destructor, and assignment (\ref{sec:datapar.overview}, \ref{sec:datapar.overview}).
    \item Wrote a new subsection for a detailed description of \type{where_expression} (\ref{sec:datapar.whereexpr}).
    \item Moved masked loads and stores from \datapar and \mask to \type{where_expression} (\ref{sec:datapar.whereexpr}).
          This required two more overloads of \code{where} to support value objects of type \mask (\ref{sec:mask.where}).
    \item Removed \type{where_expression}\code{::operator!} (\ref{sec:datapar.whereexpr}).
    \item Added aliases \type{native_datapar}, \type{native_mask}, \type{fixed_size_datapar}, \type{fixed_size_mask} (\ref{sec:datapar.syn}).
    \item Removed \bool overloads of mask reductions awaiting a better solution (\ref{sec:mask.reductions}).
    \item Removed special math functions with \code f and \code l suffix and \code l and \code{ll} prefix (\ref{sec:datapar.math}).
    \item Modified special math functions with mixed types to use \type{fixed_size} instead of \code{abi_for_size} (\ref{sec:datapar.math}).
    \item Added simple ABI cast functions \code{to_fixed_size}, \code{to_native}, and \code{to_compatible} (\ref{sec:datapar.casts}).
\end{itemize}

\subsection{Changes from R3}
Previous revision: \parencite{P0214R3}.
\begin{itemize}
  \itemheader{changes before Kona}
  \item Add special math overloads for signed char and short.
        They are important to avoid widening to multiple SIMD registers and since no integer promotion is applied for \datapar types.
  \item Editorial: Prefer \code{using} over \code{typedef}.
  \item Overload shift operators with \intt argument for the right hand side.
        This enables more efficient implementations.
        This signature is present in the Vc library, and was forgotten in the wording.
  \item Remove empty section about the omission of logical operators.
  \item Modify \mask compares to return a \mask instead of \bool (\ref{sec:mask.comparison}).
        This resolves an inconsistency with all the other binary operators.
  \item Editorial: Improve \referencetype member specification (\ref{sec:datapar.overview}).
  \item Require \code{swap(v[0], v[1])} to be valid (\ref{sec:datapar.overview}).
  \item Fixed inconsisteny of masked load constructor after move of \code{memload} to \type{where_expression} (\ref{sec:datapar.whereexpr}).
  \item Editorial: Use Requires clause instead of Remarks to require the memory argument to loads and stores to be large enough (\ref{sec:datapar.whereexpr}, \ref{sec:datapar.load}, \ref{sec:datapar.store}, \ref{sec:mask.load}, \ref{sec:mask.store}).
  \item Add a note to special math functions that precondition violation is UB (\ref{sec:datapar.math}).
  \item Bugfix: Binary operators involving two \type{datapar::reference} objects must work (\ref{sec:datapar.overview}).
  \item Editorial: Replace Note clauses in favor of \wgNote{}.
  \item Editorial: Replace UB Remarks on load/store alignment requirements with Requires clauses.
  \item Add an example section (\ref{sec:Examples}).
  \itemheader{changes after Kona}
  \item[---] design related:
  \item Readd \bool overloads of mask reductions and ensure that implicit conversions to \bool are ill-formed.
  \item Clarify effects of using an ABI parameter that is not available on the target (\ref{datapar.deleted}, \ref{mask.deleted}, \ref{datapar_size}).
  \item Split \whereexpression into \const and non-\const class templates.
  \item Add section on naming (\autoref{sec:Naming}).
  \item Discuss the questions/issues raised on \code{max_fixed_size} in Kona (\autoref{sec:maxfixedsize}).
  \item Make \code{max_fixed_size} dependent on \type{T}.
  \item Clarify that converting loads and stores only work with arrays of non-bool arithmetic type (\ref{sec:datapar.load}, \ref{sec:datapar.store}).
  \item Discuss \mask and \type{bitset} reduction interface differences (\autoref{sec:mask queries}).
  \item Relax requirements on return type of generator function for the generator constructor (\ref{sec:datapar.ctor}).
  \item Remove overly generic \code{datapar_cast} function.
  \item Add proposal for a widening cast function (\autoref{sec:widen}).
  \item Add proposal for \code{split} and \code{concat} cast functions (\autoref{sec:split and concat}).
  \item Add \code{noexcept} or “Throws: Nothing.” to most functions.

  \item[---] wording fixes \& improvements:
  \item Remove non-normative noise about ABI tag types (\ref{sec:datapar.abi}).
  \item Remove most of the text about vendor-extensions for ABI tag types, since it's QoI (\ref{sec:datapar.abi}).
  \item Clarify the differences and intent of \type{compatible<T>} vs. \type{native<T>} (\ref{sec:datapar.abi}).
  \item Move definition of \whereexpression out of the synopsis (\ref{sec:datapar.whereexpr}).
  \item Editorial: Improve \code{is_datapar} and \code{is_mask} wording (\ref{sec:datapar.traits}).
  \item Make \emph{ABI tag} a consistent term and add \code{is_abi_tag} trait (\ref{sec:datapar.traits}, \ref{sec:datapar.abi}).
  \item Clarify that \fixedsizeN must support all \code{N} matching all possible implementation-defined ABI tags (\ref{sec:datapar.abi}).
  \item Clarify \code{abi_for_size} wording (\ref{sec:datapar.traits}).
  \item Turn \code{memory_alignment} into a trait with a corresponding \code{memory_alignment_v} variable template.
  \item Clarify \code{memory_alignment} wording; when it has no \code{value} member; and imply its value through a reference to the load and store functions (\ref{sec:datapar.traits}).
  \item Remove exposition-only \type{where_expression} constructor and make exposition-only data members private (\ref{sec:datapar.whereexpr}).
  \item Editorial: use “shall not participate in overload resolution unless” consistently.
  \item Add a note about variability of \code{max_fixed_size} (\ref{sec:datapar.abi}).
  \item Editorial: use “\targetArch{}” and “\currentTarget{}” consistently.
  \item Add margin notes presenting a wording alternative that avoids “target system” and “target architecture” in normative text.
  \item Specify result of masked reduce with empty mask (\ref{sec:datapar.reductions}).
  \item Editorial: clean up the use of “supported” and resolve contradictions resulting from incorrect use of conventions in the rest of the standard text (\ref{datapar.type requirements}, \ref{mask.type requirements}, \ref{sec:datapar.traits}).

  %\todo New section on design decisions; discuss decisions made by SG1, LEWG, and through discussion with Jens; especially discuss the ABI parameter:
  %\begin{itemize}
  %  \item it provides the intended lowest level access to SIMD programming
  %  \item it allows easy high-level abstraction
  %  \item how the default was chosen, and why having a \emph{compatible} ABI be the default is the nicer default
  %  \item an ABI parameter influences the choice of function parameter passing, not necessarily the choice of instructions or even number of elements (e.g. an implementation can provide an ABI that passes via 128-bit registers but otherwise works with 256-bit registers)
  %\end{itemize}
\end{itemize}

\section{Straw Polls}
\subsection{LEWG telecon 2022-03-29}
\wgPoll
{Numeric traits can deviate from \code{numeric_limits}.}
{13&8&0&0&0}

\wgPoll
{Numeric traits should be based on representation rather than behavior (ignoring \code{reciprocal_overflow_threshold}).}
{7&5&2&0&0}

\wgPoll
{All numeric traits for bool should be disabled.}
{12&6&1&0&0}

\wgPoll
{The numeric traits that are not meaningful for \code{numeric_limits} (\code{denorm_min}, \code{epsilon}, etc) should be disabled for integral types.}
{14&3&0&0&0}

\wgPoll
{\code{max_digits10} should deviate from \code{numeric_limits} and yields \code{digits10_v + 1}.}
{6&5&2&0&0}

\subsection{LEWG telecon 2022-06-07}
\wgPoll
{Remove \code{reciprocal_overflow_threshold} from P1841.}
{6&4&1&0&0}

\section{Introduction}

Parallel Algorithms enable implementations of the existing STL algorithms to use non-sequential semantics when executing the user-supplied code (explicit callable or implicit operator call).
The first argument to the algorithm function determines this change in execution semantics via an \emph{execution policy}.
This paper introduces a new execution policy, called \dataparEP.
\dataparEP requires user-provided function objects to be callable with \datapar[<T, Abi>] arguments instead of the \type T arguments the \seqEP variant would use.
The algorithm therefore processes chunks of \datapar[<T, Abi>::size()] objects concurrently.
The execution order of the chunks retains the sequential semantics of the non-parallel algorithms.

As a consequence, the applicability of the execution policy is limited to iterators where \datapar[<Iterator::value_type>] is a valid template instantiation of \datapar.
A future extension of \datapar may lift this restriction by allowing certain (or all) user-defined types as first template argument to \datapar.


\section{Examples}\label{sec:Examples}

\subsection{Loop Vectorization}
This shows a low-level approach of manual loop chunking + epilogue for vectorization (“Leave no room for a lower-level language below \CC{} (except assembler).” \parencite{str99}).
It also shows SIMD loads, operations, write-masking (blending), and stores.
\medskip
\lstinputlisting[linerange={1-18}]{examples.cpp}

%\subsection{}


% ft=tex tw=0 et sw=2 spell

\section{Naming}\label{sec:Naming}

The name \datapar was chosen in SG1 after a short discussion, brainstorm session, and straw poll.
Names can still be changed by LEWG.
The following will present naming ideas and a bit of discussion of pros and cons and make
recommendations.

\subsection{datapar}

The class in question is an array of target-specific size, with elements of type T, and data parallel operation semantics.
The actual memory layout and storage size is unspecified.
The number of elements is influenced via the second template parameter.
If the second template parameter is \code{fixed_size<N>}, an exact number of N elements is used.
Operations on objects of the type execute the operation component-wise and concurrently.
This allows the user to communicate data parallelism inherent in the problem at hand.
An implementation might translate the data parallelism into SIMD instructions, GPU parallelism, serial execution, synchronized multi-core execution, or any mix thereof.
The implementation is expected to provide guarantees about the resulting code gen depending on compiler flags and the given ABI parameter (second template parameter), e.g. “\code{datapar<int, datapar_abi::sse>} uses \code{xmm} registers for storage and all ISA extensions enabled via compiler flags.

\subsubsection{Naming Options}

\begin{itemize}
  \item \code{vector<T>}
  \item \code{vec<T>}
  \item \code{vecpar<T>}
  \item \code{simd<T>}
  \item \code{datapar<T>}
  \item \code{pack<T>}
  \item \code{simdarray<T>} / \code{simdvector<T>} / \code{vecarray<T>}
  \item \code{vectorize<T>} / \code{simdize<T>} / \code{vectize<T>} / \code{vectorized<T>} / \code{simdized<T>} / \code{vectized<T>}
\end{itemize}

\subsubsection{Discussion}

\begin{itemize}
  \item \code{vector<T>}
    \begin{description}
      \item[pro]
        \begin{enumerate}
          \item term-of-art in the industry. We talk about "vectorization", "vector unit", "vector registers", ...
          \item does work as a mathematical vector, e.g. \code{std::reduce<std::plus>(x*y)} is the scalar product
        \end{enumerate}
      \item[con]
        \begin{enumerate}
          \item \emph{[name collision]} \code{std::vector}: the name is taken. Using a different namespace won't help: Too much confusion/conflict with \code{std::vector}, which is not constant-size.
          \item ambiguity with mathematically inclined people who may expect operators to behave differently (especially, I've heard feedback of users expecting \code{operator*} to be the dot-product)
        \end{enumerate}
    \end{description}

  \item \code{vec<T>}
    \begin{description}
      \item[pro]
        \begin{enumerate}
          \item short
          \item pronounceable
          \item usage is somewhat idiomatic: vec<T> is a vector-lookalike of T.
          \item term-of-art in the industry. We talk about "vectorization", "vector unit", "vector registers", ...
        \end{enumerate}
      \item[con]
        \begin{enumerate}
          \item abbreviation
          \item close to \code{std::vector}
          \item ambiguity with mathematically inclined people who may expect operators to behave differently (especially, I've heard feedback of users expecting \code{operator*} to be the dot-product)
        \end{enumerate}
    \end{description}

  \item \code{vecpar<T>}
    \begin{description}
      \item[pro]
        \begin{enumerate}
          \item short
          \item pronounceable
          \item term-of-art in the industry. We talk about "vectorization", "vector unit", "vector registers", ...
          \item resolves ambiguity with math vector
        \end{enumerate}
      \item[con]
        \begin{enumerate}
          \item abbreviation ("vector parallel")
          \item (\code{par_vec} - it's \code{par_unseq} now, so we should be fine)
        \end{enumerate}
    \end{description}

  \item \code{simd<T>}
    \begin{description}
      \item[pro]
        \begin{enumerate}
          \item short
          \item pronounceable
          \item usage is idiomatic: simd<T> is the SIMD thing for T.
          \item Known term in the industry
          \item maybe even more to the point than "vector" (note variable-length vector units on traditional vector computers)
        \end{enumerate}
      \item[con]
        \begin{enumerate}
          \item acronym
          \item might suggest that the type is not usable for GPUs
          \item one \code{simd<T>} object could drive multiple or partial SIMD registers, multiple partially synchronized threads, one or more non-SIMD registers, a mix of SIMD and non-SIMD registers.
        \end{enumerate}
    \end{description}

  \item \code{datapar<T>}
    \begin{description}
      \item[pro]
        \begin{enumerate}
          \item pronounceable
          \item "data parallel" hints at the intended use: Code expresses inherent data parallelism (intent). Contrast that to "code that uses SIMD registers and operations" (implementation detail).
        \end{enumerate}
      \item[con]
        \begin{enumerate}
          \item abbreviation
          \item new term
        \end{enumerate}
    \end{description}

  \item \code{pack<T>}
    \begin{description}
      \item[pro]
        \begin{enumerate}
          \item short
          \item pronounceable
          \item usage is somewhat idiomatic (e.g. \code{addpd}: “add packed double-precision”)
        \end{enumerate}
      \item[con]
        \begin{enumerate}
          \item \emph{[name collision]} Conflicts with "template parameter pack" usage in variadic templates.
            These tend to appear in the same context: "You can have a [template parameter] pack of packs [types]." (what?)
          \item no hint about concurrently executing operations in the name
        \end{enumerate}
    \end{description}

  \item \code{simdarray<T>}
    \begin{description}
      \item[pro]
        \begin{enumerate}
          \item matches constant-length \code{std:array} and math-style of \code{std::valarray}.
          \item pronounceable
          \item usage is idiomatic: SIMD operations on a fixed-size array
        \end{enumerate}
      \item[con]
        \begin{enumerate}
          \item a bit long for daily use
          \item acronym
          \item might suggest that the type is not usable for GPUs
        \end{enumerate}
      \item[variations]\ 
        \begin{enumerate}
          \item \code{simdvector<T>}: "vector" suggests \code{std::vector} behavior - prefer \code{simdarray<T>}
          \item \code{vecarray<T>}: abbreviation ("vectorized array", \emph{not} "vector array");\\
            "vector array" misleading
        \end{enumerate}
    \end{description}

  \item \code{vectorize<T>}
    \begin{description}
      \item[pro]
        \begin{enumerate}
          \item pronounceable
          \item clear meaning: produces a type that is a \emph{vectorized} \code{T}\\
            i.e. action at compile time, so being a verb is fine
          \item clear meaning if proposal is extended to support \code{std::tuple} for \code{T} (and structs/classes once we get enough reflection into the language)
        \end{enumerate}
      \item[con]
        \begin{enumerate}
          \item it is a class, it should be a noun (\code{vectorization<T>}?)
          \item a bit long for daily use
        \end{enumerate}
      \item[variations]\ 
        \begin{enumerate}
          \item \code{simdize<T>}: shorter; downsides of \code{simd} - see above
          \item \code{vectize<T>}: shorter; abbreviation
          \item \code{vectorized<T> simdized<T>}: adjective, still not a noun
        \end{enumerate}
    \end{description}
\end{itemize}

\subsubsection{Recommendation}
I recommend to short-list to:
\begin{itemize}
  \item \code{vec<T>}
  \item \code{vecpar<T>}
  \item \code{simd<T>}
  \item \code{datapar<T>}
  \item \code{simdarray<T>}
  \item \code{vectorize<T>}
  \item \code{simdize<T>}
\end{itemize}

\subsection{mask}

The class in question is an array of target-specific size with elements of boolean value.
The actual memory layout and storage size is unspecified.
This type is the equivalence of \code{bool} for the \code{datapar<T>} types.
It acts as the return type of \code{datapar} comparisons and can be used for write-masking, masked loads \& stores, and reductions to \code{bool}.

\subsubsection{Naming Options}

\begin{itemize}
  \item \code{mask<T>}
  \item \code{vecmask<T>}
  \item \code{boolvec<T>}
  \item \code{simdmask<T>}
  \item \code{simdbool<T>}
  \item \code{parmask<T>}
  \item \code{boolpack<T>}
\end{itemize}

\subsubsection{Discussion}

Depending on the name chosen for the “datapar” class, there are some natural candidates for the \code{mask} class.
In any case, the \code{mask} name is:
\begin{enumerate}
  \item a term-of-art,
  \item short,
  \item pronounceable,
  \item idiomatic,
  \item noun,
  \item no name collision with existing types (as is the case for \code{vector}).
\end{enumerate}
So I do not see a need for choosing a different (longer) name.

\subsection{where}
The “where function” wraps a \code{mask} object and a reference to a \code{datapar} or \code{mask} object to implement write-masking, and masked loads \& stores.
The function acts as special syntax to express that e.g. assignment shall only happen at the element indexes where the mask object is \code{true}.
The where function returns a temporary object (type \code{where_expression}) that implements the write-masked operations.

\subsubsection{Naming Options}
\begin{itemize}
  \item \code{where}
  \item \code{masked}
  \item \code{withmask}
  \item \code{maskedval}
  \item \code{maskedref}
\end{itemize}

\subsubsection{Discussion}

\begin{itemize}
  \item \code{where}
    \begin{description}
      \item[pro]
        \begin{enumerate}
          \item short
          \item pronounceable
          \item turns code into prose: \code{where(x < y, z) += 2;} reads as “where x is less then y, modify z by adding 2”
          \item naming reflects relation to \code{if} statements
        \end{enumerate}
      \item[con]
        \begin{enumerate}
          \item less intuitive if extended and used in the middle of expressions, e.g. \code{fun(where(mask, v))}
        \end{enumerate}
    \end{description}

  \item \code{masked}
    \begin{description}
      \item[pro]
        \begin{enumerate}
          \item short
          \item pronounceable
        \end{enumerate}
      \item[con]
        \begin{enumerate}
          \item too close to \code{mask}: ambiguous when spoken
        \end{enumerate}
    \end{description}

  \item \code{withmask}
    \begin{description}
      \item[pro]
        \begin{enumerate}
          \item pronounceable
        \end{enumerate}
      \item[con]
        \begin{enumerate}
          \item less intuitive to read: \code{withmask(x < y, z) += 2;}\\
            does something \emph{with} a mask, what?
        \end{enumerate}
    \end{description}

  \item \code{maskedval}
    \begin{description}
      \item[pro]
        \begin{enumerate}
          \item pronounceable
          \item communicates: produce a new object that is a \emph{masked value} of the given object
        \end{enumerate}
      \item[con]
        \begin{enumerate}
          \item \emph{value} is not technically correct as it actually holds a reference to the given object
          \item the object returned by \code{maskedval} may only exists as rvalue;
            the name suggests otherwise
        \end{enumerate}
    \end{description}

  \item \code{maskedref}
    \begin{description}
      \item[pro]
        \begin{enumerate}
          \item pronounceable
          \item communicates: produce a new object that is a \emph{masked reference} to the given object
        \end{enumerate}
      \item[con]
        \begin{enumerate}
          \item the object returned by \code{maskedref} may only exists as rvalue;
            the name suggests otherwise
        \end{enumerate}
    \end{description}

\end{itemize}

\subsubsection{Recommendation}
My recommendation is to go with \code{where} for what is in the wording now.
If we later want to produce lvalues that act as masked references, I believe we should use a different mechanism/name anyway.
Pablo suggested in private communication that \code{where} could be extended to:
\begin{lstlisting}[style=Vc]
where (mask, v1, v2, [](auto v1_, auto v2_) {
  // type of v1_ is a masked reference to v1
  fun(v1_, v2_);  // all operations of fun on its parameters are masked
});
\end{lstlisting}
This suggests that there might not even be a need for allowing \code{where} or a new function/type in the middle of expressions.

\subsection{memload \& memstore}
Loads and stores are the (low-level) conversions between arrays of \code{T} and objects of \code{datapar<T>}.
Converting loads and stores additionally perform widening or narrowing conversions to/from arrays of \code{U}, which is convertible to/from \code{T}.

\code{std::atomic} has member functions called \code{atomic::load} and \code{atomic::store}: \code{load} returns the value of the \code{atomic} with a given \code{memory_order}; \code{store} replaces the value of \code{atomic} with the given value using the given \code{memory_order}.
\code{datapar::load} does the reverse of \code{atomic::load}: it loads \code{datapar::size()} consecutive values starting from the given pointer into the \code{datapar} object.
\code{datapar::store} does the reverse of \code{atomic::store}: it stores \code{datapar::size()} values from the \code{datapar} object to the given pointer.

\subsubsection{Naming Options}

\begin{itemize}
  \item \code{load(const U*, Flags), store(U*, Flags)}
  \item \code{memload(const U*, Flags), memstore(U*, Flags)}
  \item \code{load_from(const U*, Flags), store_to(U*, Flags)}
  \item \code{copy_from(const U*, Flags), copy_to(U*, Flags)}
\end{itemize}

\subsubsection{Discussion}
\begin{itemize}
  \item \code{load(const U*, Flags), store(U*, Flags)}
    \begin{description}
      \item[pro]
        \begin{enumerate}
          \item short
          \item pronounceable
          \item term-of-art
        \end{enumerate}
      \item[con]
        \begin{enumerate}
          \item possibly confusing when compared with \code{load} and \code{store} functions of \code{std::atomic}
        \end{enumerate}
    \end{description}

  \item \code{memload(const U*, Flags), memstore(U*, Flags)}
    \begin{description}
      \item[pro]
        \begin{enumerate}
          \item pronounceable
          \item \code{mem} prefix hints at array behind the pointer argument
        \end{enumerate}
      \item[con]
        \begin{enumerate}
          \item abbreviation (pretty common, though)
        \end{enumerate}
    \end{description}

  \item \code{load_from(const U*, Flags), store_to(U*, Flags)}
    \begin{description}
      \item[pro]
        \begin{enumerate}
          \item pronounceable
          \item reads as prose: \code{v.load_from(mem, vector_aligned)}
        \end{enumerate}
      \item[con]
        \begin{enumerate}
        \end{enumerate}
    \end{description}

  \item \code{copy_from(const U*, Flags), copy_to(U*, Flags)}
    \begin{description}
      \item[pro]
        \begin{enumerate}
          \item pronounceable
          \item reads as prose: \code{v.copy_from(mem, vector_aligned)}
          \item clarifies that values are copied
            (user feedback implies that some people expect aliasing)
        \end{enumerate}
      \item[con]
        \begin{enumerate}
          \item avoids term-of-art (load/store)
        \end{enumerate}
    \end{description}

\end{itemize}

\subsubsection{Recommendation}
My preference is to go with \code{load} and \code{store}.
The \code{std::atomic} class is different enough.
I have never received feedback that the copy direction of the load and store functions is confusing.

My second choice is \code{copy_from/to}.
Avoid the embarrassment of using the terms load and store but having to name them differently just because of \code{std::atomic}.
I'm certain that if we choose \code{memload/memstore} or \code{load_from/store_to} the question why we didn't just use \code{load/store} will become a FAQ.

% ft=tex tw=0 et sw=2 spell

\section{Wording}

The following is a rough draft of possible wording that defines a basic set of data-parallel types.

\begin{wgText}
  \wgSection{Data-Parallel Types}{datapar.types}

  \wgSubsection{Header \code{<datapar>} synopsis}{datapar.syn}
  \lstinputlisting[]{synopsis.cpp}

  \pnum
  The header \code{<datapar>} defines two class templates (\datapar, and \mask), several tag types, and a series of related function templates for concurrent manipulation of the values in \datapar and \mask objects.

  \begin{itemdecl}
namespace datapar_abi {
  struct scalar {};
  // implementation-defined tag types, e.g. sse, avx, avx512, neon, ...
  typedef implementation_defined compatible;
  typedef implementation_defined native;
}
  \end{itemdecl}
  \begin{itemdescr}
    \pnum
    The ABI types are tag types to be used as the second template argument to \datapar and \mask.

    \pnum
    The \type{scalar} tag is present in all implementations and forces \datapar and \mask to store a single component (i.e. \datapar{}\type{<T, datapar_abi::scalar>::size()} returns \code 1).

    \pnum
    An implementation may choose to implement data-parallel execution for many different targets.
    \wgNote{There can certainly be more than one tag type per (micro-)architecture, e.g. to support different vector lengths or partial register usage.}
    All tag types an implementation supports shall be present independent of the chosen target.

    \pnum
    The \type{datapar_abi::compatible} tag is defined by the implementation to alias the tag type with the most efficient data parallel execution that ensures the highest compatibility on the target architecture.

    \pnum
    The \type{datapar_abi::native} tag is defined by the implementation to alias the tag type with the most efficient data parallel execution that is supported on the target system.
  \end{itemdescr}

  \begin{itemdecl}
template <class T, size_t N> struct abi_for_width { typedef implementation_defined type; };
  \end{itemdecl}
  \begin{itemdescr}
    \pnum
    The \type{abi_for_width} class template defines the member type \type{type} to one of the tag types in \code{datapar_abi} or not at all, depending on the value of the template parameters.

    \pnum
    \code{datapar<T, abi_for_width_t<T, N>>::size()} must return \code N or result in a substitution failure.
  \end{itemdescr}

  \begin{itemdecl}
template <class T, class Abi = datapar_abi::compatible>
struct datapar_size : public integral_constant<size_t, implementation_defined> {};
  \end{itemdecl}
  \begin{itemdescr}
    \pnum The \type{datapar_size} class template inherits from \type{integral_constant} with a value that equals \datapar{}\code{<T, Abi>::size()}.

    \pnum \code{datapar_size<T, Abi>::value} shall result in a substitution failure if any of the template arguments \type T and \type{Abi} are invalid template arguments to \datapar.
  \end{itemdescr}

  \wgSubsection{Class template \datapar}{datapar}
  \wgSubsubsection{Class template \datapar overview}{datapar.overview}
  \lstinputlisting[]{datapar.cpp}

  \pnum The class template \datapar{}\type{<T, Abi>} is a one-dimensional smart array.
  In contrast to \type{valarray} (26.6), the number of elements in the array is determined at compile time, according to the \type{Abi} template parameter.

  \pnum The first template argument \type T must be an integral or floating-point fundamental type.
  The type \bool is not allowed.

  \pnum The second template argument \type{Abi} must be a tag type from the \code{datapar_abi} namespace.

  \begin{itemdecl}
typedef implementation_defined native_handle_type;
  \end{itemdecl}
  \begin{itemdescr}
    \pnum
    The \type{native_handle_type} member type is an alias for the \code{native_handle()} member function return type.
    It is used to expose an implementation-defined handle for implementation- and target-specific extensions.
  \end{itemdescr}

  \begin{itemdecl}
typedef implementation_defined register_value_type;
  \end{itemdecl}
  \begin{itemdescr}
  \end{itemdescr}

  \wgSubsubsection{\datapar constructors}{datapar.ctor}
  \begin{itemdecl}
datapar() = default;
  \end{itemdecl}
  \begin{itemdescr}
    \pnum
    \effects
    Constructs an object with all elements initialized to \code{T()}.
    \wgNote{This zero-initializes the object.}
  \end{itemdescr}

  \begin{itemdecl}
datapar(value_type);
  \end{itemdecl}
  \begin{itemdescr}
    \pnum
    \effects
    Constructs an object with each element initialized to the value of the argument.
  \end{itemdescr}

  \begin{itemdecl}
template <typename U> datapar(datapar<U, Abi> x);
  \end{itemdecl}
  \begin{itemdescr}
    \pnum\remarks This constructor shall not participate in overload resolution unless
                  \type U and \type T are different integral types and
                  \code{make_signed<U>::type} equals \code{make_signed<T>::type}.
    \pnum\effects Constructs an object of type \datapar.
    \pnum\postcondition The $i$-th element equals \code{static_cast<T>(x[i])} for all elements.
  \end{itemdescr}

  \wgSubsubsection{\datapar load functions}{datapar.load}
  \begin{itemdecl}
static datapar load(const value_type *x);
  \end{itemdecl}
  \begin{itemdescr}
    \pnum \effects Constructs an object with each element $i$ initialized to \code{x[i]} for all elements.
    \pnum \returns The constructed object.
    \pnum \remarks If \datapar{}\code{::size()} is greater than the number of values pointed to by the argument, the behavior is undefined.
  \end{itemdescr}

  \begin{itemdecl}
template <typename Flags> static datapar load(const value_type *x, Flags);
  \end{itemdecl}
  \begin{itemdescr}
    \pnum\effects Constructs an object with each element $i$ initialized to \code{x[i]}.
    \pnum\returns The constructed object.
    \pnum\remarks If \datapar{}\code{::size()} is greater than the number of values pointed to by the first argument, the behavior is undefined.
    \pnum         If the template parameter is of type \type{aligned_tag} and the pointer value is not a multiple of \code{memory_alignment<\type T>}, the behavior is undefined.
  \end{itemdescr}

  \begin{itemdecl}
template <typename U, typename Flags = unaligned_tag> static datapar load(const U *x, Flags = Flags());
  \end{itemdecl}
  \begin{itemdescr}
    \pnum\effects Constructs an object with each element $i$ initialized to \code{static_cast<T>(x[i])}.
    \pnum\returns The constructed object.
    \pnum\remarks If \datapar{}\code{::size()} is greater than the number of values pointed to by the first argument, the behavior is undefined.
    \pnum         If the second template parameter is of type \type{aligned_tag} and the pointer value is not a multiple of \code{memory_alignment<\type U>}, the behavior is undefined.
  \end{itemdescr}

  \wgSubsubsection{\datapar store functions}{datapar.store}
  \begin{itemdecl}
void store(value_type *x);
  \end{itemdecl}
  \begin{itemdescr}
    \pnum\effects Copies each element such that the $i$-th element is stored to \code{x[i]}.
    \pnum\remarks If \datapar{}\code{::size()} is greater than the number of values pointed to by the first argument, the behavior is undefined.
  \end{itemdescr}

  \begin{itemdecl}
template <typename Flags> void store(value_type *x, Flags);
  \end{itemdecl}
  \begin{itemdescr}
    \pnum\effects Copies each element such that the $i$-th element is stored to \code{x[i]}.
    \pnum\remarks If \datapar{}\code{::size()} is greater than the number of values pointed to by the first argument, the behavior is undefined.
    \pnum         If the template parameter is of type \type{aligned_tag} and the pointer value is not a multiple of \code{memory_alignment<\type T>}, the behavior is undefined.
  \end{itemdescr}

  \begin{itemdecl}
template <typename U, typename Flags = unaligned_tag> void store(U *x, Flags = Flags());
  \end{itemdecl}
  \begin{itemdescr}
    \pnum\effects Copies each element such that the $i$-th element is first converted to \type U and then stored to \code{x[i]}.
    \pnum\remarks If \datapar{}\code{::size()} is greater than the number of values pointed to by the first argument, the behavior is undefined.
    \pnum         If the second template parameter is of type \type{aligned_tag} and the pointer value is not a multiple of \code{memory_alignment<\type U>}, the behavior is undefined.
  \end{itemdescr}

  \begin{itemdecl}
void store(value_type *x, mask_type);
  \end{itemdecl}
  \begin{itemdescr}
    \pnum\effects Copies each element where the corresponding element in the second argument is \true such that the $i$-th element is stored to \code{x[i]}.
    \pnum\remarks If the largest $i$ where the second argument is \true is greater than the number of values pointed to by the first argument, the behavior is undefined.
  \end{itemdescr}

  \begin{itemdecl}
template <typename Flags> void store(value_type *x, mask_type, Flags);
  \end{itemdecl}
  \begin{itemdescr}
    \pnum\effects Copies each element where the corresponding element in the second argument is \true such that the $i$-th element is stored to \code{x[i]}.
    \pnum\remarks If the largest $i$ where the second argument is \true is greater than the number of values pointed to by the first argument, the behavior is undefined.
    \pnum         If the template parameter is of type \type{aligned_tag} and the pointer value is not a multiple of \code{memory_alignment<\type T>}, the behavior is undefined.
  \end{itemdescr}

  \begin{itemdecl}
template <typename U, typename Flags = unaligned_tag> void store(U *x, mask_type, Flags = Flags());
  \end{itemdecl}
  \begin{itemdescr}
    \pnum\effects Copies each element where the corresponding element in the second argument is \true such that the $i$-th element is first converted to \type U and then stored to \code{x[i]}.
    \pnum\remarks If the largest $i$ where the second argument is \true is greater than the number of values pointed to by the first argument, the behavior is undefined.
    \pnum         If the template parameter is of type \type{aligned_tag} and the pointer value is not a multiple of \code{memory_alignment<\type U>}, the behavior is undefined.
  \end{itemdescr}

  \wgSubsubsection{\datapar subscript operators}{datapar.subscr}
  \begin{itemdecl}
reference operator[](size_type i);
  \end{itemdecl}
  \begin{itemdescr}
    \pnum\returns An lvalue reference to the $i$-th element.
    \pnum\postconditions Assignment of objects of type \type T modify the $i$-th element without aliasing violations.
    \pnum                Modification of \code{*this} does not invalidate references held to the return value.
    Subsequent reads from such references yield the new value of the $i$-th element.
  \end{itemdescr}

  \begin{itemdecl}
const_reference operator[](size_type) const;
  \end{itemdecl}
  \begin{itemdescr}
    \pnum\returns A \const lvalue reference to the $i$-th element.
    \pnum\postconditions Modification of \code{*this} does not invalidate references held to the return value.
    Subsequent reads from such references yield the new value of the $i$-th element.
  \end{itemdescr}

  \wgSubsubsection{\datapar unary operators}{datapar.unary}
  \begin{itemdecl}
datapar &operator++();
  \end{itemdecl}
  \begin{itemdescr}
    \pnum\effects Increments every element of \code{*this} by one.
    \pnum\returns An lvalue reference to \code{*this} after incrementing.
    \pnum\remarks Overflow semantics follow the same semantics as for \type T.
  \end{itemdescr}

  \begin{itemdecl}
datapar operator++(int);
  \end{itemdecl}
  \begin{itemdescr}
    \pnum\effects Increments every element of \code{*this} by one.
    \pnum\returns A copy of \code{*this} before incrementing.
    \pnum\remarks Overflow semantics follow the same semantics as for \type T.
  \end{itemdescr}

  \begin{itemdecl}
datapar &operator--();
  \end{itemdecl}
  \begin{itemdescr}
    \pnum\effects Decrements every element of \code{*this} by one.
    \pnum\returns An lvalue reference to \code{*this} after decrementing.
    \pnum\remarks Underflow semantics follow the same semantics as for \type T.
  \end{itemdescr}

  \begin{itemdecl}
datapar operator--(int);
  \end{itemdecl}
  \begin{itemdescr}
    \pnum\effects Decrements every element of \code{*this} by one.
    \pnum\returns A copy of \code{*this} before decrementing.
    \pnum\remarks Underflow semantics follow the same semantics as for \type T.
  \end{itemdescr}

  \begin{itemdecl}
mask_type operator!() const;
  \end{itemdecl}
  \begin{itemdescr}
    \pnum\returns A mask object with the $i$-th element set to \code{!operator[](i)} for all elements.
  \end{itemdescr}

  \begin{itemdecl}
datapar operator~() const;
  \end{itemdecl}
  \begin{itemdescr}
    \pnum\requires The first template argument \type T to \datapar must be an integral type.
    \pnum\effects Constructs an object where each bit of \code{*this} is inverted.
    \pnum\returns The new object.
    \pnum\remarks \datapar{}\code{::operator\textasciitilde{}()} shall not participate in overload resolution if \type T is a floating-point type.
  \end{itemdescr}

  \begin{itemdecl}
datapar operator+() const;
  \end{itemdecl}
  \begin{itemdescr}
    \pnum \returns A copy of \code{*this}
  \end{itemdescr}

  \begin{itemdecl}
datapar operator-() const;
  \end{itemdecl}
  \begin{itemdescr}
    \pnum\effects Constructs an object where the $i$-th element is initialized to \code{-operator[](i)} for all elements.
    \pnum\returns The new object.
  \end{itemdescr}

  \wgSubsubsection{\datapar native handles}{datapar.native}
  \begin{itemdecl}
native_handle_type &native_handle();
  \end{itemdecl}
  \begin{itemdescr}
    \pnum\returns An lvalue reference to the implementation-specific object implementing the data-parallel semantics.
  \end{itemdescr}

  \begin{itemdecl}
const native_handle_type &native_handle() const;
  \end{itemdecl}
  \begin{itemdescr}
    \pnum\returns A \const lvalue reference to the implementation-specific object implementing the data-parallel semantics.
  \end{itemdescr}

  \wgSubsection{\datapar non-member operations}{datapar.nonmembers}
  \wgSubsubsection{\datapar binary operators}{datapar.binary}
  \begin{itemdecl}
template <class L, class R> using datapar_return_type = ...;  // exposition only
template <class T, class Abi, class U>
datapar_return_type<datapar<T, Abi>, U> operator+ (datapar<T, Abi>, const U &);
template <class T, class Abi, class U>
datapar_return_type<datapar<T, Abi>, U> operator- (datapar<T, Abi>, const U &);
template <class T, class Abi, class U>
datapar_return_type<datapar<T, Abi>, U> operator* (datapar<T, Abi>, const U &);
template <class T, class Abi, class U>
datapar_return_type<datapar<T, Abi>, U> operator/ (datapar<T, Abi>, const U &);
template <class T, class Abi, class U>
datapar_return_type<datapar<T, Abi>, U> operator% (datapar<T, Abi>, const U &);
template <class T, class Abi, class U>
datapar_return_type<datapar<T, Abi>, U> operator& (datapar<T, Abi>, const U &);
template <class T, class Abi, class U>
datapar_return_type<datapar<T, Abi>, U> operator| (datapar<T, Abi>, const U &);
template <class T, class Abi, class U>
datapar_return_type<datapar<T, Abi>, U> operator^ (datapar<T, Abi>, const U &);
template <class T, class Abi, class U>
datapar_return_type<datapar<T, Abi>, U> operator<<(datapar<T, Abi>, const U &);
template <class T, class Abi, class U>
datapar_return_type<datapar<T, Abi>, U> operator>>(datapar<T, Abi>, const U &);
template <class T, class Abi, class U>
datapar_return_type<datapar<T, Abi>, U> operator+ (const U &, datapar<T, Abi>);
template <class T, class Abi, class U>
datapar_return_type<datapar<T, Abi>, U> operator- (const U &, datapar<T, Abi>);
template <class T, class Abi, class U>
datapar_return_type<datapar<T, Abi>, U> operator* (const U &, datapar<T, Abi>);
template <class T, class Abi, class U>
datapar_return_type<datapar<T, Abi>, U> operator/ (const U &, datapar<T, Abi>);
template <class T, class Abi, class U>
datapar_return_type<datapar<T, Abi>, U> operator% (const U &, datapar<T, Abi>);
template <class T, class Abi, class U>
datapar_return_type<datapar<T, Abi>, U> operator& (const U &, datapar<T, Abi>);
template <class T, class Abi, class U>
datapar_return_type<datapar<T, Abi>, U> operator| (const U &, datapar<T, Abi>);
template <class T, class Abi, class U>
datapar_return_type<datapar<T, Abi>, U> operator^ (const U &, datapar<T, Abi>);
template <class T, class Abi, class U>
datapar_return_type<datapar<T, Abi>, U> operator<<(const U &, datapar<T, Abi>);
template <class T, class Abi, class U>
datapar_return_type<datapar<T, Abi>, U> operator>>(const U &, datapar<T, Abi>);
  \end{itemdecl}
  \begin{itemdescr}
    \pnum\remarks The return type of these operators shall be deduced according to the following rules:
    \begin{itemize}
      \item If \code{is_datapar_v<U> == true}
        then the return type shall be determined from \type T and \type{U::value_type} according to the following paragraph.
      \item Otherwise, if \type T is integral and \type U is \intt
        the return type shall be \datapar{}\type{<T, Abi>}.
      \item Otherwise, if \type T is integral and \type U is \uint
        the return type shall be \datapar{}\code{<make_unsigned_t<T>, Abi>}.
      \item Otherwise, if \type U is a fundamental arithmetic type or \type U is convertible to \intt
        then the return type shall be determined from \type T and \type U according to the following paragraph.
      \item Otherwise, if \type U is implicitly convertible to \datapar{}\type{<V, Abi>}, where \type V is determined according to standard template type deduction,
        then the return type shall be determined from \type T and \type V according to the following paragraph.
      \item Otherwise, if \type U is implicitly convertible to \datapar{}\type{<T, Abi>},
        the return type shall be \datapar{}\type{<T, Abi>}.
      \item Otherwise no return type is defined (SFINAE).
    \end{itemize}

    \pnum\remarks Given the types \type T and \type{Abi} from the class template argument list and a third type \type U determined by the rules of the previous paragraph a return type is deduced according to the following rules:
    \begin{itemize}
      \item If \type U is not a fundamental arithmetic type then the return type shall be \datapar{}\type{<T, Abi>}.
      \item Otherwise, if at least one of the  types \type T and \type U is a floating-point type
        the return type shall be \datapar{}\type{<decltype(T() + U()), Abi>}.
      \item Otherwise, if \code{sizeof(T) < sizeof(U)} the return type shall be \datapar{}\type{<U, Abi>}.
      \item Otherwise, if \code{sizeof(T) > sizeof(U)} the return type shall be \datapar{}\type{<T, Abi>}.
      \item Otherwise, the type \type T or \type U that is farthest back in the list of \textit{standard integer types} (cf. [basic.fundamental]) is used as type \type V and
        the return type shall be \datapar{}\type{<V, Abi>} if both types \type T and \type U are signed, otherwise the return type shall be \datapar{}\type{<make_unsigned_t<V>, Abi>}.
    \end{itemize}

    \pnum\remarks Each of these operators only participates in overload resolution if all of the following hold:
    \begin{itemize}
      \item The indicated operator can be applied to objects of type \type{R::value_type}, with \type R the return type.
      \item \datapar{}\type{<T, Abi>} is implicitly convertible to the return type.
      \item \type U is implicitly convertible to the return type.
    \end{itemize}

    \pnum\remarks The operators with \type{const U \&} as first parameter shall not participate in overload resolution if \code{is_datapar_v<U> == true}.

    \pnum\effects Both arguments are first converted to the return type.
      Each of these operators subsequently performs the indicated operation component-wise on each of the elements of the first argument and the corresponding element of the second argument.
    \pnum\returns An object containing the results of the component-wise operator application.
  \end{itemdescr}

  \wgSubsubsection{\datapar compound assignment}{datapar.cassign}
  \begin{itemdecl}
template <class T, class Abi, class U> datapar<T, Abi> &operator+= (datapar<T, Abi> &, const U &);
template <class T, class Abi, class U> datapar<T, Abi> &operator-= (datapar<T, Abi> &, const U &);
template <class T, class Abi, class U> datapar<T, Abi> &operator*= (datapar<T, Abi> &, const U &);
template <class T, class Abi, class U> datapar<T, Abi> &operator/= (datapar<T, Abi> &, const U &);
template <class T, class Abi, class U> datapar<T, Abi> &operator%= (datapar<T, Abi> &, const U &);
template <class T, class Abi, class U> datapar<T, Abi> &operator&= (datapar<T, Abi> &, const U &);
template <class T, class Abi, class U> datapar<T, Abi> &operator|= (datapar<T, Abi> &, const U &);
template <class T, class Abi, class U> datapar<T, Abi> &operator^= (datapar<T, Abi> &, const U &);
template <class T, class Abi, class U> datapar<T, Abi> &operator<<=(datapar<T, Abi> &, const U &);
template <class T, class Abi, class U> datapar<T, Abi> &operator>>=(datapar<T, Abi> &, const U &);
  \end{itemdecl}
  \begin{itemdescr}
    \pnum\remarks Each of these operators only participates in overload resolution if all of the following hold:
    \begin{itemize}
      \item The indicated operator can be applied to objects of type \type{datapar_return_type<datapar<T, Abi>, U>::value_type}.
      \item \datapar{}\type{<T, Abi>} is implicitly convertible to \type{datapar_return_type<datapar<T, Abi>, U>}.
      \item \type U is implicitly convertible to \type{datapar_return_type<datapar<T, Abi>, U>}.
      \item \type{datapar_return_type<datapar<T, Abi>, U>} is implicitly convertible to \datapar{}\type{<T, Abi>}.
    \end{itemize}
    \pnum\effects Each of these operators performs the indicated operation component-wise on each of the elements of the first argument and the corresponding element of the second argument after conversion to \datapar{}\code{<T, Abi>}.
    \pnum\returns A reference to the first argument.
  \end{itemdescr}

  \wgSubsubsection{\datapar logical operators}{datapar.comparison}

  \wgSubsubsection{\datapar transcendentals}{datapar.transcend}

  \wgSubsection{Class template \mask}{datapar.mask}
  \lstinputlisting[]{mask.cpp}

\end{wgText}

\section{Widening Cast}\label{sec:widen}
The following presents an option for extending the above wording with a cast function that only allows "lossless" conversions of the element type.

Add to the synopsis in \ref{sec:datapar.syn}:
\begin{wgText}
  \begin{lstlisting}[style=Vc]
    template <class V, class T, class A> V datapar_cast(const datapar<T, A>&);
  \end{lstlisting}
\end{wgText}

Append to \ref{sec:datapar.casts}:
\begin{wgText}
  \begin{itemdecl}
    template <class V, class T, class A> V datapar_cast(const datapar<T, A>& x);
  \end{itemdecl}
  \begin{itemdescr}
    \pnum\remarks The function shall not participate in overload resolution unless
    \begin{itemize}
      \item \code{is_datapar_v<V>},
      \item and \code{V::size()} is equal to \code{datapar<T, A>::size()},
      \item and every possible value of type \type T can be represented with type \datapar\code{::}\valuetype.
    \end{itemize}
    \pnum\returns A \datapar object with the $i$-th element initialized to \code{static_cast<V::\valuetype>(x[i])}.
  \end{itemdescr}
\end{wgText}

\section{Split \& Concat}\label{sec:split and concat}
The following presents an option for extending the above wording with two cast functions.
The \code{split} function allows to turn one \datapar or \mask object into two or more \datapar/\mask objects with smaller element counts.
The \code{concat} function allows to combine multiple \datapar or \mask objects into a single \datapar/\mask object consisting of all the input elements.

 Here is a simple example for \code{split} and \code{concat}:
 \smallskip
\begin{lstlisting}[style=Vc]
fixed_size_datapar<float, 12> x = ...;
auto [a, b] = split<8, 4>(x);
// e.g. on x86 you'd get: decltype(a) == datapar<float, avx>
//                   and: decltype(b) == datapar<float, sse>
x = concat(a + 1, b + 2);
\end{lstlisting}

The \type{abi_for_size_t} choice below could also be changed to use the \fixedsize ABI tag unconditionally.
Since a \fixedsize \datapar is implicitly convertible to a non-\fixedsize \datapar type with equal \code{size()}, this may be the more generic solution.
I have a slight preference for \type{abi_for_size_t}, since it more naturally supports the pattern of splitting a \fixedsize object into several native \datapar objects.
That pattern is not fully covered by the second \code{split} variant (e.g. consider the example above).

The same discussion of \type{abi_for_size_t} vs. \fixedsize is valid for the return type of \code{concat}.

Add to the synopsis in \ref{sec:datapar.syn}:
\begin{wgText}
  \begin{lstlisting}[style=Vc]
    template <size_t... Sizes, class T, class A>
    tuple<datapar<T, abi_for_size_t<Sizes>>...> split(const datapar<T, A>&);
    template <size_t... Sizes, class T, class A>
    tuple<mask<T, abi_for_size_t<Sizes>>...> split(const mask<T, A>&);

    template <class V, class T, class A>
    array<V, datapar_size_v<T, A> / V::size()> split(const datapar<T, A>&);
    template <class V, class T, class A>
    array<V, datapar_size_v<T, A> / V::size()> split(const mask<T, A>&);

    template <class T, class... As>
    datapar<T, abi_for_size_t<T, (datapar_size_v<T, As> + ...)>> concat(const datapar<T, As>&...);
    template <class T, class... As>
    mask<T, abi_for_size_t<T, (datapar_size_v<T, As> + ...)>> concat(const mask<T, As>&...);
  \end{lstlisting}
\end{wgText}

Append to \ref{sec:datapar.casts}:
\begin{wgText}

  \begin{itemdecl}
    template <size_t... Sizes, class T, class A>
    tuple<datapar<T, abi_for_size_t<Sizes>>...> split(const datapar<T, A>& x);
    template <size_t... Sizes, class T, class A>
    tuple<mask<T, abi_for_size_t<Sizes>>...> split(const mask<T, A>& x);
  \end{itemdecl}
  \begin{itemdescr}
    \pnum\remarks These functions shall not participate in overload resolution unless the sum of all values in the \code{Sizes} pack is equal to \code{datapar_size_v<T, A>}.
    \pnum\returns A \type{tuple} of \datapar/\mask objects with the $i$-th \datapar/\mask element of the $j$-th \type{tuple} element initialized to the value of the element in \code x with index $i$ + partial sum of the first $j$ values in the \code{Sizes} pack.
  \end{itemdescr}

  \begin{itemdecl}
    template <class V, class T, class A>
    array<V, datapar_size_v<T, A> / V::size()> split(const datapar<T, A>& x);
    template <class V, class T, class A>
    array<V, datapar_size_v<T, A> / V::size()> split(const mask<T, A>& x);
  \end{itemdecl}
  \begin{itemdescr}
    \pnum\remarks These functions shall not participate in overload resolution unless
    \begin{itemize}
      \item \code{is_datapar_v<V>} for the first signature / \code{is_mask_v<V>} for the second signature,
      \item and \code{datapar_size_v<T, A>} is an integral multiple of \code{V::size()}.
    \end{itemize}

    \pnum\returns An \type{array} of \datapar/\mask objects with the $i$-th \datapar/\mask element of the $j$-th \type{array} element initialized to the value of the element in \code x with index $i + j \cdot $\code{V::size()}.
  \end{itemdescr}

  \begin{itemdecl}
    template <class T, class... As>
    datapar<T, abi_for_size_t<T, (datapar_size_v<T, As> + ...)>> concat(const datapar<T, As>&... xs);
    template <class T, class... As>
    mask<T, abi_for_size_t<T, (datapar_size_v<T, As> + ...)>> concat(const mask<T, As>&... xs);
  \end{itemdecl}
  \begin{itemdescr}
    \pnum\returns A \datapar/\mask object initialized with the concatenated values in the \code{xs} pack of \datapar/\mask objects.
    The $i$-th \datapar/\mask element of the $j$-th parameter in the \code{xs} pack is copied to the return value's element with index $i$ + partial sum of the \code{size()} of the first $j$ parameters in the \code{xs} pack.
  \end{itemdescr}

\end{wgText}

% vim: tw=0 sw=2

\section{Discussion}

\subsection{Member Types}
The member types may not seem obvious.
Rationales:
\begin{typelist*}
  \item[value_type]
    In the spirit of the \valuetype member of STL containers, this type denotes the \emph{logical} type of the values in the vector.

  \item[reference]
    Used as the return type of the non-const scalar subscript operator.

  \item[mask_type]
    The natural \mask type for this \simd instantiation.
    This type is used as return type of compares and write-mask on assignments.

  \item[simd_type]
    The natural \simd type for this \mask instantiation.

  \item[size_type]
    Standard member type used for \code{size()} and \code{operator[]}.

  \item[abi_type]
    The \type{Abi} template parameter to \simd.

\end{typelist*}

\subsection{Conversions}
The \simd conversion constructor only allows implicit conversion from \simd template instantiations with the same \type{Abi} type and compatible \valuetype.
Discussion in SG1 showed clear preference for only allowing implicit conversion between integral types that only differ in signedness.
All other conversions could be implemented via an explicit conversion constructor.
The alternative (preferred) is to use \simdcast consistently for all other conversions.

After more discussion on the LEWG reflector, in Issaquah, and between me and Jens, we modified conversions to be even more conservative.
No implicit conversion will ever allow a narrowing conversion of the element type (and signed - unsigned is narrowing in both directions).

\subsection{Broadcast Constructor}
The \simd broadcast constructor is not declared \code{explicit} to ease the use of scalar prvalues in expressions involving data-parallel operations.
The operations where such a conversion should not be implicit consequently need to use SFINAE / concepts to inhibit the conversion.

Experience from Vc shows that the situation is different for \mask, where an implicit conversion from \bool typically hides an error.
(Since there is little use for broadcasting \true or \false.)

\subsection{Aliasing of Subscript Operators}
The subscript operators return an rvalue.
The const overload returns a copy of the element.
The non-const overload returns a smart reference.
This reference behaves mostly like an lvalue reference, but without the requirement to implement assignment via type punning.
At this point the specification of the smart reference is very conservative / restrictive:
The reference type is neither copyable nor movable.
The intention is to avoid users to program like the operator returned an lvalue reference.
The return type is significantly larger than an lvalue reference and harder to optimize when passed around.
The restriction thus forces users to do element modification directly on the \simd / \mask objects.

Guidance from SG1 at JAX 2016:\\
\wgPoll{Should subscript operator return an lvalue reference?}{0  & 6 & 10 & 2 & 1}

\wgPoll{Should subscript operator return a “smart reference”?}{1  & 7 & 10 & 0 & 0}

\subsection{Compound Assignment}
The semantics of compound assignment would allow less strict implicit conversion rules.
Consider \code{simd<int>() *= double()}: the corresponding binary multiplication operator would not compile because the implicit conversion to \simd[<double>] is non-portable.
Compound assignment, on the other hand, implies an implicit conversion back to the type of the expression on the left of the assignment operator.
Thus, it is possible to define compound operators that execute the operation correctly on the promoted type without sacrificing portability.
There are two arguments for not relaxing the rules for compound assignment, though:
\begin{enumerate}
  \item Consistency: The conversion of an expression with compound assignment to a binary operator might make it ill-formed.
  \item The implicit conversion in the \code{int * double} case could be expensive and unintended.
    This is already a problem for builtin types, where many developers multiply \float variables with \double prvalues, though.
\end{enumerate}

\subsection{Return Type of Masked Assignment Operators}
The assignment operators of the type returned by \code{where(mask, simd)} could return one of:
\begin{itemize}
  \item A reference to the \simd object that was modified.
  \item A temporary \simd object that only contains the elements where the \mask is \true.
  \item A reference to the \type{where_expression} object.
  \item Nothing (\ie \void).
\end{itemize}
My first choice was a reference to the modified \simd object.
However, then the statement \code{(where(x < 0, x) *= -1) += 2} may be surprising: it adds \code 2 to all vector entries, independent of the mask.
Likewise, \code{y += (where(x < 0, x) *= -1)} has a possibly confusing interpretation because of the \mask in the middle of the expression.

Consider that write-masked assignment is used as a replacement for \code{if}-statements.
Using \void as return type therefore is a more fitting choice because \code{if}-statements have no return value.
By declaring the return type as \void the above expressions become ill-formed, which seems to be the best solution for guiding users to write maintainable code and express intent clearly.

\subsection{Fundamental SIMD Type or Not?}
\subsubsection{The Issue}
There was substantial discussion on the reflectors and SG1 meetings over the question whether \CC{} should define a fundamental, native SIMD type (let us call it \type{fundamental<T>}) and additionally a generic data-parallel type which supports an arbitrary number of elements (call it \type{arbitrary<T, N>}).
The alternative to defining both types is to only define \type{arbitrary<T, N = default_size<T>>}, since it encompasses the \type{fundamental<T>} type.

With regard to this proposal this second approach would add a third template parameter to \simd and \mask as shown in \lst{simd N}.
\begin{lstlisting}[style=Vc,numbers=left,float,label=lst:simd N,caption={
  Possible declaration of the class template parameters of a \simd class with arbitrary width.
}]
template <class T, size_t N = simd_size_v<T, simd_abi::compatible<T>>,
          class Abi = simd_abi::compatible<T>>
class simd;
\end{lstlisting}

\subsubsection{Standpoints}
The controversy is about how the flexibility of a type with arbitrary \code N is presented to the users.
Is there a (clear) distinction between a “fundamental” type with target-dependent (i.e. fixed) \code N and a higher-level abstraction with arbitrary \code N which can potentially compile to inefficient machine code?
Or should the \CC{} standard only define \type{arbitrary} and set it to a default \code N value that corresponds to the target-dependent \code N.
Thus, the default \code N, of \type{arbitrary} would correspond to \type{fundamental}.

It is interesting to note that \type{arbitrary<T, 1>} is the class variant of \type T.
Consequently, if we say there is no need for a \type{fundamental} type then we could argue for the deprecation of the builtin arithmetic types, in favor of \type{arbitrary<T, 1>}. \wgNote{This is an academic discussion, of course.}

The author has implemented a library where a clear distinction is made between \type{fundamental<T, Abi>} and \type{arbitrary<T, N>}.
The documentation and all teaching material says that the user should program with \type{fundamental}.
The \type{arbitrary} type should be used in special circumstances, or wherever \type{fundamental} works with the \type{arbitrary} type in its interfaces (e.g. for gather \& scatter or the \code{ldexp} \& \code{frexp} functions).

\subsubsection{Issues}
The definition of two separate class templates can alleviate some source compatibility issues resulting from different \code N on different target systems.
Consider the simplest example of a multiplication of an \intt vector with a \float vector:
\smallskip\begin{lstlisting}[style=Vc]
arbitrary<float>() * arbitrary<int>();  // compiles for some targets, fails for others
fundamental<float>() * fundamental<int>();  // never compiles, requires explicit cast
\end{lstlisting}
The \simd[<T>] operators are specified in such a way that source compatibility is ensured.
For a type with user definable \code N, the binary operators should work slightly different with regard to implicit conversions.
Most importantly, \type{arbitrary<T, N>} solves the issue of portable code containing mixed integral and floating-point values.
A user would typically create aliases such as:
\smallskip\begin{lstlisting}[style=Vc]
using floatvec = simd<float>;
using intvec = arbitrary<int, floatvec::size()>;
using doublevec = arbitrary<int, floatvec::size()>;
\end{lstlisting}
Objects of types \type{floatvec}, \type{intvec}, and \type{doublevec} will work together, independent of the target system.

Obviously, these type aliases are basically the same if the \code N parameter of \type{arbitrary} has a default value:
\smallskip\begin{lstlisting}[style=Vc]
using floatvec = arbitrary<float>;
using intvec = arbitrary<int, floatvec::size()>;
using doublevec = arbitrary<int, floatvec::size()>;
\end{lstlisting}
The ability to create these aliases is not the issue.
Seeing the need for using such a pattern is the issue.
Typically, a developer will think no more of it if his code compiles on his machine.
If \code{arbitrary<float>() * arbitrary<int>()} just happens to compile (which is likely), then this is the code that will get checked in to the repository.
Note that with the existence of the \type{fundamental} class template, the \code N parameter of the \type{arbitrary} class would not have a default value and thus force the user to think a second longer about portability.

\subsubsection{Progress}\label{sec:fixedsize progress}
\newcommand\common[2]{\code{\textit{common}(\type{#1}, \type{#2})}}
\newcommand\commonabi[3]{\code{\textit{commonabi}(\type{#1}, \type{#2}, \type{#3})}}

SG1 Guidance at JAX 2016:\\
\wgPoll{Specify simd width using ABI tag, with a special template tag for fixed size.}{3 & 7 & 0 & 0 & 1}
\wgPoll{Specify simd width using <T, N, abi>, where abi is not specified by the user.}{1 & 2 & 5 & 2 & 1}

At the Jacksonville meeting, SG1 decided to continue with the \simd[<T, Abi>] class template, with the addition of a new Abi type that denotes a user-requested number of elements in the vector (\fixedsizeN).
This has the following implications:
\begin{itemize}
  \item There is only one class template with a common interface for \textit{fundamental} and \textit{arbitrary} (\fixedsize) vector types.
  \item There are slight differences in the conversion semantics for \simd types with the \fixedsize Abi type.
    This may look like the \code{vector<bool>} mistake all over again.
    I'll argue below why I believe this is not the case.
  \item The \textit{fundamental} class instances could be implemented in such a way that they do not guarantee ABI compatibility on a given architecture where translation units are compiled with different compiler flags (for micro-architectural differences).
  \item The \fixedsize class instances, on the other hand, could be implemented to be the ABI stable types (if an implementation thinks this is an important feature).
    In implementation terms this means that \textit{fundamental} types are allowed to be passed via registers on function calls.
    \fixedsize types can be implemented in such a way that they are only passed via the stack, and thus an implementation only needs to ensure equal alignment and memory representation across TU borders for a given \type T, \code N.
\end{itemize}

The conversion differences between the \textit{fundamental} and \fixedsize class template instances are the main motivation for having a distinction (cf. discussion above).
The differences are chosen such that, in general, \textit{fundamental} types are more restrictive and do not turn into \fixedsize types on any operation that involves no \fixedsize types.
Operations of \fixedsize types allow easier use of mixed precision code as long as no elements need to be dropped / generated (i.e. the number of elements of all involved \simd objects is equal or a builtin arithmetic type is broadcast).

Examples:

\begin{enumerate}
  \item Mixed \intt--\float operations
\smallskip\begin{lstlisting}[style=Vc,numbers=left]
using floatv = simd<float>;  // native ABI
using float_sized_abi = simd_abi::fixed_size<floatv::size()>;
using intv = simd<int, float_sized_abi>;

auto x = floatv() + intv();/*! \label{lstline:1} */
intv y = floatv() + intv();/*! \label{lstline:2} */
\end{lstlisting}
    Line \ref{lstline:1} is well-formed:
    It states that $N$ (= \type{floatv}\code{::size()}) additions shall be executed concurrently.
    The type of \code{x} is \simd[<\float{}>], because it stores $N$ elements and both types \type{intv} and \type{floatv} are implicitly convertible to \simd[<\float{}>].
    Line \ref{lstline:2} is also well-formed because implicit conversion from \simd[<\type T, \type{Abi}>] to \simd[<\type U, \fixedsizeN{}>] is allowed whenever \code{N == \simd{}<\type T, \type{Abi}>::size()}.

  \item Native \intt vectors
\smallskip\begin{lstlisting}[style=Vc,numbers=left]
using intv = simd<int>;  // native ABI
using int_sized_abi = simd_abi::fixed_size<intv::size()>;
using floatv = simd<float, int_sized_abi>;

auto x = floatv() + intv();/*! \label{lstline:3} */
intv y = floatv() + intv();/*! \label{lstline:4} */
\end{lstlisting}
    Line \ref{lstline:3} is well-formed:
    It states that $N$ (= \type{intv}\code{::size()}) additions shall be executed concurrently.
    The type of \code{x} is \simd[<\floatv{}, \type{int_sized_abi}>] (i.e. \type{floatv}) and never \simd[<\float{}>], because \ldots
    \begin{enumerate}
      \item[\ldots] the \type{Abi} types of \type{intv} and \type{floatv} are not equal.
      \item[\ldots] either \code{\simd[<\float{}>]::size() != N} or \type{intv} is not implicitly convertible to \simd[<\float{}>].
      \item[\ldots] the last rule for \commonabi{V0}{V1}{T} sets the \type{Abi} type to \type{int_sized_abi}.
    \end{enumerate}
    Line \ref{lstline:4} is also well-formed because implicit conversion from \simd[<\type T, \fixedsizeN{}>] to \simd[<\type U, \type{Abi}>] is allowed whenever \code{N == \simd{}<\type U, \type{Abi}>::size()}.
\end{enumerate}


\subsection{Native Handle}\label{sec:native}
The presence of a \code{native_handle} function for accessing an internal data member such as e.g. a vector builtin or SIMD intrinsic type is seen as an important feature for adoption in the target communities.
Without such a handle the user is constrained to work within the (limited) API defined by the standard.
Many SIMD instruction sets have domain-specific instructions that will not easily be usable (if at all) via the standardized interface.
A user considering whether to use \simd or a SIMD extension such as vector builtins or SIMD intrinsics might decide against \simd just for fear of not being able to access all functionality.\footnote{
  Whether that's a reasonable fear is a different discussion.
}

I would be happy to settle on an alternative to exposing an lvalue reference to a data member.
Consider implementation-defined support casting (\code{static_cast}?) between \simd and non-standard SIMD extension types.
My understanding is that there could not be any normative wording about such a feature.
However, I think it could be useful to add a non-normative note about making \code{static_cast}(?) able to convert between such non-standard extensions and \simd.

Guidance from SG1 at Oulu 2016:\\
\wgPoll{Keep \code{native_handle} in the wording?}{0 & 6 & 3 & 3 & 0}

\subsection{Load \& Store Flags}\label{sec:flags}
SIMD loads and stores require at least an alignment option.
This is in contrast to implicit loads and stores present in \CC{}, where alignment is always assumed.
Many SIMD instruction sets allow more options, though:
\begin{itemize}
  \item Streaming, non-temporal loads and stores
  \item Software prefetching
\end{itemize}
In the Vc library I have added these as options in the load store flag parameter of the \code{load} and \code{store} functions.
However, non-temporal loads \& stores and prefetching are also useful for the existing builtin types.
I would like guidance on this question: should the general direction be to stick to \emph{only} alignment options for \simd loads and stores?

The other question is on the default of the load and store flags.
Some argue for setting the default to \code{aligned}, as that's what the user should always aim for and is most efficient.
Others argue for \code{unaligned} since this is safe per default.
The Vc library before version 1.0 used aligned loads and stores per default.
After the guidance from SG1 I changed the default to unaligned loads and stores with the Vc 1.0 release.
Changing the default is probably the worst that could be done, though.\footnote{As I realized too late.}
For Vc 2.0 I will drop the default.

For \simd I prefer no default:
\begin{itemize}
  \item This makes it obvious that the API has the alignment option.
    Users should not just take the default and think no more of it.
  \item If we decide to keep the load constructor, the alignment parameter (without default) nicely disambiguate the load from the broadcast.
  \item The right default would be application/domain/experience specific.
  \item Users can write their own load/store wrapper functions that implement their chosen default.
\end{itemize}

Guidance from SG1 at Oulu 2016:\\
\wgPoll{Should the interface provide a way to specify a number for over-alignment?}{2 & 6 & 5 & 0 & 0}
\wgPoll{Should loads and stores have a default load/store flag?}{0 & 0 & 7 & 4 & 1}
The discussion made it clear that we only want to support alignment flags in the load and store operations.
The other functionality is orthogonal.

\subsection{Unary Minus Return Type}\label{sec:unary minus}
The return type of \simd[<\type T, \type{Abi}>::operator-()] is \simd[<\type T, \type{Abi}>].
This is slightly different to the behavior of the underlying element type \type T, if \type T is an integral type of lower integer conversion rank than \intt.
In this case integral promotion promotes the type to \intt before applying unary minus.
Thus, the expression \code{-T()} is of type \intt for all \type T with lower integer conversion rank than \intt.
This is widening of the element size is likely unintended for SIMD vector types.

Fundamental types with integer conversion rank greater than \intt are not promoted and thus a unary minus expression has unchanged type.
This behavior is copied to element types of lower integer conversion rank for \simd.

There may be one interesting alternative to pursue here:
We can make it ill-formed to apply unary minus to unsigned integral types.
Anyone who wants to have the modulo behavior of a unary minus could still write \code{0u - x}.

\subsection{\code{max_fixed_size}}\label{sec:maxfixedsize}
In Kona, LEWG asked why \code{max_fixed_size} is not dependent on \type T.
After some consideration I am convinced that the correct solution is to make \code{max_fixed_size} a variable template, dependent on \type T.

The reason to restrict the number of elements $N$ in a fixed-size \simd type at all, is to inhibit misuse of the feature.
The intended use of the fixed-size ABI, is to work with a number of elements that is somewhere in the region of the number of elements that can be processed efficiently concurrently in hardware.
Implementations may want to use recursion to implement the fixed-size ABI.
While such an implementation can, in theory, scale to any $N$, experience shows that compiler memory usage and compile times grow significantly for “too large” $N$.
The optimizer also has a hard time to optimize register / stack allocation optimally if $N$ becomes “too large”.
Unsuspecting users might not think of such issues and try to map their complete problem to a single \simd object.
Allowing implementations to restrict $N$ to a value that they can and want to support thus is useful for users and implementations.
The value itself should not be prescribed by the standard as it is really just a QoI issue.

However, why should the user be able to query the maximum $N$ supported by the implementation?
\begin{itemize}
  \item In principle, a user can always determine the number using SFINAE to find the maximum $N$ that he can still instantiate without substitution failure.
    Not providing the number thus provides no “safety” against “bad usage”.
  \item A developer may want to use the value to document assumptions / requirements about the implementation, e.g. with a static assertion.
  \item A developer may want to use the value to make code portable between implementations that use a different \code{max_fixed_size}.
\end{itemize}

Making the \code{max_fixed_size} dependent on \type T makes sense because most hardware can process a different number of elements in parallel depending on \type T.
Thus, if an implementation wants to restrict $N$ to some sensible multiple of the hardware capabilities, the number must be dependent on \type T.

In Kona, LEWG also asked whether there should be a provision in the standard to ensure that a native \simd of 8-bit element type is convertible to a fixed-size \simd of 64-bit element type.
It was already there (\ref{simd.fixedsize.def}: “for every supported \simd[<T, A>] (see \ref{simd.type requirements}), where \type A is an implementation-defined ABI tag, \code N $=$ \simd[<T, A>::size()] must be supported”).
Note that this does not place a lower bound on \code{max_fixed_size}.
The wording allows implementations to support values for fixed-size \simd that are larger than \code{max_fixed_size}.
I.e. $N \leq $ \code{max_fixed_size} works; whether $N > $ \code{max_fixed_size} works is unspecified.

% vim: tw=0 sw=2 spell

\end{document}
% vim: sw=2 sts=2 ai et tw=0
