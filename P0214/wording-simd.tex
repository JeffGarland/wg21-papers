\wgSubsection{Class template \type{simd}}{simd.class}
\wgSubsubsection{Class template \simd overview}{simd.overview}
\lstinputlisting[]{simd.cpp}

\pnum The class template \simd{} is a \dataparalleltype.
The \width of a given \simd specialization is a constant expression, determined by the template parameters.

\newcommand\simdTypeRequirements[1]{
\pnum\label{#1.type requirements}\label{#1.deleted}%
Every specialization of \type{#1} shall be a complete type.
The specialization \type{#1<T, Abi>} is supported if \type T is a \realArithmeticType and
  \begin{itemize}
    \item \type{Abi} is \type{simd_abi::scalar}, or
    \item \type{Abi} is \fixedsizeN, with \code N constrained as defined in \ref{simd.fixedsize.def}.
  \end{itemize}

\noindent If \type{Abi} is an extended ABI tag, it is implementation-defined whether \type{#1<T, Abi>} is supported.
  \wgNote{The intent is for implementations to decide on the basis of the \currentTarget.}

\noindent If \type{#1<T, Abi>} is not supported, the specialization shall have a deleted default constructor, deleted destructor, deleted copy constructor, and deleted copy assignment.
}
\simdTypeRequirements{simd}

\noindent\wgExample{
  Consider an implementation that defines the extended ABI tags \type{simd_x} and \type{gpu_y}.
  When the compiler is invoked to translate to a machine that has support for the \type{simd_x} ABI tag for all arithmetic types other than \type{long double} and no support for the \type{gpu_y} ABI tag, then:
  \begin{itemize}
    \item \simd[<T, simd_abi::gpu_y>] is not supported for any \type T and has a deleted constructor.
    \item \simd[<long double, simd_abi::simd_x>] is not supported and has a deleted constructor.
    \item \simd[<double, simd_abi::simd_x>] is supported.
    \item \simd[<long double, simd_abi::scalar>] is supported.
  \end{itemize}
}

\pnum Default initialization performs no initialization of the elements; value-initialization initializes each element with \code{T()}.
\wgNote{Thus, default initialization leaves the elements in an indeterminate state.}

\begin{itemdecl}
static constexpr size_t size() noexcept;
\end{itemdecl}
\begin{itemdescr}
  \pnum\returns The width of \simd[<T, Abi>].
\end{itemdescr}

\pnum Implementations should enable explicit conversion from and to implementation-defined types.
This adds one or more of the following declarations to class \simd:
\begin{itemdecl}
  explicit operator implementation-defined() const;
  explicit simd(const implementation-defined& init);
\end{itemdecl}

\begin{exampleEnv}\label{impl-conversion-example}
  Consider an implementation that supports the type \type{__vec4f} and the function \code{__vec4f, _vec4f_addsub(__vec4f, __vec4f)} for the \currentTarget.
  A user may require the use of \code{_vec4f_addsub} for maximum performance and thus writes:
  \begin{lstlisting}[style=Vc]
using V = simd<float, simd_abi::Simd128>;
V addsub(V a, V b) {
  return static_cast<V>(_vec4f_addsub(static_cast<__vec4f>(a), static_cast<__vec4f>(b)));
}
  \end{lstlisting}
\end{exampleEnv}

\wgSubsubsection{Element references}{simd.reference}\label{sec:reference type}
\pnum A \code{reference} is an object that refers to an element in a \simd or \mask object.

\pnum Class \code{reference} is for exposition only.
An implementation is permitted to provide equivalent functionality without providing a class with this name.

\lstinputlisting[]{reference.cpp}

\begin{itemdecl}
operator value_type() const noexcept;
\end{itemdecl}
\begin{itemdescr}
  \pnum\returns The value of the element referred to by \code{*this}.
\end{itemdescr}

\begin{itemdecl}
template <class U> reference operator=(U&& x) &&;
\end{itemdecl}
\begin{itemdescr}
  \pnum\effects Replaces the referred to element in \simd or \mask with \code{static_cast<value_type>(std::forward<U>(x))}.

  \pnum\returns A copy of \code{*this}.

  \pnum\remarks This function shall not participate in overload resolution unless \code{declval<value_type \&>() = std::forward<U>(x)} is well-formed.
\end{itemdescr}

\begin{itemdecl}
template <class U> reference operator+=(U&& x) &&;
template <class U> reference operator-=(U&& x) &&;
template <class U> reference operator*=(U&& x) &&;
template <class U> reference operator/=(U&& x) &&;
template <class U> reference operator%=(U&& x) &&;
template <class U> reference operator|=(U&& x) &&;
template <class U> reference operator&=(U&& x) &&;
template <class U> reference operator^=(U&& x) &&;
template <class U> reference operator<<=(U&& x) &&;
template <class U> reference operator>>=(U&& x) &&;
\end{itemdecl}
\begin{itemdescr}
  \pnum\effects Applies the indicated compound operator to the referred to element in \simd or \mask and \code{std::forward<U>(x)}.

  \pnum\returns A copy of \code{*this}.

  \pnum\remarks This function shall not participate in overload resolution unless \code{declval<value_type \&>() @= std::forward<U>(x)} is well-formed.
\end{itemdescr}

\begin{itemdecl}
reference operator++() &&;
reference operator--() &&;
\end{itemdecl}
\begin{itemdescr}
  \pnum\effects Applies the indicated operator to the referred to element in \simd or \mask.

  \pnum\returns A copy of \code{*this}.

  \pnum\remarks This function shall not participate in overload resolution unless the indicated operator can be applied to objects of type \valuetype.% i.e. break it for bool
\end{itemdescr}

\begin{itemdecl}
value_type operator++(int) &&;
value_type operator--(int) &&;
\end{itemdecl}
\begin{itemdescr}
  \pnum\effects Applies the indicated operator to the referred to element in \simd or \mask.

  \pnum\returns A copy of the referred to element before applying the indicated operator.

  \pnum\remarks This function shall not participate in overload resolution unless the indicated operator can be applied to objects of type \valuetype.% i.e. break it for bool
\end{itemdescr}

\begin{itemdecl}
friend void swap(reference&& a, reference&& b) noexcept;
friend void swap(value_type& a, reference&& b) noexcept;
friend void swap(reference&& a, value_type& b) noexcept;
\end{itemdecl}
\begin{itemdescr}
  \pnum\effects Exchanges the values \code{a} and \code{b} refer to.
\end{itemdescr}

% vim: ft=tex tw=0 spell sw=2

%\pnum The member type \referencetype is an unspecified type acting as a reference to an element of a data-parallel type with the following properties:
\label{sec:reference type}
\begin{itemize}
  \item The type has a deleted default constructor, copy constructor, and copy assignment operator.

  \item Assignment, compound assignment, increment, and decrement operators shall not participate in overload resolution unless the \referencetype object is an rvalue and the corresponding operator for \valuetype is usable.

  \item Application of an assignment, compound assignment, increment, or decrement operator on a \referencetype object is applied to the referenced element.

  \item Objects of type \referencetype are implicitly convertible to \valuetype returning the value of the referenced element.

  \item If a binary operator is applied to an object of type \referencetype, the operator is only applied after converting the \referencetype object to \valuetype.

  \item Calls to \code{swap(\referencetype \&\&, \valuetype \&)} and \code{swap(\valuetype \&, \referencetype \&\&)} exchange the values referred to by the \referencetype object and the \valuetype reference.
  Calls to \code{swap(\referencetype \&\&, \referencetype \&\&)} exchange the values referred to by the \referencetype objects.
\end{itemize}

% vim: ft=tex tw=0 spell sw=2


\wgSubsubsection{\simd constructors}{simd.ctor}
\begin{itemdecl}
template <class U> simd(U&&);
\end{itemdecl}
\begin{itemdescr}
  \pnum\effects Constructs an object with each element initialized to the value of the argument after conversion to \valuetype.

  \pnum\throws Any exception thrown while converting the argument to \valuetype.

  \pnum\remarks
  Let \type{From} identify the type \code{remove_cv_t<remove_reference_t<U>>}.
  This constructor shall not participate in overload resolution unless:
  \begin{itemize}
    \item \type{From} is a \realArithmeticType and every possible value of \type{From} can be represented with type \valuetype, or
    \item \type{From} is not an arithmetic type and is implicitly convertible to \valuetype, or
    \item \type{From} is \intt, or
    \item \type{From} is \uint and \valuetype is an unsigned integral type.
  \end{itemize}
\end{itemdescr}

\begin{itemdecl}
template <class U> simd(const simd<U, simd_abi::fixed_size<size()>>& x);
\end{itemdecl}
\begin{itemdescr}
  \pnum\effects Constructs an object where the $i$-th element equals \code{static_cast<T>(x[i])} \foralli.

  \pnum\remarks This constructor shall not participate in overload resolution unless
  \begin{itemize}
    \item \type{abi_type} is \fixedsizescoped{}\code{<size()>}, and
    \item every possible value of \type U can be represented with type \valuetype, and
    \item if both \type U and \valuetype are integral, the integer conversion rank [conv.rank] of \valuetype is greater than the integer conversion rank of \type U.
  \end{itemize}
\end{itemdescr}

\begin{itemdecl}
template <class G> simd(G&& gen);
\end{itemdecl}
\begin{itemdescr}
  \pnum\effects Constructs an object where the $i$-th element is initialized to \code{gen(integral_constant<size_t, i>())}.

  \pnum\remarks This constructor shall not participate in overload resolution unless \code{simd(gen(integral_constant<size_t, i>()))} is well-formed \foralli.

  \pnum%\remarks
  The calls to \code{gen} are unsequenced with respect to each other.
  Vectorization-unsafe standard library functions may not be invoked by \code{gen} ([algorithms.parallel.exec]).
\end{itemdescr}

\begin{itemdecl}
template <class U, class Flags> simd(const U* mem, Flags);
\end{itemdecl}
\begin{itemdescr}
  \flagsRequires{\simd, \type U}
  %\pnum\requires
  \code{[mem, mem + size())} is a valid range.

  \pnum\effects Constructs an object where the $i$-th element is initialized to \code{static_cast<T>(mem[i])} \foralli.

  \pnum\remarks This constructor shall not participate in overload resolution unless
  \begin{itemize}
      \item \code{is_simd_flag_type_v<Flags>} is \true, and
      \item \type U is a \realArithmeticType.
  \end{itemize}
\end{itemdescr}

\wgSubsubsection{\simd copy functions}{simd.copy}
\begin{itemdecl}
template <class U, class Flags> void copy_from(const U* mem, Flags);
\end{itemdecl}
\begin{itemdescr}
  \flagsRequires{\simd, \type U}
  %\pnum\requires
  \code{[mem, mem + size())} is a valid range.

  \pnum\effects Replaces the elements of the \simd object such that the $i$-th element is assigned with \code{static_cast<T>(mem[i])} \foralli.

  \pnum\remarks This function shall not participate in overload resolution unless
  \begin{itemize}
      \item \code{is_simd_flag_type_v<Flags>} is \true, and
      \item \type U is a \realArithmeticType.
  \end{itemize}
\end{itemdescr}

\begin{itemdecl}
template <class U, class Flags> void copy_to(U* mem, Flags) const;
\end{itemdecl}
\begin{itemdescr}
  \flagsRequires{\simd, \type U}
  %\pnum\requires
  \code{[mem, mem + size())} is a valid range.

  \pnum\effects Copies all \simd elements as if \code{mem[i] = static_cast<U>(operator[](i))} \foralli.

  \pnum\remarks This function shall not participate in overload resolution unless
  \begin{itemize}
      \item \code{is_simd_flag_type_v<Flags>} is \true, and
      \item \type U is a \realArithmeticType.
  \end{itemize}
\end{itemdescr}

\wgSubsubsection{\simd subscript operators}{simd.subscr}
\newcommand\simdElementReference[1]{
  \pnum\requires \code{i < size()}.

  \pnum\returns A \code{reference} (see \ref{sec:reference type}) referring to the $i$-th element.
  %with the following effects:
  %\begin{itemize}
    %\item The assignment, compound assignment, increment, and decrement operators of \referencetype execute the indicated operation on the $i$-th element of the #1 object.
%
    %\item Conversion to \valuetype returns a copy of the $i$-th element.
  %\end{itemize}

  \pnum\throws Nothing.
}
\begin{itemdecl}
reference operator[](size_t i);
\end{itemdecl}
\begin{itemdescr}
  \simdElementReference{\simd{}}
\end{itemdescr}

\begin{itemdecl}
value_type operator[](size_t i) const;
\end{itemdecl}
\begin{itemdescr}
  \pnum\requires \code{i < size()}.

  \pnum\returns The value of the $i$-th element.

  \pnum\throws Nothing.
\end{itemdescr}

\wgSubsubsection{\simd unary operators}{simd.unary}
\pnum Effects in this subclause are applied as unary element-wise operations.

\begin{itemdecl}
simd& operator++();
\end{itemdecl}
\begin{itemdescr}
  \pnum\effects Increments every element by one.

  \pnum\returns \code{*this}.

  \pnum\throws Nothing.
\end{itemdescr}

\begin{itemdecl}
simd operator++(int);
\end{itemdecl}
\begin{itemdescr}
  \pnum\effects Increments every element by one.

  \pnum\returns A copy of \code{*this} before incrementing.

  \pnum\throws Nothing.
\end{itemdescr}

\begin{itemdecl}
simd& operator--();
\end{itemdecl}
\begin{itemdescr}
  \pnum\effects Decrements every element by one.

  \pnum\returns \code{*this}.

  \pnum\throws Nothing.
\end{itemdescr}

\begin{itemdecl}
simd operator--(int);
\end{itemdecl}
\begin{itemdescr}
  \pnum\effects Decrements every element by one.

  \pnum\returns A copy of \code{*this} before decrementing.

  \pnum\throws Nothing.
\end{itemdescr}

\begin{itemdecl}
mask_type operator!() const;
\end{itemdecl}
\begin{itemdescr}
  \pnum\returns A \mask object with the $i$-th element set to \code{!operator[](i)} \foralli.

  \pnum\throws Nothing.
\end{itemdescr}

\begin{itemdecl}
simd operator~() const;
\end{itemdecl}
\begin{itemdescr}
  \pnum\returns A \simd object where each bit is the inverse of the corresponding bit in \code{*this}.

  \pnum\throws Nothing.

  \pnum\remarks \simd{}\code{::operator\textasciitilde{}()} \specialsfinae unless \type T is an integral type.
\end{itemdescr}

\begin{itemdecl}
simd operator+() const;
\end{itemdecl}
\begin{itemdescr}
  \pnum\returns \code{*this}.

  \pnum\throws Nothing.
\end{itemdescr}

\begin{itemdecl}
simd operator-() const;
\end{itemdecl}
\begin{itemdescr}
  \pnum\returns A \simd object where the $i$-th element is initialized to \code{-operator[](i)} \foralli.

  \pnum\throws Nothing.
\end{itemdescr}

\wgSubsection{\type{simd} non-member operations}{simd.nonmembers}

\wgSubsubsection{\simd binary operators}{simd.binary}
\begin{itemdecl}
friend simd operator+ (const simd& lhs, const simd& rhs);
friend simd operator- (const simd& lhs, const simd& rhs);
friend simd operator* (const simd& lhs, const simd& rhs);
friend simd operator/ (const simd& lhs, const simd& rhs);
friend simd operator% (const simd& lhs, const simd& rhs);
friend simd operator& (const simd& lhs, const simd& rhs);
friend simd operator| (const simd& lhs, const simd& rhs);
friend simd operator^ (const simd& lhs, const simd& rhs);
friend simd operator<<(const simd& lhs, const simd& rhs);
friend simd operator>>(const simd& lhs, const simd& rhs);
\end{itemdecl}
\begin{itemdescr}
  \pnum\returns A \simd object initialized with the results of the element-wise application of the indicated operator.

  \pnum\throws Nothing.

  \pnum\remarks Each of these operators \specialsfinae unless the indicated operator can be applied to objects of type \type{value_type}.
\end{itemdescr}

\begin{itemdecl}
friend simd operator<<(const simd& v, int n);
friend simd operator>>(const simd& v, int n);
\end{itemdecl}
\begin{itemdescr}
  \pnum\returns A \simd object where the $i$-th element is initialized to the result of applying the indicated operator to \code{v[i]} and \code n \foralli.

  \pnum\throws Nothing.

  \pnum\remarks These operators \specialsfinae unless the indicated operator can be applied to objects of type \type{value_type}.
\end{itemdescr}

\wgSubsubsection{\simd compound assignment}{simd.cassign}
\begin{itemdecl}
friend simd& operator+= (simd& lhs, const simd& rhs);
friend simd& operator-= (simd& lhs, const simd& rhs);
friend simd& operator*= (simd& lhs, const simd& rhs);
friend simd& operator/= (simd& lhs, const simd& rhs);
friend simd& operator%= (simd& lhs, const simd& rhs);
friend simd& operator&= (simd& lhs, const simd& rhs);
friend simd& operator|= (simd& lhs, const simd& rhs);
friend simd& operator^= (simd& lhs, const simd& rhs);
friend simd& operator<<=(simd& lhs, const simd& rhs);
friend simd& operator>>=(simd& lhs, const simd& rhs);
friend simd& operator<<=(simd& lhs, int n);
friend simd& operator>>=(simd& lhs, int n);
\end{itemdecl}
\begin{itemdescr}
  \pnum\effects These operators perform the indicated binary element-wise operation.

  \pnum\returns \code{lhs}.

  \pnum\throws Nothing.

  \pnum\remarks These operators \specialsfinae unless the indicated operator can be applied to objects of type \type{value_type}.
\end{itemdescr}

\wgSubsubsection{\simd compare operators}{simd.comparison}
\begin{itemdecl}
friend mask_type operator==(const simd&, const simd&);
friend mask_type operator!=(const simd&, const simd&);
friend mask_type operator>=(const simd&, const simd&);
friend mask_type operator<=(const simd&, const simd&);
friend mask_type operator> (const simd&, const simd&);
friend mask_type operator< (const simd&, const simd&);
\end{itemdecl}
\begin{itemdescr}
  \pnum\returns A \mask object initialized with the results of the element-wise application of the indicated operator.

  \pnum\throws Nothing.
\end{itemdescr}

\wgSubsubsection{\simd reductions}{simd.reductions}
\pnum In this subclause, \code{BinaryOperation} shall be a binary element-wise operation.

\begin{itemdecl}
template <class T, class Abi, class BinaryOperation = plus<>>
T reduce(const simd<T, Abi>& x, BinaryOperation binary_op = {});
\end{itemdecl}
\begin{itemdescr}
  \pnum\requires \code{binary_op} shall be callable with two arguments of type \type T returning \type T, or callable with two arguments of type \simd[<T, A1>] returning \simd[<T, A1>] for every \type{A1} that is an ABI tag type.

  \pnum\returns \code{\textit{GENERALIZED_SUM}(binary_op, x.data[i], \ldots)} \foralli.

  \pnum\throws Any exception thrown from \code{binary_op}.

  \pnum\wgNote{This overload of \code{reduce} does not require an initial value because \code x is guaranteed to be non-empty.}
\end{itemdescr}

\begin{itemdecl}
template <class M, class V, class BinaryOperation>
typename V::value_type reduce(const const_where_expression<M, V>& x, typename V::value_type identity_element,
                              BinaryOperation binary_op);
\end{itemdecl}
\begin{itemdescr}
  \pnum\requires \code{binary_op} shall be callable with two arguments of type \type T returning \type T, or callable with two arguments of type \simd[<T, A1>] returning \simd[<T, A1>] for every \type{A1} that is an ABI tag type.
  The results of \code{binary_op(identity_element, x)} and \code{binary_op(x, identity_element)} shall be equal to \code{x} for all finite values \code{x} representable by \type{V::value_type}.

  \pnum\returns
  If \code{none_of(x.mask)}, returns \code{identity_element}.
  Otherwise, returns \code{\textit{GENERALIZED_SUM}(binary_op, x.data[i], \ldots)} \forallmaskedi{x.mask}.

  \pnum\throws Any exception thrown from \code{binary_op}.
\end{itemdescr}

\begin{itemdecl}
template <class M, class V>
typename V::value_type reduce(const const_where_expression<M, V>& x, plus<> binary_op = {});
\end{itemdecl}
\begin{itemdescr}
  \pnum\returns
  If \code{none_of(x.mask)}, returns \code 0.
  Otherwise, returns \code{\textit{GENERALIZED_SUM}(binary_op, x.data[i], \ldots)} \forallmaskedi{x.mask}.

  \pnum\throws Nothing.
\end{itemdescr}

\begin{itemdecl}
template <class M, class V>
typename V::value_type reduce(const const_where_expression<M, V>& x, multiplies<> binary_op);
\end{itemdecl}
\begin{itemdescr}
  \pnum\returns
  If \code{none_of(x.mask)}, returns \code 1.
  Otherwise, returns \code{\textit{GENERALIZED_SUM}(binary_op, x.data[i], \ldots)} \forallmaskedi{x.mask}.

  \pnum\throws Nothing.
\end{itemdescr}

\begin{itemdecl}
template <class M, class V>
typename V::value_type reduce(const const_where_expression<M, V>& x, bit_and<> binary_op);
\end{itemdecl}
\begin{itemdescr}
  \pnum\requires \code{is_integral_v<V::value_type>} is \true.

  \pnum\returns
  If \code{none_of(x.mask)}, returns \code{\textasciitilde{}V::value_type()}.
  Otherwise, returns \code{\textit{GENERALIZED_SUM}(binary_op, x.data[i], \ldots)} \forallmaskedi{x.mask}.

  \pnum\throws Nothing.
\end{itemdescr}

\begin{itemdecl}
template <class M, class V>
typename V::value_type reduce(const const_where_expression<M, V>& x, bit_or<> binary_op);
template <class M, class V>
typename V::value_type reduce(const const_where_expression<M, V>& x, bit_xor<> binary_op);
\end{itemdecl}
\begin{itemdescr}
  \pnum\requires \code{is_integral_v<V::value_type>} is \true.

  \pnum\returns
  If \code{none_of(x.mask)}, returns \code 0.
  Otherwise, returns \code{\textit{GENERALIZED_SUM}(binary_op, x.data[i], \ldots)} \forallmaskedi{x.mask}.

  \pnum\throws Nothing.
\end{itemdescr}

\begin{itemdecl}
template <class T, class Abi> T hmin(const simd<T, Abi>& x);
\end{itemdecl}
\begin{itemdescr}
  \pnum\returns The value of an element \code{x[j]} for which \code{x[j] <= x[i]} \foralli.

  \pnum\throws Nothing.
\end{itemdescr}

\begin{itemdecl}
template <class M, class V> typename V::value_type hmin(const const_where_expression<M, V>& x);
\end{itemdecl}
\begin{itemdescr}
  \pnum\returns If \code{none_of(x.mask)}, the return value is \code{numeric_limits<V::value_type>::max()}.
  Otherwise, returns the value of an element \code{x.data[j]} for which \code{x.mask[j] == true} and \code{x.data[j] <= x.data[i]} \forallmaskedi{x.mask}.

  \pnum\throws Nothing.
\end{itemdescr}

\begin{itemdecl}
template <class T, class Abi> T hmax(const simd<T, Abi>& x);
\end{itemdecl}
\begin{itemdescr}
  \pnum\returns The value of an element \code{x[j]} for which \code{x[j] >= x[i]} \foralli.

  \pnum\throws Nothing.
\end{itemdescr}

\begin{itemdecl}
template <class M, class V> typename V::value_type hmax(const const_where_expression<M, V>& x);
\end{itemdecl}
\begin{itemdescr}
  \pnum\returns If \code{none_of(x.mask)}, the return value is \code{numeric_limits<V::value_type>::lowest()}.
  Otherwise, returns the value of an element \code{x.data[j]} for which \code{x.mask[j] == true} and \code{x.data[j] >= x.data[i]} \forallmaskedi{x.mask}.

  \pnum\throws Nothing.
\end{itemdescr}


\wgSubsubsection{\simd casts}{simd.casts}
\begin{itemdecl}
  template <class T, class U, class Abi> @\emph{see below}@ simd_cast(const simd<U, Abi>& x);
\end{itemdecl}
\begin{itemdescr}
  \pnum Let \type{To} identify \type{T::\valuetype} if \code{is_simd_v<T>} is \true, or \type T otherwise.

  \pnum\returns A \simd object with the $i$-th element initialized to \code{static_cast<To>(x[i])} \foralli.

  \pnum\throws Nothing.

  \pnum\remarks The function shall not participate in overload resolution unless
  \begin{itemize}
    \item every possible value of type \type U can be represented with type \type{To}, and
    \item either \code{is_simd_v<T>} is \false, or \code{T::size() == simd<U, Abi>::size()} is \true.
  \end{itemize}

  \pnum
  The return type is
  \begin{itemize}
    \item \type T if \code{is_simd_v<T>} is \true, otherwise
    \item \simd[<T, Abi>] if \type U is \type T, otherwise
    \item \simd[<T, \fixedsizescoped{}<\simd{}<U, Abi>::size()>>].
  \end{itemize}
  %If \code{is_simd_v<T>} is \true, the return type is \type T.
  %Otherwise, if \type U is \type T, the return type is \simd[<T, Abi>].
  %Otherwise, the return type is \simd[<T, \fixedsizescoped{}<\simd{}<U, Abi>::size()>>].

\end{itemdescr}

\begin{itemdecl}
template <class T, class U, class Abi> @\emph{see below}@ static_simd_cast(const simd<U, Abi>& x);
\end{itemdecl}
\begin{itemdescr}
  \pnum Let \type{To} identify \type{T::\valuetype} if \code{is_simd_v<T>} is \true or \type T otherwise.

  \pnum\returns A \simd object with the $i$-th element initialized to \code{static_cast<To>(x[i])} \foralli.

  \pnum\throws Nothing.

  \pnum\remarks The function shall not participate in overload resolution unless either \code{is_simd_v<T>} is \false or \code{T::size() == simd<U, Abi>::size()} is \true.

  \pnum
  If \code{is_simd_v<T>} is \true, the return type is \type T.
  Otherwise, if either \type U is \type T or \type U and \type T are integral types that only differ in signedness, the return type is \simd[<T, Abi>].
  Otherwise, the return type is \simd[<T, \fixedsizescoped{}<\simd{}<U, Abi>::size()>>].
\end{itemdescr}

\begin{itemdecl}
template <class T, class Abi>
fixed_size_simd<T, simd_size_v<T, Abi>> to_fixed_size(const simd<T, Abi>& x) noexcept;
template <class T, class Abi>
fixed_size_simd_mask<T, simd_size_v<T, Abi>> to_fixed_size(const simd_mask<T, Abi>& x) noexcept;
\end{itemdecl}
\begin{itemdescr}
  \pnum\returns A data-parallel object with the $i$-th element initialized to \code{x[i]} \foralli.
\end{itemdescr}

\begin{itemdecl}
template <class T, int N> native_simd<T> to_native(const fixed_size_simd<T, N>& x) noexcept;
template <class T, int N> native_simd_mask<T> to_native(const fixed_size_simd_mask<T, N>& x) noexcept;
\end{itemdecl}
\begin{itemdescr}
  \pnum\returns A data-parallel object with the $i$-th element initialized to \code{x[i]} \foralli.

  \pnum\remarks These functions shall not participate in overload resolution unless \code{simd_size_v<T, simd_abi::native<T>> == N} is \true.
\end{itemdescr}

\begin{itemdecl}
template <class T, int N> simd<T> to_compatible(const fixed_size_simd<T, N>& x) noexcept;
template <class T, int N> simd_mask<T> to_compatible(const fixed_size_simd_mask<T, N>& x) noexcept;
\end{itemdecl}
\begin{itemdescr}
  \pnum\returns A data-parallel object with the $i$-th element initialized to \code{x[i]} \foralli.

  \pnum\remarks These functions shall not participate in overload resolution unless \code{simd_size_v<T, simd_abi::compatible<T>> == N} is \true.
\end{itemdescr}

\begin{itemdecl}
template <size_t... Sizes, class T, class Abi>
tuple<simd<T, simd_abi::deduce_t<T, Sizes>>...> split(const simd<T, Abi>& x);
template <size_t... Sizes, class T, class Abi>
tuple<simd_mask<T, simd_abi::deduce_t<T, Sizes>>...> split(const simd_mask<T, Abi>& x);
\end{itemdecl}
\begin{itemdescr}
  \pnum\returns A \type{tuple} of data-parallel objects with the $i$-th \simd/\mask element of the $j$-th \type{tuple} element initialized to the value of the element in \code x with index $i$ + sum of the first $j$ values in the \code{Sizes} pack.

  \pnum\remarks These functions shall not participate in overload resolution unless the sum of all values in the \code{Sizes} pack is equal to \code{simd_size_v<T, Abi>}.
\end{itemdescr}

\begin{itemdecl}
template <class V, class Abi>
array<V, simd_size_v<typename V::value_type, Abi> / V::size()> split(
    const simd<typename V::value_type, Abi>& x);
template <class V, class Abi>
array<V, simd_size_v<typename V::value_type, Abi> / V::size()> split(
    const simd_mask<typename V::value_type, Abi>& x);
\end{itemdecl}
\begin{itemdescr}
  \pnum\returns An \type{array} of data-parallel objects with the $i$-th \simd/\mask element of the $j$-th \type{array} element initialized to the value of the element in \code x with index \code{$i$ + $j$ * V::size()}.

  \pnum\remarks These functions shall not participate in overload resolution unless
  \begin{itemize}
    \item for the first signature \code{is_simd_v<V>} is \true, for the second signature \code{is_simd_mask_v<V>} is \true, and
    \item \code{simd_size_v<typename V::value_type, Abi>} is an integral multiple of \code{V::size()}.
  \end{itemize}
\end{itemdescr}

\begin{itemdecl}
template <class T, class... Abis>
simd<T, simd_abi::deduce_t<T, (simd_size_v<T, Abis> + ...)>> concat(const simd<T, Abis>&... xs);
template <class T, class... Abis>
simd_mask<T, simd_abi::deduce_t<T, (simd_size_v<T, Abis> + ...)>> concat(const simd_mask<T, Abis>&... xs);
\end{itemdecl}
\begin{itemdescr}
  \pnum\returns A data-parallel object initialized with the concatenated values in the \code{xs} pack of data-parallel objects:
  The $i$-th \simd/\mask element of the $j$-th parameter in the \code{xs} pack is copied to the return value's element with index $i$ + the sum of the width of the first $j$ parameters in the \code{xs} pack.
\end{itemdescr}

\wgSubsubsection{\simd algorithms}{simd.alg}
\begin{itemdecl}
template <class T, class Abi> simd<T, Abi> min(const simd<T, Abi>& a, const simd<T, Abi>& b) noexcept;
\end{itemdecl}
\begin{itemdescr}
  \pnum\returns The result of element-wise application of \code{std::min(a[i], b[i])} \foralli.
  %An object with the $i$-th element initialized to the value of \code{std::min(a[i], b[i])} \foralli.
\end{itemdescr}

\begin{itemdecl}
template <class T, class Abi> simd<T, Abi> max(const simd<T, Abi>& a, const simd<T, Abi>& b) noexcept;
\end{itemdecl}
\begin{itemdescr}
  \pnum\returns The result of element-wise application of \code{std::max(a[i], b[i])} \foralli.
  %\pnum\returns An object with the $i$-th element initialized to the value of \code{std::max(a[i], b[i])} \foralli.
\end{itemdescr}

\begin{itemdecl}
template <class T, class Abi>
pair<simd<T, Abi>, simd<T, Abi>> minmax(const simd<T, Abi>& a, const simd<T, Abi>& b) noexcept;
\end{itemdecl}
\begin{itemdescr}
  \pnum\returns A pair initialized with
  \begin{itemize}
    \item the result of element-wise application of \code{std::min(a[i], b[i])} \foralli in the \code{first} member, and
    \item the result of element-wise application of \code{std::max(a[i], b[i])} \foralli in the \code{second} member.
  \end{itemize}
\end{itemdescr}

\begin{itemdecl}
template <class T, class Abi>
simd<T, Abi> clamp(const simd<T, Abi>& v, const simd<T, Abi>& lo, const simd<T, Abi>& hi);
\end{itemdecl}
\begin{itemdescr}
  \pnum\requires No element in \code{lo} shall be greater than the corresponding element in \code{hi}.

  \pnum\returns The result of element-wise application of \code{std::clamp(v[i], lo[i], hi[i])} \foralli.
\end{itemdescr}

\wgSubsubsection{\simd math library}{simd.math}
%\lstinputlisting[]{math.cpp}

\pnum \label{cl:cmath-spec} For each set of overloaded functions within \code{<cmath>}, there shall be additional overloads sufficient to ensure that if any argument corresponding to a \double parameter has type \simd[<T, Abi>], where \code{is_floating_point_v<T>} is \true, then:
\begin{itemize}
  \item All arguments corresponding to \double parameters shall be convertible to \simd[<T, Abi>].

  \item All arguments corresponding to \type{double*} parameters shall be of type \simd[<T, Abi>*].

  \item All arguments corresponding to parameters of integral type \type U shall be convertible to \code{fixed_size_simd<U, simd_size_v<T, Abi>>}.

  \item All arguments corresponding to \type{U*}, where \type U is integral, shall be of type \code{fixed_size_simd<U, simd_size_v<T, Abi>>*}.

  \item If the corresponding return type is \double, the return type of the additional overload is \simd[<T, Abi>].
    Otherwise, if the corresponding return type is \bool, the return type of the additional overload is \mask[<T, Abi>].
    Otherwise, the return type is \code{fixed_size_simd<R, simd_size_v<T, Abi>>}, with \type R denoting the corresponding return type.
\end{itemize}
It is unspecified whether a call to these overloads with arguments that are all convertible to \simd[<T, Abi>] but are not of type \simd[<T, Abi>] is well-formed.

\pnum Each function overload produced by the above rules applies the indicated \code{<cmath>} function element-wise.
The results per element are not required to be bitwise equal to the application of the function which is overloaded for the element type.

\pnum\label{cl:precond-violation}%
  The behavior is undefined if a domain, pole, or range error occurs when the input argument(s) are applied to the indicated \code{<cmath>} function.
  %\begin{itemize}
    %\item an input argument is outside the domain over which the mathematical function is defined, or if
    %\item the mathematical function has an exact infinite result as the finite input argument(s) are approached in the limit, or if
    %\item the mathematical result of the function cannot be represented in an object of the specified type, due to extreme magnitude, or if
    %\item a domain, pole, or range error occurs when the input argument(s) are applied to the indicated \code{<cmath>} function.
  %\end{itemize}
  %If, any of the elements in the function arguments, when applied to the corresponding function in \code{<cmath>}
  %If the corresponding function in \code{<cmath>} produces a domain, pole, or range error for one of the elements in the function arguments, the behavior is undefined.

%The precondition of the indicated mathematical function shall not be violated.
%If a domain, pole, or range error occurred for one the inputs 
%\begin{itemize}
  %%\item If the corresponding function in \code{<cmath>} can produce a domain error, 
    %one of the input elements to the
%\end{itemize}
% leverage “that no domain, pole, or range error occurs”

%\comment{Neither the C nor the \CC{} standard say anything about expected error/precision.  It seems returning 0 from all functions is a conforming implementation --- just bad QoI.}

\pnum If \code{abs} is called with an argument of type \simd[<X, Abi>] for which \code{is_unsigned<X>::value} is \true, the program is ill-formed.

% vim: tw=0 spell sw=2
