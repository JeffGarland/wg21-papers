\wgSubsection{Class template \type{simd}}{simd}
\wgSubsubsection{Class template \simd overview}{simd.overview}
\lstinputlisting[]{simd.cpp}

\pnum The class template \simd{}\type{<T, Abi>} is a one-dimensional smart array.
The number of elements in the array is a constant expression, according to the \type{Abi} template parameter.

\newcommand\simdTypeRequirements[1]{
\pnum\label{#1.type requirements}\label{#1.deleted}%
The resulting class shall be a complete type with deleted default constructor, deleted destructor, deleted copy constructor, and deleted copy assignment unless all of the following hold:
\begin{itemize}
  \item The first template argument \type T is a cv-unqualified integral or floating-point type except \bool (3.9.1 [basic.fundamental]).
  \item The second template argument \type{Abi} is an ABI tag so that \code{is_abi_tag_v<Abi>} is \true.
  \item The \type{Abi} type is a supported ABI tag.
    It is supported if
    \begin{itemize}
      \item \type{Abi} is \type{simd_abi::scalar}, or
      \item \type{Abi} is \fixedsizeN with \code N $\le 32$ or implementation-defined additional valid values for \code N (see \ref{simd.fixedsize.def}).
    \end{itemize}
    It is implementation-defined whether a given combination of \type T and an implementation-defined ABI tag is supported.
    \wgNote{The intent is for implementations to decide on the basis of the \currentTarget.}
\end{itemize}
}
\simdTypeRequirements{simd}

\wgExample{
  Consider an implementation that defines the implementation-defined ABI tags \type{simd_x} and \type{gpu_y}.
  When the compiler is invoked to translate to a machine that has support for the \type{simd_x} ABI tag for all arithmetic types except \type{long double} and no support for the \type{gpu_y} ABI tag, then:
  \begin{itemize}
    \item \simd[<T, simd_abi::gpu_y>] is not supported for any \type T and results in a type with deleted constructor
    \item \simd[<long double, simd_abi::simd_x>] is not supported and results in a type with deleted constructor
    \item \simd[<double, simd_abi::simd_x>] is supported
    \item \simd[<long double, simd_abi::scalar>] is supported
  \end{itemize}
}

\pnum Default initialization performs no initialization of the elements; value-initialization initializes each element with \code{T()}.
\wgNote{Thus, default initialization leaves the elements in an indeterminate state.}

\pnum The member type \referencetype is an implementation-defined type acting as a reference to an element of type \valuetype with the following properties:
\label{sec:reference type}
\begin{itemize}
  \item The type has a deleted default constructor, copy constructor, and copy assignment operator.

  \item Assignment, compound assignment, increment, and decrement operators shall not participate in overload resolution unless the \referencetype object is an rvalue and the corresponding operator of type \valuetype is usable.

  \item Objects of type \referencetype are implicitly convertible to \valuetype.

  \item If a binary operator is applied to an object of type \referencetype, the operator is only applied after converting the \referencetype object to \valuetype.

  \item Calls to \code{swap(\referencetype \&\&, \valuetype \&)} and \code{swap(\valuetype \&, \referencetype \&\&)} exchange the values referred to by the \referencetype object and the \valuetype reference.
  Calls to \code{swap(\referencetype \&\&, \referencetype \&\&)} exchange the values referred to by the \referencetype objects.
\end{itemize}

\begin{itemdecl}
static constexpr size_type size() noexcept;
\end{itemdecl}
\begin{itemdescr}
  \pnum\returns the number of elements stored in objects of the given \simd[<T, Abi>] type.
\end{itemdescr}

\pnum\begin{noteEnv} Implementations are encouraged to enable \code{static_cast}ing from/to (an) implementation-defined SIMD type(s).
This would add one or more of the following declarations to class \simd:
\begin{itemdecl}
explicit operator implementation-defined() const;
explicit simd(const implementation-defined& init);
\end{itemdecl}
\end{noteEnv}

\wgSubsubsection{\simd constructors}{simd.ctor}
\begin{itemdecl}
template <class U> simd(U&&);
\end{itemdecl}
\begin{itemdescr}
  \pnum\remarks This constructor shall not participate in overload resolution unless either:
  \comment[Q]{Mention forwarding on conversion to \valuetype?}%
  \comment[Q]{\type U is cv- and ref-qualified, is the wording below OK?}
  \begin{itemize}
    \item \type U is an \realArithmeticType and every possible value of type \type U can be represented with type \valuetype,
    \item or \type U is not an arithmetic type and is implicitly convertible to \valuetype,
    \item or \type U is \intt,
    \item or \type U is \uint and \valuetype is an unsigned integral type.
  \end{itemize}
  \pnum\effects Constructs an object with each element initialized to the value of the argument after conversion to \valuetype.

  \pnum\throws Any exception thrown while converting the argument to \valuetype.
\end{itemdescr}

\begin{itemdecl}
template <class U> simd(const simd<U, simd_abi::fixed_size<size()>>& x);
\end{itemdecl}
\begin{itemdescr}
  \pnum\remarks This constructor shall not participate in overload resolution unless
  \begin{itemize}
    \item \type{abi_type} equals \fixedsizescoped{}\code{<size()>},
    \item and every possible value of \type U can be represented with type \valuetype,
    \item and, if both \type U and \valuetype are integral, the integer conversion rank \parencite[(4.15)]{N4618} of \valuetype is greater than the integer conversion rank of \type U.
  \end{itemize}
  \pnum\effects Constructs an object where the $i$-th element equals \code{static_cast<T>(x[i])} \foralli.
\end{itemdescr}

\begin{itemdecl}
template <class G> simd(G&& gen);
\end{itemdecl}
\begin{itemdescr}
  \pnum\remarks This constructor shall not participate in overload resolution unless \code{simd(gen(integral_constant<size_t, 0>()))} is well-formed.%
  \comment{
    To be 100\% correct this needs \code{<size_t, i>} \foralli.
  }
  \pnum\effects Constructs an object where the $i$-th element is initialized to \code{gen(integral_constant<size_t, i>())}.
  \pnum\remarks The order of calls to \code{gen} is unspecified.
\end{itemdescr}

\begin{itemdecl}
template <class U, class Flags> simd(const U *mem, Flags);
\end{itemdecl}
\begin{itemdescr}
  \pnum\remarks This constructor shall not participate in overload resolution unless \type U is an \realArithmeticType.
  \pnum\effects Constructs an object where the $i$-th element is initialized to \code{static_cast<T>(mem[i])} \foralli.
  \pnum\requires \code{size()} is less than or equal to the number of values pointed to by \code{mem}.
  \flagsRemarks{\simd, \type U}
\end{itemdescr}

\wgSubsubsection{\simd load function}{simd.load}
\begin{itemdecl}
template <class U, class Flags> void copy_from(const U *mem, Flags);
\end{itemdecl}
\begin{itemdescr}
  \pnum\remarks This function shall not participate in overload resolution unless \type U is an \realArithmeticType.
  \pnum\effects Replaces the elements of the \simd object such that the $i$-th element is assigned with \code{static_cast<T>(mem[i])} \foralli.
  \pnum\requires \code{size()} is less than or equal to the number of values pointed to by \code{mem}.
  \flagsRemarks{\simd, \type U}
\end{itemdescr}

\wgSubsubsection{\simd store function}{simd.store}
\begin{itemdecl}
template <class U, class Flags> void copy_to(U *mem, Flags);
\end{itemdecl}
\begin{itemdescr}
  \pnum\remarks This function shall not participate in overload resolution unless \type U is an \realArithmeticType.
  \pnum\effects Copies all \simd elements as if \code{mem[i] = static_cast<U>(operator[](i))} \foralli.
  \pnum\requires \code{size()} is less than or equal to the number of values pointed to by \code{mem}.
  \flagsRemarks{\simd, \type U}
\end{itemdescr}

\wgSubsubsection{\simd subscript operators}{simd.subscr}
\newcommand\simdElementReference[1]{
  \pnum\requires The value of \code{i} is less than \code{size()}.

  \pnum\returns A temporary object of type \referencetype (see \ref{sec:reference type}) with the following effects:

  \pnum\effects The assignment, compound assignment, increment, and decrement operators of \referencetype execute the indicated operation on the $i$-th element of the #1 object.

  \pnum\effects Conversion to \valuetype returns a copy of the $i$-th element.

  \pnum\throws Nothing.
}
\begin{itemdecl}
reference operator[](size_type i);
\end{itemdecl}
\begin{itemdescr}
  \simdElementReference{\simd{}}
\end{itemdescr}

\begin{itemdecl}
value_type operator[](size_type i) const;
\end{itemdecl}
\begin{itemdescr}
  \pnum\requires The value of \code{i} is less than \code{size()}.

  \pnum\returns A copy of the $i$-th element.

  \pnum\throws Nothing.
\end{itemdescr}

\wgSubsubsection{\simd unary operators}{simd.unary}
\begin{itemdecl}
simd& operator++();
\end{itemdecl}
\begin{itemdescr}
  \pnum\effects Increments every element of \code{*this} by one.
  \pnum\returns An lvalue reference to \code{*this} after incrementing.
  \pnum\remarks Overflow semantics follow the same semantics as for \type T.

  \pnum\throws Nothing.
\end{itemdescr}

\begin{itemdecl}
simd operator++(int);
\end{itemdecl}
\begin{itemdescr}
  \pnum\effects Increments every element of \code{*this} by one.
  \pnum\returns A copy of \code{*this} before incrementing.
  \pnum\remarks Overflow semantics follow the same semantics as for \type T.

  \pnum\throws Nothing.
\end{itemdescr}

\begin{itemdecl}
simd& operator--();
\end{itemdecl}
\begin{itemdescr}
  \pnum\effects Decrements every element of \code{*this} by one.
  \pnum\returns An lvalue reference to \code{*this} after decrementing.
  \pnum\remarks Underflow semantics follow the same semantics as for \type T.

  \pnum\throws Nothing.
\end{itemdescr}

\begin{itemdecl}
simd operator--(int);
\end{itemdecl}
\begin{itemdescr}
  \pnum\effects Decrements every element of \code{*this} by one.
  \pnum\returns A copy of \code{*this} before decrementing.
  \pnum\remarks Underflow semantics follow the same semantics as for \type T.

  \pnum\throws Nothing.
\end{itemdescr}

\begin{itemdecl}
mask_type operator!() const;
\end{itemdecl}
\begin{itemdescr}
  \pnum\returns A \mask object with the $i$-th element set to \code{!operator[](i)} \foralli.

  \pnum\throws Nothing.
\end{itemdescr}

\begin{itemdecl}
simd operator~() const;
\end{itemdecl}
\begin{itemdescr}
  \pnum\returns A \simd object where each bit is the inverse of the corresponding bit in \code{*this}.
  \pnum\remarks \simd{}\code{::operator\textasciitilde{}()} shall not participate in overload resolution unless \type T is an integral type.

  \pnum\throws Nothing.
\end{itemdescr}

\begin{itemdecl}
simd operator+() const;
\end{itemdecl}
\begin{itemdescr}
  \pnum \returns A copy of \code{*this}

  \pnum\throws Nothing.
\end{itemdescr}

\begin{itemdecl}
simd operator-() const;
\end{itemdecl}
\begin{itemdescr}
  \pnum\returns A \simd object where the $i$-th element is initialized to \code{-operator[](i)} \foralli.

  \pnum\throws Nothing.
\end{itemdescr}

\wgSubsection{\type{simd} non-member operations}{simd.nonmembers}

\wgSubsubsection{\simd binary operators}{simd.binary}
\begin{itemdecl}
friend simd operator+ (const simd&, const simd&);
friend simd operator- (const simd&, const simd&);
friend simd operator* (const simd&, const simd&);
friend simd operator/ (const simd&, const simd&);
friend simd operator% (const simd&, const simd&);
friend simd operator& (const simd&, const simd&);
friend simd operator| (const simd&, const simd&);
friend simd operator^ (const simd&, const simd&);
friend simd operator<<(const simd&, const simd&);
friend simd operator>>(const simd&, const simd&);
\end{itemdecl}
\begin{itemdescr}
  \pnum\remarks Each of these operators shall not participate in overload resolution unless the indicated operator can be applied to objects of type \type{value_type}.

  \pnum\returns A \simd object initialized with the results of the component-wise application of the indicated operator.

  \pnum\throws Nothing.
\end{itemdescr}

\begin{itemdecl}
friend simd operator<<(const simd& v, int n);
friend simd operator>>(const simd& v, int n);
\end{itemdecl}
\begin{itemdescr}
  \pnum\remarks Both operators shall not participate in overload resolution unless \valuetype is an unsigned integral type.

  \pnum\returns A \simd object where the $i$-th element is initialized to the result of applying the indicated operator to \code{v[i]} and \code n \foralli.

  \pnum\throws Nothing.
\end{itemdescr}

\wgSubsubsection{\simd compound assignment}{simd.cassign}
\begin{itemdecl}
friend simd& operator+= (simd&, const simd&);
friend simd& operator-= (simd&, const simd&);
friend simd& operator*= (simd&, const simd&);
friend simd& operator/= (simd&, const simd&);
friend simd& operator%= (simd&, const simd&);
friend simd& operator&= (simd&, const simd&);
friend simd& operator|= (simd&, const simd&);
friend simd& operator^= (simd&, const simd&);
friend simd& operator<<=(simd&, const simd&);
friend simd& operator>>=(simd&, const simd&);
\end{itemdecl}
\begin{itemdescr}
  \pnum\remarks Each of these operators shall not participate in overload resolution unless the indicated operator can be applied to objects of type \type{value_type}.
  \pnum\effects Each of these operators performs the indicated operator component-wise on each of the corresponding elements of the arguments.
  \pnum\returns A reference to the first argument.

  \pnum\throws Nothing.
\end{itemdescr}

\begin{itemdecl}
friend simd& operator<<=(simd& v, int n);
friend simd& operator>>=(simd& v, int n);
\end{itemdecl}
\begin{itemdescr}
  \pnum\remarks Both operators shall not participate in overload resolution unless \valuetype is an unsigned integral type.
  \pnum\effects Performs the indicated shift by \code n operation on the $i$-th element of \code v \foralli.
  \pnum\returns A reference to the first argument.

  \pnum\throws Nothing.
\end{itemdescr}

\wgSubsubsection{\simd compare operators}{simd.comparison}
\begin{itemdecl}
friend mask_type operator==(const simd&, const simd&);
friend mask_type operator!=(const simd&, const simd&);
friend mask_type operator>=(const simd&, const simd&);
friend mask_type operator<=(const simd&, const simd&);
friend mask_type operator> (const simd&, const simd&);
friend mask_type operator< (const simd&, const simd&);
\end{itemdecl}
\begin{itemdescr}
  \pnum\returns A \mask object initialized with the results of the component-wise application of the indicated operator.

  \pnum\throws Nothing.
\end{itemdescr}

\wgSubsubsection{\simd reductions}{simd.reductions}
\begin{itemdecl}
template <class BinaryOperation = std::plus<>, class T, class Abi>
T reduce(const simd<T, Abi>& x, BinaryOperation binary_op = BinaryOperation());
\end{itemdecl}
\begin{itemdescr}
  \pnum\returns \code{\textit{GENERALIZED_SUM}(binary_op, x.data[i], \ldots)} \foralli.
  \pnum\requires \code{binary_op} shall be callable with two arguments of type \type T or two arguments of type \simd[<T, A1>], where \type{A1} may be different to \type{Abi}.
  \pnum\wgNote{This overload of \code{reduce} does not require an initial value because \code x is guaranteed to be non-empty.}
\end{itemdescr}

\begin{itemdecl}
template <class BinaryOperation = std::plus<>, class M, class V>
typename V::value_type reduce(const const_where_expression<M, V>& x,
                              typename V::value_type neutral_element = default_neutral_element,
                              BinaryOperation binary_op = BinaryOperation());
\end{itemdecl}
\begin{itemdescr}
  \pnum\returns
  If \code{none_of(x.mask)}, returns \code{neutral_element}.
  Otherwise, returns \code{\textit{GENERALIZED_SUM}(binary_op, x.data[i], \ldots)} \forallmaskedi{x.mask}.

  \pnum\requires \code{binary_op} shall be callable with two arguments of type \type T or two arguments of type \simd[<T, A1>], where \type{A1} may be different to \type{Abi}.
  \pnum\wgNote{This overload of \code{reduce} requires a neutral value to enable a parallelized implementation:
  A temporary \simd object initialized with \code{neutral_element} is conditionally assigned from \code{x.data} using \code{x.mask}.
  Subsequently, the parallelized reduction (without mask) is applied to the temporary object.}
\end{itemdescr}

\begin{itemdecl}
template <class T, class A> T hmin(const simd<T, A>& x);
\end{itemdecl}
\begin{itemdescr}
  \pnum\returns The value of an element \code{x[j]} for which \code{x[j] <= x[i]} \foralli.

  \pnum\throws Nothing.
\end{itemdescr}

\begin{itemdecl}
template <class M, class V> T hmin(const const_where_expression<M, V>& x);
\end{itemdecl}
\begin{itemdescr}
  \pnum\returns If \code{none_of(x.mask)}, the return value is \code{numeric_limits<V::value_type>::max()}.
  Otherwise, returns the value of an element \code{x.data[j]} for which \code{x.mask[j] == true} and \code{x.data[j] <= x.data[i]} \foralli.

  \pnum\throws Nothing.
\end{itemdescr}

\begin{itemdecl}
template <class T, class A> T hmax(const simd<T, A>& x);
\end{itemdecl}
\begin{itemdescr}
  \pnum\returns The value of an element \code{x[j]} for which \code{x[j] >= x[i]} \foralli.

  \pnum\throws Nothing.
\end{itemdescr}

\begin{itemdecl}
template <class M, class V> T hmax(const const_where_expression<M, V>& x);
\end{itemdecl}
\begin{itemdescr}
  \pnum\returns If \code{none_of(x.mask)}, the return value is \code{numeric_limits<V::value_type>::min()}.
  Otherwise, returns the value of an element \code{x.data[j]} for which \code{x.mask[j] == true} and \code{x.data[j] >= x.data[i]} \foralli.

  \pnum\throws Nothing.
\end{itemdescr}


\wgSubsubsection{\simd casts}{simd.casts}
\begin{itemdecl}
template <class T, class U, class A> /*see below*/ simd_cast(const simd<U, A>& x);
\end{itemdecl}
\begin{itemdescr}
  \pnum\remarks Let \type{To} identify \type{T::\valuetype} if \code{is_simd_v<T>} or \type T otherwise.

  \pnum\remarks The function shall not participate in overload resolution unless
  \begin{itemize}
    \item every possible value of type \type U can be represented with type \type{To}, and
    \item either \code{!is_simd_v<T>} or \code{T::size()} is equal to \code{simd<U, A>::size()}.
  \end{itemize}

  \pnum\remarks If \code{is_simd_v<T>}, the return type is \type T.
  Otherwise, if either \type U and \type T are equal or \type U and \type T are integral types that only differ in signedness, the return type is \simd[<T, A>].
  Otherwise, the return type is \simd[<T, \fixedsizescoped{}<\simd<U, A>::size()>>].

  \pnum\returns A \simd object with the $i$-th element initialized to \code{static_cast<To>(x[i])}.

  \pnum\throws Nothing.
\end{itemdescr}

\begin{itemdecl}
template <class T, class U, class A> /*see below*/ static_simd_cast(const simd<U, A>& x);
\end{itemdecl}
\begin{itemdescr}
  \pnum\remarks Let \type{To} identify \type{T::\valuetype} if \code{is_simd_v<T>} or \type T otherwise.

  \pnum\remarks The function shall not participate in overload resolution unless either \code{!is_simd_v<T>} or \code{T::size()} is equal to \code{simd<U, A>::size()}.

  \pnum\remarks If \code{is_simd_v<T>}, the return type is \type T.
  Otherwise, if either \type U and \type T are equal or \type U and \type T are integral types that only differ in signedness, the return type is \simd[<T, A>].
  Otherwise, the return type is \simd[<T, \fixedsizescoped{}<\simd<U, A>::size()>>].

  \pnum\returns A \simd object with the $i$-th element initialized to \code{static_cast<To>(x[i])}.

  \pnum\throws Nothing.
\end{itemdescr}

\begin{itemdecl}
template <class T, class A>
simd<T, simd_abi::fixed_size<simd_size_v<T, A>>> to_fixed_size(const simd<T, A>& x) noexcept;
template <class T, class A>
mask<T, simd_abi::fixed_size<simd_size_v<T, A>>> to_fixed_size(const simd_mask<T, A>& x) noexcept;
\end{itemdecl}
\begin{itemdescr}
  \pnum\returns An object with the $i$-th element initialized to \code{x[i]}.
\end{itemdescr}

\begin{itemdecl}
template <class T, size_t N>
simd<T, simd_abi::native<T>> to_native(const simd<T, simd_abi::fixed_size<N>>& x) noexcept;
template <class T, size_t N>
mask<T, simd_abi::native<T>> to_native(const simd_mask<T, simd_abi::fixed_size<N>>& x) noexcept;
\end{itemdecl}
\begin{itemdescr}
  \pnum\remarks These functions shall not participate in overload resolution unless \code{simd_size_v<T, simd_abi::native<T>>} is equal to \code N.
  \pnum\returns An object with the $i$-th element initialized to \code{x[i]}.
\end{itemdescr}

\begin{itemdecl}
template <class T, size_t N>
simd<T, simd_abi::compatible<T>> to_compatible(const simd<T, simd_abi::fixed_size<N>>& x) noexcept;
template <class T, size_t N>
mask<T, simd_abi::compatible<T>> to_compatible(const simd_mask<T, simd_abi::fixed_size<N>>& x) noexcept;
\end{itemdecl}
\begin{itemdescr}
  \pnum\remarks These functions shall not participate in overload resolution unless \code{simd_size_v<T, simd_abi::compatible<T>>} is equal to \code N.
  \pnum\returns An object with the $i$-th element initialized to \code{x[i]}.
\end{itemdescr}

\begin{itemdecl}
template <size_t... Sizes, class T, class A>
tuple<simd<T, abi_for_size_t<Sizes>>...> split(const simd<T, A>& x);
template <size_t... Sizes, class T, class A>
tuple<simd_mask<T, abi_for_size_t<Sizes>>...> split(const simd_mask<T, A>& x);
\end{itemdecl}
\begin{itemdescr}
  \pnum\remarks These functions shall not participate in overload resolution unless the sum of all values in the \code{Sizes} pack is equal to \code{simd_size_v<T, A>}.
  \pnum\returns A \type{tuple} of \simd/\mask objects with the $i$-th \simd/\mask element of the $j$-th \type{tuple} element initialized to the value of the element in \code x with index $i$ + partial sum of the first $j$ values in the \code{Sizes} pack.
\end{itemdescr}

\begin{itemdecl}
template <class V, class A>
array<V, simd_size_v<typename V::value_type, A> / V::size()> split(
    const simd<typename V::value_type, A>&);
template <class V, class A>
array<V, simd_size_v<typename V::value_type, A> / V::size()> split(
    const simd_mask<typename V::value_type, A>&);
\end{itemdecl}
\begin{itemdescr}
  \pnum\remarks These functions shall not participate in overload resolution unless
  \begin{itemize}
    \item \code{is_simd_v<V>} for the first signature / \code{is_mask_v<V>} for the second signature,
    \item and \code{simd_size_v<typename V::value_type, A>} is an integral multiple of \code{V::size()}.
  \end{itemize}

  \pnum\returns An \type{array} of \simd/\mask objects with the $i$-th \simd/\mask element of the $j$-th \type{array} element initialized to the value of the element in \code x with index $i + j \cdot $\code{V::size()}.
\end{itemdescr}

\begin{itemdecl}
template <class T, class... As>
simd<T, abi_for_size_t<T, (simd_size_v<T, As> + ...)>> concat(const simd<T, As>&... xs);
template <class T, class... As>
simd_mask<T, abi_for_size_t<T, (simd_size_v<T, As> + ...)>> concat(const simd_mask<T, As>&... xs);
\end{itemdecl}
\begin{itemdescr}
  \pnum\returns A \simd/\mask object initialized with the concatenated values in the \code{xs} pack of \simd/\mask objects.
  The $i$-th \simd/\mask element of the $j$-th parameter in the \code{xs} pack is copied to the return value's element with index $i$ + partial sum of the \code{size()} of the first $j$ parameters in the \code{xs} pack.
\end{itemdescr}

\wgSubsubsection{\simd algorithms}{simd.alg}
\begin{itemdecl}
template <class T, class A>
simd<T, A> min(const simd<T, A>& a, const simd<T, A>& b) noexcept;
\end{itemdecl}
\begin{itemdescr}
  \pnum\returns An object with the $i$-th element initialized with the smaller value of \code{a[i]} and \code{b[i]} \foralli.
\end{itemdescr}

\begin{itemdecl}
template <class T, class A>
simd<T, A> max(const simd<T, A>&, const simd<T, A>&) noexcept;
\end{itemdecl}
\begin{itemdescr}
  \pnum\returns An object with the $i$-th element initialized with the larger value of \code{a[i]} and \code{b[i]} \foralli.
\end{itemdescr}

\begin{itemdecl}
template <class T, class A>
std::pair<simd<T, A>, simd<T, A>> minmax(const simd<T, A>&,
                                               const simd<T, A>&) noexcept;
\end{itemdecl}
\begin{itemdescr}
  \pnum\returns An object with the $i$-th element in the first \type{pair} member initialized with the smaller value of \code{a[i]} and \code{b[i]} \foralli.
  The $i$-th element in the second \type{pair} member is initialized with the larger value of \code{a[i]} and \code{b[i]} \foralli.
\end{itemdescr}

\begin{itemdecl}
template <class T, class A>
simd<T, A> clamp(const simd<T, A>& v, const simd<T, A>& lo,
                    const simd<T, A>& hi);
\end{itemdecl}
\begin{itemdescr}
  \pnum\requires No element in \code{lo} shall be greater than the corresponding element in \code{hi}.
  \pnum\returns An object with the $i$-th element initialized with \code{lo[i]} if \code{v[i]} is smaller than \code{lo[i]}, \code{hi[i]} if \code{v[i]} is larger than \code{hi[i]}, otherwise \code{v[i]} \foralli.
\end{itemdescr}

\wgSubsubsection{\simd math library}{simd.math}
\lstinputlisting[]{math.cpp}

\pnum Each listed function concurrently applies the indicated mathematical function component-wise.
The results per component are not required to be binary equal to the application of the function which is overloaded for the element type.
\comment{Neither the C nor the \CC{} standard say anything about expected error/precision.
It seems returning 0 from all functions is a conforming implementation --- just bad QoI.}
\wgNote{
  If a precondition of the indicated mathematical function is violated, the behavior is undefined.
}

\pnum If \code{abs()} is called with an argument of type \simd[<X, Abi>] for which \code{is_unsigned<X>::value} is \true, the program is ill-formed.

% vim: tw=0 spell sw=2
