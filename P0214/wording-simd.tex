\wgSubsection{Class template \type{simd}}{simd.class}
\wgSubsubsection{Class template \simd overview}{simd.overview}
\lstinputlisting[]{simd.cpp}

\pnum The class template \simd{} is a \dataparalleltype.
The \width of a given \simd instantiation is a constant expression, determined by the template parameters.

\newcommand\simdTypeRequirements[1]{
\pnum\label{#1.type requirements}\label{#1.deleted}%
Each instantiation of \type{#1} shall be a complete type with deleted default constructor, deleted destructor, deleted copy constructor, and deleted copy assignment unless all of the following hold:
\begin{itemize}
  \item The first template argument \type T is a cv-unqualified \realArithmeticType.
  \item The second template argument \type{Abi} is an ABI tag.
  \item The \type{Abi} type is a supported ABI tag.
    It is supported if
    \begin{itemize}
      \item \type{Abi} is \type{simd_abi::scalar}, or
      \item \type{Abi} is \fixedsizeN with \code N $\le 32$ or implementation-defined additional valid values for \code N (see \ref{simd.fixedsize.def}).
    \end{itemize}
    It is implementation-defined whether a given combination of \type T and an implementation-defined ABI tag is supported.
    \wgNote{The intent is for implementations to decide on the basis of the \currentTarget.}
\end{itemize}
}
\simdTypeRequirements{simd}

\wgExample{
  Consider an implementation that defines the implementation-defined ABI tags \type{simd_x} and \type{gpu_y}.
  When the compiler is invoked to translate to a machine that has support for the \type{simd_x} ABI tag for all arithmetic types other than \type{long double} and no support for the \type{gpu_y} ABI tag, then:
  \begin{itemize}
    \item \simd[<T, simd_abi::gpu_y>] is not supported for any \type T and results in a type with deleted constructor
    \item \simd[<long double, simd_abi::simd_x>] is not supported and results in a type with deleted constructor
    \item \simd[<double, simd_abi::simd_x>] is supported
    \item \simd[<long double, simd_abi::scalar>] is supported
  \end{itemize}
}

\pnum Default initialization performs no initialization of the elements; value-initialization initializes each element with \code{T()}.
\wgNote{Thus, default initialization leaves the elements in an indeterminate state.}

\pnum The member type \referencetype is an unspecified type acting as a reference to an element of a data-parallel type with the following properties:
\label{sec:reference type}
\begin{itemize}
  \item The type has a deleted default constructor, copy constructor, and copy assignment operator.

  \item Assignment, compound assignment, increment, and decrement operators shall not participate in overload resolution unless the \referencetype object is an rvalue and the corresponding operator for \valuetype is usable.

  \item Application of an assignment, compound assignment, increment, or decrement operator on a \referencetype object is applied to the referenced element.

  \item Objects of type \referencetype are implicitly convertible to \valuetype returning the value of the referenced element.

  \item If a binary operator is applied to an object of type \referencetype, the operator is only applied after converting the \referencetype object to \valuetype.

  \item Calls to \code{swap(\referencetype \&\&, \valuetype \&)} and \code{swap(\valuetype \&, \referencetype \&\&)} exchange the values referred to by the \referencetype object and the \valuetype reference.
  Calls to \code{swap(\referencetype \&\&, \referencetype \&\&)} exchange the values referred to by the \referencetype objects.
\end{itemize}

\begin{itemdecl}
static constexpr size_t size() noexcept;
\end{itemdecl}
\begin{itemdescr}
  \pnum\returns the number of elements stored in objects of the given \simd[<T, Abi>] type.
\end{itemdescr}

\pnum\begin{noteEnv} Implementations are encouraged to enable \code{static_cast}ing from and to implementation-defined types.
This would add one or more of the following declarations to class \simd:
\begin{itemdecl}
explicit operator implementation-defined() const;
explicit simd(const implementation-defined& init);
\end{itemdecl}
\end{noteEnv}

\wgSubsubsection{\simd constructors}{simd.ctor}
\begin{itemdecl}
template <class U> simd(U&&);
\end{itemdecl}
\begin{itemdescr}
  \pnum\effects Constructs an object with each element initialized to the value of the argument after conversion to \valuetype.

  \pnum\throws Any exception thrown while converting the argument to \valuetype.

  \pnum\remarks This constructor shall not participate in overload resolution unless:
  \comment[Q]{Mention forwarding on conversion to \valuetype?}%
  \comment[Q]{\type U is cv- and ref-qualified, is the wording below OK?}
  \begin{itemize}
    \item \type U is a \realArithmeticType and every possible value of type \type U can be represented with type \valuetype, or
    \item \type U is not an arithmetic type and is implicitly convertible to \valuetype, or
    \item \type U is \intt, or
    \item \type U is \uint and \valuetype is an unsigned integral type.
  \end{itemize}
\end{itemdescr}

\begin{itemdecl}
template <class U> simd(const simd<U, simd_abi::fixed_size<size()>>& x);
\end{itemdecl}
\begin{itemdescr}
  \pnum\effects Constructs an object where the $i$-th element equals \code{static_cast<T>(x[i])} \foralli.

  \pnum\remarks This constructor shall not participate in overload resolution unless
  \begin{itemize}
    \item \type{abi_type} is \fixedsizescoped{}\code{<size()>}, and
    \item every possible value of \type U can be represented with type \valuetype, and
    \item if both \type U and \valuetype are integral, the integer conversion rank [conv.rank] of \valuetype is greater than the integer conversion rank of \type U.
  \end{itemize}
\end{itemdescr}

\begin{itemdecl}
template <class G> simd(G&& gen);
\end{itemdecl}
\begin{itemdescr}
  \pnum\effects Constructs an object where the $i$-th element is initialized to \code{gen(integral_constant<size_t, i>())}.

  \pnum\remarks This constructor shall not participate in overload resolution unless \code{simd(gen(integral_constant<size_t, i>()))} is well-formed \foralli.
  %\pnum\remarks
  The calls to \code{gen} are unsequenced with respect to each other.
  \wgNote{This allows vectorized execution of the \code{gen} calls.}
\end{itemdescr}

\begin{itemdecl}
template <class U, class Flags> simd(const U *mem, Flags);
\end{itemdecl}
\begin{itemdescr}
  \flagsRequires{\simd, \type U}
  %\pnum\requires
  \code{size()} is less than or equal to the number of values pointed to by \code{mem}.

  \pnum\effects Constructs an object where the $i$-th element is initialized to \code{static_cast<T>(mem[i])} \foralli.

  \pnum\remarks This constructor shall not participate in overload resolution unless \type U is a \realArithmeticType and \code{is_simd_flag_type_v<Flags>} is \true.
\end{itemdescr}

\wgSubsubsection{\simd copy functions}{simd.copy}
\begin{itemdecl}
template <class U, class Flags> void copy_from(const U *mem, Flags);
\end{itemdecl}
\begin{itemdescr}
  \flagsRequires{\simd, \type U}
  %\pnum\requires
  \code{size()} is less than or equal to the number of values pointed to by \code{mem}.

  \pnum\effects Replaces the elements of the \simd object such that the $i$-th element is assigned with \code{static_cast<T>(mem[i])} \foralli.

  \pnum\remarks This function shall not participate in overload resolution unless \type U is a \realArithmeticType and \code{is_simd_flag_type_v<Flags>} is \true.
\end{itemdescr}

\begin{itemdecl}
template <class U, class Flags> void copy_to(U *mem, Flags);
\end{itemdecl}
\begin{itemdescr}
  \flagsRequires{\simd, \type U}
  %\pnum\requires
  \code{size()} is less than or equal to the number of values pointed to by \code{mem}.

  \pnum\effects Copies all \simd elements as if \code{mem[i] = static_cast<U>(operator[](i))} \foralli.

  \pnum\remarks This function shall not participate in overload resolution unless \type U is a \realArithmeticType and \code{is_simd_flag_type_v<Flags>} is \true.
\end{itemdescr}

\wgSubsubsection{\simd subscript operators}{simd.subscr}
\newcommand\simdElementReference[1]{
  \pnum\requires \code{i < size()}

  \pnum\returns A temporary object of type \referencetype (see \ref{sec:reference type}) that references the $i$-th element.
  %with the following effects:
  %\begin{itemize}
    %\item The assignment, compound assignment, increment, and decrement operators of \referencetype execute the indicated operation on the $i$-th element of the #1 object.
%
    %\item Conversion to \valuetype returns a copy of the $i$-th element.
  %\end{itemize}

  \pnum\throws Nothing.
}
\begin{itemdecl}
reference operator[](size_t i);
\end{itemdecl}
\begin{itemdescr}
  \simdElementReference{\simd{}}
\end{itemdescr}

\begin{itemdecl}
value_type operator[](size_t i) const;
\end{itemdecl}
\begin{itemdescr}
  \pnum\requires \code{i < size()}

  \pnum\returns A copy of the $i$-th element.

  \pnum\throws Nothing.
\end{itemdescr}

\wgSubsubsection{\simd unary operators}{simd.unary}
\begin{itemdecl}
simd& operator++();
\end{itemdecl}
\begin{itemdescr}
  \pnum\effects Is a unary element-wise operation that applies \code{operator++}.

  \pnum\returns \code{*this}

  \pnum\throws Nothing.
\end{itemdescr}

\begin{itemdecl}
simd operator++(int);
\end{itemdecl}
\begin{itemdescr}
  \pnum\effects Is a unary element-wise operation that applies \code{operator++}.

  \pnum\returns A copy of \code{*this} before incrementing.

  \pnum\throws Nothing.
\end{itemdescr}

\begin{itemdecl}
simd& operator--();
\end{itemdecl}
\begin{itemdescr}
  \pnum\effects Is a unary element-wise operation that applies \code{operator--}.

  \pnum\returns \code{*this}

  \pnum\throws Nothing.
\end{itemdescr}

\begin{itemdecl}
simd operator--(int);
\end{itemdecl}
\begin{itemdescr}
  \pnum\effects Is a unary element-wise operation that applies \code{operator--}.

  \pnum\returns A copy of \code{*this} before decrementing.

  \pnum\throws Nothing.
\end{itemdescr}

\begin{itemdecl}
mask_type operator!() const;
\end{itemdecl}
\begin{itemdescr}
  \pnum\returns A \mask object with the $i$-th element set to \code{!operator[](i)} \foralli.

  \pnum\throws Nothing.
\end{itemdescr}

\begin{itemdecl}
simd operator~() const;
\end{itemdecl}
\begin{itemdescr}
  \pnum\returns A \simd object where each bit is the inverse of the corresponding bit in \code{*this}.

  \pnum\throws Nothing.

  \pnum\remarks \simd{}\code{::operator\textasciitilde{}()} \specialsfinae unless \type T is an integral type.
\end{itemdescr}

\begin{itemdecl}
simd operator+() const;
\end{itemdecl}
\begin{itemdescr}
  \pnum\returns \code{*this}

  \pnum\throws Nothing.
\end{itemdescr}

\begin{itemdecl}
simd operator-() const;
\end{itemdecl}
\begin{itemdescr}
  \pnum\returns A \simd object where the $i$-th element is initialized to \code{-operator[](i)} \foralli.

  \pnum\throws Nothing.
\end{itemdescr}

\wgSubsection{\type{simd} non-member operations}{simd.nonmembers}

\wgSubsubsection{\simd binary operators}{simd.binary}
\begin{itemdecl}
friend simd operator+ (const simd& lhs, const simd& rhs);
friend simd operator- (const simd& lhs, const simd& rhs);
friend simd operator* (const simd& lhs, const simd& rhs);
friend simd operator/ (const simd& lhs, const simd& rhs);
friend simd operator% (const simd& lhs, const simd& rhs);
friend simd operator& (const simd& lhs, const simd& rhs);
friend simd operator| (const simd& lhs, const simd& rhs);
friend simd operator^ (const simd& lhs, const simd& rhs);
friend simd operator<<(const simd& lhs, const simd& rhs);
friend simd operator>>(const simd& lhs, const simd& rhs);
\end{itemdecl}
\begin{itemdescr}
  \pnum\returns A \simd object initialized with the results of the element-wise application of the indicated operator.

  \pnum\throws Nothing.

  \pnum\remarks Each of these operators \specialsfinae unless the indicated operator can be applied to objects of type \type{value_type}.
\end{itemdescr}

\begin{itemdecl}
friend simd operator<<(const simd& v, int n);
friend simd operator>>(const simd& v, int n);
\end{itemdecl}
\begin{itemdescr}
  \pnum\returns A \simd object where the $i$-th element is initialized to the result of applying the indicated operator to \code{v[i]} and \code n \foralli.

  \pnum\throws Nothing.

  \pnum\remarks Both operators \specialsfinae unless the indicated operator can be applied to objects of type \type{value_type}.
\end{itemdescr}

\wgSubsubsection{\simd compound assignment}{simd.cassign}
\begin{itemdecl}
friend simd& operator+= (simd& lhs, const simd& rhs);
friend simd& operator-= (simd& lhs, const simd& rhs);
friend simd& operator*= (simd& lhs, const simd& rhs);
friend simd& operator/= (simd& lhs, const simd& rhs);
friend simd& operator%= (simd& lhs, const simd& rhs);
friend simd& operator&= (simd& lhs, const simd& rhs);
friend simd& operator|= (simd& lhs, const simd& rhs);
friend simd& operator^= (simd& lhs, const simd& rhs);
friend simd& operator<<=(simd& lhs, const simd& rhs);
friend simd& operator>>=(simd& lhs, const simd& rhs);
\end{itemdecl}
\begin{itemdescr}
  \pnum\effects Each of these operators performs the indicated operator element-wise on each of the corresponding elements of the arguments.

  \pnum\returns \code{lhs}.

  \pnum\throws Nothing.

  \pnum\remarks Each of these operators \specialsfinae unless the indicated operator can be applied to objects of type \type{value_type}.
\end{itemdescr}

\begin{itemdecl}
friend simd& operator<<=(simd& v, int n);
friend simd& operator>>=(simd& v, int n);
\end{itemdecl}
\begin{itemdescr}
  \pnum\effects Performs the indicated shift by \code n operation on the $i$-th element of \code v \foralli.

  \pnum\returns \code v.

  \pnum\throws Nothing.

  \pnum\remarks Both operators \specialsfinae unless the indicated operator can be applied to objects of type \valuetype.
\end{itemdescr}

\wgSubsubsection{\simd compare operators}{simd.comparison}
\begin{itemdecl}
friend mask_type operator==(const simd&, const simd&);
friend mask_type operator!=(const simd&, const simd&);
friend mask_type operator>=(const simd&, const simd&);
friend mask_type operator<=(const simd&, const simd&);
friend mask_type operator> (const simd&, const simd&);
friend mask_type operator< (const simd&, const simd&);
\end{itemdecl}
\begin{itemdescr}
  \pnum\returns A \mask object initialized with the results of the element-wise application of the indicated operator.

  \pnum\throws Nothing.
\end{itemdescr}

\wgSubsubsection{\simd reductions}{simd.reductions}
\begin{itemdecl}
template <class T, class Abi, class BinaryOperation = std::plus<>>
T reduce(const simd<T, Abi>& x, BinaryOperation binary_op = BinaryOperation());
\end{itemdecl}
\begin{itemdescr}
  \pnum\requires \code{binary_op} shall be callable with two arguments of type \type T returning \type T, or callable with two arguments of type \simd[<T, A1>] returning \simd[<T, A1>] for every \type{A1} that is an ABI tag type.

  \pnum\returns \code{\textit{GENERALIZED_SUM}(binary_op, x.data[i], \ldots)} \foralli.

  \pnum\wgNote{This overload of \code{reduce} does not require an initial value because \code x is guaranteed to be non-empty.}
\end{itemdescr}

\begin{itemdecl}
template <class M, class V, class BinaryOperation>
typename V::value_type reduce(const const_where_expression<M, V>& x, typename V::value_type neutral_element,
                              BinaryOperation binary_op);
\end{itemdecl}
\begin{itemdescr}
  \pnum\requires \code{binary_op} shall be callable with two arguments of type \type T returning \type T, or callable with two arguments of type \simd[<T, A1>] returning \simd[<T, A1>] for every \type{A1} that is an ABI tag type.

  \pnum\returns
  If \code{none_of(x.mask)}, returns \code{neutral_element}.
  Otherwise, returns \code{\textit{GENERALIZED_SUM}(binary_op, x.data[i], \ldots)} \forallmaskedi{x.mask}.

  \pnum\wgNote{This overload of \code{reduce} requires a neutral value to enable a parallelized implementation:
  A temporary \simd object initialized with \code{neutral_element} is conditionally assigned from \code{x.data} using \code{x.mask}.
  Subsequently, the parallelized reduction is applied to the temporary object.}
\end{itemdescr}

\begin{itemdecl}
template <class M, class V>
typename V::value_type reduce(const const_where_expression<M, V>& x, plus<> binary_op = plus<>());
\end{itemdecl}
\begin{itemdescr}
  \pnum\returns
  If \code{none_of(x.mask)}, returns 0.
  Otherwise, returns \code{\textit{GENERALIZED_SUM}(binary_op, x.data[i], \ldots)} \forallmaskedi{x.mask}.

  \pnum\throws Nothing.
\end{itemdescr}

\begin{itemdecl}
template <class M, class V>
typename V::value_type reduce(const const_where_expression<M, V>& x, multiplies<> binary_op);
\end{itemdecl}
\begin{itemdescr}
  \pnum\returns
  If \code{none_of(x.mask)}, returns 1.
  Otherwise, returns \code{\textit{GENERALIZED_SUM}(binary_op, x.data[i], \ldots)} \forallmaskedi{x.mask}.

  \pnum\throws Nothing.
\end{itemdescr}

\begin{itemdecl}
template <class M, class V>
typename V::value_type reduce(const const_where_expression<M, V>& x, bit_and<> binary_op);
\end{itemdecl}
\begin{itemdescr}
  \pnum\requires \code{is_integral_v<V::value_type>} is \true.

  \pnum\returns
  If \code{none_of(x.mask)}, returns ~(V::value_type()).
  Otherwise, returns \code{\textit{GENERALIZED_SUM}(binary_op, x.data[i], \ldots)} \forallmaskedi{x.mask}.

  \pnum\throws Nothing.
\end{itemdescr}

\begin{itemdecl}
template <class M, class V>
typename V::value_type reduce(const const_where_expression<M, V>& x, bit_or<> binary_op);
template <class M, class V>
typename V::value_type reduce(const const_where_expression<M, V>& x, bit_xor<> binary_op);
\end{itemdecl}
\begin{itemdescr}
  \pnum\requires \code{is_integral_v<V::value_type>} is \true.

  \pnum\returns
  If \code{none_of(x.mask)}, returns 0.
  Otherwise, returns \code{\textit{GENERALIZED_SUM}(binary_op, x.data[i], \ldots)} \forallmaskedi{x.mask}.

  \pnum\throws Nothing.
\end{itemdescr}

\begin{itemdecl}
template <class T, class Abi> T hmin(const simd<T, Abi>& x);
\end{itemdecl}
\begin{itemdescr}
  \pnum\returns The value of an element \code{x[j]} for which \code{x[j] <= x[i]} \foralli.

  \pnum\throws Nothing.
\end{itemdescr}

\begin{itemdecl}
template <class M, class V> typename V::value_type hmin(const const_where_expression<M, V>& x);
\end{itemdecl}
\begin{itemdescr}
  \pnum\returns If \code{none_of(x.mask)}, the return value is \code{numeric_limits<V::value_type>::max()}.
  Otherwise, returns the value of an element \code{x.data[j]} for which \code{x.mask[j] == true} and \code{x.data[j] <= x.data[i]} \forallmaskedi{x.mask}.

  \pnum\throws Nothing.
\end{itemdescr}

\begin{itemdecl}
template <class T, class Abi> T hmax(const simd<T, Abi>& x);
\end{itemdecl}
\begin{itemdescr}
  \pnum\returns The value of an element \code{x[j]} for which \code{x[j] >= x[i]} \foralli.

  \pnum\throws Nothing.
\end{itemdescr}

\begin{itemdecl}
template <class M, class V> typename V::value_type hmax(const const_where_expression<M, V>& x);
\end{itemdecl}
\begin{itemdescr}
  \pnum\returns If \code{none_of(x.mask)}, the return value is \code{numeric_limits<V::value_type>::min()}.
  Otherwise, returns the value of an element \code{x.data[j]} for which \code{x.mask[j] == true} and \code{x.data[j] >= x.data[i]} \forallmaskedi{x.mask}.

  \pnum\throws Nothing.
\end{itemdescr}


\wgSubsubsection{\simd casts}{simd.casts}
\begin{itemdecl}
  template <class T, class U, class Abi> @\emph{see below}@ simd_cast(const simd<U, Abi>& x);
\end{itemdecl}
\begin{itemdescr}
  \pnum Let \type{To} identify \type{T::\valuetype} if \code{is_simd_v<T>} is \true, or \type T otherwise.

  \pnum\returns A \simd object with the $i$-th element initialized to \code{static_cast<To>(x[i])}.

  \pnum\throws Nothing.

  \pnum\remarks The function shall not participate in overload resolution unless
  \begin{itemize}
    \item every possible value of type \type U can be represented with type \type{To}, and
    \item either \code{is_simd_v<T>} is \false, or \code{T::size() == simd<U, Abi>::size()} is \true.
  \end{itemize}
  %\pnum\remarks
  If \code{is_simd_v<T>} is \true, the return type is \type T.
  Otherwise, if \type U is \type T, the return type is \simd[<T, Abi>].
  Otherwise, the return type is \simd[<T, \fixedsizescoped{}<\simd{}<U, Abi>::size()>>].

\end{itemdescr}

\begin{itemdecl}
template <class T, class U, class Abi> @\emph{see below}@ static_simd_cast(const simd<U, Abi>& x);
\end{itemdecl}
\begin{itemdescr}
  \pnum Let \type{To} identify \type{T::\valuetype} if \code{is_simd_v<T>} or \type T otherwise.

  \pnum\returns A \simd object with the $i$-th element initialized to \code{static_cast<To>(x[i])}.

  \pnum\throws Nothing.

  \pnum\remarks The function shall not participate in overload resolution unless either \code{is_simd_v<T>} is \false or \code{T::size() == simd<U, Abi>::size()} is \true.
  %\pnum\remarks
  If \code{is_simd_v<T>} is \true, the return type is \type T.
  Otherwise, if either \type U is \type T or \type U and \type T are integral types that only differ in signedness, the return type is \simd[<T, Abi>].
  Otherwise, the return type is \simd[<T, \fixedsizescoped{}<\simd{}<U, Abi>::size()>>].
\end{itemdescr}

\begin{itemdecl}
template <class T, class Abi>
fixed_size_simd<T, simd_size_v<T, Abi>> to_fixed_size(const simd<T, Abi>& x) noexcept;
template <class T, class Abi>
fixed_size_simd_mask<T, simd_size_v<T, Abi>> to_fixed_size(const simd_mask<T, Abi>& x) noexcept;
\end{itemdecl}
\begin{itemdescr}
  \pnum\returns An object of the return type with the $i$-th element initialized to \code{x[i]}.
\end{itemdescr}

\begin{itemdecl}
template <class T, size_t N> native_simd<T> to_native(const fixed_size_simd<T, N>& x) noexcept;
template <class T, size_t N> native_simd_mask<T> to_native(const fixed_size_simd_mask<T, N>& x) noexcept;
\end{itemdecl}
\begin{itemdescr}
  \pnum\returns An object of the return type with the $i$-th element initialized to \code{x[i]}.

  \pnum\remarks These functions shall not participate in overload resolution unless \code{simd_size_v<T, simd_abi::native<T>> == N} is \true.
\end{itemdescr}

\begin{itemdecl}
template <class T, size_t N> simd<T> to_compatible(const fixed_size_simd<T, N>& x) noexcept;
template <class T, size_t N> simd_mask<T> to_compatible(const fixed_size_simd_mask<T, N>& x) noexcept;
\end{itemdecl}
\begin{itemdescr}
  \pnum\returns An object of the return type with the $i$-th element initialized to \code{x[i]}.

  \pnum\remarks These functions shall not participate in overload resolution unless \code{simd_size_v<T, simd_abi::compatible<T>> == N} is \true.
\end{itemdescr}

\begin{itemdecl}
template <size_t... Sizes, class T, class Abi>
tuple<simd<T, abi_for_size_t<T, Sizes>>...> split(const simd<T, Abi>& x);
template <size_t... Sizes, class T, class Abi>
tuple<simd_mask<T, abi_for_size_t<T, Sizes>>...> split(const simd_mask<T, Abi>& x);
\end{itemdecl}
\begin{itemdescr}
  \pnum\returns A \type{tuple} of data-parallel objects with the $i$-th \simd/\mask element of the $j$-th \type{tuple} element initialized to the value of the element in \code x with index $i$ + partial sum of the first $j$ values in the \code{Sizes} pack.

  \pnum\remarks These functions shall not participate in overload resolution unless the sum of all values in the \code{Sizes} pack is equal to \code{simd_size_v<T, Abi>}.
\end{itemdescr}

\begin{itemdecl}
template <class V, class Abi>
array<V, simd_size_v<typename V::value_type, Abi> / V::size()> split(
    const simd<typename V::value_type, Abi>&);
template <class V, class Abi>
array<V, simd_size_v<typename V::value_type, Abi> / V::size()> split(
    const simd_mask<typename V::value_type, Abi>&);
\end{itemdecl}
\begin{itemdescr}
  \pnum\returns An \type{array} of data-parallel objects with the $i$-th \simd/\mask element of the $j$-th \type{array} element initialized to the value of the element in \code x with index $i + j \cdot $\code{V::size()}.

  \pnum\remarks These functions shall not participate in overload resolution unless
  \begin{itemize}
    \item \code{is_simd_v<V>} is \true for the first signature / \code{is_mask_v<V>} is \true for the second signature, and
    \item \code{simd_size_v<typename V::value_type, Abi>} is an integral multiple of \code{V::size()}.
  \end{itemize}
\end{itemdescr}

\begin{itemdecl}
template <class T, class... Abis>
simd<T, abi_for_size_t<T, (simd_size_v<T, Abis> + ...)>> concat(const simd<T, Abis>&... xs);
template <class T, class... Abis>
simd_mask<T, abi_for_size_t<T, (simd_size_v<T, Abis> + ...)>> concat(const simd_mask<T, Abis>&... xs);
\end{itemdecl}
\begin{itemdescr}
  \pnum\returns A data-parallel object initialized with the concatenated values in the \code{xs} pack of data-parallel objects:
  The $i$-th \simd/\mask element of the $j$-th parameter in the \code{xs} pack is copied to the return value's element with index $i$ + partial sum of the \code{size()} of the first $j$ parameters in the \code{xs} pack.
\end{itemdescr}

\wgSubsubsection{\simd algorithms}{simd.alg}
\begin{itemdecl}
template <class T, class Abi> simd<T, Abi> min(const simd<T, Abi>& a, const simd<T, Abi>& b) noexcept;
\end{itemdecl}
\begin{itemdescr}
  \pnum\returns The result of binary element-wise application of \code{std::min(a[i], b[i])} \foralli.
  %An object with the $i$-th element initialized to the value of \code{std::min(a[i], b[i])} \foralli.
\end{itemdescr}

\begin{itemdecl}
template <class T, class Abi> simd<T, Abi> max(const simd<T, Abi>&, const simd<T, Abi>&) noexcept;
\end{itemdecl}
\begin{itemdescr}
  \pnum\returns The result of binary element-wise application of \code{std::max(a[i], b[i])} \foralli.
  %\pnum\returns An object with the $i$-th element initialized to the value of \code{std::max(a[i], b[i])} \foralli.
\end{itemdescr}

\begin{itemdecl}
template <class T, class Abi>
std::pair<simd<T, Abi>, simd<T, Abi>> minmax(const simd<T, Abi>&, const simd<T, Abi>&) noexcept;
\end{itemdecl}
\begin{itemdescr}
  \pnum\returns A pair initialized with
  \begin{itemize}
    \item the result of binary element-wise application of \code{std::min(a[i], b[i])} \foralli in the \code{first} member, and
    \item the result of binary element-wise application of \code{std::max(a[i], b[i])} \foralli in the \code{second} member.
  \end{itemize}
\end{itemdescr}

\begin{itemdecl}
template <class T, class Abi>
simd<T, Abi> clamp(const simd<T, Abi>& v, const simd<T, Abi>& lo, const simd<T, Abi>& hi);
\end{itemdecl}
\begin{itemdescr}
  \pnum\requires No element in \code{lo} shall be greater than the corresponding element in \code{hi}.

  \pnum\returns The result of element-wise\comment{do we really need a definition of \emph{ternary element-wise}?} application of \code{std::clamp(a[i], lo[i], hi[i])} \foralli.
\end{itemdescr}

\wgSubsubsection{\simd math library}{simd.math}
\lstinputlisting[]{math.cpp}

\pnum Each listed function concurrently applies the indicated mathematical function element-wise.
The results per element are not required to be bitwise equal to the application of the function which is overloaded for the element type.
\comment{Neither the C nor the \CC{} standard say anything about expected error/precision.
It seems returning 0 from all functions is a conforming implementation --- just bad QoI.}
\wgNote{
  If a precondition of the indicated mathematical function is violated, the behavior is undefined.%
  \comment{Make this normative? And prefer implementation-defined?}
}

\pnum If \code{abs} is called with an argument of type \simd[<X, Abi>] for which \code{is_unsigned<X>::value} is \true, the program is ill-formed.

% vim: tw=0 spell sw=2
