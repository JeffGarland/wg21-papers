\section{Naming}\label{sec:Naming}

The name \datapar was chosen in SG1 after a short discussion, brainstorm session, and straw poll.
Names can still be changed by LEWG.
The following will present naming ideas and a bit of discussion of pros and cons and make
recommendations.

\subsection{datapar}

The class in question is an array of target-specific size, with elements of type T, and data parallel operation semantics.
The actual memory layout and storage size is unspecified.
The number of elements is influenced via the second template parameter.
If the second template parameter is \code{fixed_size<N>}, an exact number of N elements is used.
Operations on objects of the type execute the operation component-wise and concurrently.
This allows the user to communicate data parallelism inherent in the problem at hand.
An implementation might translate the data parallelism into SIMD instructions, GPU parallelism, serial execution, synchronized multi-core execution, or any mix thereof.
The implementation is expected to provide guarantees about the resulting code gen depending on compiler flags and the given ABI parameter (second template parameter), e.g. “\code{datapar<int, datapar_abi::sse>} uses \code{xmm} registers for storage and all ISA extensions enabled via compiler flags.

\subsubsection{Naming Options}

\begin{itemize}
  \item \code{vector<T>}
  \item \code{vec<T>}
  \item \code{vecpar<T>}
  \item \code{simd<T>}
  \item \code{datapar<T>}
  \item \code{pack<T>}
  \item \code{simdarray<T>} / \code{simdvector<T>} / \code{vecarray<T>}
  \item \code{vectorize<T>} / \code{simdize<T>} / \code{vectize<T>} / \code{vectorized<T>} / \code{simdized<T>} / \code{vectized<T>}
\end{itemize}

\subsubsection{Discussion}

\begin{itemize}
  \item \code{vector<T>}
    \begin{description}
      \item[pro]
        \begin{enumerate}
          \item term-of-art in the industry. We talk about "vectorization", "vector unit", "vector registers", ...
          \item does work as a mathematical vector, e.g. \code{std::reduce<std::plus>(x*y)} is the scalar product
        \end{enumerate}
      \item[con]
        \begin{enumerate}
          \item \emph{[name collision]} \code{std::vector}: the name is taken. Using a different namespace won't help: Too much confusion/conflict with \code{std::vector}, which is not constant-size.
          \item ambiguity with mathematically inclined people who may expect operators to behave differently (e.g. I've had feedback of users expecting \code{operator*} to be the dot-product)
        \end{enumerate}
    \end{description}

  \item \code{vec<T>}
    \begin{description}
      \item[pro]
        \begin{enumerate}
          \item short
          \item pronounceable
          \item usage is somewhat idiomatic: vec<T> is a vector-lookalike of T.
          \item term-of-art in the industry. We talk about "vectorization", "vector unit", "vector registers", ...
        \end{enumerate}
      \item[con]
        \begin{enumerate}
          \item abbreviation (though rather common)
          \item close to \code{std::vector}
          \item ambiguity with mathematically inclined people who may expect operators to behave differently (e.g. I've had feedback of users expecting \code{operator*} to be the dot-product)
        \end{enumerate}
    \end{description}

  \item \code{vecpar<T>}
    \begin{description}
      \item[pro]
        \begin{enumerate}
          \item short
          \item pronounceable
          \item term-of-art in the industry. We talk about "vectorization", "vector unit", "vector registers", ...
          \item resolves ambiguity with math vector
        \end{enumerate}
      \item[con]
        \begin{enumerate}
          \item abbreviation ("vector parallel")
          \item (\code{par_vec} - it's \code{par_unseq} now, so we should be fine)
        \end{enumerate}
    \end{description}

  \item \code{simd<T>}
    \begin{description}
      \item[pro]
        \begin{enumerate}
          \item short
          \item pronounceable
          \item usage is idiomatic: simd<T> is the SIMD thing for T.
          \item Known term in the industry
          \item maybe even more to the point than "vector" (note variable-length vector units on traditional vector computers)
        \end{enumerate}
      \item[con]
        \begin{enumerate}
          \item acronym
          \item might suggest that the type is not usable for GPUs
          \item one \code{simd<T>} object could drive multiple or partial SIMD registers, multiple partially synchronized threads, one or more non-SIMD registers, a mix of SIMD and non-SIMD registers.
        \end{enumerate}
    \end{description}

  \item \code{datapar<T>}
    \begin{description}
      \item[pro]
        \begin{enumerate}
          \item pronounceable
          \item "data parallel" hints at the intended use: Code expresses inherent data parallelism (intent). Contrast that to "code that uses SIMD registers and operations" (implementation detail).
        \end{enumerate}
      \item[con]
        \begin{enumerate}
          \item abbreviation
          \item new term
        \end{enumerate}
    \end{description}

  \item \code{pack<T>}
    \begin{description}
      \item[pro]
        \begin{enumerate}
          \item short
          \item pronounceable
          \item usage is somewhat idiomatic (e.g. \code{addpd}: “add packed double-precision”)
        \end{enumerate}
      \item[con]
        \begin{enumerate}
          \item \emph{[name collision]} Conflicts with "template parameter pack" usage in variadic templates.
            These tend to appear in the same context: "You can have a [template parameter] pack of packs [types]." (what?)
          \item no hint about concurrently executing operations in the name
        \end{enumerate}
    \end{description}

  \item \code{simdarray<T>}
    \begin{description}
      \item[pro]
        \begin{enumerate}
          \item matches constant-length \code{std:array} and math-style of \code{std::valarray}.
          \item pronounceable
          \item usage is idiomatic: SIMD operations on a fixed-size array
        \end{enumerate}
      \item[con]
        \begin{enumerate}
          \item a bit long for daily use
          \item acronym
          \item might suggest that the type is not usable for GPUs
        \end{enumerate}
      \item[variations]\ 
        \begin{enumerate}
          \item \code{simdvector<T>}: "vector" suggests \code{std::vector} behavior - prefer \code{simdarray<T>}
          \item \code{vecarray<T>}: abbreviation ("vectorized array", \emph{not} "vector array");\\
            "vector array" misleading
        \end{enumerate}
    \end{description}

  \item \code{vectorize<T>}
    \begin{description}
      \item[pro]
        \begin{enumerate}
          \item pronounceable
          \item clear meaning: produces a type that is a \emph{vectorized} \code{T}\\
            i.e. action at compile time, so being a verb is fine
          \item clear meaning if proposal is extended to support \code{std::tuple} for \code{T} (and structs/classes once we get enough reflection into the language)
        \end{enumerate}
      \item[con]
        \begin{enumerate}
          \item it is a class, it should be a noun (\code{vectorization<T>}?)
          \item a bit long for daily use
          \item if data structure vectorization (future extension, cf. \cite{Kretz2015}) should use a different type/mechanism it would be better to reserve this name for said extension
        \end{enumerate}
      \item[variations]\ 
        \begin{enumerate}
          \item \code{simdize<T>}: shorter; downsides of \code{simd} - see above
          \item \code{vectize<T>}: shorter; abbreviation
          \item \code{vectorized<T> simdized<T>}: adjective, still not a noun
        \end{enumerate}
    \end{description}
\end{itemize}

\subsubsection{Recommendation}
I recommend to short-list to:
\begin{itemize}
  \item \code{vec<T>}
  \item \code{datapar<T>}
  \item \code{simd<T>}
  \item \code{vecpar<T>}
  \item \code{simdarray<T>}
\end{itemize}

\subsection{mask}

The class in question is an array of target-specific size with elements of boolean value.
The actual memory layout and storage size is unspecified.
This type is the equivalence of \code{bool} for the \code{datapar<T>} types.
It acts as the return type of \code{datapar} comparisons and can be used for write-masking, masked loads \& stores, and reductions to \code{bool}.

\subsubsection{Naming Options}

\begin{itemize}
  \item \code{mask<T>}
  \item \code{vecmask<T>}
  \item \code{boolvec<T>}
  \item \code{simdmask<T>}
  \item \code{simdbool<T>}
  \item \code{parmask<T>}
  \item \code{boolpack<T>}
\end{itemize}

\subsubsection{Discussion}

Depending on the name chosen for the “datapar” class, there are some natural candidates for the \code{mask} class.
In any case, the \code{mask} name is:
\begin{enumerate}
  \item a term-of-art,
  \item short,
  \item pronounceable,
  \item idiomatic,
  \item noun,
  \item no name collision with existing types (as is the case for \code{vector}).
\end{enumerate}
So I do not see a need for choosing a different (longer) name.

\subsection{where}
The “where function” wraps a \code{mask} object and a reference to a \code{datapar} or \code{mask} object to implement write-masking, and masked loads \& stores.
The function acts as special syntax to express that e.g. assignment shall only happen at the element indexes where the mask object is \code{true}.
The where function returns a temporary object (type \code{where_expression}) that implements the write-masked operations.

\subsubsection{Naming Options}
\begin{itemize}
  \item \code{where}
  \item \code{masked}
  \item \code{withmask}
  \item \code{maskedval}
  \item \code{maskedref}
\end{itemize}

\subsubsection{Discussion}

\begin{itemize}
  \item \code{where}
    \begin{description}
      \item[pro]
        \begin{enumerate}
          \item short
          \item pronounceable
          \item turns code into prose: \code{where(x < y, z) += 2;} reads as “where x is less then y, modify z by adding 2”
          \item naming reflects relation to \code{if} statements
        \end{enumerate}
      \item[con]
        \begin{enumerate}
          \item maybe less intuitive if used in the middle of expressions, e.g. \code{reduce(where(mask, v))}
        \end{enumerate}
    \end{description}

  \item \code{masked}
    \begin{description}
      \item[pro]
        \begin{enumerate}
          \item short
          \item pronounceable
        \end{enumerate}
      \item[con]
        \begin{enumerate}
          \item too close to \code{mask}: ambiguous when spoken
        \end{enumerate}
    \end{description}

  \item \code{withmask}
    \begin{description}
      \item[pro]
        \begin{enumerate}
          \item pronounceable
        \end{enumerate}
      \item[con]
        \begin{enumerate}
          \item less intuitive to read: \code{withmask(x < y, z) += 2;}\\
            does something \emph{with} a mask, what?
        \end{enumerate}
    \end{description}

  \item \code{maskedval}
    \begin{description}
      \item[pro]
        \begin{enumerate}
          \item pronounceable
          \item communicates: produce a new object that is a \emph{masked value} of the given object
        \end{enumerate}
      \item[con]
        \begin{enumerate}
          \item \emph{value} is not technically correct as it actually holds a reference to the given object
          \item the object returned by \code{maskedval} may only exists as rvalue;
            the name suggests otherwise
        \end{enumerate}
    \end{description}

  \item \code{maskedref}
    \begin{description}
      \item[pro]
        \begin{enumerate}
          \item pronounceable
          \item communicates: produce a new object that is a \emph{masked reference} to the given object
        \end{enumerate}
      \item[con]
        \begin{enumerate}
          \item the object returned by \code{maskedref} may only exists as rvalue;
            the name suggests otherwise
        \end{enumerate}
    \end{description}

\end{itemize}

\subsubsection{Recommendation}
My recommendation is to go with \code{where} for what is in the wording now.
If we later want to produce lvalues that act as masked references, I believe we should use a different mechanism/name anyway.
Pablo suggested in private communication that \code{where} could be extended to:
\begin{lstlisting}[style=Vc]
where (mask, v1, v2, [](auto v1_, auto v2_) {
  // type of v1_ is a masked reference to v1
  fun(v1_, v2_);  // all operations of fun on its parameters are masked
});
\end{lstlisting}
This suggests that there might not even be a need for allowing \code{where} or a similar function in the middle of expressions.
If we want to follow that path, we might want to revisit masked reductions, which currently use a \code{const const_where_expression\&} parameter.

\subsection{memload \& memstore}
Loads and stores are the (low-level) conversions between arrays of \code{T} and objects of \code{datapar<T>}.
Converting loads and stores additionally perform widening or narrowing conversions to/from arrays of \code{U}, which is convertible to/from \code{T}.

\code{std::atomic} has member functions called \code{atomic::load} and \code{atomic::store}: \code{load} returns the value of the \code{atomic} with a given \code{memory_order}; \code{store} replaces the value of \code{atomic} with the given value using the given \code{memory_order}.
\code{datapar::load} does the reverse of \code{atomic::load}: it loads \code{datapar::size()} consecutive values starting from the given pointer into the \code{datapar} object.
\code{datapar::store} does the reverse of \code{atomic::store}: it stores \code{datapar::size()} values from the \code{datapar} object to the given pointer.

\subsubsection{Naming Options}

\begin{itemize}
  \item \code{load(const U*, Flags), store(U*, Flags)}
  \item \code{memload(const U*, Flags), memstore(U*, Flags)}
  \item \code{load_from(const U*, Flags), store_to(U*, Flags)}
  \item \code{copy_from(const U*, Flags), copy_to(U*, Flags)}
\end{itemize}

\subsubsection{Discussion}
\begin{itemize}
  \item \code{load(const U*, Flags), store(U*, Flags)}
    \begin{description}
      \item[pro]
        \begin{enumerate}
          \item short
          \item pronounceable
          \item term-of-art
        \end{enumerate}
      \item[con]
        \begin{enumerate}
          \item possibly confusing when compared with \code{load} and \code{store} functions of \code{std::atomic}
        \end{enumerate}
    \end{description}

  \item \code{memload(const U*, Flags), memstore(U*, Flags)}
    \begin{description}
      \item[pro]
        \begin{enumerate}
          \item pronounceable
          \item \code{mem} prefix hints at array behind the pointer argument
        \end{enumerate}
      \item[con]
        \begin{enumerate}
          \item abbreviation (pretty common, though)
        \end{enumerate}
    \end{description}

  \item \code{load_from(const U*, Flags), store_to(U*, Flags)}
    \begin{description}
      \item[pro]
        \begin{enumerate}
          \item pronounceable
          \item reads as prose: \code{v.load_from(mem, vector_aligned)}
        \end{enumerate}
      \item[con]
        \begin{enumerate}
        \end{enumerate}
    \end{description}

  \item \code{copy_from(const U*, Flags), copy_to(U*, Flags)}
    \begin{description}
      \item[pro]
        \begin{enumerate}
          \item pronounceable
          \item reads as prose: \code{v.copy_from(mem, vector_aligned)}
          \item clarifies that values are copied
            (user feedback implies that some people expect aliasing)
        \end{enumerate}
      \item[con]
        \begin{enumerate}
          \item avoids term-of-art (load/store)
        \end{enumerate}
    \end{description}

\end{itemize}

\subsubsection{Recommendation}
My preference is to go with \code{load} and \code{store}.
The \code{std::atomic} class is different enough.
I have never received feedback that the copy direction of the load and store functions is confusing.

My second choice is \code{copy_from/to}.
Avoid the embarrassment of using the terms load and store but having to name them differently just because of \code{std::atomic}.
I'm certain that if we choose \code{memload/memstore} or \code{load_from/store_to} the question why we didn't just use \code{load/store} will become a FAQ.

\subsection{Mask Queries}\label{sec:mask queries}
The free functions \code{all_of}, \code{any_of}, \code{none_of}, \code{some_of}, \code{popcount}, \code{find_first_set}, \code{find_last_set} could alternatively be member functions of \mask.
Having them as member functions would be consistent with \type{bitset}, which has the member functions \code{all}, \code{any}, \code{none}, and \code{count}.

\subsubsection{Recommendation}
I believe the current interface of \mask (free functions) is preferable over the interface of \type{bitset}.
If anything, I believe there should be additional free functions overloaded on \type{bitset} if consistency between \mask and \type{bitset} is desirable.

% ft=tex tw=0 et sw=2 spell
