\section{Widening Cast}\label{sec:widen}
The following presents an option for extending the above wording with a cast function that only allows "lossless" conversions of the element type.
I present three options:
The first option requires a \datapar type as cast type argument. This choice provides control over the returned ABI tag.
The second option requires an element type as cast type argument. This choice uses the typically much simpler cast argument.
The third option works with either.

Note that \code{static_datapar_cast}, which is defined in \ref{sec:datapar.casts}, uses an element type as cast type argument.
Thus, “Option 2” is equivalent to the \code{static_datapar_cast} function, differing only in the requirement that the conversion must be “lossless”.

Examples:
\begin{lstlisting}[style=Vc]
using floatv = native_datapar<float>;
floatv x = ...;

// Option 1:
auto a = datapar_cast<fixed_size_datapar<double, floatv::size()>>(x);

// Option 2:
auto b = datapar_cast<double>(x);

// Option 3:
auto c = datapar_cast<fixed_size_datapar<double, floatv::size()>>(x);
auto d = datapar_cast<double>(x);
\end{lstlisting}

\subsection{Option 1: datapar template argument}

Add to the synopsis in \ref{sec:datapar.syn}:
\begin{wgText}
  \begin{lstlisting}[style=Vc]
    template <class V, class T, class A> V datapar_cast(const datapar<T, A>&);
  \end{lstlisting}
\end{wgText}

Append to \ref{sec:datapar.casts}:
\begin{wgText}
  \begin{itemdecl}
    template <class V, class T, class A> V datapar_cast(const datapar<T, A>& x);
  \end{itemdecl}
  \begin{itemdescr}
    \pnum\remarks The function shall not participate in overload resolution unless
    \begin{itemize}
      \item \code{is_datapar_v<V>},
      \item and \code{V::size()} is equal to \code{datapar<T, A>::size()},
      \item and every possible value of type \type T can be represented with type \datapar\code{::}\valuetype.
    \end{itemize}
    \pnum\returns A \datapar object with the $i$-th element initialized to \code{static_cast<V::\valuetype>(x[i])}.

    \pnum\throws Nothing.
  \end{itemdescr}
\end{wgText}

\subsection{Option 2: element type template argument}
Add to the synopsis in \ref{sec:datapar.syn}:
\begin{wgText}
  \begin{lstlisting}[style=Vc]
    template <class T, class U, class A>
    datapar<T, /*see below*/> datapar_cast(const datapar<U, A>&);
  \end{lstlisting}
\end{wgText}

Append to \ref{sec:datapar.casts}:
\begin{wgText}
  \begin{itemdecl}
    template <class T, class U, class A>
    datapar<T, /*see below*/> datapar_cast(const datapar<U, A>& x);
  \end{itemdecl}
  \begin{itemdescr}
    \pnum\remarks The function shall not participate in overload resolution unless every possible value of type \type U can be represented with type \type T.

    \pnum\remarks The return type is \datapar[<T, A>] if either \type U and \type T are equal or \type U and \type T are integral types that only differ in signedness.
    Otherwise, the return type is \datapar[<T, \fixedsizescoped{}<\datapar<U, A>::size()>>].

    \pnum\returns A \datapar object with the $i$-th element initialized to \code{static_cast<T>(x[i])}.

    \pnum\throws Nothing.
  \end{itemdescr}
\end{wgText}

\subsection{Option 3: both}
Add to the synopsis in \ref{sec:datapar.syn}:
\begin{wgText}
  \begin{lstlisting}[style=Vc]
    template <class T, class U, class A>
    /*see below*/ datapar_cast(const datapar<U, A>&);
  \end{lstlisting}
\end{wgText}

Append to \ref{sec:datapar.casts}:
\begin{wgText}
  \begin{itemdecl}
    template <class T, class U, class A>
    /*see below*/ datapar_cast(const datapar<U, A>&);
  \end{itemdecl}
  \begin{itemdescr}
    \pnum\remarks Let \type{To} identify \type{T::\valuetype} if \code{is_datapar_v<T>} or \type T otherwise.

    \pnum\remarks The function shall not participate in overload resolution unless every possible value of type \type U can be represented with type \type{To}.

    \pnum\remarks If \code{is_datapar_v<T>}, the return type is \type T.
    Otherwise, if either \type U and \type T are equal or \type U and \type T are integral types that only differ in signedness, the return type is \datapar[<T, A>].
    Otherwise, the return type is \datapar[<T, \fixedsizescoped{}<\datapar<U, A>::size()>>].

    \pnum\returns A \datapar object with the $i$-th element initialized to \code{static_cast<To>(x[i])}.

    \pnum\throws Nothing.
  \end{itemdescr}
\end{wgText}

% vim: sw=2 tw=0
