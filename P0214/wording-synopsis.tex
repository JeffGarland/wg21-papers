\wgSubsection{Header \code{<datapar>} synopsis}{datapar.syn}
\lstinputlisting[]{synopsis.cpp}

\pnum
The header \code{<datapar>} defines two class templates (\datapar, and \mask), several tag types, and a series of related function templates for concurrent manipulation of the values in \datapar and \mask objects.

\wgSubsubsection{\datapar ABI tags}{datapar.abi}

\begin{itemdecl}
namespace datapar_abi {
  struct scalar {};  // always present
  template <int N> struct fixed_size {};  // always present
  // implementation-defined tag types, e.g. sse, avx, neon, altivec, ...
  template <typename T> using compatible = implementation_defined;  // always present
  template <typename T> using native = implementation_defined;  // always present
}
\end{itemdecl}
\begin{itemdescr}
  \pnum
  The ABI types are tag types to be used as the second template argument to \datapar and \mask.

  \pnum
  The \type{scalar} tag is present in all implementations and forces \datapar and \mask to store a single component (i.e. \datapar{}\type{<T, datapar_abi::scalar>::size()} returns \code 1).
  \comment{
      \type{scalar} could be an alias for \type{fixed_size<1>}.
      However, I think it is better to have a different ABI tag to make implicit conversions behave as for all the other non-fixed-size ABIs.}

  \pnum\label{datapar.fixedsize.def}
  The \fixedsize tag is present in all implementations.
  Use of \fixedsizeN forces \datapar and \mask to store and manipulate \code N components (i.e. \datapar{}\type{<T, \fixedsizeN{}>::size()} returns \code N).
  An implementation must support at least any \code N $\in [1\ldots 32]$.
  Additionally, an implementation must support any \code N $\in \left\{\code{\datapar[<U>]::size()}, \forall \type{U} \in \{\mathrm{arithmetic\ types}\} \right\}$.
  \wgNote{
    An implementation may choose to not ensure ABI compatibility for \datapar and \mask instantiations using the same \fixedsizeN tag.
    In case of ABI compatibilty between differently compiled translation units, the efficiency of \datapar[<T, Abi>] is likely to be better than for \datapar[<T, fixed_size<datapar_size_v<T, Abi>>>] (with \type{Abi} not a instance of \fixedsizescoped).
  }

  \pnum
  An implementation may choose to implement data-parallel execution for many different targets.
  An additional implementation-defined tag type should be added to the \code{datapar_abi} namespace, for each target the implementation supports.
  \wgNote{There can certainly be more than one tag type per (micro-)architecture, e.g. to support different vector lengths or partial register usage.}
  All tag types an implementation supports shall be present independent of the target architecture determined at invocation of the compiler.

  \pnum
  The \type{datapar_abi::compatible<T>} tag is defined by the implementation to alias the tag type with the most efficient data parallel execution for the element type \type T that ensures the highest compatibility on the target architecture.

  \pnum
  The \type{datapar_abi::native<T>} tag is defined by the implementation to alias the tag type with the most efficient data parallel execution for the element type \type T that is supported on the target system.
  \wgExample{
      Consider a target with the implementation-defined ABI tags \type{simd128} and \type{simd256} where hardware support for \type{simd256} only exists for floating-point types.
      In this case the \type{native<T>} alias equals \type{simd256} if \type T is a floating-point type and \type{simd128} otherwise.
  }
\end{itemdescr}

\wgSubsubsection{\datapar type traits}{datapar.traits}
\begin{itemdecl}
template <class T> struct is_datapar;
\end{itemdecl}
\begin{itemdescr}
  \pnum The \type{is_datapar} type derives from \type{true_type} if \type T is an instance of the \datapar class template.
  Otherwise it derives from \type{false_type}.
\end{itemdescr}

\begin{itemdecl}
template <class T> struct is_mask;
\end{itemdecl}
\begin{itemdescr}
  \pnum The \type{is_mask} type derives from \type{true_type} if \type T is an instance of the \mask class template.
  Otherwise it derives from \type{false_type}.
\end{itemdescr}

\begin{itemdecl}
template <class T, size_t N> struct abi_for_size { typedef implementation_defined type; };
\end{itemdecl}
\begin{itemdescr}
  \pnum
  The \type{abi_for_size} class template defines the member type \type{type} to one of the tag types in \code{datapar_abi}.
  If a tag type \type A exists that satisfies
  \begin{itemize}
    \item \code{datapar_size_v<T, A> == N},
    \item \type A is a supported \type{Abi} parameter to \datapar[<T, Abi>] for the current compilation target, and
    \item \type A is not \fixedsizeN,
  \end{itemize} then the member type \type{type} is an alias for \type A.
  Otherwise \type{type} is an alias for \fixedsizeN.

  \pnum \code{abi_for_size<T, N>::type} shall result in a substitution failure if \type T is not supported by \datapar or if \code N is not supported by the implementation (cf. [\ref{datapar.fixedsize.def}]).
\end{itemdescr}

\begin{itemdecl}
template <class T, class Abi = datapar_abi::compatible<T>>
struct datapar_size : public integral_constant<size_t, implementation_defined> {};
\end{itemdecl}
\begin{itemdescr}
  \pnum The \type{datapar_size} class template inherits from \type{integral_constant} with a value that equals \datapar{}\code{<T, Abi>::size()}.

  \pnum \code{datapar_size<T, Abi>::value} shall result in a substitution failure if any of the template arguments \type T or \type{Abi} are invalid template arguments to \datapar.
\end{itemdescr}

\begin{itemdecl}
template <class T, class U = typename T::value_type>
constexpr size_t memory_alignment = implementation_defined;
\end{itemdecl}
\begin{itemdescr}
  \pnum\requires The template parameter \type T must be a valid instantiation of either the \datapar or the \mask class template.
  \pnum\requires The template parameter \type U must be a type supported by the load and store functions for \type T.
  \pnum The value of \code{memory_alignment<T, U>} identifies the alignment restrictions on pointers used for (converting) loads and stores for the given type \type T on arrays of type \type U.
\end{itemdescr}

% vim: tw=0
