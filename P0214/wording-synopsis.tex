\wgSubsection{Header \code{<datapar>} synopsis}{datapar.syn}
\lstinputlisting[]{synopsis.cpp}

\pnum
The header \code{<datapar>} defines the class templates (\datapar, \mask, and \type{where_expression}), several tag types, and a series of related function templates for concurrent manipulation of the values in \datapar and \mask objects.

\wgSubsubsection{\datapar ABI tags}{datapar.abi}

\begin{itemdecl}
namespace datapar_abi {
  struct scalar {};  // always present
  template <int N> struct fixed_size {};  // always present
  constexpr int max_fixed_size = implementation_defined;
  // implementation-defined tag types, e.g. sse, avx, neon, altivec, ...
  template <typename T> using compatible = implementation_defined;  // always present
  template <typename T> using native = implementation_defined;  // always present
}
\end{itemdecl}
\begin{itemdescr}
  \pnum
  The ABI types are tag types to be used as the second template argument to \datapar and \mask.

  \pnum
  The \type{scalar} tag is present in all implementations and forces \datapar and \mask to store a single component (i.e. \datapar{}\type{<T, datapar_abi::scalar>::size()} returns \code 1).
  \wgNote{\type{scalar} shall not be an alias for \type{fixed_size<1>}.}

  \pnum\label{datapar.fixedsize.def}%
  The \fixedsize tag is present in all implementations.
  Use of \fixedsizeN forces \datapar and \mask to store and manipulate \code N components (i.e. \datapar{}\type{<T, \fixedsizeN{}>::size()} returns \code N).
  An implementation must support at least any \code N $\in [1\ldots 32]$.
  Additionally, an implementation must support any \code N $\in \left\{\code{\datapar[<U>]::size()}, \forall \type{U} \in \{\mathrm{arithmetic\ types}\} \right\}$.
  \wgNote{
    An implementation may choose to not ensure ABI compatibility for \datapar and \mask instantiations using the same \fixedsizeN tag.
    In case of ABI compatibilty between differently compiled translation units, the efficiency of \datapar[<T, Abi>] is likely to be better than for \datapar[<T, fixed_size<datapar_size_v<T, Abi>>>] (with \type{Abi} not a instance of \fixedsizescoped).
  }

  \pnum\label{datapar.maxfixedsize.def}%
  The value of \code{max_fixed_size} declares that an instance of \datapar[<T, fixed_size<N>>] with \code{N <= max_fixed_size} is supported by the implementation.%
  \comment{I'm afraid this makes changes to the maximum size an ABI break, no? Unfounded fear?}
  \wgNote{It is still possible for an implementation to support \datapar[<U, fixed_size<K>>] with \code{K > max_fixed_size}.}

  \pnum
  An implementation may choose to implement data-parallel execution for many different targets.
  An additional implementation-defined tag type should be added to the \code{datapar_abi} namespace, for each target the implementation supports.
  \wgNote{There can certainly be more than one tag type per (micro-)architecture, e.g. to support different vector lengths or partial register usage.}
  All tag types an implementation supports shall be present independent of the target architecture determined at invocation of the compiler.

  \pnum
  The \type{datapar_abi::compatible<T>} tag is defined by the implementation to alias the tag type with the most efficient data parallel execution for the element type \type T that ensures the highest compatibility on the target architecture.

  \pnum
  The \type{datapar_abi::native<T>} tag is defined by the implementation to alias the tag type with the most efficient data parallel execution for the element type \type T that is supported on the target system.
  \wgExample{
      Consider a target with the implementation-defined ABI tags \type{simd128} and \type{simd256} where hardware support for \type{simd256} only exists for floating-point types.
      In this case the \type{native<T>} alias equals \type{simd256} if \type T is a floating-point type and \type{simd128} otherwise.
  }
\end{itemdescr}

\wgSubsubsection{\datapar type traits}{datapar.traits}
\begin{itemdecl}
template <class T> struct is_datapar;
\end{itemdecl}
\begin{itemdescr}
  \pnum The \type{is_datapar} type derives from \type{true_type} if \type T is an instance of the \datapar class template.
  Otherwise it derives from \type{false_type}.
\end{itemdescr}

\begin{itemdecl}
template <class T> struct is_mask;
\end{itemdecl}
\begin{itemdescr}
  \pnum The \type{is_mask} type derives from \type{true_type} if \type T is an instance of the \mask class template.
  Otherwise it derives from \type{false_type}.
\end{itemdescr}

\begin{itemdecl}
template <class T, size_t N> struct abi_for_size { using type = implementation_defined; };
\end{itemdecl}
\begin{itemdescr}
  \pnum
  The \type{abi_for_size} class template defines the member type \type{type} to one of the tag types in \code{datapar_abi}.
  If a tag type \type A exists that satisfies
  \begin{itemize}
    \item \code{datapar_size_v<T, A> == N},
    \item \type A is a supported \type{Abi} parameter to \datapar[<T, Abi>] for the current compilation target, and
    \item \type A is not \fixedsizeN,
  \end{itemize} then the member type \type{type} is an alias for \type A.
  Otherwise \type{type} is an alias for \fixedsizeN.

  \pnum \code{abi_for_size<T, N>::type} shall result in a substitution failure if \type T is not supported by \datapar or if \code N is not supported by the implementation (cf. [\ref{datapar.fixedsize.def}]).
\end{itemdescr}

\begin{itemdecl}
template <class T, class Abi = datapar_abi::compatible<T>>
struct datapar_size : public integral_constant<size_t, implementation_defined> {};
\end{itemdecl}
\begin{itemdescr}
  \pnum The \type{datapar_size} class template inherits from \type{integral_constant} with a value that equals \datapar{}\code{<T, Abi>::size()}.

  \pnum \code{datapar_size<T, Abi>::value} shall result in a substitution failure if any of the template arguments \type T or \type{Abi} are invalid template arguments to \datapar.
\end{itemdescr}

\begin{itemdecl}
template <class T, class U = typename T::value_type>
constexpr size_t memory_alignment = implementation_defined;
\end{itemdecl}
\begin{itemdescr}
  \pnum\requires The template parameter \type T must be a valid instantiation of either the \datapar or the \mask class template.
  \pnum\requires The template parameter \type U must be a type supported by the load and store functions for \type T.
  \pnum The value of \code{memory_alignment<T, U>} identifies the alignment restrictions on pointers used for (converting) loads and stores for the given type \type T on arrays of type \type U.
\end{itemdescr}

\wgSubsubsection{Class template \code{where_expression}}{datapar.whereexpr}
\pnum The class template \code{where_expression<M, T>} combines a predicate and a value object to implement an interface that restricts assignments and/or operations on the value object to the elements selected via the predicate.

\pnum The first template argument \type M must be cv-unqualified \bool or a cv-unqualified \mask instantiation.

\pnum The second template argument \type T must be a cv-unqualified or \const qualified type \type{T'}.
If \type M is \bool, \type{T'} must be an arithmetic type.
Otherwise, \type{T'} must either be \type M or \type{M::datapar_type}.

\begin{itemdecl}
const M &mask;                     // exposition only
T &data;                           // exposition only
where_expression(const M &, T &);  // exposition only
\end{itemdecl}
\begin{itemdescr}
  \pnum\effects
  The implementation initializes a \type{where_expression<M, T>} object with a predicate of type \type M and a reference to a value object of type \type T.

  \pnum\realnote The predicate object may be copied by the constructor implementation.
  If \type T is const qualified the constructor may copy the value object.

  \pnum\realnote
  The following declarations refer to the predicate as data member \code{mask} and to the value reference as data member \code{data}.
\end{itemdescr}

\begin{itemdecl}
template <class U> void operator=(U &&x);
template <class U> void operator+=(U &&x);
template <class U> void operator-=(U &&x);
template <class U> void operator*=(U &&x);
template <class U> void operator/=(U &&x);
template <class U> void operator%=(U &&x);
template <class U> void operator&=(U &&x);
template <class U> void operator|=(U &&x);
template <class U> void operator^=(U &&x);
template <class U> void operator<<=(U &&x);
template <class U> void operator>>=(U &&x);
\end{itemdecl}
\begin{itemdescr}
  \pnum\remarks Each of these operators only participate in overload resolution if the indicated operator can be applied to objects of type \type T.
  \pnum\effects
  If \type M is \bool, applies the indicated operator on \code{data} and \code{forward<U>(x)} unless \code{mask} is \false.
  If \type M is not \bool, applies the indicated operator on \code{data} and \code{forward<U>(x)} without modifying the elements \code{data[i]} where \code{mask[i]} is \false \foralli[M::].
  \pnum\remarks It is unspecified whether the arithmetic/bitwise operation, which is implied by a compound assignment operator, is executed on all elements or only on the ones written back.
\end{itemdescr}

\begin{itemdecl}
void operator++();
void operator++(int);
void operator--();
void operator--(int);
\end{itemdecl}
\begin{itemdescr}
  \pnum\remarks Each of these operators only participate in overload resolution if the indicated operator can be applied to objects of type \type T.
  \pnum\effects
  If \type M is \bool, applies the indicated operator on \code{data} unless \code{mask} is \false.
  If \type M is not \bool, applies the indicated operator on \code{data} without modifying the elements \code{data[i]} where \code{mask[i]} is \false \foralli[M::].
  \pnum\realnote It is unspecified whether the inc-/decrement operation is executed on all elements or only on the ones written back.
\end{itemdescr}

\begin{itemdecl}
remove_const_t<T> operator-() const;
\end{itemdecl}
\begin{itemdescr}
  \pnum\remarks This operator only participates in overload resolution if the indicated operator can be applied to objects of type \type T.
  \pnum\returns If \type M is \bool, \code{-data} if \code{mask} is \true, \code{data} otherwise.
  If \type M is not \bool, returns an object with the $i$-th element initialized to \code{-data[i]} if \code{mask[i]} is \true and \code{data[i]} otherwise \foralli[M::].
\end{itemdescr}

\begin{itemdecl}
template <class U, class Flags>
[[nodiscard]] remove_const_t<T> memload(const U *mem, Flags) const;
\end{itemdecl}
\begin{itemdescr}
  \pnum\remarks If \type T is \bool or a \mask instantiation, the function only participates in overload resolution if \type U is \bool.
  \pnum\returns If \type M is \bool, return \code{mem[0]} if \code{mask} equals \true and return \code{data} otherwise.
  If \type M is not \bool, return an object with the $i$-th element initialized to the $i$-th element of \code{data} if \code{mask[i]} is \false and \code{static_cast<T::value_type>(mem[i])} if \code{mask[i]} is \true \foralli[M::].

  \pnum\requires If \type M is not \bool, the largest $i$ where \code{mask[i]} is \true is less than the number of values pointed to by \code{mem}.
  \flagsRemarks{\type T, \type U}
\end{itemdescr}

\begin{itemdecl}
template <class U, class Flags> void memload(const U *mem, Flags);
\end{itemdecl}
\begin{itemdescr}
  \pnum\remarks If \type T is \bool or a \mask instantiation, the function only participates in overload resolution if \type U is \bool.
  \pnum\effects If \type M is \bool, assign \code{mem[0]} to \code{data} unless \code{mask} is \false.
  If \type M is not \bool, replace the elements of \code{data} where \code{mask[i]} is \true such that the $i$-th element is assigned with \code{static_cast<T::value_type>(mem[i])} \foralli[M::].

  \pnum\requires If \type M is not \bool, the largest $i$ where \code{mask[i]} is \true is less than the number of values pointed to by \code{mem}.
  \flagsRemarks{\type T, \type U}
\end{itemdescr}

\begin{itemdecl}
template <class U, class Flags> void memstore(U *mem, Flags) const;
\end{itemdecl}
\begin{itemdescr}
  \pnum\effects If \type M is \bool, assign \code{data} to \code{mem[0]} unless \code{mask} is \false.
  \pnum\remarks If \type T is a (const qualified) \mask instantiation, the function only participates in overload resolution if \type U is \bool.
  If \type M is not \bool, copies the elements \code{data[i]} where \code{mask[i]} is \true as if \code{mem[i] = static_cast<U>(data[i])} \foralli[M::].

  \pnum\requires If \type M is not \bool, the largest $i$ where \code{mask[i]} is \true is less than the number of values pointed to by \code{mem}.
  \flagsRemarks{\type remove_const_t<T>, \type U}
\end{itemdescr}

% vim: tw=0
