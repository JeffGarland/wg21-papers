\wgSubsection{Header \code{<experimental/simd>} synopsis}{simd.syn}
\lstinputlisting[]{synopsis.cpp}

\pnum
The header \code{<experimental/simd>} defines class templates, tag types, trait types, and function templates for element-wise operations on data-parallel objects.

\wgSubsubsection{\simd ABI tags}{simd.abi}

\begin{itemdecl}
namespace simd_abi {
  struct scalar {};
  template <int N> struct fixed_size {};
  template <typename T> inline constexpr int max_fixed_size = implementation-defined;
  template <typename T> using compatible = implementation-defined;
  template <typename T> using native = implementation-defined;
}
\end{itemdecl}
\begin{itemdescr}
  \pnum
  An \emph{ABI tag} is a type in the \code{simd_abi} namespace that indicates a choice of size and binary representation for objects of \dataparalleltype.
  \wgNote{The intent is for the size and binary representation to depend on the \targetArch.}
  The ABI tag, together with a given element type implies a number of elements.
  ABI tag types are used as the second template argument to \simd and \mask.
  \wgNote{
    The ABI tag is orthogonal to selecting the machine instruction set.
    The selected machine instruction set limits the usable ABI tag types, though (see \ref{simd.type requirements}).
    The ABI tags enable users to safely pass objects of \dataparalleltype between translation unit boundaries (e.g. function calls or I/O).
  }

  \pnum
  Use of the \type{scalar} tag type requires \dataparalleltypes to store a single element (i.e., \simd{}\type{<T, simd_abi::scalar>::size()} returns \code 1).
  \wgNote{\type{scalar} is not an alias for \type{fixed_size<1>}.}

  \pnum\label{simd.fixedsize.def}%
  Use of the \fixedsizeN tag type requires \dataparalleltypes to store and manipulate \code N elements (i.e. \simd{}\type{<T, \fixedsizeN{}>::size()} returns \code N).
  An implementation shall support at least all \code N $\in [1\ldots 32]$.
  Additionally, for every supported \simd[<T, Abi>] (see \ref{simd.type requirements}), where \type{Abi} is an implementation-defined ABI tag, \code N $=$ \simd[<T, Abi>::size()] shall be supported.

  \wgNote{
    An implementation may choose to forego ABI compatibility between differently compiled translation units for \simd and \mask instantiations using the same \fixedsizeN tag.
    Otherwise, the efficiency of \simd[<T, Abi>] is likely to be better than for \simd[<T, fixed_size<simd_size_v<T, Abi>>>] (with \type{Abi} not a instance of \fixedsizescoped).
  }

  \pnum\label{simd.maxfixedsize.def}%
  The value of \code{max_fixed_size<T>} declares that an instance of \simd[<T, fixed_size<N>>] with \code{N <= max_fixed_size<T>} is supported by the implementation.
  \wgNote{
    It is unspecified whether an implementation supports \simd[<T, fixed_size<N>>] with \code{N > max_fixed_size<T>}.
    The value of \code{max_fixed_size<T>} may depend on compiler flags and may change between different compiler versions.
  }

  \pnum
  An implementation may define additional ABI tag types in the \code{simd_abi} namespace, to support other forms of data-parallel computation.

  \pnum \type{compatible<T>} is an implementation-defined alias for an ABI tag.
    \wgNote{
      The intent is to use the ABI tag producing the most efficient data-parallel execution for the element type \type T that ensures ABI compatibility between translation units on the \targetArch.
    }

  \wgExample{
    Consider a \targetArch supporting the implementation-defined ABI tags \type{simd128} and \type{simd256}, where the \type{simd256} type requires an optional ISA extension on said \targetArch.
    Also, the \targetArch does not support \type{long double} with either ABI tag.
    The implementation therefore defines
    \begin{itemize}
      \item \type{compatible<T>} as an alias for \type{simd128} for all arithmetic \type T, except \type{long double},
      \item and \type{compatible<long double>} as an alias for \type{scalar}.
    \end{itemize}
  }

  \pnum \type{native<T>} is an implementation-defined alias for an ABI tag.
  \wgNote{
    The intent is to use the ABI tag producing the most efficient data-parallel execution for the element type \type T that is supported on the \currentTarget.
    For \targetArchs without ISA extensions, the \type{native<T>} and \type{compatible<T>} aliases will likely be the same.
    For \targetArchs with ISA extensions, compiler flags may influence the \type{native<T>} alias while \type{compatible<T>} will be the same independent of such flags.
  }

  \wgExample{
      Consider a \targetArch supporting the implementation-defined ABI tags \type{simd128} and \type{simd256}, where hardware support for \type{simd256} only exists for floating-point types.
    The implementation therefore defines \type{native<T>} as an alias for
    \begin{itemize}
      \item \type{simd256} if \type T is a floating-point type,
      \item and \type{simd128} otherwise.
    \end{itemize}
  }
\end{itemdescr}

\wgSubsubsection{\simd type traits}{simd.traits}
\begin{itemdecl}
template <class T> struct is_abi_tag { @\emph{see below}@ };
\end{itemdecl}
\begin{itemdescr}
  \pnum The type \type{is_abi_tag<T>} is a \UnaryTypeTrait with a \BaseCharacteristic of \type{true_type} if \type T is a standard or implementation-defined ABI tag, and \type{false_type} otherwise.
\end{itemdescr}

\begin{itemdecl}
template <class T> struct is_simd { @\emph{see below}@ };
\end{itemdecl}
\begin{itemdescr}
  \pnum The type \type{is_simd<T>} is a \UnaryTypeTrait with a \BaseCharacteristic of \type{true_type} if \type T is an instance of the \simd class template, and \type{false_type} otherwise.
\end{itemdescr}

\begin{itemdecl}
template <class T> struct is_simd_mask { @\emph{see below}@ };
\end{itemdecl}
\begin{itemdescr}
  \pnum The type \type{is_simd_mask<T>} is a \UnaryTypeTrait with a \BaseCharacteristic of \type{true_type} if \type T is an instance of the \mask class template, and \type{false_type} otherwise.
\end{itemdescr}

\begin{itemdecl}
template <class T> struct is_simd_flag_type { @\emph{see below}@ };
\end{itemdecl}
\begin{itemdescr}
  \pnum The type \type{is_simd_flag_type<T>} is a \UnaryTypeTrait with a \BaseCharacteristic of \type{true_type} if \type T is one of
  \begin{itemize}
    \item \code{element_aligned_tag}, or
    \item \code{vector_aligned_tag}, or
    \item \code{overaligned_tag<N>} with arbitrary value \code{N}
  \end{itemize}
  and \type{false_type} otherwise.
\end{itemdescr}

\begin{itemdecl}
template <class T, size_t N> struct abi_for_size { using type = @\emph{see below}@; };
\end{itemdecl}
\begin{itemdescr}
  \pnum The member \type{type} shall be omitted unless
  \begin{itemize}
    \item \type T is a cv-unqualified type, and
    \item \type T is a \realArithmeticType{}, and
    \item \fixedsizeN is supported (see \ref{simd.fixedsize.def}).
  \end{itemize}

  \pnum Where present, the member typedef \type{type} shall name an ABI tag type that satisfies
  \begin{itemize}
    \item \code{simd_size_v<T, type> == N}, and
    \item \simd[<T, type>] is default constructible (see \ref{simd.type requirements}),
  \end{itemize}
  \code{simd_abi::scalar} takes precedence over \code{simd_abi::}\fixedsize\code{<1>}.
  The precedence of implementation-defined ABI tags over \fixedsizeN is implementation-defined.
  \wgNote{
    It is expected that implementation-defined ABI tags can produce better optimizations and thus take precedence over \fixedsizeN.
  }
\end{itemdescr}

\begin{itemdecl}
template <class T, class Abi = simd_abi::compatible<T>> struct simd_size { @\emph{see below}@ };
\end{itemdecl}
\begin{itemdescr}
  \pnum\label{simd_size}%
  \type{simd_size<T, Abi>} shall have no member \code{value} unless
  \begin{itemize}
    \item \type T is a cv-unqualified type, and
    \item \type T is a \realArithmeticType, and
    \item \code{is_abi_tag_v<Abi>} is \true.
  \end{itemize}
  \wgNote{The rules are different from \ref{simd.deleted}}

  \pnum
  Otherwise, the type \type{simd_size<T, Abi>} is a \BinaryTypeTrait with a \BaseCharacteristic of \type{integral_constant<size_t, N>} with \code{N} equal to the number of elements in a \simd[<T, Abi>] object.
  \wgNote{
    If \simd[<T, Abi>] is not supported for the \currentTarget, \type{simd_size<T, Abi>::value} produces the value \simd[<T, Abi>::size()] would return if it were supported.
  }

\end{itemdescr}

\begin{itemdecl}
template <class T, class U = typename T::value_type> struct memory_alignment { @\emph{see below}@ };
\end{itemdecl}
\begin{itemdescr}
  \pnum
  \type{memory_alignment<T, U>} shall have no member \code{value} if either
  \begin{itemize}
    \item \type T is cv-qualified, or
    \item \type U is cv-qualified, or
    \item \code{!is_simd_v<T> \&\& !is_simd_mask_v<T>}, or
    \item \code{is_simd_v<T>} and \type U is not an arithmetic type or \type U is \bool, or
    \item \code{is_simd_mask_v<T>} and \type U is not \bool.
  \end{itemize}

  \pnum
  Otherwise, the type \type{memory_alignment<T, U>} is a \BinaryTypeTrait with a \BaseCharacteristic of \type{integral_constant<size_t, N>} for some implementation-defined \code{N}.
  \wgNote{
    \code{value} identifies the alignment restrictions on pointers used for (converting) loads and stores for the given type \type T on arrays of type \type U (see \ref{sec:simd.copy} and \ref{sec:simd.mask.copy}).
  }
\end{itemdescr}

\wgSubsubsection{Class templates \code{const_where_expression} and \code{where_expression}}{simd.whereexpr}
\lstinputlisting[]{whereexpression.cpp}

\pnum
The following\comment{i.e. subclauses \ref{sec:simd.whereexpr}, \ref{sec:simd.reductions}, and \ref{sec:simd.mask.where}} refers to an object of type \type M as exposition-only data member \code{mask} and to a reference to \type T as exposition-only data member \code{data}.

\pnum In the following,\comment{i.e. subclause \ref{sec:simd.whereexpr}} if \type M is \bool, \code{data[0]} is used interchangeably for \code{data},
\code{mask[0]} is used interchangeably for \code{mask}, and
\code{M::size()} is used interchangeably for \code{1}.

\pnum The class templates \code{const_where_expression} and \code{where_expression} abstract the notion of \emph{selected elements} of a given object of arithmetic or data-parallel type.
%If \code{is_simd_mask_v<M>} is \true,
Selected elements signifies the elements \code{data[i]} \forallmaskedi[M::]{mask}.
%If \type M is \bool, selected elements signifies either \code{data} if \code{mask} is \true, or no element if \code{mask} is \false.

\pnum The first template argument \type M shall be cv-unqualified \bool or a cv-unqualified \mask instantiation.

\pnum
If \type M is \bool, \type{T} shall be a non-volatile arithmetic type.
Otherwise, \type{T} shall either be \type M, \type{const M}, \type{typename M::simd_type}, or \type{const typename M::simd_type}.

\pnum In the following,\comment{i.e. subclause \ref{sec:simd.whereexpr}} the type \valuetype is an alias for \code{remove_const_t<T>} if \type M is \bool, or an alias for \code{typename T::value_type} if \code{is_simd_mask_v<M>} is \true.

\pnum\wgNote{
  The \code{where} functions [simd.mask.where] initialize \code{mask} with the first argument to \code{where} and \code{data} with the second argument to \code{where}.
}

\begin{itemdecl}
remove_const_t<T> operator-() const &&;
\end{itemdecl}
\begin{itemdescr}
  \pnum\returns A copy of \code{data} with unary minus applied to all selected elements.

  \pnum\throws Nothing.
\end{itemdescr}

\begin{itemdecl}
template <class U, class Flags> void copy_to(U *mem, Flags) const &&;
\end{itemdecl}
\begin{itemdescr}
  \flagsRequires{\type remove_const_t<T>, \type U}
  %\pnum\requires
   If \type M is not \bool, the largest i $\in$ \code{[0, M::size())} where \code{mask[i]} is \true is less than the number of values pointed to by \code{mem}.

  \pnum\effects Copies the selected elements as if \code{mem[i] = static_cast<U>(data[i])} \forallmaskedi[M::]{mask}.

  \pnum\throws Nothing.

  \pnum\remarks If \code{remove_const_t<T>} is \bool or \code{is_simd_mask_v<remove_const_t<T>>}, the function shall not participate in overload resolution unless \type U is \bool.
  Otherwise, the function shall not participate in overload resolution unless \type U is a \realArithmeticType and \code{is_simd_flag_type_v<Flags>} is \true.
\end{itemdescr}

\begin{itemdecl}
template <class U> void operator=(U&& x);
template <class U> void operator+=(U&& x);
template <class U> void operator-=(U&& x);
template <class U> void operator*=(U&& x);
template <class U> void operator/=(U&& x);
template <class U> void operator%=(U&& x);
template <class U> void operator&=(U&& x);
template <class U> void operator|=(U&& x);
template <class U> void operator^=(U&& x);
template <class U> void operator<<=(U&& x);
template <class U> void operator>>=(U&& x);
\end{itemdecl}
\begin{itemdescr}
  \pnum\effects
  Overwrites the selected elements with the corresponding elements from the application of the indicated operator on \code{data} and \code{forward<U>(x)}.

  \pnum\remarks Each of these operators shall not participate in overload resolution unless the indicated operator can be applied to objects of type \type T.
  %\pnum\remarks
  It is unspecified whether the binary operator, implied by the compound assignment operator, is executed on all elements or only on the selected elements.
\end{itemdescr}

\begin{itemdecl}
void operator++();
void operator++(int);
void operator--();
void operator--(int);
\end{itemdecl}
\begin{itemdescr}
  \pnum\effects Applies the indicated operator to the selected elements.

  \pnum\remarks Each of these operators shall not participate in overload resolution unless the indicated operator can be applied to objects of type \type T.
\end{itemdescr}

\begin{itemdecl}
template <class U, class Flags> void copy_from(const U *mem, Flags);
\end{itemdecl}
\begin{itemdescr}
  \flagsRequires{\type T, \type U}
  %\pnum\requires
  If \type M is not \bool, the largest i $\in$ \code{[0, M::size())} where \code{mask[i]} is \true is less than the number of values pointed to by \code{mem}.

  \pnum\effects Overwrites the selected elements with the corresponding values \code{static_cast<value_type>(mem[i])} \forallmaskedi[M::]{mask}.
  %If \type M is \bool, assign \code{mem[0]} to \code{data} unless \code{mask} is \false.
  %If \type M is not \bool, replace the elements of \code{data} where \code{mask[i]} is \true such that the $i$-th element is assigned with \code{static_cast<T::value_type>(mem[i])} \foralli[M::].

  \pnum\remarks If \type T is \bool or \code{is_simd_mask_v<T>}, this function shall not participate in overload resolution unless \type U is \bool.
\end{itemdescr}

% vim: tw=0
