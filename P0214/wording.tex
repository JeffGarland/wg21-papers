\section{Wording}

The following is a draft targetting inclusion into the Parallelism TS 2.
It defines a basic set of data-parallel types and operations.

\newcommand\clause{Clause\xspace}
\newcommand\width{width\xspace}

\begin{wgText}
  \setcounter{WGClause}{7}
  \wgSection{Data-Parallel Types}{simd.types}
  \wgSubsection{General}{simd.general}
  \pnum
  The data-parallel library consists of data-parallel types and operations on these types.
  A data-parallel type consists of elements of an underlying arithmetic type, called the \emph{element type}.
  The number of elements is a constant for each data-parallel type and called the \width of that type.

  \pnum
  Throughout this \clause, the term \emph{data-parallel type} refers to all \emph{supported} \ref{sec:simd.overview} instantiations of the \simd and \mask class templates.
  A \emph{data-parallel object} is an object of \emph{data-parallel type}.

  \pnum
  An \emph{element-wise} operation applies a specified operation to the elements of one or more data-parallel objects.
  Each per-element operation is unsequenced with respect to one another.\comment{this should allow GPU execution}
  A \emph{unary element-wise} operation is an element-wise operation that applies a unary operation to each element of a data-parallel object.
  A \emph{binary element-wise} operation is an element-wise operation that applies a binary operation to corresponding elements of two data-parallel objects.

  \pnum Throughout this \clause, the set of \emph{vectorizable types} for a data-parallel type comprises all cv-unqualified arithmetic types other than \bool.

  \pnum \label{cl:intent-note}\wgNote{
      The intent is to support acceleration through data-parallel execution resources, such as SIMD registers and instructions or execution units driven by a common instruction decoder.
      If such execution resources are unavailable, the interfaces support a transparent fallback to sequential execution.
  }

  \wgSubsection{Header \code{<datapar>} synopsis}{datapar.syn}
\lstinputlisting[]{synopsis.cpp}

\pnum
The header \code{<datapar>} defines the class templates (\datapar, \mask, and \type{where_expression}), several tag types, and a series of related function templates for concurrent manipulation of the values in \datapar and \mask objects.

\wgSubsubsection{\datapar ABI tags}{datapar.abi}

\begin{itemdecl}
namespace datapar_abi {
  struct scalar {};  // always present
  template <int N> struct fixed_size {};  // always present
  constexpr int max_fixed_size = implementation_defined;
  // implementation-defined tag types, e.g. sse, avx, neon, altivec, ...
  template <typename T> using compatible = implementation_defined;  // always present
  template <typename T> using native = implementation_defined;  // always present
}
\end{itemdecl}
\begin{itemdescr}
  \pnum
  The ABI types are tag types to be used as the second template argument to \datapar and \mask.

  \pnum
  The \type{scalar} tag is present in all implementations and forces \datapar and \mask to store a single component (i.e. \datapar{}\type{<T, datapar_abi::scalar>::size()} returns \code 1).
  \wgNote{\type{scalar} shall not be an alias for \type{fixed_size<1>}.}

  \pnum\label{datapar.fixedsize.def}%
  The \fixedsize tag is present in all implementations.
  Use of \fixedsizeN forces \datapar and \mask to store and manipulate \code N components (i.e. \datapar{}\type{<T, \fixedsizeN{}>::size()} returns \code N).
  An implementation must support at least any \code N $\in [1\ldots 32]$.
  Additionally, an implementation must support any \code N $\in \left\{\code{\datapar[<U>]::size()}, \forall \type{U} \in \{\mathrm{arithmetic\ types}\} \right\}$.
  \wgNote{
    An implementation may choose to not ensure ABI compatibility for \datapar and \mask instantiations using the same \fixedsizeN tag.
    In case of ABI compatibilty between differently compiled translation units, the efficiency of \datapar[<T, Abi>] is likely to be better than for \datapar[<T, fixed_size<datapar_size_v<T, Abi>>>] (with \type{Abi} not a instance of \fixedsizescoped).
  }

  \pnum\label{datapar.maxfixedsize.def}%
  The value of \code{max_fixed_size} declares that an instance of \datapar[<T, fixed_size<N>>] with \code{N <= max_fixed_size} is supported by the implementation.%
  \comment{I'm afraid this makes changes to the maximum size an ABI break, no? Unfounded fear?}
  \wgNote{It is still possible for an implementation to support \datapar[<U, fixed_size<K>>] with \code{K > max_fixed_size}.}

  \pnum
  An implementation may choose to implement data-parallel execution for many different targets.
  An additional implementation-defined tag type should be added to the \code{datapar_abi} namespace, for each target the implementation supports.
  \wgNote{There can certainly be more than one tag type per (micro-)architecture, e.g. to support different vector lengths or partial register usage.}
  All tag types an implementation supports shall be present independent of the target architecture determined at invocation of the compiler.

  \pnum
  The \type{datapar_abi::compatible<T>} tag is defined by the implementation to alias the tag type with the most efficient data parallel execution for the element type \type T that ensures the highest compatibility on the target architecture.

  \pnum
  The \type{datapar_abi::native<T>} tag is defined by the implementation to alias the tag type with the most efficient data parallel execution for the element type \type T that is supported on the target system.
  \wgExample{
      Consider a target with the implementation-defined ABI tags \type{simd128} and \type{simd256} where hardware support for \type{simd256} only exists for floating-point types.
      In this case the \type{native<T>} alias equals \type{simd256} if \type T is a floating-point type and \type{simd128} otherwise.
  }
\end{itemdescr}

\wgSubsubsection{\datapar type traits}{datapar.traits}
\begin{itemdecl}
template <class T> struct is_datapar;
\end{itemdecl}
\begin{itemdescr}
  \pnum The \type{is_datapar} type derives from \type{true_type} if \type T is an instance of the \datapar class template.
  Otherwise it derives from \type{false_type}.
\end{itemdescr}

\begin{itemdecl}
template <class T> struct is_mask;
\end{itemdecl}
\begin{itemdescr}
  \pnum The \type{is_mask} type derives from \type{true_type} if \type T is an instance of the \mask class template.
  Otherwise it derives from \type{false_type}.
\end{itemdescr}

\begin{itemdecl}
template <class T, size_t N> struct abi_for_size { using type = implementation_defined; };
\end{itemdecl}
\begin{itemdescr}
  \pnum
  The \type{abi_for_size} class template defines the member type \type{type} to one of the tag types in \code{datapar_abi}.
  If a tag type \type A exists that satisfies
  \begin{itemize}
    \item \code{datapar_size_v<T, A> == N},
    \item \type A is a supported \type{Abi} parameter to \datapar[<T, Abi>] for the current compilation target, and
    \item \type A is not \fixedsizeN,
  \end{itemize} then the member type \type{type} is an alias for \type A.
  Otherwise \type{type} is an alias for \fixedsizeN.

  \pnum \code{abi_for_size<T, N>::type} shall result in a substitution failure if \type T is not supported by \datapar or if \code N is not supported by the implementation (cf. [\ref{datapar.fixedsize.def}]).
\end{itemdescr}

\begin{itemdecl}
template <class T, class Abi = datapar_abi::compatible<T>>
struct datapar_size : public integral_constant<size_t, implementation_defined> {};
\end{itemdecl}
\begin{itemdescr}
  \pnum The \type{datapar_size} class template inherits from \type{integral_constant} with a value that equals \datapar{}\code{<T, Abi>::size()}.

  \pnum \code{datapar_size<T, Abi>::value} shall result in a substitution failure if any of the template arguments \type T or \type{Abi} are invalid template arguments to \datapar.
\end{itemdescr}

\begin{itemdecl}
template <class T, class U = typename T::value_type>
constexpr size_t memory_alignment = implementation_defined;
\end{itemdecl}
\begin{itemdescr}
  \pnum\requires The template parameter \type T must be a valid instantiation of either the \datapar or the \mask class template.
  \pnum\requires The template parameter \type U must be a type supported by the load and store functions for \type T.
  \pnum The value of \code{memory_alignment<T, U>} identifies the alignment restrictions on pointers used for (converting) loads and stores for the given type \type T on arrays of type \type U.
\end{itemdescr}

\wgSubsubsection{Class template \code{where_expression}}{datapar.whereexpr}
\pnum The class template \code{where_expression<M, T>} combines a predicate and a value object to implement an interface that restricts assignments and/or operations on the value object to the elements selected via the predicate.

\pnum The first template argument \type M must be cv-unqualified \bool or a cv-unqualified \mask instantiation.

\pnum The second template argument \type T must be a cv-unqualified or \const qualified type \type{T'}.
If \type M is \bool, \type{T'} must be an arithmetic type.
Otherwise, \type{T'} must either be \type M or \type{M::datapar_type}.

\begin{itemdecl}
const M &mask;                     // exposition only
T &data;                           // exposition only
where_expression(const M &, T &);  // exposition only
\end{itemdecl}
\begin{itemdescr}
  \pnum\effects
  The implementation initializes a \type{where_expression<M, T>} object with a predicate of type \type M and a reference to a value object of type \type T.

  \pnum\realnote The predicate object may be copied by the constructor implementation.
  If \type T is const qualified the constructor may copy the value object.

  \pnum\realnote
  The following declarations refer to the predicate as data member \code{mask} and to the value reference as data member \code{data}.
\end{itemdescr}

\begin{itemdecl}
template <class U> void operator=(U &&x);
template <class U> void operator+=(U &&x);
template <class U> void operator-=(U &&x);
template <class U> void operator*=(U &&x);
template <class U> void operator/=(U &&x);
template <class U> void operator%=(U &&x);
template <class U> void operator&=(U &&x);
template <class U> void operator|=(U &&x);
template <class U> void operator^=(U &&x);
template <class U> void operator<<=(U &&x);
template <class U> void operator>>=(U &&x);
\end{itemdecl}
\begin{itemdescr}
  \pnum\remarks Each of these operators only participate in overload resolution if the indicated operator can be applied to objects of type \type T.
  \pnum\effects
  If \type M is \bool, applies the indicated operator on \code{data} and \code{forward<U>(x)} unless \code{mask} is \false.
  If \type M is not \bool, applies the indicated operator on \code{data} and \code{forward<U>(x)} without modifying the elements \code{data[i]} where \code{mask[i]} is \false \foralli[M::].
  \pnum\remarks It is unspecified whether the arithmetic/bitwise operation, which is implied by a compound assignment operator, is executed on all elements or only on the ones written back.
\end{itemdescr}

\begin{itemdecl}
void operator++();
void operator++(int);
void operator--();
void operator--(int);
\end{itemdecl}
\begin{itemdescr}
  \pnum\remarks Each of these operators only participate in overload resolution if the indicated operator can be applied to objects of type \type T.
  \pnum\effects
  If \type M is \bool, applies the indicated operator on \code{data} unless \code{mask} is \false.
  If \type M is not \bool, applies the indicated operator on \code{data} without modifying the elements \code{data[i]} where \code{mask[i]} is \false \foralli[M::].
  \pnum\realnote It is unspecified whether the inc-/decrement operation is executed on all elements or only on the ones written back.
\end{itemdescr}

\begin{itemdecl}
remove_const_t<T> operator-() const;
\end{itemdecl}
\begin{itemdescr}
  \pnum\remarks This operator only participates in overload resolution if the indicated operator can be applied to objects of type \type T.
  \pnum\returns If \type M is \bool, \code{-data} if \code{mask} is \true, \code{data} otherwise.
  If \type M is not \bool, returns an object with the $i$-th element initialized to \code{-data[i]} if \code{mask[i]} is \true and \code{data[i]} otherwise \foralli[M::].
\end{itemdescr}

\begin{itemdecl}
template <class U, class Flags>
[[nodiscard]] remove_const_t<T> memload(const U *mem, Flags) const;
\end{itemdecl}
\begin{itemdescr}
  \pnum\remarks If \type T is \bool or a \mask instantiation, the function only participates in overload resolution if \type U is \bool.
  \pnum\returns If \type M is \bool, return \code{mem[0]} if \code{mask} equals \true and return \code{data} otherwise.
  If \type M is not \bool, return an object with the $i$-th element initialized to the $i$-th element of \code{data} if \code{mask[i]} is \false and \code{static_cast<T::value_type>(mem[i])} if \code{mask[i]} is \true \foralli[M::].

  \pnum\requires If \type M is not \bool, the largest $i$ where \code{mask[i]} is \true is less than the number of values pointed to by \code{mem}.
  \flagsRemarks{\type T, \type U}
\end{itemdescr}

\begin{itemdecl}
template <class U, class Flags> void memload(const U *mem, Flags);
\end{itemdecl}
\begin{itemdescr}
  \pnum\remarks If \type T is \bool or a \mask instantiation, the function only participates in overload resolution if \type U is \bool.
  \pnum\effects If \type M is \bool, assign \code{mem[0]} to \code{data} unless \code{mask} is \false.
  If \type M is not \bool, replace the elements of \code{data} where \code{mask[i]} is \true such that the $i$-th element is assigned with \code{static_cast<T::value_type>(mem[i])} \foralli[M::].

  \pnum\requires If \type M is not \bool, the largest $i$ where \code{mask[i]} is \true is less than the number of values pointed to by \code{mem}.
  \flagsRemarks{\type T, \type U}
\end{itemdescr}

\begin{itemdecl}
template <class U, class Flags> void memstore(U *mem, Flags) const;
\end{itemdecl}
\begin{itemdescr}
  \pnum\effects If \type M is \bool, assign \code{data} to \code{mem[0]} unless \code{mask} is \false.
  \pnum\remarks If \type T is a (const qualified) \mask instantiation, the function only participates in overload resolution if \type U is \bool.
  If \type M is not \bool, copies the elements \code{data[i]} where \code{mask[i]} is \true as if \code{mem[i] = static_cast<U>(data[i])} \foralli[M::].

  \pnum\requires If \type M is not \bool, the largest $i$ where \code{mask[i]} is \true is less than the number of values pointed to by \code{mem}.
  \flagsRemarks{\type remove_const_t<T>, \type U}
\end{itemdescr}

% vim: tw=0

  \wgSubsection{Class template \type{simd}}{simd.class}
\wgSubsubsection{Class template \simd overview}{simd.overview}
\lstinputlisting[]{simd.cpp}

\pnum The class template \simd{}\type{<T, Abi>} is a one-dimensional smart array.
The number of elements in the array is a constant expression, according to the \type{Abi} template parameter.

\newcommand\simdTypeRequirements[1]{
\pnum\label{#1.type requirements}\label{#1.deleted}%
The resulting class shall be a complete type with deleted default constructor, deleted destructor, deleted copy constructor, and deleted copy assignment unless all of the following hold:
\begin{itemize}
  \item The first template argument \type T is a cv-unqualified integral or floating-point type except \bool (3.9.1 [basic.fundamental]).
  \item The second template argument \type{Abi} is an ABI tag so that \code{is_abi_tag_v<Abi>} is \true.
  \item The \type{Abi} type is a supported ABI tag.
    It is supported if
    \begin{itemize}
      \item \type{Abi} is \type{simd_abi::scalar}, or
      \item \type{Abi} is \fixedsizeN with \code N $\le 32$ or implementation-defined additional valid values for \code N (see \ref{simd.fixedsize.def}).
    \end{itemize}
    It is implementation-defined whether a given combination of \type T and an implementation-defined ABI tag is supported.
    \wgNote{The intent is for implementations to decide on the basis of the \currentTarget.}
\end{itemize}
}
\simdTypeRequirements{simd}

\wgExample{
  Consider an implementation that defines the implementation-defined ABI tags \type{simd_x} and \type{gpu_y}.
  When the compiler is invoked to translate to a machine that has support for the \type{simd_x} ABI tag for all arithmetic types except \type{long double} and no support for the \type{gpu_y} ABI tag, then:
  \begin{itemize}
    \item \simd[<T, simd_abi::gpu_y>] is not supported for any \type T and results in a type with deleted constructor
    \item \simd[<long double, simd_abi::simd_x>] is not supported and results in a type with deleted constructor
    \item \simd[<double, simd_abi::simd_x>] is supported
    \item \simd[<long double, simd_abi::scalar>] is supported
  \end{itemize}
}

\pnum Default initialization performs no initialization of the elements; value-initialization initializes each element with \code{T()}.
\wgNote{Thus, default initialization leaves the elements in an indeterminate state.}

\pnum The member type \referencetype is an implementation-defined type acting as a reference to an element of type \valuetype with the following properties:
\label{sec:reference type}
\begin{itemize}
  \item The type has a deleted default constructor, copy constructor, and copy assignment operator.

  \item Assignment, compound assignment, increment, and decrement operators shall not participate in overload resolution unless the \referencetype object is an rvalue and the corresponding operator of type \valuetype is usable.

  \item Objects of type \referencetype are implicitly convertible to \valuetype.

  \item If a binary operator is applied to an object of type \referencetype, the operator is only applied after converting the \referencetype object to \valuetype.

  \item Calls to \code{swap(\referencetype \&\&, \valuetype \&)} and \code{swap(\valuetype \&, \referencetype \&\&)} exchange the values referred to by the \referencetype object and the \valuetype reference.
  Calls to \code{swap(\referencetype \&\&, \referencetype \&\&)} exchange the values referred to by the \referencetype objects.
\end{itemize}

\begin{itemdecl}
static constexpr size_type size() noexcept;
\end{itemdecl}
\begin{itemdescr}
  \pnum\returns the number of elements stored in objects of the given \simd[<T, Abi>] type.
\end{itemdescr}

\pnum\begin{noteEnv} Implementations are encouraged to enable \code{static_cast}ing from/to (an) implementation-defined SIMD type(s).
This would add one or more of the following declarations to class \simd:
\begin{itemdecl}
explicit operator implementation-defined() const;
explicit simd(const implementation-defined& init);
\end{itemdecl}
\end{noteEnv}

\wgSubsubsection{\simd constructors}{simd.ctor}
\begin{itemdecl}
template <class U> simd(U&&);
\end{itemdecl}
\begin{itemdescr}
  \pnum\remarks This constructor shall not participate in overload resolution unless either:
  \comment[Q]{Mention forwarding on conversion to \valuetype?}%
  \comment[Q]{\type U is cv- and ref-qualified, is the wording below OK?}
  \begin{itemize}
    \item \type U is an \realArithmeticType and every possible value of type \type U can be represented with type \valuetype,
    \item or \type U is not an arithmetic type and is implicitly convertible to \valuetype,
    \item or \type U is \intt,
    \item or \type U is \uint and \valuetype is an unsigned integral type.
  \end{itemize}
  \pnum\effects Constructs an object with each element initialized to the value of the argument after conversion to \valuetype.

  \pnum\throws Any exception thrown while converting the argument to \valuetype.
\end{itemdescr}

\begin{itemdecl}
template <class U> simd(const simd<U, simd_abi::fixed_size<size()>>& x);
\end{itemdecl}
\begin{itemdescr}
  \pnum\remarks This constructor shall not participate in overload resolution unless
  \begin{itemize}
    \item \type{abi_type} equals \fixedsizescoped{}\code{<size()>},
    \item and every possible value of \type U can be represented with type \valuetype,
    \item and, if both \type U and \valuetype are integral, the integer conversion rank \parencite[(4.15)]{N4618} of \valuetype is greater than the integer conversion rank of \type U.
  \end{itemize}
  \pnum\effects Constructs an object where the $i$-th element equals \code{static_cast<T>(x[i])} \foralli.
\end{itemdescr}

\begin{itemdecl}
template <class G> simd(G&& gen);
\end{itemdecl}
\begin{itemdescr}
  \pnum\remarks This constructor shall not participate in overload resolution unless \code{simd(gen(integral_constant<size_t, 0>()))} is well-formed.%
  \comment{
    To be 100\% correct this needs \code{<size_t, i>} \foralli.
  }
  \pnum\effects Constructs an object where the $i$-th element is initialized to \code{gen(integral_constant<size_t, i>())}.
  \pnum\remarks The order of calls to \code{gen} is unspecified.
\end{itemdescr}

\begin{itemdecl}
template <class U, class Flags> simd(const U *mem, Flags);
\end{itemdecl}
\begin{itemdescr}
  \pnum\remarks This constructor shall not participate in overload resolution unless \type U is an \realArithmeticType.
  \pnum\effects Constructs an object where the $i$-th element is initialized to \code{static_cast<T>(mem[i])} \foralli.
  \pnum\requires \code{size()} is less than or equal to the number of values pointed to by \code{mem}.
  \flagsRemarks{\simd, \type U}
\end{itemdescr}

\wgSubsubsection{\simd load function}{simd.load}
\begin{itemdecl}
template <class U, class Flags> void copy_from(const U *mem, Flags);
\end{itemdecl}
\begin{itemdescr}
  \pnum\remarks This function shall not participate in overload resolution unless \type U is an \realArithmeticType.
  \pnum\effects Replaces the elements of the \simd object such that the $i$-th element is assigned with \code{static_cast<T>(mem[i])} \foralli.
  \pnum\requires \code{size()} is less than or equal to the number of values pointed to by \code{mem}.
  \flagsRemarks{\simd, \type U}
\end{itemdescr}

\wgSubsubsection{\simd store function}{simd.store}
\begin{itemdecl}
template <class U, class Flags> void copy_to(U *mem, Flags);
\end{itemdecl}
\begin{itemdescr}
  \pnum\remarks This function shall not participate in overload resolution unless \type U is an \realArithmeticType.
  \pnum\effects Copies all \simd elements as if \code{mem[i] = static_cast<U>(operator[](i))} \foralli.
  \pnum\requires \code{size()} is less than or equal to the number of values pointed to by \code{mem}.
  \flagsRemarks{\simd, \type U}
\end{itemdescr}

\wgSubsubsection{\simd subscript operators}{simd.subscr}
\newcommand\simdElementReference[1]{
  \pnum\requires The value of \code{i} is less than \code{size()}.

  \pnum\returns A temporary object of type \referencetype (see \ref{sec:reference type}) with the following effects:

  \pnum\effects The assignment, compound assignment, increment, and decrement operators of \referencetype execute the indicated operation on the $i$-th element of the #1 object.

  \pnum\effects Conversion to \valuetype returns a copy of the $i$-th element.

  \pnum\throws Nothing.
}
\begin{itemdecl}
reference operator[](size_type i);
\end{itemdecl}
\begin{itemdescr}
  \simdElementReference{\simd{}}
\end{itemdescr}

\begin{itemdecl}
value_type operator[](size_type i) const;
\end{itemdecl}
\begin{itemdescr}
  \pnum\requires The value of \code{i} is less than \code{size()}.

  \pnum\returns A copy of the $i$-th element.

  \pnum\throws Nothing.
\end{itemdescr}

\wgSubsubsection{\simd unary operators}{simd.unary}
\begin{itemdecl}
simd& operator++();
\end{itemdecl}
\begin{itemdescr}
  \pnum\effects Increments every element of \code{*this} by one.
  \pnum\returns An lvalue reference to \code{*this} after incrementing.
  \pnum\remarks Overflow semantics follow the same semantics as for \type T.

  \pnum\throws Nothing.
\end{itemdescr}

\begin{itemdecl}
simd operator++(int);
\end{itemdecl}
\begin{itemdescr}
  \pnum\effects Increments every element of \code{*this} by one.
  \pnum\returns A copy of \code{*this} before incrementing.
  \pnum\remarks Overflow semantics follow the same semantics as for \type T.

  \pnum\throws Nothing.
\end{itemdescr}

\begin{itemdecl}
simd& operator--();
\end{itemdecl}
\begin{itemdescr}
  \pnum\effects Decrements every element of \code{*this} by one.
  \pnum\returns An lvalue reference to \code{*this} after decrementing.
  \pnum\remarks Underflow semantics follow the same semantics as for \type T.

  \pnum\throws Nothing.
\end{itemdescr}

\begin{itemdecl}
simd operator--(int);
\end{itemdecl}
\begin{itemdescr}
  \pnum\effects Decrements every element of \code{*this} by one.
  \pnum\returns A copy of \code{*this} before decrementing.
  \pnum\remarks Underflow semantics follow the same semantics as for \type T.

  \pnum\throws Nothing.
\end{itemdescr}

\begin{itemdecl}
mask_type operator!() const;
\end{itemdecl}
\begin{itemdescr}
  \pnum\returns A \mask object with the $i$-th element set to \code{!operator[](i)} \foralli.

  \pnum\throws Nothing.
\end{itemdescr}

\begin{itemdecl}
simd operator~() const;
\end{itemdecl}
\begin{itemdescr}
  \pnum\returns A \simd object where each bit is the inverse of the corresponding bit in \code{*this}.
  \pnum\remarks \simd{}\code{::operator\textasciitilde{}()} shall not participate in overload resolution unless \type T is an integral type.

  \pnum\throws Nothing.
\end{itemdescr}

\begin{itemdecl}
simd operator+() const;
\end{itemdecl}
\begin{itemdescr}
  \pnum \returns A copy of \code{*this}

  \pnum\throws Nothing.
\end{itemdescr}

\begin{itemdecl}
simd operator-() const;
\end{itemdecl}
\begin{itemdescr}
  \pnum\returns A \simd object where the $i$-th element is initialized to \code{-operator[](i)} \foralli.

  \pnum\throws Nothing.
\end{itemdescr}

\wgSubsection{\type{simd} non-member operations}{simd.nonmembers}

\wgSubsubsection{\simd binary operators}{simd.binary}
\begin{itemdecl}
friend simd operator+ (const simd&, const simd&);
friend simd operator- (const simd&, const simd&);
friend simd operator* (const simd&, const simd&);
friend simd operator/ (const simd&, const simd&);
friend simd operator% (const simd&, const simd&);
friend simd operator& (const simd&, const simd&);
friend simd operator| (const simd&, const simd&);
friend simd operator^ (const simd&, const simd&);
friend simd operator<<(const simd&, const simd&);
friend simd operator>>(const simd&, const simd&);
\end{itemdecl}
\begin{itemdescr}
  \pnum\remarks Each of these operators shall not participate in overload resolution unless the indicated operator can be applied to objects of type \type{value_type}.

  \pnum\returns A \simd object initialized with the results of the component-wise application of the indicated operator.

  \pnum\throws Nothing.
\end{itemdescr}

\begin{itemdecl}
friend simd operator<<(const simd& v, int n);
friend simd operator>>(const simd& v, int n);
\end{itemdecl}
\begin{itemdescr}
  \pnum\remarks Both operators shall not participate in overload resolution unless \valuetype is an unsigned integral type.

  \pnum\returns A \simd object where the $i$-th element is initialized to the result of applying the indicated operator to \code{v[i]} and \code n \foralli.

  \pnum\throws Nothing.
\end{itemdescr}

\wgSubsubsection{\simd compound assignment}{simd.cassign}
\begin{itemdecl}
friend simd& operator+= (simd&, const simd&);
friend simd& operator-= (simd&, const simd&);
friend simd& operator*= (simd&, const simd&);
friend simd& operator/= (simd&, const simd&);
friend simd& operator%= (simd&, const simd&);
friend simd& operator&= (simd&, const simd&);
friend simd& operator|= (simd&, const simd&);
friend simd& operator^= (simd&, const simd&);
friend simd& operator<<=(simd&, const simd&);
friend simd& operator>>=(simd&, const simd&);
\end{itemdecl}
\begin{itemdescr}
  \pnum\remarks Each of these operators shall not participate in overload resolution unless the indicated operator can be applied to objects of type \type{value_type}.
  \pnum\effects Each of these operators performs the indicated operator component-wise on each of the corresponding elements of the arguments.
  \pnum\returns A reference to the first argument.

  \pnum\throws Nothing.
\end{itemdescr}

\begin{itemdecl}
friend simd& operator<<=(simd& v, int n);
friend simd& operator>>=(simd& v, int n);
\end{itemdecl}
\begin{itemdescr}
  \pnum\remarks Both operators shall not participate in overload resolution unless \valuetype is an unsigned integral type.
  \pnum\effects Performs the indicated shift by \code n operation on the $i$-th element of \code v \foralli.
  \pnum\returns A reference to the first argument.

  \pnum\throws Nothing.
\end{itemdescr}

\wgSubsubsection{\simd compare operators}{simd.comparison}
\begin{itemdecl}
friend mask_type operator==(const simd&, const simd&);
friend mask_type operator!=(const simd&, const simd&);
friend mask_type operator>=(const simd&, const simd&);
friend mask_type operator<=(const simd&, const simd&);
friend mask_type operator> (const simd&, const simd&);
friend mask_type operator< (const simd&, const simd&);
\end{itemdecl}
\begin{itemdescr}
  \pnum\returns A \mask object initialized with the results of the component-wise application of the indicated operator.

  \pnum\throws Nothing.
\end{itemdescr}

\wgSubsubsection{\simd reductions}{simd.reductions}
\begin{itemdecl}
template <class T, class Abi, class BinaryOperation = std::plus<>>
T reduce(const simd<T, Abi>& x, BinaryOperation binary_op = BinaryOperation());
\end{itemdecl}
\begin{itemdescr}
  \pnum\returns \code{\textit{GENERALIZED_SUM}(binary_op, x.data[i], \ldots)} \foralli.
  \pnum\requires \code{binary_op} shall be callable with two arguments of type \type T or two arguments of type \simd[<T, A1>], where \type{A1} may be different to \type{Abi}.
  \pnum\wgNote{This overload of \code{reduce} does not require an initial value because \code x is guaranteed to be non-empty.}
\end{itemdescr}

\begin{itemdecl}
template <class M, class V, class BinaryOperation = std::plus<>>
typename V::value_type reduce(const const_where_expression<M, V>& x,
                              typename V::value_type neutral_element = implementation-defined,
                              BinaryOperation binary_op = BinaryOperation());
\end{itemdecl}
\begin{itemdescr}
  \pnum\returns
  If \code{none_of(x.mask)}, returns \code{neutral_element}.
  Otherwise, returns \code{\textit{GENERALIZED_SUM}(binary_op, x.data[i], \ldots)} \forallmaskedi{x.mask}.

  \pnum\requires \code{binary_op} shall be callable with two arguments of type \type T or two arguments of type \simd[<T, A1>], where \type{A1} may be different to \type{Abi}.

  \pnum\requires The value of \code{neutral_element} is default-initialized when \code{BinaryOperation} is one of \code{plus<U>}, \code{multiplies<U>}, \code{bit_and<U>}, \code{bit_or<U>}, \code{bit_xor<U>}, with arbitrary \type U.
  Otherwise, default initialization shall result in a substitution failure.

  \pnum\wgNote{This overload of \code{reduce} requires a neutral value to enable a parallelized implementation:
  A temporary \simd object initialized with \code{neutral_element} is conditionally assigned from \code{x.data} using \code{x.mask}.
  Subsequently, the parallelized reduction (without mask) is applied to the temporary object.}
\end{itemdescr}

\begin{itemdecl}
template <class T, class Abi> T hmin(const simd<T, Abi>& x);
\end{itemdecl}
\begin{itemdescr}
  \pnum\returns The value of an element \code{x[j]} for which \code{x[j] <= x[i]} \foralli.

  \pnum\throws Nothing.
\end{itemdescr}

\begin{itemdecl}
template <class M, class V> T hmin(const const_where_expression<M, V>& x);
\end{itemdecl}
\begin{itemdescr}
  \pnum\returns If \code{none_of(x.mask)}, the return value is \code{numeric_limits<V::value_type>::max()}.
  Otherwise, returns the value of an element \code{x.data[j]} for which \code{x.mask[j] == true} and \code{x.data[j] <= x.data[i]} \foralli.

  \pnum\throws Nothing.
\end{itemdescr}

\begin{itemdecl}
template <class T, class Abi> T hmax(const simd<T, Abi>& x);
\end{itemdecl}
\begin{itemdescr}
  \pnum\returns The value of an element \code{x[j]} for which \code{x[j] >= x[i]} \foralli.

  \pnum\throws Nothing.
\end{itemdescr}

\begin{itemdecl}
template <class M, class V> T hmax(const const_where_expression<M, V>& x);
\end{itemdecl}
\begin{itemdescr}
  \pnum\returns If \code{none_of(x.mask)}, the return value is \code{numeric_limits<V::value_type>::min()}.
  Otherwise, returns the value of an element \code{x.data[j]} for which \code{x.mask[j] == true} and \code{x.data[j] >= x.data[i]} \foralli.

  \pnum\throws Nothing.
\end{itemdescr}


\wgSubsubsection{\simd casts}{simd.casts}
\begin{itemdecl}
  template <class T, class U, class Abi> @\emph{see below}@ simd_cast(const simd<U, Abi>& x);
\end{itemdecl}
\begin{itemdescr}
  \pnum\remarks Let \type{To} identify \type{T::\valuetype} if \code{is_simd_v<T>} or \type T otherwise.

  \pnum\remarks The function shall not participate in overload resolution unless
  \begin{itemize}
    \item every possible value of type \type U can be represented with type \type{To}, and
    \item either \code{!is_simd_v<T>} or \code{T::size()} is equal to \code{simd<U, Abi>::size()}.
  \end{itemize}

  \pnum\remarks If \code{is_simd_v<T>}, the return type is \type T.
  Otherwise, if either \type U and \type T are equal or \type U and \type T are integral types that only differ in signedness, the return type is \simd[<T, Abi>].
  Otherwise, the return type is \simd[<T, \fixedsizescoped{}<\simd<U, Abi>::size()>>].

  \pnum\returns A \simd object with the $i$-th element initialized to \code{static_cast<To>(x[i])}.

  \pnum\throws Nothing.
\end{itemdescr}

\begin{itemdecl}
template <class T, class U, class Abi> @\emph{see below}@ static_simd_cast(const simd<U, Abi>& x);
\end{itemdecl}
\begin{itemdescr}
  \pnum\remarks Let \type{To} identify \type{T::\valuetype} if \code{is_simd_v<T>} or \type T otherwise.

  \pnum\remarks The function shall not participate in overload resolution unless either \code{!is_simd_v<T>} or \code{T::size()} is equal to \code{simd<U, Abi>::size()}.

  \pnum\remarks If \code{is_simd_v<T>}, the return type is \type T.
  Otherwise, if either \type U and \type T are equal or \type U and \type T are integral types that only differ in signedness, the return type is \simd[<T, Abi>].
  Otherwise, the return type is \simd[<T, \fixedsizescoped{}<\simd<U, Abi>::size()>>].

  \pnum\returns A \simd object with the $i$-th element initialized to \code{static_cast<To>(x[i])}.

  \pnum\throws Nothing.
\end{itemdescr}

\begin{itemdecl}
template <class T, class Abi>
fixed_size_simd<T, simd_size_v<T, Abi>> to_fixed_size(const simd<T, Abi>& x) noexcept;
template <class T, class Abi>
fixed_size_simd_mask<T, simd_size_v<T, Abi>> to_fixed_size(const simd_mask<T, Abi>& x) noexcept;
\end{itemdecl}
\begin{itemdescr}
  \pnum\returns An object with the $i$-th element initialized to \code{x[i]}.
\end{itemdescr}

\begin{itemdecl}
template <class T, size_t N> native_simd<T> to_native(const fixed_size_simd<T, N>& x) noexcept;
template <class T, size_t N> native_simd_mask<T> to_native(const fixed_size_simd_mask<T, N>> &x) noexcept;
\end{itemdecl}
\begin{itemdescr}
  \pnum\remarks These functions shall not participate in overload resolution unless \code{simd_size_v<T, simd_abi::native<T>>} is equal to \code N.
  \pnum\returns An object with the $i$-th element initialized to \code{x[i]}.
\end{itemdescr}

\begin{itemdecl}
template <class T, size_t N> simd<T> to_compatible(const fixed_size_simd<T, N>& x) noexcept;
template <class T, size_t N> simd_mask<T> to_compatible(const fixed_size_simd_mask<T, N>& x) noexcept;
\end{itemdecl}
\begin{itemdescr}
  \pnum\remarks These functions shall not participate in overload resolution unless \code{simd_size_v<T, simd_abi::compatible<T>>} is equal to \code N.
  \pnum\returns An object with the $i$-th element initialized to \code{x[i]}.
\end{itemdescr}

\begin{itemdecl}
template <size_t... Sizes, class T, class Abi>
tuple<simd<T, abi_for_size_t<Sizes>>...> split(const simd<T, Abi>& x);
template <size_t... Sizes, class T, class Abi>
tuple<simd_mask<T, abi_for_size_t<Sizes>>...> split(const simd_mask<T, Abi>& x);
\end{itemdecl}
\begin{itemdescr}
  \pnum\remarks These functions shall not participate in overload resolution unless the sum of all values in the \code{Sizes} pack is equal to \code{simd_size_v<T, Abi>}.
  \pnum\returns A \type{tuple} of \simd/\mask objects with the $i$-th \simd/\mask element of the $j$-th \type{tuple} element initialized to the value of the element in \code x with index $i$ + partial sum of the first $j$ values in the \code{Sizes} pack.
\end{itemdescr}

\begin{itemdecl}
template <class V, class Abi>
array<V, simd_size_v<typename V::value_type, Abi> / V::size()> split(
    const simd<typename V::value_type, Abi>&);
template <class V, class Abi>
array<V, simd_size_v<typename V::value_type, Abi> / V::size()> split(
    const simd_mask<typename V::value_type, Abi>&);
\end{itemdecl}
\begin{itemdescr}
  \pnum\remarks These functions shall not participate in overload resolution unless
  \begin{itemize}
    \item \code{is_simd_v<V>} for the first signature / \code{is_mask_v<V>} for the second signature,
    \item and \code{simd_size_v<typename V::value_type, Abi>} is an integral multiple of \code{V::size()}.
  \end{itemize}

  \pnum\returns An \type{array} of \simd/\mask objects with the $i$-th \simd/\mask element of the $j$-th \type{array} element initialized to the value of the element in \code x with index $i + j \cdot $\code{V::size()}.
\end{itemdescr}

\begin{itemdecl}
template <class T, class... Abis>
simd<T, abi_for_size_t<T, (simd_size_v<T, Abis> + ...)>> concat(const simd<T, Abis>&... xs);
template <class T, class... Abis>
simd_mask<T, abi_for_size_t<T, (simd_size_v<T, Abis> + ...)>> concat(const simd_mask<T, Abis>&... xs);
\end{itemdecl}
\begin{itemdescr}
  \pnum\returns A \simd/\mask object initialized with the concatenated values in the \code{xs} pack of \simd/\mask objects.
  The $i$-th \simd/\mask element of the $j$-th parameter in the \code{xs} pack is copied to the return value's element with index $i$ + partial sum of the \code{size()} of the first $j$ parameters in the \code{xs} pack.
\end{itemdescr}

\wgSubsubsection{\simd algorithms}{simd.alg}
\begin{itemdecl}
template <class T, class Abi> simd<T, Abi> min(const simd<T, Abi>& a, const simd<T, Abi>& b) noexcept;
\end{itemdecl}
\begin{itemdescr}
  \pnum\returns An object with the $i$-th element initialized with the smaller value of \code{a[i]} and \code{b[i]} \foralli.
\end{itemdescr}

\begin{itemdecl}
template <class T, class Abi> simd<T, Abi> max(const simd<T, Abi>&, const simd<T, Abi>&) noexcept;
\end{itemdecl}
\begin{itemdescr}
  \pnum\returns An object with the $i$-th element initialized with the larger value of \code{a[i]} and \code{b[i]} \foralli.
\end{itemdescr}

\begin{itemdecl}
template <class T, class Abi>
std::pair<simd<T, Abi>, simd<T, Abi>> minmax(const simd<T, Abi>&, const simd<T, Abi>&) noexcept;
\end{itemdecl}
\begin{itemdescr}
  \pnum\returns An object with the $i$-th element in the first \type{pair} member initialized with the smaller value of \code{a[i]} and \code{b[i]} \foralli.
  The $i$-th element in the second \type{pair} member is initialized with the larger value of \code{a[i]} and \code{b[i]} \foralli.
\end{itemdescr}

\begin{itemdecl}
template <class T, class Abi>
simd<T, Abi> clamp(const simd<T, Abi>& v, const simd<T, Abi>& lo, const simd<T, Abi>& hi);
\end{itemdecl}
\begin{itemdescr}
  \pnum\requires No element in \code{lo} shall be greater than the corresponding element in \code{hi}.
  \pnum\returns An object with the $i$-th element initialized with \code{lo[i]} if \code{v[i]} is smaller than \code{lo[i]}, \code{hi[i]} if \code{v[i]} is larger than \code{hi[i]}, otherwise \code{v[i]} \foralli.
\end{itemdescr}

\wgSubsubsection{\simd math library}{simd.math}
\lstinputlisting[]{math.cpp}

\pnum Each listed function concurrently applies the indicated mathematical function component-wise.
The results per component are not required to be binary equal to the application of the function which is overloaded for the element type.
\comment{Neither the C nor the \CC{} standard say anything about expected error/precision.
It seems returning 0 from all functions is a conforming implementation --- just bad QoI.}
\wgNote{
  If a precondition of the indicated mathematical function is violated, the behavior is undefined.
}

\pnum If \code{abs()} is called with an argument of type \simd[<X, Abi>] for which \code{is_unsigned<X>::value} is \true, the program is ill-formed.

% vim: tw=0 spell sw=2

  \wgSubsection{Class template \type{mask}}{mask}
\wgSubsubsection{Class template \mask overview}{mask.overview}
\lstinputlisting[]{mask.cpp}

\pnum The class template \mask[<T, Abi>] is a one-dimensional smart array of booleans.
The number of elements in the array is a constant expression, equal to the number of elements in \datapar[<T, Abi>].

\dataparTypeRequirements{mask}

\dataparDeleted{mask}

\pnum Default initialization performs no initialization of the elements; value-initialization initializes each element with \code{bool()}.
\wgNote{Thus, default initialization leaves the elements in an indeterminate state.}

\begin{itemdecl}
static constexpr size_type size();
\end{itemdecl}
\begin{itemdescr}
  \pnum\returns the number of boolean elements stored in objects of the given \mask[<T, Abi>] type.
\end{itemdescr}

\pnum\begin{noteEnv}Implementations are encouraged to enable \code{static_cast}ing from/to (an) implementation-defined SIMD mask type(s).
This would add one or more of the following declarations to class \mask:
\begin{itemdecl}
explicit operator implementation-defined() const;
explicit datapar(const implementation-defined& init);
\end{itemdecl}
\end{noteEnv}

\wgSubsubsection{\mask constructors}{mask.ctor}
\begin{itemdecl}
explicit mask(value_type);
\end{itemdecl}
\begin{itemdescr}
  \pnum\effects Constructs an object with each element initialized to the value of the argument.
\end{itemdescr}

\begin{itemdecl}
template <class U> mask(const mask<U, datapar_abi::fixed_size<size()>>& x);
\end{itemdecl}
\begin{itemdescr}
  \pnum\remarks This constructor shall not participate in overload resolution unless
    \type{abi_type} equals \fixedsizescoped{}\code{<size()>}.
  \pnum\effects Constructs an object of type \mask where the $i$-th element equals \code{x[i]} \foralli.
\end{itemdescr}

\begin{itemdecl}
template <class Flags> mask(const value_type *mem, Flags);
\end{itemdecl}
\begin{itemdescr}
  \pnum\effects Constructs an object where the $i$-th element is initialized to \code{mem[i]} \foralli.
  \pnum\requires \code{size()} is less than or equal to the number of values pointed to by \code{mem}.
  \flagsRemarks{\mask{}}
\end{itemdescr}

\wgSubsubsection{\mask load function}{mask.load}
\begin{itemdecl}
template <class Flags> void memload(const value_type *mem, Flags);
\end{itemdecl}
\begin{itemdescr}
  \pnum\effects Replaces the elements of the \mask object such that the $i$-th element is assigned with \code{mem[i]} \foralli.
  \pnum\requires \code{size()} is less than or equal to the number of values pointed to by \code{mem}.
  \flagsRemarks{\mask{}}
\end{itemdescr}

\wgSubsubsection{\mask store function}{mask.store}
\begin{itemdecl}
template <class Flags> void memstore(value_type *mem, Flags);
\end{itemdecl}
\begin{itemdescr}
  \pnum\effects Copies all \mask elements as if \code{mem[i] = operator[](i)} \foralli.
  \pnum\requires \code{size()} is less than or equal to the number of values pointed to by \code{mem}.
  \flagsRemarks{\mask{}}
\end{itemdescr}

\wgSubsubsection{\mask{} subscript operators}{mask.subscr}
\begin{itemdecl}
reference operator[](size_type i);
\end{itemdecl}
\begin{itemdescr}
  \dataparElementReference{\mask{}}
\end{itemdescr}

\begin{itemdecl}
value_type operator[](size_type i) const;
\end{itemdecl}
\begin{itemdescr}
  \pnum\requires The value of \code{i} is less than \code{size()}.

  \pnum\returns A copy of the $i$-th element.
\end{itemdescr}

\wgSubsubsection{\mask unary operators}{mask.unary}
\begin{itemdecl}
mask operator!() const;
\end{itemdecl}
\begin{itemdescr}
  \pnum\returns A mask object with the $i$-th element set to the logical negation \foralli.
\end{itemdescr}

\wgSubsection{\type{mask} non-member operations}{mask.nonmembers}

\wgSubsubsection{\mask binary operators}{mask.binary}
\begin{itemdecl}
friend mask operator&&(const mask&, const mask&);
friend mask operator||(const mask&, const mask&);
friend mask operator& (const mask&, const mask&);
friend mask operator| (const mask&, const mask&);
friend mask operator^ (const mask&, const mask&);
\end{itemdecl}
\begin{itemdescr}
  \pnum\returns A \mask object initialized with the results of the component-wise application of the indicated operator.
\end{itemdescr}

\wgSubsubsection{\mask compound assignment}{mask.cassign}
\begin{itemdecl}
friend mask& operator&=(mask&, const mask&);
friend mask& operator|=(mask&, const mask&);
friend mask& operator^=(mask&, const mask&);
\end{itemdecl}
\begin{itemdescr}
  \pnum\effects Each of these operators performs the indicated operator component-wise on each of the corresponding elements of the arguments.
  \pnum\returns A reference to the first argument.
\end{itemdescr}

\wgSubsubsection{\mask compares}{mask.comparison}
\begin{itemdecl}
friend mask operator==(const mask&, const mask&);
friend mask operator!=(const mask&, const mask&);
\end{itemdecl}
\begin{itemdescr}
  \pnum\returns A \mask object initialized with the results of the component-wise application of the indicated operator.
\end{itemdescr}

\wgSubsubsection{\mask reductions}{mask.reductions}
\begin{itemdecl}
template <class T, class Abi> bool  all_of(mask<T, Abi>);
\end{itemdecl}
\begin{itemdescr}
  \pnum\returns \true if all boolean elements in the function argument equal \true, \false otherwise.
\end{itemdescr}

\begin{itemdecl}
template <class T, class Abi> bool  any_of(mask<T, Abi>);
\end{itemdecl}
\begin{itemdescr}
  \pnum\returns \true if at least one boolean element in the function argument equals \true, \false otherwise.
\end{itemdescr}

\begin{itemdecl}
template <class T, class Abi> bool none_of(mask<T, Abi>);
\end{itemdecl}
\begin{itemdescr}
  \pnum\returns \true if none of the boolean element in the function argument equals \true, \false otherwise.
\end{itemdescr}

\begin{itemdecl}
template <class T, class Abi> bool some_of(mask<T, Abi>);
\end{itemdecl}
\begin{itemdescr}
  \pnum\returns \true if at least one of the boolean elements in the function argument equals \true and at least one of the boolean elements in the function argument equals \false, \false otherwise.
\end{itemdescr}

\begin{itemdecl}
template <class T, class Abi> int popcount(mask<T, Abi>);
\end{itemdecl}
\begin{itemdescr}
  \pnum\returns The number of boolean elements that are \true.
\end{itemdescr}

\begin{itemdecl}
template <class T, class Abi> int find_first_set(mask<T, Abi> m);
\end{itemdecl}
\begin{itemdescr}
  \pnum\requires \code{any_of(m)} returns \true
  \pnum\returns The lowest element index \code i where \code{m[i] == true}.
\end{itemdescr}

\begin{itemdecl}
template <class T, class Abi> int find_last_set(mask<T, Abi> m);
\end{itemdecl}
\begin{itemdescr}
  \pnum\requires \code{any_of(m)} returns \true
  \pnum\returns The highest element index \code i where \code{m[i] == true}.
\end{itemdescr}

\begin{itemdecl}
template <class T, class Abi> bool  all_of(implementation-defined);
template <class T, class Abi> bool  any_of(implementation-defined);
template <class T, class Abi> bool none_of(implementation-defined);
template <class T, class Abi> bool some_of(implementation-defined);
template <class T, class Abi> int popcount(implementation-defined);
template <class T, class Abi> int find_first_set(implementation-defined);
template <class T, class Abi> int find_last_set(implementation-defined);
\end{itemdecl}
\begin{itemdescr}
  \pnum\remarks The functions shall not participate in overload resolution unless the argument is of type \bool.
  \pnum\returns \code{all_of} and \code{any_of} return their arguments; \code{none_of} returns the negation of its argument; \code{some_of} returns \false; \code{popcount} returns the integral representation of its argument; \code{find_first_set} and \code{find_last_set} return 0.
\end{itemdescr}

\wgSubsubsection{Masked assigment}{mask.where}
\begin{itemdecl}
template <class T, class A>
where_expression<mask<T, A>, datapar<T, A>> where(
    const typename datapar<T, A>::mask_type& k, datapar<T, A>& v);
template <class T, class A>
const where_expression<mask<T, A>, const datapar<T, A>> where(
    const typename datapar<T, A>::mask_type& k, const datapar<T, A>& v);

template <class T, class A>
where_expression<mask<T, A>, mask<T, A>> where(const remove_const_t<mask<T, A>>& k,
                                               mask<T, A>& v);
template <class T, class A>
const where_expression<mask<T, A>, const mask<T, A>> where(
    const remove_const_t<mask<T, A>>& k, const mask<T, A>& v);
\end{itemdecl}
\comment{\code{remove_const} is only used in place of a missing \code{template <class T> struct id \{ using type = T; \};} for inhibiting type deduction.}
\begin{itemdescr}
  \pnum\returns An object of type \type{where_expression} (see \ref{sec:datapar.whereexpr}) initialized with the predicate \code k and the value reference \code v.
\end{itemdescr}

\begin{itemdecl}
template <class T> where_expression<bool, T> where(implementation-defined k, T& v);
\end{itemdecl}
\begin{itemdescr}
  \pnum\remarks The function shall not participate in overload resolution unless
  \begin{itemize}
    \item \type T is neither a \datapar nor a \mask instantiation, and
    \item the first argument is of type \bool.
  \end{itemize}
  \pnum\returns An object of type \type{where_expression} (see \ref{sec:datapar.whereexpr}) initialized with the predicate \code k and the value reference \code v.
\end{itemdescr}

% vim: tw=0 spell sw=2

\end{wgText}

% vim: tw=0
