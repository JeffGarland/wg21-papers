\wgSubsection{Class template \type{mask}}{mask}
\wgSubsubsection{Class template \mask overview}{mask.overview}
\lstinputlisting[]{mask.cpp}

\pnum The class template \mask[<T, Abi>] is a one-dimensional smart array of booleans.
The number of elements in the array is determined at compile time, equal to the number of elements in \datapar[<T, Abi>].

\pnum The first template argument \type T must be a cv-unqualified integral or floating-point type (3.9.1 [basic.fundamental]).
The type \bool is not allowed.

\pnum The second template argument \type{Abi} must be a tag type from the \code{datapar_abi} namespace.

\pnum If any of the template arguments does not conform to the above requirements, the resulting class shall be a complete type with deleted default constructor, deleted destructor, deleted copy constructor, and deleted copy assignment.

\pnum Default initialization performs no initialization of the elements; value-initialization initializes each element with \code{bool()}.
\wgNote{Thus, default initialization leaves the elements in an indeterminate state.}

\begin{itemdecl}
static constexpr size_type size();
\end{itemdecl}
\begin{itemdescr}
  \pnum\returns the number of boolean elements stored in objects of the given \mask[<T, Abi>] type.
\end{itemdescr}

\pnum\realnote Implementations are encouraged to enable \code{static_cast}ing from/to (an) implementation-defined SIMD mask type(s).
This would add one or more of the following declarations to class \mask:
\begin{itemdecl}
explicit operator implementation_defined() const;
explicit datapar(const implementation_defined &init);
\end{itemdecl}

\wgSubsubsection{\mask constructors}{mask.ctor}
\begin{itemdecl}
explicit mask(value_type);
\end{itemdecl}
\begin{itemdescr}
  \pnum\effects Constructs an object with each element initialized to the value of the argument.
\end{itemdescr}

\begin{itemdecl}
template <class U> mask(const mask<U, datapar_abi::fixed_size<size()>> &x);
\end{itemdecl}
\begin{itemdescr}
  \pnum\remarks This constructor shall not participate in overload resolution unless
    \type{abi_type} equals \fixedsizescoped{}\code{<size()>}.
  \pnum\effects Constructs an object of type \mask where the $i$-th element equals \code{x[i]} \foralli.
\end{itemdescr}

\begin{itemdecl}
template <class Flags> mask(const value_type *mem, Flags);
\end{itemdecl}
\begin{itemdescr}
  \pnum\effects Constructs an object where the $i$-th element is initialized to \code{mem[i]} \foralli.
  \pnum\remarks If \code{size()} returns a value greater than the number of values pointed to by the first argument, the behavior is undefined.
  \flagsRemarks{\mask{}}
\end{itemdescr}

\begin{itemdecl}
template <class Flags> mask(const value_type *mem, mask k, Flags);
\end{itemdecl}
\begin{itemdescr}
  \pnum\effects Constructs an object where the $i$-th element is initialized to \code{k[i] ? mem[i] : false} \foralli.
  \pnum\remarks If the largest $i$ where \code{k[i]} is \true is greater than the number of values pointed to by the first argument, the behavior is undefined.
  \flagsRemarks{\mask{}}
\end{itemdescr}

\wgSubsubsection{\mask load function}{mask.load}
\begin{itemdecl}
template <class Flags> void memload(const value_type *mem, Flags);
\end{itemdecl}
\begin{itemdescr}
  \pnum\effects Replaces the elements of the \mask object such that the $i$-th element is assigned with \code{mem[i]} \foralli.
  \pnum\remarks If \code{size()} returns a value greater than the number of values pointed to by the first argument, the behavior is undefined.
  \flagsRemarks{\mask{}}
\end{itemdescr}

\begin{itemdecl}
template <class Flags> void memload(const value_type *mem, mask k, Flags);
\end{itemdecl}
\begin{itemdescr}
  \pnum\effects Replaces all elements of the \mask object where $k[i]$ is \true such that the $i$-th element is assigned with \code{mem[i]} \foralli.
  \pnum\remarks If the largest $i$ where \code{k[i]} is \true is greater than the number of values pointed to by the first argument, the behavior is undefined.
  \flagsRemarks{\mask{}}
\end{itemdescr}

\wgSubsubsection{\mask store functions}{mask.store}
\begin{itemdecl}
template <class Flags> void memstore(value_type *mem, Flags);
\end{itemdecl}
\begin{itemdescr}
  \pnum\effects Copies all \mask elements as if \code{mem[i] = operator[](i)} \foralli.
  \pnum\remarks If \code{size()} returns a value greater than the number of values pointed to by \code{mem}, the behavior is undefined.
  \flagsRemarks{\mask{}}
\end{itemdescr}

\begin{itemdecl}
template <class Flags> void memstore(value_type *mem, mask k, Flags);
\end{itemdecl}
\begin{itemdescr}
  \pnum\effects Copies each \mask element $i$ where \code{k[i]} is \true as if \code{mem[i] = operator[](i)} \foralli.
  \pnum\remarks If the largest $i$ where \code{k[i]} is \true is greater than the number of values pointed to by \code{mem}, the behavior is undefined.
  \wgNote{
    Masked stores only write to the bytes in memory selected by the \code k argument.
    This prohibits implementations that load, blend, and store the complete vector.
  }
  \flagsRemarks{\mask{}}
\end{itemdescr}

\wgSubsubsection{\mask{} subscript operators}{mask.subscr}
\begin{itemdecl}
reference operator[](size_type i);
\end{itemdecl}
\begin{itemdescr}
  \dataparElementReference
\end{itemdescr}

\begin{itemdecl}
value_type operator[](size_type) const;
\end{itemdecl}
\begin{itemdescr}
  \pnum\returns A copy of the $i$-th element.
\end{itemdescr}

\wgSubsubsection{\mask unary operators}{mask.unary}
\begin{itemdecl}
mask operator!() const;
\end{itemdecl}
\begin{itemdescr}
  \pnum\returns A mask object with the $i$-th element set to the logical negation \foralli.
\end{itemdescr}

\wgSubsection{\type{mask} non-member operations}{mask.nonmembers}

\wgSubsubsection{\mask binary operators}{mask.binary}
\begin{itemdecl}
friend mask operator&&(const mask &, const mask &);
friend mask operator||(const mask &, const mask &);
friend mask operator& (const mask &, const mask &);
friend mask operator| (const mask &, const mask &);
friend mask operator^ (const mask &, const mask &);
\end{itemdecl}
\begin{itemdescr}
  \pnum\returns A \mask object initialized with the results of the component-wise application of the indicated operator.
\end{itemdescr}

\wgSubsubsection{\mask compound assignment}{mask.cassign}
\begin{itemdecl}
friend mask &operator&=(mask &, const mask &);
friend mask &operator|=(mask &, const mask &);
friend mask &operator^=(mask &, const mask &);
\end{itemdecl}
\begin{itemdescr}
  \pnum\effects Each of these operators performs the indicated operator component-wise on each of the corresponding elements of the arguments.
  \pnum\returns A reference to the first argument.
\end{itemdescr}

\wgSubsubsection{\mask compares}{mask.comparison}
\begin{itemdecl}
friend bool operator==(const mask &, const mask &);
\end{itemdecl}
\begin{itemdescr}
  \pnum\returns \true if all boolean elements of the first argument equal the corresponding element of the second argument.
  It returns \false otherwise.
\end{itemdescr}

\begin{itemdecl}
friend bool operator!=(const mask &, const mask &);
\end{itemdecl}
\begin{itemdescr}
  \pnum\returns \code{!operator==(a, b)}.
\end{itemdescr}

\wgSubsubsection{\mask reductions}{mask.reductions}
\begin{itemdecl}
template <class T, class Abi> bool  all_of(mask<T, Abi>);
\end{itemdecl}
\begin{itemdescr}
  \pnum\returns \true if all boolean elements in the function argument equal \true, \false otherwise.
\end{itemdescr}

\begin{itemdecl}
template <class T, class Abi> bool  any_of(mask<T, Abi>);
\end{itemdecl}
\begin{itemdescr}
  \pnum\returns \true if at least one boolean element in the function argument equals \true, \false otherwise.
\end{itemdescr}

\begin{itemdecl}
template <class T, class Abi> bool none_of(mask<T, Abi>);
\end{itemdecl}
\begin{itemdescr}
  \pnum\returns \true if none of the boolean element in the function argument equals \true, \false otherwise.
\end{itemdescr}

\begin{itemdecl}
template <class T, class Abi> bool some_of(mask<T, Abi>);
\end{itemdecl}
\begin{itemdescr}
  \pnum\returns \true if at least one of the boolean elements in the function argument equals \true and at least one of the boolean elements in the function argument equals \false, \false otherwise.
\end{itemdescr}

\begin{itemdecl}
template <class T, class Abi> int popcount(mask<T, Abi>);
\end{itemdecl}
\begin{itemdescr}
  \pnum\returns The number of boolean elements that are \true.
\end{itemdescr}

\begin{itemdecl}
template <class T, class Abi> int find_first_set(mask<T, Abi> m);
\end{itemdecl}
\begin{itemdescr}
  \pnum\returns The lowest element index \code i where \code{m[i] == true}.
  \pnum\remarks If \code{none_of(m) == true} the behavior is undefined.
\end{itemdescr}

\begin{itemdecl}
template <class T, class Abi> int find_last_set(mask<T, Abi> m);
\end{itemdecl}
\begin{itemdescr}
  \pnum\returns The highest element index \code i where \code{m[i] == true}.
  \pnum\remarks If \code{none_of(m) == true} the behavior is undefined.
\end{itemdescr}

\wgSubsubsection{Masked assigment}{mask.where}
\begin{itemdecl}
template <class T, class A>
where_expression<mask<T, A>, datapar<T, A>> where(
    const typename datapar<T, A>::mask_type &k, datapar<T, A> &v);
template <class T, class A>
const where_expression<mask<T, A>, const datapar<T, A>> where(
    const typename datapar<T, A>::mask_type &k, const datapar<T, A> &v);

template <class T, class A>
where_expression<mask<T, A>, mask<T, A>> where(const remove_const_t<mask<T, A>> &k,
                                               mask<T, A> &v);
template <class T, class A>
const where_expression<mask<T, A>, const mask<T, A>> where(
    const remove_const_t<mask<T, A>> &k, const mask<T, A> &v);
\end{itemdecl}
\comment{\code{remove_const} is only used in place of a missing \code{template <class T> struct id \{ using type = T; \};} for inhibiting type deduction.}
\begin{itemdescr}
  \pnum\returns An object of type \type{where_expression} (see \ref{sec:datapar.whereexpr}) initialized with the predicate \code k and the value reference \code v.
\end{itemdescr}

\begin{itemdecl}
template <class T> where_expression<bool, T> where(bool k, T &v);
\end{itemdecl}
\begin{itemdescr}
  \pnum\remarks The function only participates in overload resolution if \type T is neither a \datapar nor a \mask instantiation.
  \pnum\returns An object of type \type{where_expression} (see \ref{sec:datapar.whereexpr}) initialized with the predicate \code k and the value reference \code v.
\end{itemdescr}

% vim: tw=0 spell sw=2
