\wgSubsection{Class template \type{mask}}{mask}
\wgSubsubsection{Class template \mask overview}{mask.overview}
\lstinputlisting[]{mask.cpp}

\pnum The class template \mask[<T, Abi>] is a one-dimensional smart array of booleans.
The number of elements in the array is a constant expression, equal to the number of elements in \datapar[<T, Abi>].

\dataparTypeRequirements{mask}

\dataparDeleted{mask}

\pnum Default initialization performs no initialization of the elements; value-initialization initializes each element with \code{bool()}.
\wgNote{Thus, default initialization leaves the elements in an indeterminate state.}

\begin{itemdecl}
static constexpr size_type size();
\end{itemdecl}
\begin{itemdescr}
  \pnum\returns the number of boolean elements stored in objects of the given \mask[<T, Abi>] type.
\end{itemdescr}

\pnum\begin{noteEnv}Implementations are encouraged to enable \code{static_cast}ing from/to (an) implementation-defined SIMD mask type(s).
This would add one or more of the following declarations to class \mask:
\begin{itemdecl}
explicit operator implementation-defined() const;
explicit datapar(const implementation-defined& init);
\end{itemdecl}
\end{noteEnv}

\wgSubsubsection{\mask constructors}{mask.ctor}
\begin{itemdecl}
explicit mask(value_type);
\end{itemdecl}
\begin{itemdescr}
  \pnum\effects Constructs an object with each element initialized to the value of the argument.
\end{itemdescr}

\begin{itemdecl}
template <class U> mask(const mask<U, datapar_abi::fixed_size<size()>>& x);
\end{itemdecl}
\begin{itemdescr}
  \pnum\remarks This constructor shall not participate in overload resolution unless
    \type{abi_type} equals \fixedsizescoped{}\code{<size()>}.
  \pnum\effects Constructs an object of type \mask where the $i$-th element equals \code{x[i]} \foralli.
\end{itemdescr}

\begin{itemdecl}
template <class Flags> mask(const value_type *mem, Flags);
\end{itemdecl}
\begin{itemdescr}
  \pnum\effects Constructs an object where the $i$-th element is initialized to \code{mem[i]} \foralli.
  \pnum\requires \code{size()} is less than or equal to the number of values pointed to by \code{mem}.
  \flagsRemarks{\mask{}}
\end{itemdescr}

\wgSubsubsection{\mask load function}{mask.load}
\begin{itemdecl}
template <class Flags> void memload(const value_type *mem, Flags);
\end{itemdecl}
\begin{itemdescr}
  \pnum\effects Replaces the elements of the \mask object such that the $i$-th element is assigned with \code{mem[i]} \foralli.
  \pnum\requires \code{size()} is less than or equal to the number of values pointed to by \code{mem}.
  \flagsRemarks{\mask{}}
\end{itemdescr}

\wgSubsubsection{\mask store function}{mask.store}
\begin{itemdecl}
template <class Flags> void memstore(value_type *mem, Flags);
\end{itemdecl}
\begin{itemdescr}
  \pnum\effects Copies all \mask elements as if \code{mem[i] = operator[](i)} \foralli.
  \pnum\requires \code{size()} is less than or equal to the number of values pointed to by \code{mem}.
  \flagsRemarks{\mask{}}
\end{itemdescr}

\wgSubsubsection{\mask{} subscript operators}{mask.subscr}
\begin{itemdecl}
reference operator[](size_type i);
\end{itemdecl}
\begin{itemdescr}
  \dataparElementReference{\mask}
\end{itemdescr}

\begin{itemdecl}
value_type operator[](size_type) const;
\end{itemdecl}
\begin{itemdescr}
  \pnum\returns A copy of the $i$-th element.
\end{itemdescr}

\wgSubsubsection{\mask unary operators}{mask.unary}
\begin{itemdecl}
mask operator!() const;
\end{itemdecl}
\begin{itemdescr}
  \pnum\returns A mask object with the $i$-th element set to the logical negation \foralli.
\end{itemdescr}

\wgSubsection{\type{mask} non-member operations}{mask.nonmembers}

\wgSubsubsection{\mask binary operators}{mask.binary}
\begin{itemdecl}
friend mask operator&&(const mask&, const mask&);
friend mask operator||(const mask&, const mask&);
friend mask operator& (const mask&, const mask&);
friend mask operator| (const mask&, const mask&);
friend mask operator^ (const mask&, const mask&);
\end{itemdecl}
\begin{itemdescr}
  \pnum\returns A \mask object initialized with the results of the component-wise application of the indicated operator.
\end{itemdescr}

\wgSubsubsection{\mask compound assignment}{mask.cassign}
\begin{itemdecl}
friend mask& operator&=(mask&, const mask&);
friend mask& operator|=(mask&, const mask&);
friend mask& operator^=(mask&, const mask&);
\end{itemdecl}
\begin{itemdescr}
  \pnum\effects Each of these operators performs the indicated operator component-wise on each of the corresponding elements of the arguments.
  \pnum\returns A reference to the first argument.
\end{itemdescr}

\wgSubsubsection{\mask compares}{mask.comparison}
\begin{itemdecl}
friend mask operator==(const mask&, const mask&);
friend mask operator!=(const mask&, const mask&);
\end{itemdecl}
\begin{itemdescr}
  \pnum\returns A \mask object initialized with the results of the component-wise application of the indicated operator.
\end{itemdescr}

\wgSubsubsection{\mask reductions}{mask.reductions}
\begin{itemdecl}
template <class T, class Abi> bool  all_of(mask<T, Abi>);
\end{itemdecl}
\begin{itemdescr}
  \pnum\returns \true if all boolean elements in the function argument equal \true, \false otherwise.
\end{itemdescr}

\begin{itemdecl}
template <class T, class Abi> bool  any_of(mask<T, Abi>);
\end{itemdecl}
\begin{itemdescr}
  \pnum\returns \true if at least one boolean element in the function argument equals \true, \false otherwise.
\end{itemdescr}

\begin{itemdecl}
template <class T, class Abi> bool none_of(mask<T, Abi>);
\end{itemdecl}
\begin{itemdescr}
  \pnum\returns \true if none of the boolean element in the function argument equals \true, \false otherwise.
\end{itemdescr}

\begin{itemdecl}
template <class T, class Abi> bool some_of(mask<T, Abi>);
\end{itemdecl}
\begin{itemdescr}
  \pnum\returns \true if at least one of the boolean elements in the function argument equals \true and at least one of the boolean elements in the function argument equals \false, \false otherwise.
\end{itemdescr}

\begin{itemdecl}
template <class T, class Abi> int popcount(mask<T, Abi>);
\end{itemdecl}
\begin{itemdescr}
  \pnum\returns The number of boolean elements that are \true.
\end{itemdescr}

\begin{itemdecl}
template <class T, class Abi> int find_first_set(mask<T, Abi> m);
\end{itemdecl}
\begin{itemdescr}
  \pnum\requires \code{any_of(m)} returns \true
  \pnum\returns The lowest element index \code i where \code{m[i] == true}.
\end{itemdescr}

\begin{itemdecl}
template <class T, class Abi> int find_last_set(mask<T, Abi> m);
\end{itemdecl}
\begin{itemdescr}
  \pnum\requires \code{any_of(m)} returns \true
  \pnum\returns The highest element index \code i where \code{m[i] == true}.
\end{itemdescr}

\begin{itemdecl}
template <class T, class Abi> bool  all_of(implementation-defined);
template <class T, class Abi> bool  any_of(implementation-defined);
template <class T, class Abi> bool none_of(implementation-defined);
template <class T, class Abi> bool some_of(implementation-defined);
template <class T, class Abi> int popcount(implementation-defined);
template <class T, class Abi> int find_first_set(implementation-defined);
template <class T, class Abi> int find_last_set(implementation-defined);
\end{itemdecl}
\begin{itemdescr}
  \pnum\remarks The functions shall not participate in overload resolution unless the argument is of type \bool.
  \pnum\returns \code{all_of} and \code{any_of} return their arguments; \code{none_of} returns the negation of its argument; \code{some_of} returns \false; \code{popcount} returns the integral representation of its argument; \code{find_first_set} and \code{find_last_set} return 0.
\end{itemdescr}

\wgSubsubsection{Masked assigment}{mask.where}
\begin{itemdecl}
template <class T, class A>
where_expression<mask<T, A>, datapar<T, A>> where(
    const typename datapar<T, A>::mask_type& k, datapar<T, A>& v);
template <class T, class A>
const where_expression<mask<T, A>, const datapar<T, A>> where(
    const typename datapar<T, A>::mask_type& k, const datapar<T, A>& v);

template <class T, class A>
where_expression<mask<T, A>, mask<T, A>> where(const remove_const_t<mask<T, A>>& k,
                                               mask<T, A>& v);
template <class T, class A>
const where_expression<mask<T, A>, const mask<T, A>> where(
    const remove_const_t<mask<T, A>>& k, const mask<T, A>& v);
\end{itemdecl}
\comment{\code{remove_const} is only used in place of a missing \code{template <class T> struct id \{ using type = T; \};} for inhibiting type deduction.}
\begin{itemdescr}
  \pnum\returns An object of type \type{where_expression} (see \ref{sec:datapar.whereexpr}) initialized with the predicate \code k and the value reference \code v.
\end{itemdescr}

\begin{itemdecl}
template <class T> where_expression<bool, T> where(implementation-defined k, T& v);
\end{itemdecl}
\begin{itemdescr}
  \pnum\remarks The function shall not participate in overload resolution unless
  \begin{itemize}
    \item \type T is neither a \datapar nor a \mask instantiation, and
    \item the first argument is of type \bool.
  \end{itemize}
  \pnum\returns An object of type \type{where_expression} (see \ref{sec:datapar.whereexpr}) initialized with the predicate \code k and the value reference \code v.
\end{itemdescr}

% vim: tw=0 spell sw=2
