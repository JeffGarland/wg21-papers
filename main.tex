\newcommand\wgTitle{Data-Parallel Vector Types \& Operations}
\newcommand\wgName{Matthias Kretz <kretz@compeng.uni-frankfurt.de>}
\newcommand\wgDocumentNumber{D0214R0}
\newcommand\wgPreviousDocumentNumber{N4184, N4185, N4395}
\newcommand\wgAcknowledgements{
  This work was supported by GSI Helmholtzzentrum für Schwerionenforschung
  and the Hessian LOEWE initiative through the Helmholtz International Center
  for FAIR (HIC for FAIR).
}

\usepackage{underscore}
\usepackage{typenames}
\usepackage{mymacros}
\usepackage{wg21}

\addbibresource{extra.bib}

\newcommand\datapar[1][]{\type{datapar#1}\xspace}
\newcommand\valuetype{\type{value\_type}\xspace}
\newcommand\simdcast{\code{datapar\_cast}\xspace}
\newcommand\mask{\type{mask}\xspace}

\begin{document}
\selectlanguage{american}
\begin{wgTitlepage}
  This paper describes class templates for portable data-parallel (e.g. SIMD) programming via vector types.
\end{wgTitlepage}

\pagestyle{scrheadings}
\addtocounter{section}{-1}
\section{Remarks}
\begin{itemize}
  \item This documents talks about “vector” types/objects.
    In general this will not refer to the \std\type{vector} class template.
    References to the container type will explicitly call out the \code{std} prefix to avoid confusion.
  \item In the following \VSize{T} denotes the number of scalar values (width) in a vector of type \type T (sometimes also called the number of SIMD lanes)
\end{itemize}

\section{Introduction}

Parallel Algorithms enable implementations of the existing STL algorithms to use non-sequential semantics when executing the user-supplied code (explicit callable or implicit operator call).
The first argument to the algorithm function determines this change in execution semantics via an \emph{execution policy}.
This paper introduces a new execution policy, called \dataparEP.
\dataparEP requires user-provided function objects to be callable with \datapar[<T, Abi>] arguments instead of the \type T arguments the \seqEP variant would use.
The algorithm therefore processes chunks of \datapar[<T, Abi>::size()] objects concurrently.
The execution order of the chunks retains the sequential semantics of the non-parallel algorithms.

As a consequence, the applicability of the execution policy is limited to iterators where \datapar[<Iterator::value_type>] is a valid template instantiation of \datapar.
A future extension of \datapar may lift this restriction by allowing certain (or all) user-defined types as first template argument to \datapar.


\section{Wording}

The following is a rough draft of possible wording that defines a basic set of data-parallel types.

\begin{wgText}
  \wgSection{Data-Parallel Types}{datapar.types}

  \wgSubsection{Header \code{<datapar>} synopsis}{datapar.syn}
  \lstinputlisting[]{synopsis.cpp}

  \pnum
  The header \code{<datapar>} defines two class templates (\datapar, and \mask), several tag types, and a series of related function templates for concurrent manipulation of the values in \datapar and \mask objects.

  \begin{itemdecl}
namespace datapar_abi {
  struct scalar {};
  // implementation-defined tag types, e.g. sse, avx, avx512, neon, ...
  typedef implementation_defined compatible;
  typedef implementation_defined native;
}
  \end{itemdecl}
  \begin{itemdescr}
    \pnum
    The ABI types are tag types to be used as the second template argument to \datapar and \mask.

    \pnum
    The \type{scalar} tag is present in all implementations and forces \datapar and \mask to store a single component (i.e. \datapar{}\type{<T, datapar_abi::scalar>::size()} returns \code 1).

    \pnum
    An implementation may choose to implement data-parallel execution for many different targets.
    \wgNote{There can certainly be more than one tag type per (micro-)architecture, e.g. to support different vector lengths or partial register usage.}
    All tag types an implementation supports shall be present independent of the chosen target.

    \pnum
    The \type{datapar_abi::compatible} tag is defined by the implementation to alias the tag type with the most efficient data parallel execution that ensures the highest compatibility on the target architecture.

    \pnum
    The \type{datapar_abi::native} tag is defined by the implementation to alias the tag type with the most efficient data parallel execution that is supported on the target system.
  \end{itemdescr}

  \begin{itemdecl}
template <class T, size_t N> struct abi_for_width { typedef implementation_defined type; };
  \end{itemdecl}
  \begin{itemdescr}
    \pnum
    The \type{abi_for_width} class template defines the member type \type{type} to one of the tag types in \code{datapar_abi} or not at all, depending on the value of the template parameters.

    \pnum
    \code{datapar<T, abi_for_width_t<T, N>>::size()} must return \code N or result in a substitution failure.
  \end{itemdescr}

  \begin{itemdecl}
template <class T, class Abi = datapar_abi::compatible>
struct datapar_size : public integral_constant<size_t, implementation_defined> {};
  \end{itemdecl}
  \begin{itemdescr}
    \pnum The \type{datapar_size} class template inherits from \type{integral_constant} with a value that equals \datapar{}\code{<T, Abi>::size()}.

    \pnum \code{datapar_size<T, Abi>::value} shall result in a substitution failure if any of the template arguments \type T and \type{Abi} are invalid template arguments to \datapar.
  \end{itemdescr}

  \wgSubsection{Class template \datapar}{datapar}
  \wgSubsubsection{Class template \datapar overview}{datapar.overview}
  \lstinputlisting[]{datapar.cpp}

  \pnum The class template \datapar{}\type{<T, Abi>} is a one-dimensional smart array.
  In contrast to \type{valarray} (26.6), the number of elements in the array is determined at compile time, according to the \type{Abi} template parameter.

  \pnum The first template argument \type T must be an integral or floating-point fundamental type.
  The type \bool is not allowed.

  \pnum The second template argument \type{Abi} must be a tag type from the \code{datapar_abi} namespace.

  \begin{itemdecl}
typedef implementation_defined native_handle_type;
  \end{itemdecl}
  \begin{itemdescr}
    \pnum
    The \type{native_handle_type} member type is an alias for the \code{native_handle()} member function return type.
    It is used to expose an implementation-defined handle for implementation- and target-specific extensions.
  \end{itemdescr}

  \begin{itemdecl}
typedef implementation_defined register_value_type;
  \end{itemdecl}
  \begin{itemdescr}
  \end{itemdescr}

  \wgSubsubsection{\datapar constructors}{datapar.ctor}
  \begin{itemdecl}
datapar() = default;
  \end{itemdecl}
  \begin{itemdescr}
    \pnum
    \effects
    Constructs an object with all elements initialized to \code{T()}.
    \wgNote{This zero-initializes the object.}
  \end{itemdescr}

  \begin{itemdecl}
datapar(value_type);
  \end{itemdecl}
  \begin{itemdescr}
    \pnum
    \effects
    Constructs an object with each element initialized to the value of the argument.
  \end{itemdescr}

  \begin{itemdecl}
template <typename U> datapar(datapar<U, Abi> x);
  \end{itemdecl}
  \begin{itemdescr}
    \pnum\remarks This constructor shall not participate in overload resolution unless
                  \type U and \type T are different integral types and
                  \code{make_signed<U>::type} equals \code{make_signed<T>::type}.
    \pnum\effects Constructs an object of type \datapar.
    \pnum\postcondition The $i$-th element equals \code{static_cast<T>(x[i])} for all elements.
  \end{itemdescr}

  \wgSubsubsection{\datapar load functions}{datapar.load}
  \begin{itemdecl}
static datapar load(const value_type *x);
  \end{itemdecl}
  \begin{itemdescr}
    \pnum \effects Constructs an object with each element $i$ initialized to \code{x[i]} for all elements.
    \pnum \returns The constructed object.
    \pnum \remarks If \datapar{}\code{::size()} is greater than the number of values pointed to by the argument, the behavior is undefined.
  \end{itemdescr}

  \begin{itemdecl}
template <typename Flags> static datapar load(const value_type *x, Flags);
  \end{itemdecl}
  \begin{itemdescr}
    \pnum\effects Constructs an object with each element $i$ initialized to \code{x[i]}.
    \pnum\returns The constructed object.
    \pnum\remarks If \datapar{}\code{::size()} is greater than the number of values pointed to by the first argument, the behavior is undefined.
    \pnum         If the template parameter is of type \type{aligned_tag} and the pointer value is not a multiple of \code{memory_alignment<\type T>}, the behavior is undefined.
  \end{itemdescr}

  \begin{itemdecl}
template <typename U, typename Flags = unaligned_tag> static datapar load(const U *x, Flags = Flags());
  \end{itemdecl}
  \begin{itemdescr}
    \pnum\effects Constructs an object with each element $i$ initialized to \code{static_cast<T>(x[i])}.
    \pnum\returns The constructed object.
    \pnum\remarks If \datapar{}\code{::size()} is greater than the number of values pointed to by the first argument, the behavior is undefined.
    \pnum         If the second template parameter is of type \type{aligned_tag} and the pointer value is not a multiple of \code{memory_alignment<\type U>}, the behavior is undefined.
  \end{itemdescr}

  \wgSubsubsection{\datapar store functions}{datapar.store}
  \begin{itemdecl}
void store(value_type *x);
  \end{itemdecl}
  \begin{itemdescr}
    \pnum\effects Copies each element such that the $i$-th element is stored to \code{x[i]}.
    \pnum\remarks If \datapar{}\code{::size()} is greater than the number of values pointed to by the first argument, the behavior is undefined.
  \end{itemdescr}

  \begin{itemdecl}
template <typename Flags> void store(value_type *x, Flags);
  \end{itemdecl}
  \begin{itemdescr}
    \pnum\effects Copies each element such that the $i$-th element is stored to \code{x[i]}.
    \pnum\remarks If \datapar{}\code{::size()} is greater than the number of values pointed to by the first argument, the behavior is undefined.
    \pnum         If the template parameter is of type \type{aligned_tag} and the pointer value is not a multiple of \code{memory_alignment<\type T>}, the behavior is undefined.
  \end{itemdescr}

  \begin{itemdecl}
template <typename U, typename Flags = unaligned_tag> void store(U *x, Flags = Flags());
  \end{itemdecl}
  \begin{itemdescr}
    \pnum\effects Copies each element such that the $i$-th element is first converted to \type U and then stored to \code{x[i]}.
    \pnum\remarks If \datapar{}\code{::size()} is greater than the number of values pointed to by the first argument, the behavior is undefined.
    \pnum         If the second template parameter is of type \type{aligned_tag} and the pointer value is not a multiple of \code{memory_alignment<\type U>}, the behavior is undefined.
  \end{itemdescr}

  \begin{itemdecl}
void store(value_type *x, mask_type);
  \end{itemdecl}
  \begin{itemdescr}
    \pnum\effects Copies each element where the corresponding element in the second argument is \true such that the $i$-th element is stored to \code{x[i]}.
    \pnum\remarks If the largest $i$ where the second argument is \true is greater than the number of values pointed to by the first argument, the behavior is undefined.
  \end{itemdescr}

  \begin{itemdecl}
template <typename Flags> void store(value_type *x, mask_type, Flags);
  \end{itemdecl}
  \begin{itemdescr}
    \pnum\effects Copies each element where the corresponding element in the second argument is \true such that the $i$-th element is stored to \code{x[i]}.
    \pnum\remarks If the largest $i$ where the second argument is \true is greater than the number of values pointed to by the first argument, the behavior is undefined.
    \pnum         If the template parameter is of type \type{aligned_tag} and the pointer value is not a multiple of \code{memory_alignment<\type T>}, the behavior is undefined.
  \end{itemdescr}

  \begin{itemdecl}
template <typename U, typename Flags = unaligned_tag> void store(U *x, mask_type, Flags = Flags());
  \end{itemdecl}
  \begin{itemdescr}
    \pnum\effects Copies each element where the corresponding element in the second argument is \true such that the $i$-th element is first converted to \type U and then stored to \code{x[i]}.
    \pnum\remarks If the largest $i$ where the second argument is \true is greater than the number of values pointed to by the first argument, the behavior is undefined.
    \pnum         If the template parameter is of type \type{aligned_tag} and the pointer value is not a multiple of \code{memory_alignment<\type U>}, the behavior is undefined.
  \end{itemdescr}

  \wgSubsubsection{\datapar subscript operators}{datapar.subscr}
  \begin{itemdecl}
reference operator[](size_type i);
  \end{itemdecl}
  \begin{itemdescr}
    \pnum\returns An lvalue reference to the $i$-th element.
    \pnum\postconditions Assignment of objects of type \type T modify the $i$-th element without aliasing violations.
    \pnum                Modification of \code{*this} does not invalidate references held to the return value.
    Subsequent reads from such references yield the new value of the $i$-th element.
  \end{itemdescr}

  \begin{itemdecl}
const_reference operator[](size_type) const;
  \end{itemdecl}
  \begin{itemdescr}
    \pnum\returns A \const lvalue reference to the $i$-th element.
    \pnum\postconditions Modification of \code{*this} does not invalidate references held to the return value.
    Subsequent reads from such references yield the new value of the $i$-th element.
  \end{itemdescr}

  \wgSubsubsection{\datapar unary operators}{datapar.unary}
  \begin{itemdecl}
datapar &operator++();
  \end{itemdecl}
  \begin{itemdescr}
    \pnum\effects Increments every element of \code{*this} by one.
    \pnum\returns An lvalue reference to \code{*this} after incrementing.
    \pnum\remarks Overflow semantics follow the same semantics as for \type T.
  \end{itemdescr}

  \begin{itemdecl}
datapar operator++(int);
  \end{itemdecl}
  \begin{itemdescr}
    \pnum\effects Increments every element of \code{*this} by one.
    \pnum\returns A copy of \code{*this} before incrementing.
    \pnum\remarks Overflow semantics follow the same semantics as for \type T.
  \end{itemdescr}

  \begin{itemdecl}
datapar &operator--();
  \end{itemdecl}
  \begin{itemdescr}
    \pnum\effects Decrements every element of \code{*this} by one.
    \pnum\returns An lvalue reference to \code{*this} after decrementing.
    \pnum\remarks Underflow semantics follow the same semantics as for \type T.
  \end{itemdescr}

  \begin{itemdecl}
datapar operator--(int);
  \end{itemdecl}
  \begin{itemdescr}
    \pnum\effects Decrements every element of \code{*this} by one.
    \pnum\returns A copy of \code{*this} before decrementing.
    \pnum\remarks Underflow semantics follow the same semantics as for \type T.
  \end{itemdescr}

  \begin{itemdecl}
mask_type operator!() const;
  \end{itemdecl}
  \begin{itemdescr}
    \pnum\returns A mask object with the $i$-th element set to \code{!operator[](i)} for all elements.
  \end{itemdescr}

  \begin{itemdecl}
datapar operator~() const;
  \end{itemdecl}
  \begin{itemdescr}
    \pnum\requires The first template argument \type T to \datapar must be an integral type.
    \pnum\effects Constructs an object where each bit of \code{*this} is inverted.
    \pnum\returns The new object.
    \pnum\remarks \datapar{}\code{::operator\textasciitilde{}()} shall not participate in overload resolution if \type T is a floating-point type.
  \end{itemdescr}

  \begin{itemdecl}
datapar operator+() const;
  \end{itemdecl}
  \begin{itemdescr}
    \pnum \returns A copy of \code{*this}
  \end{itemdescr}

  \begin{itemdecl}
datapar operator-() const;
  \end{itemdecl}
  \begin{itemdescr}
    \pnum\effects Constructs an object where the $i$-th element is initialized to \code{-operator[](i)} for all elements.
    \pnum\returns The new object.
  \end{itemdescr}

  \wgSubsubsection{\datapar native handles}{datapar.native}
  \begin{itemdecl}
native_handle_type &native_handle();
  \end{itemdecl}
  \begin{itemdescr}
    \pnum\returns An lvalue reference to the implementation-specific object implementing the data-parallel semantics.
  \end{itemdescr}

  \begin{itemdecl}
const native_handle_type &native_handle() const;
  \end{itemdecl}
  \begin{itemdescr}
    \pnum\returns A \const lvalue reference to the implementation-specific object implementing the data-parallel semantics.
  \end{itemdescr}

  \wgSubsection{\datapar non-member operations}{datapar.nonmembers}
  \wgSubsubsection{\datapar binary operators}{datapar.binary}
  \begin{itemdecl}
template <class L, class R> using datapar_return_type = ...;  // exposition only
template <class T, class Abi, class U>
datapar_return_type<datapar<T, Abi>, U> operator+ (datapar<T, Abi>, const U &);
template <class T, class Abi, class U>
datapar_return_type<datapar<T, Abi>, U> operator- (datapar<T, Abi>, const U &);
template <class T, class Abi, class U>
datapar_return_type<datapar<T, Abi>, U> operator* (datapar<T, Abi>, const U &);
template <class T, class Abi, class U>
datapar_return_type<datapar<T, Abi>, U> operator/ (datapar<T, Abi>, const U &);
template <class T, class Abi, class U>
datapar_return_type<datapar<T, Abi>, U> operator% (datapar<T, Abi>, const U &);
template <class T, class Abi, class U>
datapar_return_type<datapar<T, Abi>, U> operator& (datapar<T, Abi>, const U &);
template <class T, class Abi, class U>
datapar_return_type<datapar<T, Abi>, U> operator| (datapar<T, Abi>, const U &);
template <class T, class Abi, class U>
datapar_return_type<datapar<T, Abi>, U> operator^ (datapar<T, Abi>, const U &);
template <class T, class Abi, class U>
datapar_return_type<datapar<T, Abi>, U> operator<<(datapar<T, Abi>, const U &);
template <class T, class Abi, class U>
datapar_return_type<datapar<T, Abi>, U> operator>>(datapar<T, Abi>, const U &);
template <class T, class Abi, class U>
datapar_return_type<datapar<T, Abi>, U> operator+ (const U &, datapar<T, Abi>);
template <class T, class Abi, class U>
datapar_return_type<datapar<T, Abi>, U> operator- (const U &, datapar<T, Abi>);
template <class T, class Abi, class U>
datapar_return_type<datapar<T, Abi>, U> operator* (const U &, datapar<T, Abi>);
template <class T, class Abi, class U>
datapar_return_type<datapar<T, Abi>, U> operator/ (const U &, datapar<T, Abi>);
template <class T, class Abi, class U>
datapar_return_type<datapar<T, Abi>, U> operator% (const U &, datapar<T, Abi>);
template <class T, class Abi, class U>
datapar_return_type<datapar<T, Abi>, U> operator& (const U &, datapar<T, Abi>);
template <class T, class Abi, class U>
datapar_return_type<datapar<T, Abi>, U> operator| (const U &, datapar<T, Abi>);
template <class T, class Abi, class U>
datapar_return_type<datapar<T, Abi>, U> operator^ (const U &, datapar<T, Abi>);
template <class T, class Abi, class U>
datapar_return_type<datapar<T, Abi>, U> operator<<(const U &, datapar<T, Abi>);
template <class T, class Abi, class U>
datapar_return_type<datapar<T, Abi>, U> operator>>(const U &, datapar<T, Abi>);
  \end{itemdecl}
  \begin{itemdescr}
    \pnum\remarks The return type of these operators shall be deduced according to the following rules:
    \begin{itemize}
      \item If \code{is_datapar_v<U> == true}
        then the return type shall be determined from \type T and \type{U::value_type} according to the following paragraph.
      \item Otherwise, if \type T is integral and \type U is \intt
        the return type shall be \datapar{}\type{<T, Abi>}.
      \item Otherwise, if \type T is integral and \type U is \uint
        the return type shall be \datapar{}\code{<make_unsigned_t<T>, Abi>}.
      \item Otherwise, if \type U is a fundamental arithmetic type or \type U is convertible to \intt
        then the return type shall be determined from \type T and \type U according to the following paragraph.
      \item Otherwise, if \type U is implicitly convertible to \datapar{}\type{<V, Abi>}, where \type V is determined according to standard template type deduction,
        then the return type shall be determined from \type T and \type V according to the following paragraph.
      \item Otherwise, if \type U is implicitly convertible to \datapar{}\type{<T, Abi>},
        the return type shall be \datapar{}\type{<T, Abi>}.
      \item Otherwise no return type is defined (SFINAE).
    \end{itemize}

    \pnum\remarks Given the types \type T and \type{Abi} from the class template argument list and a third type \type U determined by the rules of the previous paragraph a return type is deduced according to the following rules:
    \begin{itemize}
      \item If \type U is not a fundamental arithmetic type then the return type shall be \datapar{}\type{<T, Abi>}.
      \item Otherwise, if at least one of the  types \type T and \type U is a floating-point type
        the return type shall be \datapar{}\type{<decltype(T() + U()), Abi>}.
      \item Otherwise, if \code{sizeof(T) < sizeof(U)} the return type shall be \datapar{}\type{<U, Abi>}.
      \item Otherwise, if \code{sizeof(T) > sizeof(U)} the return type shall be \datapar{}\type{<T, Abi>}.
      \item Otherwise, the type \type T or \type U that is farthest back in the list of \textit{standard integer types} (cf. [basic.fundamental]) is used as type \type V and
        the return type shall be \datapar{}\type{<V, Abi>} if both types \type T and \type U are signed, otherwise the return type shall be \datapar{}\type{<make_unsigned_t<V>, Abi>}.
    \end{itemize}

    \pnum\remarks Each of these operators only participates in overload resolution if all of the following hold:
    \begin{itemize}
      \item The indicated operator can be applied to objects of type \type{R::value_type}, with \type R the return type.
      \item \datapar{}\type{<T, Abi>} is implicitly convertible to the return type.
      \item \type U is implicitly convertible to the return type.
    \end{itemize}

    \pnum\remarks The operators with \type{const U \&} as first parameter shall not participate in overload resolution if \code{is_datapar_v<U> == true}.

    \pnum\effects Both arguments are first converted to the return type.
      Each of these operators subsequently performs the indicated operation component-wise on each of the elements of the first argument and the corresponding element of the second argument.
    \pnum\returns An object containing the results of the component-wise operator application.
  \end{itemdescr}

  \wgSubsubsection{\datapar compound assignment}{datapar.cassign}
  \begin{itemdecl}
template <class T, class Abi, class U> datapar<T, Abi> &operator+= (datapar<T, Abi> &, const U &);
template <class T, class Abi, class U> datapar<T, Abi> &operator-= (datapar<T, Abi> &, const U &);
template <class T, class Abi, class U> datapar<T, Abi> &operator*= (datapar<T, Abi> &, const U &);
template <class T, class Abi, class U> datapar<T, Abi> &operator/= (datapar<T, Abi> &, const U &);
template <class T, class Abi, class U> datapar<T, Abi> &operator%= (datapar<T, Abi> &, const U &);
template <class T, class Abi, class U> datapar<T, Abi> &operator&= (datapar<T, Abi> &, const U &);
template <class T, class Abi, class U> datapar<T, Abi> &operator|= (datapar<T, Abi> &, const U &);
template <class T, class Abi, class U> datapar<T, Abi> &operator^= (datapar<T, Abi> &, const U &);
template <class T, class Abi, class U> datapar<T, Abi> &operator<<=(datapar<T, Abi> &, const U &);
template <class T, class Abi, class U> datapar<T, Abi> &operator>>=(datapar<T, Abi> &, const U &);
  \end{itemdecl}
  \begin{itemdescr}
    \pnum\remarks Each of these operators only participates in overload resolution if all of the following hold:
    \begin{itemize}
      \item The indicated operator can be applied to objects of type \type{datapar_return_type<datapar<T, Abi>, U>::value_type}.
      \item \datapar{}\type{<T, Abi>} is implicitly convertible to \type{datapar_return_type<datapar<T, Abi>, U>}.
      \item \type U is implicitly convertible to \type{datapar_return_type<datapar<T, Abi>, U>}.
      \item \type{datapar_return_type<datapar<T, Abi>, U>} is implicitly convertible to \datapar{}\type{<T, Abi>}.
    \end{itemize}
    \pnum\effects Each of these operators performs the indicated operation component-wise on each of the elements of the first argument and the corresponding element of the second argument after conversion to \datapar{}\code{<T, Abi>}.
    \pnum\returns A reference to the first argument.
  \end{itemdescr}

  \wgSubsubsection{\datapar logical operators}{datapar.comparison}

  \wgSubsubsection{\datapar transcendentals}{datapar.transcend}

  \wgSubsection{Class template \mask}{datapar.mask}
  \lstinputlisting[]{mask.cpp}

\end{wgText}

\section{Discussion}

\subsection{Member Types}
The member types may not seem obvious.
Rationales:
\begin{typelist*}
  \item[value_type]
    In the spirit of the \valuetype member of STL containers, this type denotes the \emph{logical} type of the values in the vector.

  \item[reference]
    Used as the return type of the non-const scalar subscript operator.

  \item[mask_type]
    The natural \mask type for this \simd instantiation.
    This type is used as return type of compares and write-mask on assignments.

  \item[simd_type]
    The natural \simd type for this \mask instantiation.

  \item[size_type]
    Standard member type used for \code{size()} and \code{operator[]}.

  \item[abi_type]
    The \type{Abi} template parameter to \simd.

\end{typelist*}

\subsection{Conversions}
The \simd conversion constructor only allows implicit conversion from \simd template instantiations with the same \type{Abi} type and compatible \valuetype.
Discussion in SG1 showed clear preference for only allowing implicit conversion between integral types that only differ in signedness.
All other conversions could be implemented via an explicit conversion constructor.
The alternative (preferred) is to use \simdcast consistently for all other conversions.

After more discussion on the LEWG reflector, in Issaquah, and between me and Jens, we modified conversions to be even more conservative.
No implicit conversion will ever allow a narrowing conversion of the element type (and signed - unsigned is narrowing in both directions).

\subsection{Broadcast Constructor}
The \simd broadcast constructor is not declared \code{explicit} to ease the use of scalar prvalues in expressions involving data-parallel operations.
The operations where such a conversion should not be implicit consequently need to use SFINAE / concepts to inhibit the conversion.

Experience from Vc shows that the situation is different for \mask, where an implicit conversion from \bool typically hides an error.
(Since there is little use for broadcasting \true or \false.)

\subsection{Aliasing of Subscript Operators}
The subscript operators return an rvalue.
The const overload returns a copy of the element.
The non-const overload returns a smart reference.
This reference behaves mostly like an lvalue reference, but without the requirement to implement assignment via type punning.
At this point the specification of the smart reference is very conservative / restrictive:
The reference type is neither copyable nor movable.
The intention is to avoid users to program like the operator returned an lvalue reference.
The return type is significantly larger than an lvalue reference and harder to optimize when passed around.
The restriction thus forces users to do element modification directly on the \simd / \mask objects.

Guidance from SG1 at JAX 2016:\\
\wgPoll{Should subscript operator return an lvalue reference?}{0  & 6 & 10 & 2 & 1}

\wgPoll{Should subscript operator return a “smart reference”?}{1  & 7 & 10 & 0 & 0}

\subsection{Compound Assignment}
The semantics of compound assignment would allow less strict implicit conversion rules.
Consider \code{simd<int>() *= double()}: the corresponding binary multiplication operator would not compile because the implicit conversion to \simd[<double>] is non-portable.
Compound assignment, on the other hand, implies an implicit conversion back to the type of the expression on the left of the assignment operator.
Thus, it is possible to define compound operators that execute the operation correctly on the promoted type without sacrificing portability.
There are two arguments for not relaxing the rules for compound assignment, though:
\begin{enumerate}
  \item Consistency: The conversion of an expression with compound assignment to a binary operator might make it ill-formed.
  \item The implicit conversion in the \code{int * double} case could be expensive and unintended.
    This is already a problem for builtin types, where many developers multiply \float variables with \double prvalues, though.
\end{enumerate}

\subsection{Return Type of Masked Assignment Operators}
The assignment operators of the type returned by \code{where(mask, simd)} could return one of:
\begin{itemize}
  \item A reference to the \simd object that was modified.
  \item A temporary \simd object that only contains the elements where the \mask is \true.
  \item A reference to the \type{where_expression} object.
  \item Nothing (\ie \void).
\end{itemize}
My first choice was a reference to the modified \simd object.
However, then the statement \code{(where(x < 0, x) *= -1) += 2} may be surprising: it adds \code 2 to all vector entries, independent of the mask.
Likewise, \code{y += (where(x < 0, x) *= -1)} has a possibly confusing interpretation because of the \mask in the middle of the expression.

Consider that write-masked assignment is used as a replacement for \code{if}-statements.
Using \void as return type therefore is a more fitting choice because \code{if}-statements have no return value.
By declaring the return type as \void the above expressions become ill-formed, which seems to be the best solution for guiding users to write maintainable code and express intent clearly.

\subsection{Fundamental SIMD Type or Not?}
\subsubsection{The Issue}
There was substantial discussion on the reflectors and SG1 meetings over the question whether \CC{} should define a fundamental, native SIMD type (let us call it \type{fundamental<T>}) and additionally a generic data-parallel type which supports an arbitrary number of elements (call it \type{arbitrary<T, N>}).
The alternative to defining both types is to only define \type{arbitrary<T, N = default_size<T>>}, since it encompasses the \type{fundamental<T>} type.

With regard to this proposal this second approach would add a third template parameter to \simd and \mask as shown in \lst{simd N}.
\begin{lstlisting}[style=Vc,numbers=left,float,label=lst:simd N,caption={
  Possible declaration of the class template parameters of a \simd class with arbitrary width.
}]
template <class T, size_t N = simd_size_v<T, simd_abi::compatible<T>>,
          class Abi = simd_abi::compatible<T>>
class simd;
\end{lstlisting}

\subsubsection{Standpoints}
The controversy is about how the flexibility of a type with arbitrary \code N is presented to the users.
Is there a (clear) distinction between a “fundamental” type with target-dependent (i.e. fixed) \code N and a higher-level abstraction with arbitrary \code N which can potentially compile to inefficient machine code?
Or should the \CC{} standard only define \type{arbitrary} and set it to a default \code N value that corresponds to the target-dependent \code N.
Thus, the default \code N, of \type{arbitrary} would correspond to \type{fundamental}.

It is interesting to note that \type{arbitrary<T, 1>} is the class variant of \type T.
Consequently, if we say there is no need for a \type{fundamental} type then we could argue for the deprecation of the builtin arithmetic types, in favor of \type{arbitrary<T, 1>}. \wgNote{This is an academic discussion, of course.}

The author has implemented a library where a clear distinction is made between \type{fundamental<T, Abi>} and \type{arbitrary<T, N>}.
The documentation and all teaching material says that the user should program with \type{fundamental}.
The \type{arbitrary} type should be used in special circumstances, or wherever \type{fundamental} works with the \type{arbitrary} type in its interfaces (e.g. for gather \& scatter or the \code{ldexp} \& \code{frexp} functions).

\subsubsection{Issues}
The definition of two separate class templates can alleviate some source compatibility issues resulting from different \code N on different target systems.
Consider the simplest example of a multiplication of an \intt vector with a \float vector:
\smallskip\begin{lstlisting}[style=Vc]
arbitrary<float>() * arbitrary<int>();  // compiles for some targets, fails for others
fundamental<float>() * fundamental<int>();  // never compiles, requires explicit cast
\end{lstlisting}
The \simd[<T>] operators are specified in such a way that source compatibility is ensured.
For a type with user definable \code N, the binary operators should work slightly different with regard to implicit conversions.
Most importantly, \type{arbitrary<T, N>} solves the issue of portable code containing mixed integral and floating-point values.
A user would typically create aliases such as:
\smallskip\begin{lstlisting}[style=Vc]
using floatvec = simd<float>;
using intvec = arbitrary<int, floatvec::size()>;
using doublevec = arbitrary<int, floatvec::size()>;
\end{lstlisting}
Objects of types \type{floatvec}, \type{intvec}, and \type{doublevec} will work together, independent of the target system.

Obviously, these type aliases are basically the same if the \code N parameter of \type{arbitrary} has a default value:
\smallskip\begin{lstlisting}[style=Vc]
using floatvec = arbitrary<float>;
using intvec = arbitrary<int, floatvec::size()>;
using doublevec = arbitrary<int, floatvec::size()>;
\end{lstlisting}
The ability to create these aliases is not the issue.
Seeing the need for using such a pattern is the issue.
Typically, a developer will think no more of it if his code compiles on his machine.
If \code{arbitrary<float>() * arbitrary<int>()} just happens to compile (which is likely), then this is the code that will get checked in to the repository.
Note that with the existence of the \type{fundamental} class template, the \code N parameter of the \type{arbitrary} class would not have a default value and thus force the user to think a second longer about portability.

\subsubsection{Progress}\label{sec:fixedsize progress}
\newcommand\common[2]{\code{\textit{common}(\type{#1}, \type{#2})}}
\newcommand\commonabi[3]{\code{\textit{commonabi}(\type{#1}, \type{#2}, \type{#3})}}

SG1 Guidance at JAX 2016:\\
\wgPoll{Specify simd width using ABI tag, with a special template tag for fixed size.}{3 & 7 & 0 & 0 & 1}
\wgPoll{Specify simd width using <T, N, abi>, where abi is not specified by the user.}{1 & 2 & 5 & 2 & 1}

At the Jacksonville meeting, SG1 decided to continue with the \simd[<T, Abi>] class template, with the addition of a new Abi type that denotes a user-requested number of elements in the vector (\fixedsizeN).
This has the following implications:
\begin{itemize}
  \item There is only one class template with a common interface for \textit{fundamental} and \textit{arbitrary} (\fixedsize) vector types.
  \item There are slight differences in the conversion semantics for \simd types with the \fixedsize Abi type.
    This may look like the \code{vector<bool>} mistake all over again.
    I'll argue below why I believe this is not the case.
  \item The \textit{fundamental} class instances could be implemented in such a way that they do not guarantee ABI compatibility on a given architecture where translation units are compiled with different compiler flags (for micro-architectural differences).
  \item The \fixedsize class instances, on the other hand, could be implemented to be the ABI stable types (if an implementation thinks this is an important feature).
    In implementation terms this means that \textit{fundamental} types are allowed to be passed via registers on function calls.
    \fixedsize types can be implemented in such a way that they are only passed via the stack, and thus an implementation only needs to ensure equal alignment and memory representation across TU borders for a given \type T, \code N.
\end{itemize}

The conversion differences between the \textit{fundamental} and \fixedsize class template instances are the main motivation for having a distinction (cf. discussion above).
The differences are chosen such that, in general, \textit{fundamental} types are more restrictive and do not turn into \fixedsize types on any operation that involves no \fixedsize types.
Operations of \fixedsize types allow easier use of mixed precision code as long as no elements need to be dropped / generated (i.e. the number of elements of all involved \simd objects is equal or a builtin arithmetic type is broadcast).

Examples:

\begin{enumerate}
  \item Mixed \intt--\float operations
\smallskip\begin{lstlisting}[style=Vc,numbers=left]
using floatv = simd<float>;  // native ABI
using float_sized_abi = simd_abi::fixed_size<floatv::size()>;
using intv = simd<int, float_sized_abi>;

auto x = floatv() + intv();/*! \label{lstline:1} */
intv y = floatv() + intv();/*! \label{lstline:2} */
\end{lstlisting}
    Line \ref{lstline:1} is well-formed:
    It states that $N$ (= \type{floatv}\code{::size()}) additions shall be executed concurrently.
    The type of \code{x} is \simd[<\float{}>], because it stores $N$ elements and both types \type{intv} and \type{floatv} are implicitly convertible to \simd[<\float{}>].
    Line \ref{lstline:2} is also well-formed because implicit conversion from \simd[<\type T, \type{Abi}>] to \simd[<\type U, \fixedsizeN{}>] is allowed whenever \code{N == \simd{}<\type T, \type{Abi}>::size()}.

  \item Native \intt vectors
\smallskip\begin{lstlisting}[style=Vc,numbers=left]
using intv = simd<int>;  // native ABI
using int_sized_abi = simd_abi::fixed_size<intv::size()>;
using floatv = simd<float, int_sized_abi>;

auto x = floatv() + intv();/*! \label{lstline:3} */
intv y = floatv() + intv();/*! \label{lstline:4} */
\end{lstlisting}
    Line \ref{lstline:3} is well-formed:
    It states that $N$ (= \type{intv}\code{::size()}) additions shall be executed concurrently.
    The type of \code{x} is \simd[<\floatv{}, \type{int_sized_abi}>] (i.e. \type{floatv}) and never \simd[<\float{}>], because \ldots
    \begin{enumerate}
      \item[\ldots] the \type{Abi} types of \type{intv} and \type{floatv} are not equal.
      \item[\ldots] either \code{\simd[<\float{}>]::size() != N} or \type{intv} is not implicitly convertible to \simd[<\float{}>].
      \item[\ldots] the last rule for \commonabi{V0}{V1}{T} sets the \type{Abi} type to \type{int_sized_abi}.
    \end{enumerate}
    Line \ref{lstline:4} is also well-formed because implicit conversion from \simd[<\type T, \fixedsizeN{}>] to \simd[<\type U, \type{Abi}>] is allowed whenever \code{N == \simd{}<\type U, \type{Abi}>::size()}.
\end{enumerate}


\subsection{Native Handle}\label{sec:native}
The presence of a \code{native_handle} function for accessing an internal data member such as e.g. a vector builtin or SIMD intrinsic type is seen as an important feature for adoption in the target communities.
Without such a handle the user is constrained to work within the (limited) API defined by the standard.
Many SIMD instruction sets have domain-specific instructions that will not easily be usable (if at all) via the standardized interface.
A user considering whether to use \simd or a SIMD extension such as vector builtins or SIMD intrinsics might decide against \simd just for fear of not being able to access all functionality.\footnote{
  Whether that's a reasonable fear is a different discussion.
}

I would be happy to settle on an alternative to exposing an lvalue reference to a data member.
Consider implementation-defined support casting (\code{static_cast}?) between \simd and non-standard SIMD extension types.
My understanding is that there could not be any normative wording about such a feature.
However, I think it could be useful to add a non-normative note about making \code{static_cast}(?) able to convert between such non-standard extensions and \simd.

Guidance from SG1 at Oulu 2016:\\
\wgPoll{Keep \code{native_handle} in the wording?}{0 & 6 & 3 & 3 & 0}

\subsection{Load \& Store Flags}\label{sec:flags}
SIMD loads and stores require at least an alignment option.
This is in contrast to implicit loads and stores present in \CC{}, where alignment is always assumed.
Many SIMD instruction sets allow more options, though:
\begin{itemize}
  \item Streaming, non-temporal loads and stores
  \item Software prefetching
\end{itemize}
In the Vc library I have added these as options in the load store flag parameter of the \code{load} and \code{store} functions.
However, non-temporal loads \& stores and prefetching are also useful for the existing builtin types.
I would like guidance on this question: should the general direction be to stick to \emph{only} alignment options for \simd loads and stores?

The other question is on the default of the load and store flags.
Some argue for setting the default to \code{aligned}, as that's what the user should always aim for and is most efficient.
Others argue for \code{unaligned} since this is safe per default.
The Vc library before version 1.0 used aligned loads and stores per default.
After the guidance from SG1 I changed the default to unaligned loads and stores with the Vc 1.0 release.
Changing the default is probably the worst that could be done, though.\footnote{As I realized too late.}
For Vc 2.0 I will drop the default.

For \simd I prefer no default:
\begin{itemize}
  \item This makes it obvious that the API has the alignment option.
    Users should not just take the default and think no more of it.
  \item If we decide to keep the load constructor, the alignment parameter (without default) nicely disambiguate the load from the broadcast.
  \item The right default would be application/domain/experience specific.
  \item Users can write their own load/store wrapper functions that implement their chosen default.
\end{itemize}

Guidance from SG1 at Oulu 2016:\\
\wgPoll{Should the interface provide a way to specify a number for over-alignment?}{2 & 6 & 5 & 0 & 0}
\wgPoll{Should loads and stores have a default load/store flag?}{0 & 0 & 7 & 4 & 1}
The discussion made it clear that we only want to support alignment flags in the load and store operations.
The other functionality is orthogonal.

\subsection{Unary Minus Return Type}\label{sec:unary minus}
The return type of \simd[<\type T, \type{Abi}>::operator-()] is \simd[<\type T, \type{Abi}>].
This is slightly different to the behavior of the underlying element type \type T, if \type T is an integral type of lower integer conversion rank than \intt.
In this case integral promotion promotes the type to \intt before applying unary minus.
Thus, the expression \code{-T()} is of type \intt for all \type T with lower integer conversion rank than \intt.
This is widening of the element size is likely unintended for SIMD vector types.

Fundamental types with integer conversion rank greater than \intt are not promoted and thus a unary minus expression has unchanged type.
This behavior is copied to element types of lower integer conversion rank for \simd.

There may be one interesting alternative to pursue here:
We can make it ill-formed to apply unary minus to unsigned integral types.
Anyone who wants to have the modulo behavior of a unary minus could still write \code{0u - x}.

\subsection{\code{max_fixed_size}}\label{sec:maxfixedsize}
In Kona, LEWG asked why \code{max_fixed_size} is not dependent on \type T.
After some consideration I am convinced that the correct solution is to make \code{max_fixed_size} a variable template, dependent on \type T.

The reason to restrict the number of elements $N$ in a fixed-size \simd type at all, is to inhibit misuse of the feature.
The intended use of the fixed-size ABI, is to work with a number of elements that is somewhere in the region of the number of elements that can be processed efficiently concurrently in hardware.
Implementations may want to use recursion to implement the fixed-size ABI.
While such an implementation can, in theory, scale to any $N$, experience shows that compiler memory usage and compile times grow significantly for “too large” $N$.
The optimizer also has a hard time to optimize register / stack allocation optimally if $N$ becomes “too large”.
Unsuspecting users might not think of such issues and try to map their complete problem to a single \simd object.
Allowing implementations to restrict $N$ to a value that they can and want to support thus is useful for users and implementations.
The value itself should not be prescribed by the standard as it is really just a QoI issue.

However, why should the user be able to query the maximum $N$ supported by the implementation?
\begin{itemize}
  \item In principle, a user can always determine the number using SFINAE to find the maximum $N$ that he can still instantiate without substitution failure.
    Not providing the number thus provides no “safety” against “bad usage”.
  \item A developer may want to use the value to document assumptions / requirements about the implementation, e.g. with a static assertion.
  \item A developer may want to use the value to make code portable between implementations that use a different \code{max_fixed_size}.
\end{itemize}

Making the \code{max_fixed_size} dependent on \type T makes sense because most hardware can process a different number of elements in parallel depending on \type T.
Thus, if an implementation wants to restrict $N$ to some sensible multiple of the hardware capabilities, the number must be dependent on \type T.

In Kona, LEWG also asked whether there should be a provision in the standard to ensure that a native \simd of 8-bit element type is convertible to a fixed-size \simd of 64-bit element type.
It was already there (\ref{simd.fixedsize.def}: “for every supported \simd[<T, A>] (see \ref{simd.type requirements}), where \type A is an implementation-defined ABI tag, \code N $=$ \simd[<T, A>::size()] must be supported”).
Note that this does not place a lower bound on \code{max_fixed_size}.
The wording allows implementations to support values for fixed-size \simd that are larger than \code{max_fixed_size}.
I.e. $N \leq $ \code{max_fixed_size} works; whether $N > $ \code{max_fixed_size} works is unspecified.

% vim: tw=0 sw=2 spell

\end{document}
% vim: sw=2 sts=2 ai et tw=0
