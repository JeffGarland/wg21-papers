\newcommand\wgTitle{Data-Parallel Vector Types \& Operations}
\newcommand\wgName{Matthias Kretz <kretz@compeng.uni-frankfurt.de>}
\newcommand\wgVersion{Version 1}
\newcommand\wgDocumentNumber{D4XXX}
\newcommand\wgPreviousDocumentNumber{N4184, N4185, N4395}
\newcommand\wgAcknowledgements{
  This work was supported by GSI Helmholtzzentrum für Schwerionenforschung
  and the Hessian LOEWE initiative through the Helmholtz International Center
  for FAIR (HIC for FAIR).
}

\usepackage{underscore}
\usepackage{typenames}
\usepackage{mymacros}
\usepackage{wg21}

\addbibresource{extra.bib}

\begin{document}
\selectlanguage{american}
\begin{wgTitlepage}
  This paper describes class templates for portable data-parallel (e.g. SIMD) programming via vector types.
\end{wgTitlepage}

\pagestyle{scrheadings}
\addtocounter{section}{-1}
\section{Remarks}
\begin{itemize}
  \item This documents talks about “vector” types/objects and in general does not refer to the \std\type{vector} class template in such cases.
    References to the container type will explicitly call out the \code{std} prefix to avoid confusion.
  \item In the following \VSize{T} denotes the number of scalar values (width) in a vector of type \type T (sometimes also called the number of SIMD lanes)
\end{itemize}

\section{Introduction}

Parallel Algorithms enable implementations of the existing STL algorithms to use non-sequential semantics when executing the user-supplied code (explicit callable or implicit operator call).
The first argument to the algorithm function determines this change in execution semantics via an \emph{execution policy}.
This paper introduces a new execution policy, called \dataparEP.
\dataparEP requires user-provided function objects to be callable with \datapar[<T, Abi>] arguments instead of the \type T arguments the \seqEP variant would use.
The algorithm therefore processes chunks of \datapar[<T, Abi>::size()] objects concurrently.
The execution order of the chunks retains the sequential semantics of the non-parallel algorithms.

As a consequence, the applicability of the execution policy is limited to iterators where \datapar[<Iterator::value_type>] is a valid template instantiation of \datapar.
A future extension of \datapar may lift this restriction by allowing certain (or all) user-defined types as first template argument to \datapar.


%\input{vectorclass}
\end{document}
% vim: sw=2 sts=2 ai et tw=0
