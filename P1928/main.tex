\newcommand\wgTitle{Merge data-parallel types from the Parallelism TS 2}
\newcommand\wgName{Matthias Kretz <m.kretz@gsi.de>}
\newcommand\wgDocumentNumber{D1928R1}
\newcommand\wgGroup{SG1}
\newcommand\wgTarget{\CC{}23}
%\newcommand\wgAcknowledgements{ }

\usepackage{mymacros}
\usepackage{wg21}
\setcounter{tocdepth}{2} % show sections and subsections in TOC
\hypersetup{bookmarksdepth=5}
\usepackage{changelog}
\usepackage{underscore}

\addbibresource{extra.bib}

\newcommand\simd[1][]{\type{simd#1}\xspace}
\newcommand\simdT{\type{simd<T>}\xspace}
\newcommand\valuetype{\type{value\_type}\xspace}
\newcommand\referencetype{\type{reference}\xspace}
\newcommand\whereexpression{\type{where\_expression}\xspace}
\newcommand\simdcast{\code{simd\_cast}\xspace}
\newcommand\mask[1][]{\type{simd\_mask#1}\xspace}
\newcommand\maskT{\type{simd\_mask<T>}\xspace}
\newcommand\fixedsizeN{\type{simd\_abi::fixed\_size<N>}\xspace}
\newcommand\fixedsizescoped{\type{simd\_abi::fixed\_size}\xspace}
\newcommand\fixedsize{\type{fixed\_size}\xspace}
\newcommand\wglink[1]{\href{https://wg21.link/#1}{#1}}

\begin{document}
\selectlanguage{american}
\begin{wgTitlepage}
  After the Parallelism TS 2 was published in 2018, data-parallel types (\simdT) have been implemented and used, and we are receiving feedback, this paper proposes to merge Section 9 of the Parallelism TS 2 into the IS working draft.
\end{wgTitlepage}

\pagestyle{scrheadings}

\section{Changelog}
\begin{revision}
  \item Added Wording.
\end{revision}

%\section{Straw Polls}
\subsection{LEWG telecon 2022-03-29}
\wgPoll
{Numeric traits can deviate from \code{numeric_limits}.}
{13&8&0&0&0}

\wgPoll
{Numeric traits should be based on representation rather than behavior (ignoring \code{reciprocal_overflow_threshold}).}
{7&5&2&0&0}

\wgPoll
{All numeric traits for bool should be disabled.}
{12&6&1&0&0}

\wgPoll
{The numeric traits that are not meaningful for \code{numeric_limits} (\code{denorm_min}, \code{epsilon}, etc) should be disabled for integral types.}
{14&3&0&0&0}

\wgPoll
{\code{max_digits10} should deviate from \code{numeric_limits} and yields \code{digits10_v + 1}.}
{6&5&2&0&0}

\subsection{LEWG telecon 2022-06-07}
\wgPoll
{Remove \code{reciprocal_overflow_threshold} from P1841.}
{6&4&1&0&0}


\section{Introduction}
\cite{P0214R9} introduced \simdT and related types and functions into the Parallelism TS 2 Section 9.
The TS was published in 2018.
An incomplete and non-conforming (because P0214 evolved) implementation existed for the whole time P0214 progressed through the committee.
Shortly after the GCC 9 release, a complete implementation of Section 9 of the TS was made available.
Since GCC 11 a complete \code{simd} implementation of the TS is part of its standard library.

Note: The first revision of this paper did not proposing a merge.
In the meantime the TS feedback progressed to a point where a merge should happen ASAP.

\subsection{Related papers}
\begin{description}
  \item[\wglink{P0350}] Before publication of the TS, SG1 approved \cite{P0350R0} which did not progress in time in LEWG to make it into the TS.
    \wglink{P0350} is moving forward independently.

  \item[\wglink{P0918}] After publication of the TS, SG1 approved \cite{P0918R2} which adds \code{shuffle}, \code{interleave}, \code{sum_to}, \code{multiply_sum_to}, and \code{saturated_simd_cast}.
    \wglink{P0918} will move forward independently.

  \item[\wglink{P1068}] R3 of the paper removed discussion/proposal of a \code{simd} based API because it was targeting \CC{}23 with the understanding of \code{simd} not being ready for \CC{}23.
    This is unfortunate as the presence of \code{simd} in the IS might lead to a considerably different assessment of the iterator/range-based API proposed in P1068.

  \item[\wglink{P0917}] The ability to write code that is generic wrt. arithmetic types and \code{simd} types is considered to be of high value (TS feedback).
    Conditional expressions via the \code{where} function were not all too well received.
    Conditional expressions via the conditional operator would provide a solution deemed perfect by those giving feedback (myself included).

  \item[draft on non-member {operator[]}] TODO

  \item[\wglink{P2600}] The fix for ADL is important to ensure the above two papers do not break existing code.

  \item[\wglink{P0543}] The paper proposing functions for saturation arithmetic expects \code{simd} overloads as soon as \code{simd} is merged to the IS.

  \item[\wglink{P0553}] The bit operations that are part of \CC{}20 expects \code{simd} overloads as soon as \code{simd} is merged to the IS.

\end{description}
The papers \wglink{P0350}, \wglink{P0918}, and the \code{simd}-based \wglink{P1068} fork currently have no shipping vehicle and are basically blocked on this paper.

\section{Changes after TS feedback}
This section is mostly a stub.
\cite{P1915R0} (Expected Feedback from \code{simd} in the Parallelism TS 2) was just published and asks for specific feedback.
After gathering feedback, the relevant changes will be added to a new revision of this paper.

\subsection{Missing \code{simd_mask} generator constructor}
The \code{simd} generator constructor is very useful for initializing objects from scalars in a portable (different \code{size()}) fashion.
The need for a similar constructor for \code{simd_mask} is less frequent, but if only for consistency, there should be one.
Besides consistency, it is also useful, of course.
Consider a predicate function that is given without \code{simd} interface (e.g. from a library).
How do you construct a \code{simd_mask} from it?
With a generator constructor it is easy:
\medskip\begin{lstlisting}[style=Vc]
simd<T> f(simd<T> x, Predicate p) {
  const simd_mask<T> k([&](auto i) { return p(x[i]); });
  where(k, x) = 0;
  return x;
}
\end{lstlisting}
Without the generator constructor one has to write e.g.:
\medskip\begin{lstlisting}[style=Vc]
simd<T> f(simd<T> x, Predicate p) {
  simd_mask<T> k;
  for (size_t i = 0; i < simd<T>::size(); ++i) {
    k[i] = p(x[i]);
  }
  where(k, x) = 0;
  return x;
}
\end{lstlisting}
The latter solution makes it hard to initialize the \code{simd_mask} as \code{const}, is more verbose, is harder to optimize, and cannot use the sequencing properties the generator constructor allows.

Therefore add:
\begin{wgText}
\begin{itemdecl}
template<class G> simd_mask(G&& gen) noexcept;
\end{itemdecl}
\end{wgText}

\subsection{Add missing casts for \code{simd_mask}}
The \code{simd_cast} and \code{static_simd_cast} overloads for \code{simd_mask} were forgotten for the TS.
Without those casts (and no casts via constructors) mixing different arithmetic types is painful.
There is no motivation for forbidding casts on \code{simd_mask}.

Therefore add the following overloads:
\begin{wgText}
\begin{codeblock}
  template<class T, class U, class Abi> @\seebelow@ simd_cast(const simd_mask<U, Abi>&) noexcept;
  template<class T, class U, class Abi> @\seebelow@ static_simd_cast(const simd_mask<U, Abi>&) noexcept;
\end{codeblock}
\end{wgText}

\subsection{\code{element_reference} is overspecified}
\code{element_reference} is spelled out in a lot of detail.
It may be better to define its requirements in a table instead.

This change is not reflected in the wording, pending encouragement from WG21 (mostly LWG).

\subsection{Clean up math function overloads}
The wording that produces \code{simd} overloads misses a few cases and leaves room for ambiguity.
There is also no explicit mention of integral overloads that are supported in \code{<cmath>} (e.g. \code{std::cos(1)} calling \code{std::cos(double)}).

This needs more work and is not reflected in the wording at this point.


\section{Open questions}

\subsection{Integration with ranges}
\code{simd} itself is not a range.
The value of a data-parallel object is not an array of elements but rather needs to be understood as a single opaque value that happens to have means for reading and writing element values.
I.e. \code{simd<int> x = \{\};} does not start the lifetime of \type{int} objects.
The \code{element_reference} for \code{operator[]} is bad enough as is.
Let's not exacerbate the problem by adding iterators to \code{simd}.

Instead, a \code{simd} can be converted into an array (e.g. \lst{simdtoarray}).
Conversely, a \std\code{span} with static extent can be converted into a \std\type{simd}.
\begin{lstlisting}[numbers=left,float={hbtp},label=lst:simdtoarray,caption={
  \code{simd} to \code{array} conversion
}]
template <class T, class A>
std::array<T, std::simd_size_v<T, A>> to_array(std::simd<T, A> x)
{
  std::array<T, std::simd_size_v<T, A>> r;
  x.copy_to(r.data());
  return r;
}
\end{lstlisting}

I plan to pursue conversions to array and from span with static extent in a follow-up paper.
I believe it is not necessary to resolve this question before merging \code{simd} from the TS.

\subsection{Correct place for \code{simd} in the IS?}

While \code{simd} is certainly very important for numerics and therefore fits into the “Numerics library” clause, it is also more than that.
E.g. \code{simd} can be used for vectorization of text processing.
In principle \code{simd} should be understood similar to fundamental types.
Is the “General utilities library” clause a better place?
Or rename “Concurrency support library” to “Parallelism and concurrency support library” and put it there?
Alternatively, add a new library clause?

\section{Wording}

The following section presents the wording to be applied against the \CC{} working draft.
The subsequent \sect{diff} reproduces the same wording as a diff against the Parallelism TS 2.

\subsection{Add Section 9 of N4808 with modifications}\label{sec:wording}

\begin{wgText}[Add a new subclause after §28.8 {[numbers]}]
  \def\wgRem#1{}
  \def\wgAdd#1{#1}
  \def\wgChange#1#2{#2}
  \colorlet{WgAdd}{black}
  \colorlet{WgRem}{white}
  \setcounter{WGClause}{28}
  \setcounter{WGSubSection}{8}
  \lstset{%
    columns=fullflexible,
    deletedelim=**[is]{|-}{-|},
    moredelim=[is][\color{white}\fontsize{0.1pt}{0.1pt}\selectfont{}]{|-}{-|}
  }
  \section{Wording}

The following is a rough draft of possible wording that defines a basic set of data-parallel types.

\begin{wgText}
  \wgSection{Data-Parallel Types}{datapar.types}

  \wgSubsection{Header \code{<datapar>} synopsis}{datapar.syn}
  \lstinputlisting[]{synopsis.cpp}

  \pnum
  The header \code{<datapar>} defines two class templates (\datapar, and \mask), several tag types, and a series of related function templates for concurrent manipulation of the values in \datapar and \mask objects.

  \begin{itemdecl}
namespace datapar_abi {
  struct scalar {};
  // implementation-defined tag types, e.g. sse, avx, avx512, neon, ...
  typedef implementation_defined compatible;
  typedef implementation_defined native;
}
  \end{itemdecl}
  \begin{itemdescr}
    \pnum
    The ABI types are tag types to be used as the second template argument to \datapar and \mask.

    \pnum
    The \type{scalar} tag is present in all implementations and forces \datapar and \mask to store a single component (i.e. \datapar{}\type{<T, datapar_abi::scalar>::size()} returns \code 1).

    \pnum
    An implementation may choose to implement data-parallel execution for many different targets.
    \wgNote{There can certainly be more than one tag type per (micro-)architecture, e.g. to support different vector lengths or partial register usage.}
    All tag types an implementation supports shall be present independent of the chosen target.

    \pnum
    The \type{datapar_abi::compatible} tag is defined by the implementation to alias the tag type with the most efficient data parallel execution that ensures the highest compatibility on the target architecture.

    \pnum
    The \type{datapar_abi::native} tag is defined by the implementation to alias the tag type with the most efficient data parallel execution that is supported on the target system.
  \end{itemdescr}

  \begin{itemdecl}
template <class T, size_t N> struct abi_for_width { typedef implementation_defined type; };
  \end{itemdecl}
  \begin{itemdescr}
    \pnum
    The \type{abi_for_width} class template defines the member type \type{type} to one of the tag types in \code{datapar_abi} or not at all, depending on the value of the template parameters.

    \pnum
    \code{datapar<T, abi_for_width_t<T, N>>::size()} must return \code N or result in a substitution failure.
  \end{itemdescr}

  \begin{itemdecl}
template <class T, class Abi = datapar_abi::compatible>
struct datapar_size : public integral_constant<size_t, implementation_defined> {};
  \end{itemdecl}
  \begin{itemdescr}
    \pnum The \type{datapar_size} class template inherits from \type{integral_constant} with a value that equals \datapar{}\code{<T, Abi>::size()}.

    \pnum \code{datapar_size<T, Abi>::value} shall result in a substitution failure if any of the template arguments \type T and \type{Abi} are invalid template arguments to \datapar.
  \end{itemdescr}

  \wgSubsection{Class template \datapar}{datapar}
  \wgSubsubsection{Class template \datapar overview}{datapar.overview}
  \lstinputlisting[]{datapar.cpp}

  \pnum The class template \datapar{}\type{<T, Abi>} is a one-dimensional smart array.
  In contrast to \type{valarray} (26.6), the number of elements in the array is determined at compile time, according to the \type{Abi} template parameter.

  \pnum The first template argument \type T must be an integral or floating-point fundamental type.
  The type \bool is not allowed.

  \pnum The second template argument \type{Abi} must be a tag type from the \code{datapar_abi} namespace.

  \begin{itemdecl}
typedef implementation_defined native_handle_type;
  \end{itemdecl}
  \begin{itemdescr}
    \pnum
    The \type{native_handle_type} member type is an alias for the \code{native_handle()} member function return type.
    It is used to expose an implementation-defined handle for implementation- and target-specific extensions.
  \end{itemdescr}

  \begin{itemdecl}
typedef implementation_defined register_value_type;
  \end{itemdecl}
  \begin{itemdescr}
  \end{itemdescr}

  \wgSubsubsection{\datapar constructors}{datapar.ctor}
  \begin{itemdecl}
datapar() = default;
  \end{itemdecl}
  \begin{itemdescr}
    \pnum
    \effects
    Constructs an object with all elements initialized to \code{T()}.
    \wgNote{This zero-initializes the object.}
  \end{itemdescr}

  \begin{itemdecl}
datapar(value_type);
  \end{itemdecl}
  \begin{itemdescr}
    \pnum
    \effects
    Constructs an object with each element initialized to the value of the argument.
  \end{itemdescr}

  \begin{itemdecl}
template <typename U> datapar(datapar<U, Abi> x);
  \end{itemdecl}
  \begin{itemdescr}
    \pnum\remarks This constructor shall not participate in overload resolution unless
                  \type U and \type T are different integral types and
                  \code{make_signed<U>::type} equals \code{make_signed<T>::type}.
    \pnum\effects Constructs an object of type \datapar.
    \pnum\postcondition The $i$-th element equals \code{static_cast<T>(x[i])} for all elements.
  \end{itemdescr}

  \wgSubsubsection{\datapar load functions}{datapar.load}
  \begin{itemdecl}
static datapar load(const value_type *x);
  \end{itemdecl}
  \begin{itemdescr}
    \pnum \effects Constructs an object with each element $i$ initialized to \code{x[i]} for all elements.
    \pnum \returns The constructed object.
    \pnum \remarks If \datapar{}\code{::size()} is greater than the number of values pointed to by the argument, the behavior is undefined.
  \end{itemdescr}

  \begin{itemdecl}
template <typename Flags> static datapar load(const value_type *x, Flags);
  \end{itemdecl}
  \begin{itemdescr}
    \pnum\effects Constructs an object with each element $i$ initialized to \code{x[i]}.
    \pnum\returns The constructed object.
    \pnum\remarks If \datapar{}\code{::size()} is greater than the number of values pointed to by the first argument, the behavior is undefined.
    \pnum         If the template parameter is of type \type{aligned_tag} and the pointer value is not a multiple of \code{memory_alignment<\type T>}, the behavior is undefined.
  \end{itemdescr}

  \begin{itemdecl}
template <typename U, typename Flags = unaligned_tag> static datapar load(const U *x, Flags = Flags());
  \end{itemdecl}
  \begin{itemdescr}
    \pnum\effects Constructs an object with each element $i$ initialized to \code{static_cast<T>(x[i])}.
    \pnum\returns The constructed object.
    \pnum\remarks If \datapar{}\code{::size()} is greater than the number of values pointed to by the first argument, the behavior is undefined.
    \pnum         If the second template parameter is of type \type{aligned_tag} and the pointer value is not a multiple of \code{memory_alignment<\type U>}, the behavior is undefined.
  \end{itemdescr}

  \wgSubsubsection{\datapar store functions}{datapar.store}
  \begin{itemdecl}
void store(value_type *x);
  \end{itemdecl}
  \begin{itemdescr}
    \pnum\effects Copies each element such that the $i$-th element is stored to \code{x[i]}.
    \pnum\remarks If \datapar{}\code{::size()} is greater than the number of values pointed to by the first argument, the behavior is undefined.
  \end{itemdescr}

  \begin{itemdecl}
template <typename Flags> void store(value_type *x, Flags);
  \end{itemdecl}
  \begin{itemdescr}
    \pnum\effects Copies each element such that the $i$-th element is stored to \code{x[i]}.
    \pnum\remarks If \datapar{}\code{::size()} is greater than the number of values pointed to by the first argument, the behavior is undefined.
    \pnum         If the template parameter is of type \type{aligned_tag} and the pointer value is not a multiple of \code{memory_alignment<\type T>}, the behavior is undefined.
  \end{itemdescr}

  \begin{itemdecl}
template <typename U, typename Flags = unaligned_tag> void store(U *x, Flags = Flags());
  \end{itemdecl}
  \begin{itemdescr}
    \pnum\effects Copies each element such that the $i$-th element is first converted to \type U and then stored to \code{x[i]}.
    \pnum\remarks If \datapar{}\code{::size()} is greater than the number of values pointed to by the first argument, the behavior is undefined.
    \pnum         If the second template parameter is of type \type{aligned_tag} and the pointer value is not a multiple of \code{memory_alignment<\type U>}, the behavior is undefined.
  \end{itemdescr}

  \begin{itemdecl}
void store(value_type *x, mask_type);
  \end{itemdecl}
  \begin{itemdescr}
    \pnum\effects Copies each element where the corresponding element in the second argument is \true such that the $i$-th element is stored to \code{x[i]}.
    \pnum\remarks If the largest $i$ where the second argument is \true is greater than the number of values pointed to by the first argument, the behavior is undefined.
  \end{itemdescr}

  \begin{itemdecl}
template <typename Flags> void store(value_type *x, mask_type, Flags);
  \end{itemdecl}
  \begin{itemdescr}
    \pnum\effects Copies each element where the corresponding element in the second argument is \true such that the $i$-th element is stored to \code{x[i]}.
    \pnum\remarks If the largest $i$ where the second argument is \true is greater than the number of values pointed to by the first argument, the behavior is undefined.
    \pnum         If the template parameter is of type \type{aligned_tag} and the pointer value is not a multiple of \code{memory_alignment<\type T>}, the behavior is undefined.
  \end{itemdescr}

  \begin{itemdecl}
template <typename U, typename Flags = unaligned_tag> void store(U *x, mask_type, Flags = Flags());
  \end{itemdecl}
  \begin{itemdescr}
    \pnum\effects Copies each element where the corresponding element in the second argument is \true such that the $i$-th element is first converted to \type U and then stored to \code{x[i]}.
    \pnum\remarks If the largest $i$ where the second argument is \true is greater than the number of values pointed to by the first argument, the behavior is undefined.
    \pnum         If the template parameter is of type \type{aligned_tag} and the pointer value is not a multiple of \code{memory_alignment<\type U>}, the behavior is undefined.
  \end{itemdescr}

  \wgSubsubsection{\datapar subscript operators}{datapar.subscr}
  \begin{itemdecl}
reference operator[](size_type i);
  \end{itemdecl}
  \begin{itemdescr}
    \pnum\returns An lvalue reference to the $i$-th element.
    \pnum\postconditions Assignment of objects of type \type T modify the $i$-th element without aliasing violations.
    \pnum                Modification of \code{*this} does not invalidate references held to the return value.
    Subsequent reads from such references yield the new value of the $i$-th element.
  \end{itemdescr}

  \begin{itemdecl}
const_reference operator[](size_type) const;
  \end{itemdecl}
  \begin{itemdescr}
    \pnum\returns A \const lvalue reference to the $i$-th element.
    \pnum\postconditions Modification of \code{*this} does not invalidate references held to the return value.
    Subsequent reads from such references yield the new value of the $i$-th element.
  \end{itemdescr}

  \wgSubsubsection{\datapar unary operators}{datapar.unary}
  \begin{itemdecl}
datapar &operator++();
  \end{itemdecl}
  \begin{itemdescr}
    \pnum\effects Increments every element of \code{*this} by one.
    \pnum\returns An lvalue reference to \code{*this} after incrementing.
    \pnum\remarks Overflow semantics follow the same semantics as for \type T.
  \end{itemdescr}

  \begin{itemdecl}
datapar operator++(int);
  \end{itemdecl}
  \begin{itemdescr}
    \pnum\effects Increments every element of \code{*this} by one.
    \pnum\returns A copy of \code{*this} before incrementing.
    \pnum\remarks Overflow semantics follow the same semantics as for \type T.
  \end{itemdescr}

  \begin{itemdecl}
datapar &operator--();
  \end{itemdecl}
  \begin{itemdescr}
    \pnum\effects Decrements every element of \code{*this} by one.
    \pnum\returns An lvalue reference to \code{*this} after decrementing.
    \pnum\remarks Underflow semantics follow the same semantics as for \type T.
  \end{itemdescr}

  \begin{itemdecl}
datapar operator--(int);
  \end{itemdecl}
  \begin{itemdescr}
    \pnum\effects Decrements every element of \code{*this} by one.
    \pnum\returns A copy of \code{*this} before decrementing.
    \pnum\remarks Underflow semantics follow the same semantics as for \type T.
  \end{itemdescr}

  \begin{itemdecl}
mask_type operator!() const;
  \end{itemdecl}
  \begin{itemdescr}
    \pnum\returns A mask object with the $i$-th element set to \code{!operator[](i)} for all elements.
  \end{itemdescr}

  \begin{itemdecl}
datapar operator~() const;
  \end{itemdecl}
  \begin{itemdescr}
    \pnum\requires The first template argument \type T to \datapar must be an integral type.
    \pnum\effects Constructs an object where each bit of \code{*this} is inverted.
    \pnum\returns The new object.
    \pnum\remarks \datapar{}\code{::operator\textasciitilde{}()} shall not participate in overload resolution if \type T is a floating-point type.
  \end{itemdescr}

  \begin{itemdecl}
datapar operator+() const;
  \end{itemdecl}
  \begin{itemdescr}
    \pnum \returns A copy of \code{*this}
  \end{itemdescr}

  \begin{itemdecl}
datapar operator-() const;
  \end{itemdecl}
  \begin{itemdescr}
    \pnum\effects Constructs an object where the $i$-th element is initialized to \code{-operator[](i)} for all elements.
    \pnum\returns The new object.
  \end{itemdescr}

  \wgSubsubsection{\datapar native handles}{datapar.native}
  \begin{itemdecl}
native_handle_type &native_handle();
  \end{itemdecl}
  \begin{itemdescr}
    \pnum\returns An lvalue reference to the implementation-specific object implementing the data-parallel semantics.
  \end{itemdescr}

  \begin{itemdecl}
const native_handle_type &native_handle() const;
  \end{itemdecl}
  \begin{itemdescr}
    \pnum\returns A \const lvalue reference to the implementation-specific object implementing the data-parallel semantics.
  \end{itemdescr}

  \wgSubsection{\datapar non-member operations}{datapar.nonmembers}
  \wgSubsubsection{\datapar binary operators}{datapar.binary}
  \begin{itemdecl}
template <class L, class R> using datapar_return_type = ...;  // exposition only
template <class T, class Abi, class U>
datapar_return_type<datapar<T, Abi>, U> operator+ (datapar<T, Abi>, const U &);
template <class T, class Abi, class U>
datapar_return_type<datapar<T, Abi>, U> operator- (datapar<T, Abi>, const U &);
template <class T, class Abi, class U>
datapar_return_type<datapar<T, Abi>, U> operator* (datapar<T, Abi>, const U &);
template <class T, class Abi, class U>
datapar_return_type<datapar<T, Abi>, U> operator/ (datapar<T, Abi>, const U &);
template <class T, class Abi, class U>
datapar_return_type<datapar<T, Abi>, U> operator% (datapar<T, Abi>, const U &);
template <class T, class Abi, class U>
datapar_return_type<datapar<T, Abi>, U> operator& (datapar<T, Abi>, const U &);
template <class T, class Abi, class U>
datapar_return_type<datapar<T, Abi>, U> operator| (datapar<T, Abi>, const U &);
template <class T, class Abi, class U>
datapar_return_type<datapar<T, Abi>, U> operator^ (datapar<T, Abi>, const U &);
template <class T, class Abi, class U>
datapar_return_type<datapar<T, Abi>, U> operator<<(datapar<T, Abi>, const U &);
template <class T, class Abi, class U>
datapar_return_type<datapar<T, Abi>, U> operator>>(datapar<T, Abi>, const U &);
template <class T, class Abi, class U>
datapar_return_type<datapar<T, Abi>, U> operator+ (const U &, datapar<T, Abi>);
template <class T, class Abi, class U>
datapar_return_type<datapar<T, Abi>, U> operator- (const U &, datapar<T, Abi>);
template <class T, class Abi, class U>
datapar_return_type<datapar<T, Abi>, U> operator* (const U &, datapar<T, Abi>);
template <class T, class Abi, class U>
datapar_return_type<datapar<T, Abi>, U> operator/ (const U &, datapar<T, Abi>);
template <class T, class Abi, class U>
datapar_return_type<datapar<T, Abi>, U> operator% (const U &, datapar<T, Abi>);
template <class T, class Abi, class U>
datapar_return_type<datapar<T, Abi>, U> operator& (const U &, datapar<T, Abi>);
template <class T, class Abi, class U>
datapar_return_type<datapar<T, Abi>, U> operator| (const U &, datapar<T, Abi>);
template <class T, class Abi, class U>
datapar_return_type<datapar<T, Abi>, U> operator^ (const U &, datapar<T, Abi>);
template <class T, class Abi, class U>
datapar_return_type<datapar<T, Abi>, U> operator<<(const U &, datapar<T, Abi>);
template <class T, class Abi, class U>
datapar_return_type<datapar<T, Abi>, U> operator>>(const U &, datapar<T, Abi>);
  \end{itemdecl}
  \begin{itemdescr}
    \pnum\remarks The return type of these operators shall be deduced according to the following rules:
    \begin{itemize}
      \item If \code{is_datapar_v<U> == true}
        then the return type shall be determined from \type T and \type{U::value_type} according to the following paragraph.
      \item Otherwise, if \type T is integral and \type U is \intt
        the return type shall be \datapar{}\type{<T, Abi>}.
      \item Otherwise, if \type T is integral and \type U is \uint
        the return type shall be \datapar{}\code{<make_unsigned_t<T>, Abi>}.
      \item Otherwise, if \type U is a fundamental arithmetic type or \type U is convertible to \intt
        then the return type shall be determined from \type T and \type U according to the following paragraph.
      \item Otherwise, if \type U is implicitly convertible to \datapar{}\type{<V, Abi>}, where \type V is determined according to standard template type deduction,
        then the return type shall be determined from \type T and \type V according to the following paragraph.
      \item Otherwise, if \type U is implicitly convertible to \datapar{}\type{<T, Abi>},
        the return type shall be \datapar{}\type{<T, Abi>}.
      \item Otherwise no return type is defined (SFINAE).
    \end{itemize}

    \pnum\remarks Given the types \type T and \type{Abi} from the class template argument list and a third type \type U determined by the rules of the previous paragraph a return type is deduced according to the following rules:
    \begin{itemize}
      \item If \type U is not a fundamental arithmetic type then the return type shall be \datapar{}\type{<T, Abi>}.
      \item Otherwise, if at least one of the  types \type T and \type U is a floating-point type
        the return type shall be \datapar{}\type{<decltype(T() + U()), Abi>}.
      \item Otherwise, if \code{sizeof(T) < sizeof(U)} the return type shall be \datapar{}\type{<U, Abi>}.
      \item Otherwise, if \code{sizeof(T) > sizeof(U)} the return type shall be \datapar{}\type{<T, Abi>}.
      \item Otherwise, the type \type T or \type U that is farthest back in the list of \textit{standard integer types} (cf. [basic.fundamental]) is used as type \type V and
        the return type shall be \datapar{}\type{<V, Abi>} if both types \type T and \type U are signed, otherwise the return type shall be \datapar{}\type{<make_unsigned_t<V>, Abi>}.
    \end{itemize}

    \pnum\remarks Each of these operators only participates in overload resolution if all of the following hold:
    \begin{itemize}
      \item The indicated operator can be applied to objects of type \type{R::value_type}, with \type R the return type.
      \item \datapar{}\type{<T, Abi>} is implicitly convertible to the return type.
      \item \type U is implicitly convertible to the return type.
    \end{itemize}

    \pnum\remarks The operators with \type{const U \&} as first parameter shall not participate in overload resolution if \code{is_datapar_v<U> == true}.

    \pnum\effects Both arguments are first converted to the return type.
      Each of these operators subsequently performs the indicated operation component-wise on each of the elements of the first argument and the corresponding element of the second argument.
    \pnum\returns An object containing the results of the component-wise operator application.
  \end{itemdescr}

  \wgSubsubsection{\datapar compound assignment}{datapar.cassign}
  \begin{itemdecl}
template <class T, class Abi, class U> datapar<T, Abi> &operator+= (datapar<T, Abi> &, const U &);
template <class T, class Abi, class U> datapar<T, Abi> &operator-= (datapar<T, Abi> &, const U &);
template <class T, class Abi, class U> datapar<T, Abi> &operator*= (datapar<T, Abi> &, const U &);
template <class T, class Abi, class U> datapar<T, Abi> &operator/= (datapar<T, Abi> &, const U &);
template <class T, class Abi, class U> datapar<T, Abi> &operator%= (datapar<T, Abi> &, const U &);
template <class T, class Abi, class U> datapar<T, Abi> &operator&= (datapar<T, Abi> &, const U &);
template <class T, class Abi, class U> datapar<T, Abi> &operator|= (datapar<T, Abi> &, const U &);
template <class T, class Abi, class U> datapar<T, Abi> &operator^= (datapar<T, Abi> &, const U &);
template <class T, class Abi, class U> datapar<T, Abi> &operator<<=(datapar<T, Abi> &, const U &);
template <class T, class Abi, class U> datapar<T, Abi> &operator>>=(datapar<T, Abi> &, const U &);
  \end{itemdecl}
  \begin{itemdescr}
    \pnum\remarks Each of these operators only participates in overload resolution if all of the following hold:
    \begin{itemize}
      \item The indicated operator can be applied to objects of type \type{datapar_return_type<datapar<T, Abi>, U>::value_type}.
      \item \datapar{}\type{<T, Abi>} is implicitly convertible to \type{datapar_return_type<datapar<T, Abi>, U>}.
      \item \type U is implicitly convertible to \type{datapar_return_type<datapar<T, Abi>, U>}.
      \item \type{datapar_return_type<datapar<T, Abi>, U>} is implicitly convertible to \datapar{}\type{<T, Abi>}.
    \end{itemize}
    \pnum\effects Each of these operators performs the indicated operation component-wise on each of the elements of the first argument and the corresponding element of the second argument after conversion to \datapar{}\code{<T, Abi>}.
    \pnum\returns A reference to the first argument.
  \end{itemdescr}

  \wgSubsubsection{\datapar logical operators}{datapar.comparison}

  \wgSubsubsection{\datapar transcendentals}{datapar.transcend}

  \wgSubsection{Class template \mask}{datapar.mask}
  \lstinputlisting[]{mask.cpp}

\end{wgText}

\end{wgText}

\subsection{Diff against Parallelism TS 2 (N4808)}\label{sec:diff}

In the following, the wording from \sect{wording} is repeated with additional indications of differences with regard to N4808.
Changes relative to N4808, which contains editorial changes after the publication of the TS, are marked using color for \textcolor{WgAdd}{additions} and \textcolor{WgRem}{removals}.

\def\wgLabelPrefix{diff}%
\begin{wgText}
  \setcounter{WGClause}{28}
  \setcounter{WGSubSection}{8}
  \section{Wording}

The following is a rough draft of possible wording that defines a basic set of data-parallel types.

\begin{wgText}
  \wgSection{Data-Parallel Types}{datapar.types}

  \wgSubsection{Header \code{<datapar>} synopsis}{datapar.syn}
  \lstinputlisting[]{synopsis.cpp}

  \pnum
  The header \code{<datapar>} defines two class templates (\datapar, and \mask), several tag types, and a series of related function templates for concurrent manipulation of the values in \datapar and \mask objects.

  \begin{itemdecl}
namespace datapar_abi {
  struct scalar {};
  // implementation-defined tag types, e.g. sse, avx, avx512, neon, ...
  typedef implementation_defined compatible;
  typedef implementation_defined native;
}
  \end{itemdecl}
  \begin{itemdescr}
    \pnum
    The ABI types are tag types to be used as the second template argument to \datapar and \mask.

    \pnum
    The \type{scalar} tag is present in all implementations and forces \datapar and \mask to store a single component (i.e. \datapar{}\type{<T, datapar_abi::scalar>::size()} returns \code 1).

    \pnum
    An implementation may choose to implement data-parallel execution for many different targets.
    \wgNote{There can certainly be more than one tag type per (micro-)architecture, e.g. to support different vector lengths or partial register usage.}
    All tag types an implementation supports shall be present independent of the chosen target.

    \pnum
    The \type{datapar_abi::compatible} tag is defined by the implementation to alias the tag type with the most efficient data parallel execution that ensures the highest compatibility on the target architecture.

    \pnum
    The \type{datapar_abi::native} tag is defined by the implementation to alias the tag type with the most efficient data parallel execution that is supported on the target system.
  \end{itemdescr}

  \begin{itemdecl}
template <class T, size_t N> struct abi_for_width { typedef implementation_defined type; };
  \end{itemdecl}
  \begin{itemdescr}
    \pnum
    The \type{abi_for_width} class template defines the member type \type{type} to one of the tag types in \code{datapar_abi} or not at all, depending on the value of the template parameters.

    \pnum
    \code{datapar<T, abi_for_width_t<T, N>>::size()} must return \code N or result in a substitution failure.
  \end{itemdescr}

  \begin{itemdecl}
template <class T, class Abi = datapar_abi::compatible>
struct datapar_size : public integral_constant<size_t, implementation_defined> {};
  \end{itemdecl}
  \begin{itemdescr}
    \pnum The \type{datapar_size} class template inherits from \type{integral_constant} with a value that equals \datapar{}\code{<T, Abi>::size()}.

    \pnum \code{datapar_size<T, Abi>::value} shall result in a substitution failure if any of the template arguments \type T and \type{Abi} are invalid template arguments to \datapar.
  \end{itemdescr}

  \wgSubsection{Class template \datapar}{datapar}
  \wgSubsubsection{Class template \datapar overview}{datapar.overview}
  \lstinputlisting[]{datapar.cpp}

  \pnum The class template \datapar{}\type{<T, Abi>} is a one-dimensional smart array.
  In contrast to \type{valarray} (26.6), the number of elements in the array is determined at compile time, according to the \type{Abi} template parameter.

  \pnum The first template argument \type T must be an integral or floating-point fundamental type.
  The type \bool is not allowed.

  \pnum The second template argument \type{Abi} must be a tag type from the \code{datapar_abi} namespace.

  \begin{itemdecl}
typedef implementation_defined native_handle_type;
  \end{itemdecl}
  \begin{itemdescr}
    \pnum
    The \type{native_handle_type} member type is an alias for the \code{native_handle()} member function return type.
    It is used to expose an implementation-defined handle for implementation- and target-specific extensions.
  \end{itemdescr}

  \begin{itemdecl}
typedef implementation_defined register_value_type;
  \end{itemdecl}
  \begin{itemdescr}
  \end{itemdescr}

  \wgSubsubsection{\datapar constructors}{datapar.ctor}
  \begin{itemdecl}
datapar() = default;
  \end{itemdecl}
  \begin{itemdescr}
    \pnum
    \effects
    Constructs an object with all elements initialized to \code{T()}.
    \wgNote{This zero-initializes the object.}
  \end{itemdescr}

  \begin{itemdecl}
datapar(value_type);
  \end{itemdecl}
  \begin{itemdescr}
    \pnum
    \effects
    Constructs an object with each element initialized to the value of the argument.
  \end{itemdescr}

  \begin{itemdecl}
template <typename U> datapar(datapar<U, Abi> x);
  \end{itemdecl}
  \begin{itemdescr}
    \pnum\remarks This constructor shall not participate in overload resolution unless
                  \type U and \type T are different integral types and
                  \code{make_signed<U>::type} equals \code{make_signed<T>::type}.
    \pnum\effects Constructs an object of type \datapar.
    \pnum\postcondition The $i$-th element equals \code{static_cast<T>(x[i])} for all elements.
  \end{itemdescr}

  \wgSubsubsection{\datapar load functions}{datapar.load}
  \begin{itemdecl}
static datapar load(const value_type *x);
  \end{itemdecl}
  \begin{itemdescr}
    \pnum \effects Constructs an object with each element $i$ initialized to \code{x[i]} for all elements.
    \pnum \returns The constructed object.
    \pnum \remarks If \datapar{}\code{::size()} is greater than the number of values pointed to by the argument, the behavior is undefined.
  \end{itemdescr}

  \begin{itemdecl}
template <typename Flags> static datapar load(const value_type *x, Flags);
  \end{itemdecl}
  \begin{itemdescr}
    \pnum\effects Constructs an object with each element $i$ initialized to \code{x[i]}.
    \pnum\returns The constructed object.
    \pnum\remarks If \datapar{}\code{::size()} is greater than the number of values pointed to by the first argument, the behavior is undefined.
    \pnum         If the template parameter is of type \type{aligned_tag} and the pointer value is not a multiple of \code{memory_alignment<\type T>}, the behavior is undefined.
  \end{itemdescr}

  \begin{itemdecl}
template <typename U, typename Flags = unaligned_tag> static datapar load(const U *x, Flags = Flags());
  \end{itemdecl}
  \begin{itemdescr}
    \pnum\effects Constructs an object with each element $i$ initialized to \code{static_cast<T>(x[i])}.
    \pnum\returns The constructed object.
    \pnum\remarks If \datapar{}\code{::size()} is greater than the number of values pointed to by the first argument, the behavior is undefined.
    \pnum         If the second template parameter is of type \type{aligned_tag} and the pointer value is not a multiple of \code{memory_alignment<\type U>}, the behavior is undefined.
  \end{itemdescr}

  \wgSubsubsection{\datapar store functions}{datapar.store}
  \begin{itemdecl}
void store(value_type *x);
  \end{itemdecl}
  \begin{itemdescr}
    \pnum\effects Copies each element such that the $i$-th element is stored to \code{x[i]}.
    \pnum\remarks If \datapar{}\code{::size()} is greater than the number of values pointed to by the first argument, the behavior is undefined.
  \end{itemdescr}

  \begin{itemdecl}
template <typename Flags> void store(value_type *x, Flags);
  \end{itemdecl}
  \begin{itemdescr}
    \pnum\effects Copies each element such that the $i$-th element is stored to \code{x[i]}.
    \pnum\remarks If \datapar{}\code{::size()} is greater than the number of values pointed to by the first argument, the behavior is undefined.
    \pnum         If the template parameter is of type \type{aligned_tag} and the pointer value is not a multiple of \code{memory_alignment<\type T>}, the behavior is undefined.
  \end{itemdescr}

  \begin{itemdecl}
template <typename U, typename Flags = unaligned_tag> void store(U *x, Flags = Flags());
  \end{itemdecl}
  \begin{itemdescr}
    \pnum\effects Copies each element such that the $i$-th element is first converted to \type U and then stored to \code{x[i]}.
    \pnum\remarks If \datapar{}\code{::size()} is greater than the number of values pointed to by the first argument, the behavior is undefined.
    \pnum         If the second template parameter is of type \type{aligned_tag} and the pointer value is not a multiple of \code{memory_alignment<\type U>}, the behavior is undefined.
  \end{itemdescr}

  \begin{itemdecl}
void store(value_type *x, mask_type);
  \end{itemdecl}
  \begin{itemdescr}
    \pnum\effects Copies each element where the corresponding element in the second argument is \true such that the $i$-th element is stored to \code{x[i]}.
    \pnum\remarks If the largest $i$ where the second argument is \true is greater than the number of values pointed to by the first argument, the behavior is undefined.
  \end{itemdescr}

  \begin{itemdecl}
template <typename Flags> void store(value_type *x, mask_type, Flags);
  \end{itemdecl}
  \begin{itemdescr}
    \pnum\effects Copies each element where the corresponding element in the second argument is \true such that the $i$-th element is stored to \code{x[i]}.
    \pnum\remarks If the largest $i$ where the second argument is \true is greater than the number of values pointed to by the first argument, the behavior is undefined.
    \pnum         If the template parameter is of type \type{aligned_tag} and the pointer value is not a multiple of \code{memory_alignment<\type T>}, the behavior is undefined.
  \end{itemdescr}

  \begin{itemdecl}
template <typename U, typename Flags = unaligned_tag> void store(U *x, mask_type, Flags = Flags());
  \end{itemdecl}
  \begin{itemdescr}
    \pnum\effects Copies each element where the corresponding element in the second argument is \true such that the $i$-th element is first converted to \type U and then stored to \code{x[i]}.
    \pnum\remarks If the largest $i$ where the second argument is \true is greater than the number of values pointed to by the first argument, the behavior is undefined.
    \pnum         If the template parameter is of type \type{aligned_tag} and the pointer value is not a multiple of \code{memory_alignment<\type U>}, the behavior is undefined.
  \end{itemdescr}

  \wgSubsubsection{\datapar subscript operators}{datapar.subscr}
  \begin{itemdecl}
reference operator[](size_type i);
  \end{itemdecl}
  \begin{itemdescr}
    \pnum\returns An lvalue reference to the $i$-th element.
    \pnum\postconditions Assignment of objects of type \type T modify the $i$-th element without aliasing violations.
    \pnum                Modification of \code{*this} does not invalidate references held to the return value.
    Subsequent reads from such references yield the new value of the $i$-th element.
  \end{itemdescr}

  \begin{itemdecl}
const_reference operator[](size_type) const;
  \end{itemdecl}
  \begin{itemdescr}
    \pnum\returns A \const lvalue reference to the $i$-th element.
    \pnum\postconditions Modification of \code{*this} does not invalidate references held to the return value.
    Subsequent reads from such references yield the new value of the $i$-th element.
  \end{itemdescr}

  \wgSubsubsection{\datapar unary operators}{datapar.unary}
  \begin{itemdecl}
datapar &operator++();
  \end{itemdecl}
  \begin{itemdescr}
    \pnum\effects Increments every element of \code{*this} by one.
    \pnum\returns An lvalue reference to \code{*this} after incrementing.
    \pnum\remarks Overflow semantics follow the same semantics as for \type T.
  \end{itemdescr}

  \begin{itemdecl}
datapar operator++(int);
  \end{itemdecl}
  \begin{itemdescr}
    \pnum\effects Increments every element of \code{*this} by one.
    \pnum\returns A copy of \code{*this} before incrementing.
    \pnum\remarks Overflow semantics follow the same semantics as for \type T.
  \end{itemdescr}

  \begin{itemdecl}
datapar &operator--();
  \end{itemdecl}
  \begin{itemdescr}
    \pnum\effects Decrements every element of \code{*this} by one.
    \pnum\returns An lvalue reference to \code{*this} after decrementing.
    \pnum\remarks Underflow semantics follow the same semantics as for \type T.
  \end{itemdescr}

  \begin{itemdecl}
datapar operator--(int);
  \end{itemdecl}
  \begin{itemdescr}
    \pnum\effects Decrements every element of \code{*this} by one.
    \pnum\returns A copy of \code{*this} before decrementing.
    \pnum\remarks Underflow semantics follow the same semantics as for \type T.
  \end{itemdescr}

  \begin{itemdecl}
mask_type operator!() const;
  \end{itemdecl}
  \begin{itemdescr}
    \pnum\returns A mask object with the $i$-th element set to \code{!operator[](i)} for all elements.
  \end{itemdescr}

  \begin{itemdecl}
datapar operator~() const;
  \end{itemdecl}
  \begin{itemdescr}
    \pnum\requires The first template argument \type T to \datapar must be an integral type.
    \pnum\effects Constructs an object where each bit of \code{*this} is inverted.
    \pnum\returns The new object.
    \pnum\remarks \datapar{}\code{::operator\textasciitilde{}()} shall not participate in overload resolution if \type T is a floating-point type.
  \end{itemdescr}

  \begin{itemdecl}
datapar operator+() const;
  \end{itemdecl}
  \begin{itemdescr}
    \pnum \returns A copy of \code{*this}
  \end{itemdescr}

  \begin{itemdecl}
datapar operator-() const;
  \end{itemdecl}
  \begin{itemdescr}
    \pnum\effects Constructs an object where the $i$-th element is initialized to \code{-operator[](i)} for all elements.
    \pnum\returns The new object.
  \end{itemdescr}

  \wgSubsubsection{\datapar native handles}{datapar.native}
  \begin{itemdecl}
native_handle_type &native_handle();
  \end{itemdecl}
  \begin{itemdescr}
    \pnum\returns An lvalue reference to the implementation-specific object implementing the data-parallel semantics.
  \end{itemdescr}

  \begin{itemdecl}
const native_handle_type &native_handle() const;
  \end{itemdecl}
  \begin{itemdescr}
    \pnum\returns A \const lvalue reference to the implementation-specific object implementing the data-parallel semantics.
  \end{itemdescr}

  \wgSubsection{\datapar non-member operations}{datapar.nonmembers}
  \wgSubsubsection{\datapar binary operators}{datapar.binary}
  \begin{itemdecl}
template <class L, class R> using datapar_return_type = ...;  // exposition only
template <class T, class Abi, class U>
datapar_return_type<datapar<T, Abi>, U> operator+ (datapar<T, Abi>, const U &);
template <class T, class Abi, class U>
datapar_return_type<datapar<T, Abi>, U> operator- (datapar<T, Abi>, const U &);
template <class T, class Abi, class U>
datapar_return_type<datapar<T, Abi>, U> operator* (datapar<T, Abi>, const U &);
template <class T, class Abi, class U>
datapar_return_type<datapar<T, Abi>, U> operator/ (datapar<T, Abi>, const U &);
template <class T, class Abi, class U>
datapar_return_type<datapar<T, Abi>, U> operator% (datapar<T, Abi>, const U &);
template <class T, class Abi, class U>
datapar_return_type<datapar<T, Abi>, U> operator& (datapar<T, Abi>, const U &);
template <class T, class Abi, class U>
datapar_return_type<datapar<T, Abi>, U> operator| (datapar<T, Abi>, const U &);
template <class T, class Abi, class U>
datapar_return_type<datapar<T, Abi>, U> operator^ (datapar<T, Abi>, const U &);
template <class T, class Abi, class U>
datapar_return_type<datapar<T, Abi>, U> operator<<(datapar<T, Abi>, const U &);
template <class T, class Abi, class U>
datapar_return_type<datapar<T, Abi>, U> operator>>(datapar<T, Abi>, const U &);
template <class T, class Abi, class U>
datapar_return_type<datapar<T, Abi>, U> operator+ (const U &, datapar<T, Abi>);
template <class T, class Abi, class U>
datapar_return_type<datapar<T, Abi>, U> operator- (const U &, datapar<T, Abi>);
template <class T, class Abi, class U>
datapar_return_type<datapar<T, Abi>, U> operator* (const U &, datapar<T, Abi>);
template <class T, class Abi, class U>
datapar_return_type<datapar<T, Abi>, U> operator/ (const U &, datapar<T, Abi>);
template <class T, class Abi, class U>
datapar_return_type<datapar<T, Abi>, U> operator% (const U &, datapar<T, Abi>);
template <class T, class Abi, class U>
datapar_return_type<datapar<T, Abi>, U> operator& (const U &, datapar<T, Abi>);
template <class T, class Abi, class U>
datapar_return_type<datapar<T, Abi>, U> operator| (const U &, datapar<T, Abi>);
template <class T, class Abi, class U>
datapar_return_type<datapar<T, Abi>, U> operator^ (const U &, datapar<T, Abi>);
template <class T, class Abi, class U>
datapar_return_type<datapar<T, Abi>, U> operator<<(const U &, datapar<T, Abi>);
template <class T, class Abi, class U>
datapar_return_type<datapar<T, Abi>, U> operator>>(const U &, datapar<T, Abi>);
  \end{itemdecl}
  \begin{itemdescr}
    \pnum\remarks The return type of these operators shall be deduced according to the following rules:
    \begin{itemize}
      \item If \code{is_datapar_v<U> == true}
        then the return type shall be determined from \type T and \type{U::value_type} according to the following paragraph.
      \item Otherwise, if \type T is integral and \type U is \intt
        the return type shall be \datapar{}\type{<T, Abi>}.
      \item Otherwise, if \type T is integral and \type U is \uint
        the return type shall be \datapar{}\code{<make_unsigned_t<T>, Abi>}.
      \item Otherwise, if \type U is a fundamental arithmetic type or \type U is convertible to \intt
        then the return type shall be determined from \type T and \type U according to the following paragraph.
      \item Otherwise, if \type U is implicitly convertible to \datapar{}\type{<V, Abi>}, where \type V is determined according to standard template type deduction,
        then the return type shall be determined from \type T and \type V according to the following paragraph.
      \item Otherwise, if \type U is implicitly convertible to \datapar{}\type{<T, Abi>},
        the return type shall be \datapar{}\type{<T, Abi>}.
      \item Otherwise no return type is defined (SFINAE).
    \end{itemize}

    \pnum\remarks Given the types \type T and \type{Abi} from the class template argument list and a third type \type U determined by the rules of the previous paragraph a return type is deduced according to the following rules:
    \begin{itemize}
      \item If \type U is not a fundamental arithmetic type then the return type shall be \datapar{}\type{<T, Abi>}.
      \item Otherwise, if at least one of the  types \type T and \type U is a floating-point type
        the return type shall be \datapar{}\type{<decltype(T() + U()), Abi>}.
      \item Otherwise, if \code{sizeof(T) < sizeof(U)} the return type shall be \datapar{}\type{<U, Abi>}.
      \item Otherwise, if \code{sizeof(T) > sizeof(U)} the return type shall be \datapar{}\type{<T, Abi>}.
      \item Otherwise, the type \type T or \type U that is farthest back in the list of \textit{standard integer types} (cf. [basic.fundamental]) is used as type \type V and
        the return type shall be \datapar{}\type{<V, Abi>} if both types \type T and \type U are signed, otherwise the return type shall be \datapar{}\type{<make_unsigned_t<V>, Abi>}.
    \end{itemize}

    \pnum\remarks Each of these operators only participates in overload resolution if all of the following hold:
    \begin{itemize}
      \item The indicated operator can be applied to objects of type \type{R::value_type}, with \type R the return type.
      \item \datapar{}\type{<T, Abi>} is implicitly convertible to the return type.
      \item \type U is implicitly convertible to the return type.
    \end{itemize}

    \pnum\remarks The operators with \type{const U \&} as first parameter shall not participate in overload resolution if \code{is_datapar_v<U> == true}.

    \pnum\effects Both arguments are first converted to the return type.
      Each of these operators subsequently performs the indicated operation component-wise on each of the elements of the first argument and the corresponding element of the second argument.
    \pnum\returns An object containing the results of the component-wise operator application.
  \end{itemdescr}

  \wgSubsubsection{\datapar compound assignment}{datapar.cassign}
  \begin{itemdecl}
template <class T, class Abi, class U> datapar<T, Abi> &operator+= (datapar<T, Abi> &, const U &);
template <class T, class Abi, class U> datapar<T, Abi> &operator-= (datapar<T, Abi> &, const U &);
template <class T, class Abi, class U> datapar<T, Abi> &operator*= (datapar<T, Abi> &, const U &);
template <class T, class Abi, class U> datapar<T, Abi> &operator/= (datapar<T, Abi> &, const U &);
template <class T, class Abi, class U> datapar<T, Abi> &operator%= (datapar<T, Abi> &, const U &);
template <class T, class Abi, class U> datapar<T, Abi> &operator&= (datapar<T, Abi> &, const U &);
template <class T, class Abi, class U> datapar<T, Abi> &operator|= (datapar<T, Abi> &, const U &);
template <class T, class Abi, class U> datapar<T, Abi> &operator^= (datapar<T, Abi> &, const U &);
template <class T, class Abi, class U> datapar<T, Abi> &operator<<=(datapar<T, Abi> &, const U &);
template <class T, class Abi, class U> datapar<T, Abi> &operator>>=(datapar<T, Abi> &, const U &);
  \end{itemdecl}
  \begin{itemdescr}
    \pnum\remarks Each of these operators only participates in overload resolution if all of the following hold:
    \begin{itemize}
      \item The indicated operator can be applied to objects of type \type{datapar_return_type<datapar<T, Abi>, U>::value_type}.
      \item \datapar{}\type{<T, Abi>} is implicitly convertible to \type{datapar_return_type<datapar<T, Abi>, U>}.
      \item \type U is implicitly convertible to \type{datapar_return_type<datapar<T, Abi>, U>}.
      \item \type{datapar_return_type<datapar<T, Abi>, U>} is implicitly convertible to \datapar{}\type{<T, Abi>}.
    \end{itemize}
    \pnum\effects Each of these operators performs the indicated operation component-wise on each of the elements of the first argument and the corresponding element of the second argument after conversion to \datapar{}\code{<T, Abi>}.
    \pnum\returns A reference to the first argument.
  \end{itemdescr}

  \wgSubsubsection{\datapar logical operators}{datapar.comparison}

  \wgSubsubsection{\datapar transcendentals}{datapar.transcend}

  \wgSubsection{Class template \mask}{datapar.mask}
  \lstinputlisting[]{mask.cpp}

\end{wgText}

\end{wgText}

\end{document}
% vim: sw=2 sts=2 ai et tw=0
