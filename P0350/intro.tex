\section{Introduction}

Parallel Algorithms enable implementations of the existing STL algorithms to use non-sequential semantics when executing the user-supplied code (explicit callable or implicit operator call).
The first argument to the algorithm function determines this change in execution semantics via an \emph{execution policy}.
This paper introduces a new execution policy, called \simdEP\footnote{
  An alternative suggestion for the name is \code{execution::simd_type}.
}.
\simdEP requires user-provided function objects to be callable with \simd[<T, Abi>] arguments instead of the \type T arguments the \seqEP variant would use.
The algorithm therefore processes chunks of \simd[<T, Abi>::size()] objects concurrently.
The execution order of the chunks retains the sequential semantics of the non-parallel algorithms.

As a consequence, the applicability of the execution policy is limited to iterators where \code{Iterator::value_type} is a vectorizable type \cite[{[parallel.simd.general]}]{N4744}.
A future extension of \simd may lift this restriction by allowing certain (or all) user-defined types as first template argument to \simd.
A different conceivable extensions is a recursive destructuring applied inside the algorithm, subsequent creation of a corresponding number of \simd objects, and a call to the function object with a corresponding number of arguments.
(E.g. application of an algorithm on \code{std::vector<std::pair<float, float>>} calls the function object with \code{simd<float>, simd<\MayBreak{}float>} instead of \code{simd<std::pair<float, float>>}.)

